\begin{frame}
  \frametitle{Propagation: The \emph{Edge-Finding} Rule}

\vfill  

\begin{itemize}
  \item Make deductions w.r.t. \nooverlap  and project them on the variables' bounds
\end{itemize}

\vfill

\begin{columns}

\column{.35\textwidth}

\vbox to .5\textheight{%

\uncover<2->{
\begin{itemize}
  \item Suppose that a task in $\Omega$ is not last

\vfill
\begin{itemize}
  \item Make a hypothetical overload test
  \end{itemize}

\vfill\uncover<3->{
  \item Proof by contradiction
  \vfill
  \begin{itemize}
    \item That task \myemph{must be last} in $\Omega$
  \end{itemize}
  }

\vfill\uncover<4->{
  \item Symmetrical backward pass
  }

  \vfill\uncover<4->{
  \item Algorithm in $O(n \log n)$ \citation{Vil\`{i}m 04}
  }
  \end{itemize}
}
}%

\column{.65\textwidth}


\begin{center}
\begin{colorschedfigure}{.7}
\PrintGrid{11}{4}
\node[] (es) at (2.3, 2) {{\scriptsize \begin{tabular}{c}\textcolor{blue!50!black}{earliest}\\\textcolor{blue!50!black}{start}\end{tabular}}};
\node[] (ls) at (7.7, 2) {{\scriptsize \begin{tabular}{c}\textcolor{blue!50!black}{latest}\\\textcolor{blue!50!black}{start}\end{tabular}}};
\node[] (ee) at (5.3, 2) {{\scriptsize \begin{tabular}{c}\textcolor{red!50!black}{earliest}\\\textcolor{red!50!black}{end}\end{tabular}}};
\node[] (le) at (10.7, 2) {{\scriptsize \begin{tabular}{c}\textcolor{red!50!black}{latest}\\\textcolor{red!50!black}{end}\end{tabular}}};
\draw[->, shorten >=3pt,color=blue!50!black] (ls) -- (7.5,0);
\Pruning{10}{11}{1}{0}
\Pruning{6}{11}{2}{1}
\Pruning{8}{11}{1}{3}
\Pruning{0}{2}{1}{0}
\Pruning{0}{3}{1}{1}
\Pruning{0}{2}{1}{2}


\only<1-2>{
\draw[->, shorten >=3pt,color=red!50!black] (ee) -- (5.5,0);
\draw[->, shorten >=3pt,color=blue!50!black] (es) -- (2.5,0);
}

\only<3->{
\draw[->, shorten >=3pt, color=red!50!black] (ee.south east) to[bend left=10] (9.5,0);
\draw[->, shorten >=3pt,color=blue!50!black] (es.south east) to[bend left=10] (6.5,0);
}



\draw[->, shorten >=3pt,color=red!50!black] (le) -- (10.5,0);

\uncover<1-2>{
\PreemptiveVariableTask{2}{8}{5}{11}{3}{1}{0}{A}{}
}
\uncover<1-3>{
\PreemptiveVariableTask{0}{7}{2}{9}{2}{1}{3}{A}{}
}
\PreemptiveVariableTask{2}{5}{4}{7}{2}{1}{2}{A}{}
\PreemptiveVariableTask{3}{5}{5}{7}{2}{1}{1}{A}{}

\uncover<2>{
\draw[color=black, thick, rounded corners] (-\tasksep,\tasksep) rectangle (11+\tasksep,-1-\tasksep);
  % \PruningUnder{8}{10}{4}{0}
  \Pruning{8}{10}{1}{0}
}

\uncover<3->{
\PreemptiveVariableTask{6}{8}{9}{11}{3}{1}{0}{A}{}
\PruningOver{2}{6}{1}{0}
}

\uncover<4->{
% \GroundTask[]{0}{3}{1}{3}{A}{}
\PreemptiveVariableTask{0}{3}{1}{4}{2}{1}{3}{A}{}
\PruningUnder{4}{8}{1}{3}
}


\end{colorschedfigure}

\end{center}

\end{columns}

\vfill  
      
\end{frame}


\begin{frame}
\frametitle{Overload Checking -- Theta Tree}
\begin{itemize}
  \item Order the tasks by non-decreasing \memph{due date} to compute \memph{$\leftcut{{\cal T}}{j}$} for all \memph{$j \in {\cal T}$}
  \item Order the tasks by non-decreasing \memph{release date} to compute \memph{$\ect{\leftcut{{\cal T}}{j}}$}
\end{itemize}

\begin{myblock}{Solutions}
  \begin{itemize}
    \item Theta tree \citation{Vil\'{i}m et al. 04}
    \begin{itemize}
      \item Explore nested sets of tasks in any order (here non-decreasing due dates)
      \item Incrementally compute a property (here \memph{$\ect{\leftcut{{\cal T}}{j}}$}) requiring another order
    \end{itemize}
  \end{itemize}
\end{myblock}
\end{frame}


\begin{frame}
\frametitle{Theta Tree}
\begin{columns}

\column{.5\textwidth}
  \begin{colorschedfigure}{.3}
    \uncover<1>{
\PrintTics{0,5,...,26}{1.0000}
\PrintGrid{25}{6}
\LeftExtensibleVariableTask{0.000000}{18.750000}{6.250000}{6.250000}{1}{0}{C}{A}
\LeftExtensibleVariableTask{5.000000}{12.500000}{5.000000}{5.000000}{1}{1}{C}{B}
\LeftExtensibleVariableTask{2.500000}{15.000000}{3.750000}{3.750000}{1}{2}{C}{C}
\LeftExtensibleVariableTask{8.750000}{17.500000}{5.000000}{5.000000}{1}{3}{C}{D}
\LeftExtensibleVariableTask{3.750000}{25.000000}{7.500000}{7.500000}{1}{4}{C}{E}
\LeftExtensibleVariableTask{12.500000}{20.000000}{3.750000}{3.750000}{1}{5}{C}{F}
}
\uncover<2>{
\PrintTics{0,5,...,26}{1.0000}
\PrintGrid{25}{6}
\LeftExtensibleVariableTask{0.000000}{18.750000}{6.250000}{6.250000}{1}{0}{C}{A}
\LeftExtensibleVariableTask{2.500000}{15.000000}{3.750000}{3.750000}{1}{1}{C}{C}
\LeftExtensibleVariableTask{3.750000}{25.000000}{7.500000}{7.500000}{1}{2}{C}{E}
\LeftExtensibleVariableTask{5.000000}{12.500000}{5.000000}{5.000000}{1}{3}{C}{B}
\LeftExtensibleVariableTask{8.750000}{17.500000}{5.000000}{5.000000}{1}{4}{C}{D}
\LeftExtensibleVariableTask{12.500000}{20.000000}{3.750000}{3.750000}{1}{5}{C}{F}
}
\uncover<3->{
\PrintTics{0,5,...,26}{1.0000}
\PrintGrid{25}{6}
}
\uncover<3->{\LeftExtensibleVariableTask{5.000000}{12.500000}{5.000000}{5.000000}{1}{0}{C}{B}}
\uncover<4->{\LeftExtensibleVariableTask{2.500000}{15.000000}{3.750000}{3.750000}{1}{1}{C}{C}}
\uncover<5->{\LeftExtensibleVariableTask{8.750000}{17.500000}{5.000000}{5.000000}{1}{2}{C}{D}}
\uncover<6->{\LeftExtensibleVariableTask{0.000000}{18.750000}{6.250000}{6.250000}{1}{3}{C}{A}}
\uncover<7->{\LeftExtensibleVariableTask{12.500000}{20.000000}{3.750000}{3.750000}{1}{4}{C}{F}}
\uncover<8->{\LeftExtensibleVariableTask{3.750000}{25.000000}{7.500000}{7.500000}{1}{5}{C}{E}}

  \end{colorschedfigure}

\uncover<10->{
\begin{itemize}
  \item Explanation $s_A \geq 0 \land s_B \geq 0 \land \ldots \land s_F \geq 0$ and $e_A \leq 25 \land e_B \leq 25 \land \ldots \land e_F \leq 25$
  \vfill
  \uncover<11->{
  \item There can be ``holes''
  }
\end{itemize}
}

\uncover<11->{
   \begin{colorschedfigure}{.3}
    
\PrintTics{0,5,...,26}{1.0000}
\PrintGrid{25}{6}
\LeftExtensibleVariableTask{1.500000}{12.500000}{4.000000}{4.000000}{1}{0}{C}{B}
\LeftExtensibleVariableTask{1.500000}{15.000000}{2.0000}{2.0000}{1}{1}{C}{C}
\LeftExtensibleVariableTask{13.0000}{22.00000}{3.000000}{3.000000}{1}{2}{C}{D}
\LeftExtensibleVariableTask{0.000000}{19.0000}{3.0000}{3.0000}{1}{3}{C}{A}
\LeftExtensibleVariableTask{15.00000}{23.000000}{4.0000}{4.0000}{1}{4}{C}{F}
\LeftExtensibleVariableTask{12.0000}{25.000000}{5.00000}{5.00000}{1}{5}{C}{E}
  \end{colorschedfigure}
  }

\column{.5\textwidth}
  \uncover<2->{
  \begin{downthreelvltree}
    \thetaroot{\only<3>{\textcolor{red!85!black}{4}}\only<4>{\textcolor{red!85!black}{7}}\only<5>{\textcolor{red!85!black}{11}}\only<6>{\textcolor{red!85!black}{16}}\only<7>{\textcolor{red!85!black}{19}}\only<8>{\textcolor{red!85!black}{25}}}{\only<3>{\textcolor{red!85!black}{8}}\only<4>{\textcolor{red!85!black}{9}}\only<5>{\textcolor{red!85!black}{13}}\only<6>{\textcolor{red!85!black}{16}}\only<7>{\textcolor{red!85!black}{19}}\only<8>{\textcolor{red!85!black}{25}}}{3}
child{
\ttnode{\only<3>{\textcolor{red!85!black}{4}}\only<4>{\textcolor{red!85!black}{7}}\only<5>{7}\only<6>{\textcolor{red!85!black}{12}}\only<7>{12}\only<8>{\textcolor{red!85!black}{18}}}{\only<3>{\textcolor{red!85!black}{8}}\only<4>{\textcolor{red!85!black}{9}}\only<5>{9}\only<6>{\textcolor{red!85!black}{12}}\only<7>{12}\only<8>{\textcolor{red!85!black}{18}}}{3}
child{
\ttnode{\only<4>{\textcolor{red!85!black}{3}}\only<5>{3}\only<6>{\textcolor{red!85!black}{8}}\only<7>{8}\only<8>{8}}{\only<4>{\textcolor{red!85!black}{5}}\only<5>{5}\only<6>{\textcolor{red!85!black}{8}}\only<7>{8}\only<8>{8}}{4}
child{
\ttnode{\only<6>{\textcolor{red!85!black}{5}}\only<7>{5}\only<8>{5}}{\only<6>{\textcolor{red!85!black}{5}}\only<7>{5}\only<8>{5}}{6}
edge from parent
       node[kant, left, yshift=1mm, pos=.6] {A}
}
child{
\ttnode{\only<4>{\textcolor{red!85!black}{3}}\only<5>{3}\only<6>{3}\only<7>{3}\only<8>{3}}{\only<4>{\textcolor{red!85!black}{5}}\only<5>{5}\only<6>{5}\only<7>{5}\only<8>{5}}{4}
edge from parent
       node[kant, right, yshift=1mm, pos=.6] {C}
}
}
child{
\ttnode{\only<3>{\textcolor{red!85!black}{4}}\only<4>{4}\only<5>{4}\only<6>{4}\only<7>{4}\only<8>{\textcolor{red!85!black}{10}}}{\only<3>{\textcolor{red!85!black}{8}}\only<4>{8}\only<5>{8}\only<6>{8}\only<7>{8}\only<8>{\textcolor{red!85!black}{13}}}{3}
child{
\ttnode{\only<8>{\textcolor{red!85!black}{6}}}{\only<8>{\textcolor{red!85!black}{9}}}{8}
edge from parent
       node[kant, left, yshift=1mm, pos=.6] {E}
}
child{
\ttnode{\only<3>{\textcolor{red!85!black}{4}}\only<4>{4}\only<5>{4}\only<6>{4}\only<7>{4}\only<8>{4}}{\only<3>{\textcolor{red!85!black}{8}}\only<4>{8}\only<5>{8}\only<6>{8}\only<7>{8}\only<8>{8}}{3}
edge from parent
       node[kant, right, yshift=1mm, pos=.6] {B}
}
}
}
child{
\ttnode{\only<5>{\textcolor{red!85!black}{4}}\only<6>{4}\only<7>{\textcolor{red!85!black}{7}}\only<8>{7}}{\only<5>{\textcolor{red!85!black}{11}}\only<6>{11}\only<7>{\textcolor{red!85!black}{14}}\only<8>{14}}{5}
child{
\ttnode{\only<5>{\textcolor{red!85!black}{4}}\only<6>{4}\only<7>{\textcolor{red!85!black}{7}}\only<8>{7}}{\only<5>{\textcolor{red!85!black}{11}}\only<6>{11}\only<7>{\textcolor{red!85!black}{14}}\only<8>{14}}{5}
child{
\ttnode{\only<5>{\textcolor{red!85!black}{4}}\only<6>{4}\only<7>{4}\only<8>{4}}{\only<5>{\textcolor{red!85!black}{11}}\only<6>{11}\only<7>{11}\only<8>{11}}{5}
edge from parent
       node[kant, left, yshift=1mm, pos=.6] {D}
}
child{
\ttnode{\only<7>{\textcolor{red!85!black}{3}}\only<8>{3}}{\only<7>{\textcolor{red!85!black}{13}}\only<8>{13}}{7}
edge from parent
       node[kant, right, yshift=1mm, pos=.6] {F}
}
}
};

  \end{downthreelvltree}
  }

\end{columns}
\end{frame}



\begin{frame}
  \frametitle{Propagation: The \emph{Edge-Finding} Rule}

\vfill 


\vfill  
      
\end{frame}

