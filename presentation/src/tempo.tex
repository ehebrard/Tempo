%----------------------------------------------------------------------------------------
%	PACKAGES AND THEMES
%----------------------------------------------------------------------------------------
\documentclass[aspectratio=169,xcolor=dvipsnames]{beamer}
\usetheme{laas}

\definecolor{bittersweet}{rgb}{1.0, 0.44, 0.37}

\definecolor{airforceblue}{rgb}{0.36, 0.54, 0.66}
\definecolor{burntsienna}{rgb}{0.91, 0.45, 0.32}
\definecolor{darkseagreen}{rgb}{0.56, 0.74, 0.56}

\definecolor{bostonuniversityred}{rgb}{0.8, 0.0, 0.0}
\definecolor{islamicgreen}{rgb}{0.0, 0.56, 0.0}
\definecolor{darkseagreen}{rgb}{0.56, 0.74, 0.56}
\definecolor{darkcyan}{rgb}{0.0, 0.55, 0.55}
\definecolor{cerulean}{rgb}{0.0, 0.48, 0.65}
\definecolor{coolblack}{rgb}{0.0, 0.18, 0.39}
\definecolor{darkcerulean}{rgb}{0.03, 0.27, 0.49}

% \definecolor{alizarin}{rgb}{0.82, 0.1, 0.26}
% \definecolor{emerald}{rgb}{0.31, 0.78, 0.47}

\definecolor{emerald}{rgb}{0.01, 1, 0.14}
\definecolor{alizarin}{rgb}{1.0, 0.01, 0.14}

\definecolor{pearl}{rgb}{0.94, 0.92, 0.84}

% \alt<2->{\newcolumntype{R}{>{\columncolor{burntsienna}}r}}{\newcolumntype{R}{r}}
% \alt<3->{\newcolumntype{B}{>{\columncolor{airforceblue}}r}}{\newcolumntype{B}{r}}
% \alt<4->{\newcolumntype{G}{>{\columncolor{darkseagreen}}r}}{\newcolumntype{G}{r}}

\newcommand{\backupbegin}{
   \newcounter{finalframe}
   \setcounter{finalframe}{\value{framenumber}}
}
\newcommand{\backupend}{
   \setcounter{framenumber}{\value{finalframe}}
}


\newcommand<>{\HiLi}{\onslide#1{\leavevmode\rlap{\hbox to \hsize{\color{yellow!50}\leaders\hrule height .8\baselineskip depth .5ex\hfill}}}}





% Use the postscript times font!
% \usepackage{times}

\usepackage{array}
\newcolumntype{L}[1]{>{\raggedright\let\newline\\\arraybackslash\hspace{0pt}}m{#1}}
\newcolumntype{C}[1]{>{\centering\let\newline\\\arraybackslash\hspace{0pt}}m{#1}}
\newcolumntype{R}[1]{>{\raggedleft\let\newline\\\arraybackslash\hspace{0pt}}m{#1}}


\usepackage{xcolor}
\usepackage{soul}
\usepackage[utf8]{inputenc}
\usepackage[small]{caption}
\usepackage{url}

\usepackage{xspace}
\usepackage[ruled,vlined]{algorithm2e}
\usepackage{amssymb}
\usepackage{amsthm}
\usepackage{amsmath}
\usepackage{longtable}
\usepackage{booktabs}
\usepackage{multirow}
\usepackage{tikz}
\usepackage{fp}
\usepackage{pgfplots}
\usepackage{subfig}
\usetikzlibrary{arrows,shadows,fit,calc,positioning,decorations.pathreplacing,matrix,shapes,petri,topaths,fadings,mindmap,backgrounds}


\usepackage{animate}
\usepackage{marvosym}

\usepackage{forest}

\usepackage{colortbl}

%
% %\usepackage{pdfrender}
%
%
% \usepackage{tikz}
% \usepackage{fp}
% \usetikzlibrary{arrows,shadows,fit,calc,positioning,decorations.pathmorphing,decorations.pathreplacing,matrix,shapes,petri,topaths,fadings,mindmap,backgrounds,patterns}
% %\usepackage{tkz-berge}

\usepackage{tikzsymbols}


% \newcommand\solver[1]{\texttt{#1}}
% \def\wlmc{\solver{wlmc}}
% \def\cdcl{\solver{cdcl-quick}}
% \def\dyn{\solver{cdcl-dyn}}
% \def\dscdcl{\solver{ds-cdcl-quick}}
% \def\dsdyn{\solver{ds-cdcl-dyn}}
% \def\weak{\solver{cdcl}}
% \def\cliquer{\solver{cliquer}}
% \def\nopruning{\cdcl$\setminus$\solver{pru}}
% \def\nodominance{\cdcl$\setminus$\solver{dom}}
% \def\nolearning{\cdcl$\setminus$\solver{learn}}
% \def\weaksol{\weak$^+$}
% \def\dynsol{\dyn$^+$}
% \def\otclique{\solver{OTClique}}
% \def\tavares{\solver{Tavares}}

% \usetheme{laas}





\definecolor{cit}{rgb}{0.1,0.6,0.2}
\definecolor{mc}{rgb}{0.73,0.13,0.13}
\definecolor{bc}{rgb}{0.83,0.13,0.13}
\definecolor{bst}{rgb}{0.13,0.13,0.73}
\definecolor{gr}{rgb}{0.5,0.5,0.5}
\definecolor{emp}{rgb}{0,0.25,0.4}


\tikzstyle{npvc}=[circle, fill=black!50!MidnightBlue, text=white, drop shadow={shadow xshift=.5mm, shadow yshift=-.5mm}, draw=black, font=\scriptsize, minimum width=4mm, inner sep=3pt]
\tikzstyle{npis}=[circle, fill=black!50!Red, text=white, drop shadow={shadow xshift=.5mm, shadow yshift=-.5mm}, draw=black, font=\scriptsize, minimum width=4mm, inner sep=3pt]
\tikzstyle{npone}=[circle, fill=white, drop shadow={shadow xshift=.5mm, shadow yshift=-.5mm}, draw=black, font=\scriptsize, minimum width=4mm, inner sep=3pt]
\tikzstyle{nptwo}=[circle, fill=mc, text=white, drop shadow={shadow xshift=.5mm, shadow yshift=-.5mm}, draw=black, font=\scriptsize, minimum width=4mm, inner sep=3pt]
\tikzstyle{npthree}=[circle, fill=bst, text=white, drop shadow={shadow xshift=.5mm, shadow yshift=-.5mm}, draw=black, font=\scriptsize, minimum width=4mm, inner sep=3pt]
\tikzstyle{npfour}=[circle, fill=Green, text=white, drop shadow={shadow xshift=.5mm, shadow yshift=-.5mm}, draw=black, font=\scriptsize, minimum width=4mm, inner sep=3pt]
\tikzstyle{npfive}=[circle, fill=Dandelion, text=white, drop shadow={shadow xshift=.5mm, shadow yshift=-.5mm}, draw=black, font=\scriptsize, minimum width=4mm, inner sep=3pt]
\tikzstyle{npsix}=[circle, fill=RedOrange, text=white, drop shadow={shadow xshift=.5mm, shadow yshift=-.5mm}, draw=black, font=\scriptsize, minimum width=4mm, inner sep=3pt]
\tikzstyle{npseven}=[circle, fill=Purple, text=white, drop shadow={shadow xshift=.5mm, shadow yshift=-.5mm}, draw=black, font=\scriptsize, minimum width=4mm, inner sep=3pt]


\tikzstyle{assigned}=[thick, circle, draw=black, fill=black!30, font=\scriptsize, minimum width=4mm, inner sep=3pt]




\tikzstyle{uedge}=[thick, draw=black, shorten <=1.5mm, shorten >=1.5mm]
\tikzstyle{redge}=[thick, draw=black, densely dotted, shorten <=1.5mm, shorten >=1.5mm]
\tikzstyle{lbl}=[font=\scriptsize, color=mc!50!black]






\def\citation#1{{\color{cit} [{#1}]}}
\def\invisible#1{{\color{white} {#1}}}
\def\name#1{{\color{bst} {#1}}}
\def\hidden#1{{\color{gr} {#1}}}
\def\math#1{{\color{mc} {#1}}}
\def\cemph#1{{\color{emp} {#1}}}
\def\bemph#1{{\bf {\large {\color{bc} {#1}}}}}
% \def\best#1{{\color{bst} {#1}}}
\def\greened#1{{\color{Green} {#1}}}




\newcommand\cactus[5]{%%
\begin{tikzpicture}[scale=.9]
  \begin{axis}[scatter/classes={
    a={mark=square*,blue!50,draw=blue!50!black},%%
    b={mark=triangle*,red!50,draw=red!50!black},%%
    c={mark=*,black!50!white,draw=black},%%
    d={mark=diamond*,green!50,draw=green!50!black},%%
    e={mark=pentagon*,orange!50,draw=orange!50!black},%%
    f={mark=10-pointed star,draw=blue!60!black},%%
    g={mark=x,draw=black,thick},%%
    h={mark=+,draw=red!50!black},%%
    i={mark=Mercedes star,draw=green!50!black},%%
    j={mark=square,draw=orange!50!black},%%
    k={mark=diamond,draw=red!50!black},%%
    l={mark=o,draw=blue!50!black},%%
    m={mark=asterisk,draw=magenta!50!black},%%
    n={mark=triangle,draw=green!50!black},%%
    o={mark=pentagon,draw=red!50!black},%%
    p={mark=asterisk,draw=black},%%
    q={mark=Mercedes star,draw=magenta!50!black},%%
    r={mark=pentagon*,blue!50,draw=blue!50!black},%%
    s={mark=x,draw=green!50!black},%%
    t={mark=diamond*,red!50,draw=red!50!black}},%%
    mark size=1.5pt,
    ymode=log,
    ylabel=#2,
    xlabel=#1,
    font=\scriptsize,
    {#5}
    ]

    \foreach \bidule in #4 {

    \addplot[scatter, scatter src=explicit symbolic] coordinates {\bidule};

    }
    \legend{#3}
  \end{axis}
\end{tikzpicture}
}



\author{Emmanuel Hebrard}
\institute{LAAS-CNRS, Université de Toulouse}
\title{On How to Have the \textnormal{\emph{Edge}} on CPOptimizer}


% \newcommand{\backupbegin}{
%    \newcounter{finalframe}
%    \setcounter{finalframe}{\value{framenumber}}
% }
% \newcommand{\backupend}{
%    \setcounter{framenumber}{\value{finalframe}}
% }


\tikzset{
    invisible/.style={opacity=0,text opacity=0},
    visible on/.style={alt=#1{}{invisible}},
    alt/.code args={<#1>#2#3}{%
      \alt<#1>{\pgfkeysalso{#2}}{\pgfkeysalso{#3}} % \pgfkeysalso doesn't change the path
    },
}
\forestset{
  visible on/.style={
    for tree={
      /tikz/visible on={#1},
      edge+={/tikz/visible on={#1}}}}}

% \tikzset{% set up for transitions using tikz with beamer overlays
%   invisible/.style={opacity=0,text opacity=0},
%   visible on/.style={alt=#1{}{invisible}},
%   alt/.code args={<#1>#2#3}{%
%     \alt<#1>{\pgfkeysalso{#2}}{\pgfkeysalso{#3}} % \pgfkeysalso doesn't change the path
%   },
%   transparent/.style={opacity=0.1,text opacity=0.1},
%   opaque on/.style={alt=#1{}{transparent}},
%   alerted/.style={color=alerted text.fg},
%   alert on/.style={alt=#1{alerted}{}},
% }
% \forestset{%
%   visible on/.style={%
%     for tree={%
%       /tikz/visible on={#1},
%       edge={/tikz/visible on={#1}}}},
%   opaque on/.style={%
%     for tree={%
%       /tikz/opaque on={#1},
%       edge={/tikz/opaque on={#1}}}},
%   alerted on/.style={%
%     for tree={%
%       /tikz/alerted on={#1},
%       edge={/tikz/alerted on={#1}}}},
% }




\def\numjob{n}
\def\nummachine{m}
\newcommand{\numtask}[1][]{\ensuremath{n_{#1}}}
\def\alljobs{\ensuremath{\mathcal J}}
\def\allmachines{\ensuremath{\mathcal M}}
\newcommand{\job}[1]{\ensuremath{J_{#1}}\xspace}
% \newcommand{\task}[1]{\ensuremath{t_{#1}}\xspace}
\newcommand{\dur}[1]{\ensuremath{p_{#1}}\xspace}
\newcommand{\machine}[1]{\ensuremath{M_{#1}}\xspace}
\def\gid{i}
\def\jid{i}
\def\tid{j}
\def\maid{m}
\def\pid{k}
\def\cmax{C_{\max}}

\def\true{\texttt{true}}
\def\false{\texttt{false}}




% \input{../solvers}

\def\ntask{\ensuremath{n}}
\def\nresource{\ensuremath{k}}
\def\njob{\ensuremath{m}}

\def\ti{\ensuremath{i}}
\def\ji{\ensuremath{l}}
\def\mi{\ensuremath{j}}
\def\alltasks{\ensuremath{\mathcal{T}}}
\def\schedule{\ensuremath{\mathcl{S}}}


\def\atask{\ensuremath{t}}
\def\aresource{\ensuremath{R}}


\usepackage{xintexpr}
\def\roundandprint #1{\xinttheiexpr #1\relax }


\usetikzlibrary{shapes,positioning,patterns,shadows}



\newcommand{\task}[1]{\ensuremath{\atask_{#1}}}
\newcommand{\mach}[1]{\ensuremath{\amach_{#1}}}
\newcommand{\resource}[1]{\ensuremath{\aresource_{#1}}}


\def\itask{\task{1}}
\def\iresource{\resource{1}}

\def\otask{\task{2}}
\def\oresource{\resource{2}}

\def\yatask{\task{3}}
\def\yaresource{\resource{3}}


\newdimen\XCoord
\newdimen\YCoord
\newdimen\ACoord
\newdimen\BCoord
\newcommand*{\ExtractXYCoordinate}[1]{\path (#1); \pgfgetlastxy{\XCoord}{\YCoord};}%
\newcommand*{\ExtractABCoordinate}[1]{\path (#1); \pgfgetlastxy{\ACoord}{\BCoord};}%


\newcommand{\tasksof}[1]{#1}
\newcommand{\resourceof}[1]{\ensuremath{res(#1)}}
\newcommand{\stof}[1]{\ensuremath{s_{#1}}}
\newcommand{\ctof}[1]{\ensuremath{e_{#1}}}

\newcommand{\patvar}[2]{\ensuremath{a^{#2}_{#1}}}
\newcommand{\precvar}[2]{\ensuremath{b_{{#1}{#2}}}}

% \newcommand{\dur}[1]{\ensuremath{p_{#1}}}
\newcommand{\demand}[1]{\ensuremath{c_{#1}}}
\newcommand{\est}[1]{\ensuremath{r_{#1}}}
\newcommand{\lst}[1]{\ensuremath{lst_{#1}}}
\newcommand{\ect}[1]{\ensuremath{ect_{#1}}}
\newcommand{\lct}[1]{\ensuremath{d_{#1}}}
\newcommand{\energy}[1]{\ensuremath{w_{#1}}}
\newcommand{\energyon}[3]{\ensuremath{w_{#1}({#2},{#3})}}
\newcommand{\totalenergy}[2]{\ensuremath{W({#1},{#2})}}
\newcommand{\leftshift}[2]{\ensuremath{\dur{#1}^+({#2})}}
\newcommand{\rightshift}[2]{\ensuremath{\dur{#1}^-({#2})}}

\def\capacity{\ensuremath{C}}

\newcommand{\leftcut}[2]{\ensuremath{{#1}|_{{#2}}}}



  \newcounter{x}
  \newcounter{y}
  \newcounter{px}
  \newcounter{py}
  \newcounter{task}
  \newcounter{time}
	
  \newcounter{resA}
  \newcounter{resB}
  \newcounter{resC}
  \newcounter{resD}
  \newcounter{resE}
  \newcounter{resF}
	\newcounter{resG}
	\newcounter{resH}
	\newcounter{resI}
	\newcounter{resJ}
	\newcounter{resK}
	\newcounter{resL}
	% \newcounter{resAB}
	
  \newcounter{joba}
  \newcounter{jobb}
  \newcounter{jobc}
  \newcounter{jobd}
  \newcounter{jobe}
 	\newcounter{jobf}
	\newcounter{jobg}
	\newcounter{jobh}
	\newcounter{jobi}
	\newcounter{jobj}
	\newcounter{jobk}
	\newcounter{jobl}
	
  \newcounter{crowa}
  \newcounter{crowb}
  \newcounter{crowc}
  \newcounter{crowd}
  \newcounter{crowe}
 	\newcounter{crowf}
	\newcounter{crowg}
	\newcounter{crowh}
	\newcounter{crowi}
	\newcounter{crowj}
	\newcounter{crowk}
	\newcounter{crowl}
	
	% \newcommand{\getId}[1]{\ifthenelse{\equal{#1}{1}}{A}{\ifthenelse{\equal{#1}{2}}{B}{\ifthenelse{\equal{#1}{3}}{C}{\ifthenelse{\equal{#1}{4}}{D}{\ifthenelse{\equal{#1}{5}}{E}{\ifthenelse{\equal{#1}{6}}{F}{\ifthenelse{\equal{#1}{7}}{G}{\ifthenelse{\equal{#1}{8}}{H}{\ifthenelse{\equal{#1}{9}}{I}{\ifthenelse{\equal{#1}{10}}{J}{\ifthenelse{\equal{#1}{11}}{K}{\ifthenelse{\equal{#1}{12}}{L}{\ifthenelse{\equal{#1}{13}}{M}{\ifthenelse{\equal{#1}{14}}{N}{\ifthenelse{\equal{#1}{15}}{O}{\ifthenelse{\equal{#1}{16}}{P}{}}}}}}}}}}}}}}}}}
	%
	
	\newcommand{\getStyle}[1]{stres#1}
	\newcommand{\getExtStyle}[1]{extst#1}
	\newcommand{\getResource}[1]{res#1}
	\newcommand{\getJob}[1]{job#1}
	\newcommand{\getRow}[1]{crow#1}
	
	
\newcommand\getmytransformmatrix{%
  \pgfgettransformentries{\mya}{\myb}{\myc}{\myd}{\mys}{\myt}%
% coordinate (x,y) is transformed to (ax + by + s, cx + dy + t)  
}


\newenvironment{bwschedfigure}[1]{%
  \begin{tikzpicture}[scale={#1},
      stresA/.style={top color=white,bottom color=white!50!black},
      stresB/.style={color=black, fill=white},
      stresC/.style={color=black, pattern color=white!50!black, pattern=north east lines },
      stresD/.style={color=black, pattern color=white!50!black, pattern=crosshatch },
      stresE/.style={color=black, pattern color=white!50!black, pattern=grid },
      stresF/.style={color=black, pattern color=white!50!black, pattern=vertical lines },
      stresG/.style={color=black, pattern color=white!50!black, pattern=dots },
      stresH/.style={dashed},
    ]
}{%
  \end{tikzpicture}
}

\definecolor{blue(munsell)}{rgb}{0.0, 0.5, 0.69}
\definecolor{airforceblue}{rgb}{0.36, 0.54, 0.66}
% \definecolor{blue(ncs)}{rgb}{0.0, 0.53, 0.74}
\definecolor{egyptianblue}{rgb}{0.06, 0.2, 0.65}


\definecolor{cadmiumred}{rgb}{0.89, 0.0, 0.13}
\definecolor{carnelian}{rgb}{0.7, 0.11, 0.11}


\definecolor{chromeyellow}{rgb}{1.0, 0.65, 0.0}
\definecolor{chocolate(web)}{rgb}{0.82, 0.41, 0.12}


\definecolor{green(pigment)}{rgb}{0.0, 0.65, 0.31}
\definecolor{darkgreen}{rgb}{0.0, 0.2, 0.13}


\definecolor{platinum}{rgb}{0.9, 0.89, 0.89}
\definecolor{onyx}{rgb}{0.06, 0.06, 0.06}


\definecolor{cinnamon}{rgb}{0.82, 0.41, 0.12}
\definecolor{darkbrown}{rgb}{0.4, 0.26, 0.13}


\definecolor{verdigris}{rgb}{0.26, 0.7, 0.68}
\definecolor{darkcyan}{rgb}{0.0, 0.55, 0.55}


\definecolor{darkpastelpurple}{rgb}{0.59, 0.44, 0.84}
\definecolor{byzantium}{rgb}{0.44, 0.16, 0.39}

\definecolor{brinkpink}{rgb}{0.98, 0.38, 0.5}
\definecolor{burgundy}{rgb}{0.5, 0.0, 0.13}


\definecolor{bostonuniversityred}{rgb}{0.8, 0.0, 0.0}

\definecolor{deepchampagne}{rgb}{0.98, 0.84, 0.65}


\def\tasksep{.1}



\makeatletter
\newenvironment{colorfadedschedfigure}[2]{%
  \begin{tikzpicture}[scale={#1},
stresA/.style={color=egyptianblue, fill=white!#2!blue(munsell)},
stresAA/.style={color=egyptianblue, fill=white!#2!blue(munsell)},
stresAB/.style={color=egyptianblue, top color=white!#2!blue(munsell), bottom color=white!#2!cadmiumred},
stresAC/.style={color=egyptianblue, bottom color=white!#2!chromeyellow, top color=white!#2!blue(munsell)},
stresAD/.style={color=egyptianblue, bottom color=white!#2!green(pigment), top color=white!#2!blue(munsell)},
stresAE/.style={color=onyx, fill=white!#2!blue(munsell)},
stresAF/.style={color=darkbrown, fill=white!#2!blue(munsell)},
stresAG/.style={color=darkcyan, fill=white!#2!blue(munsell)},
stresAH/.style={color=byzantium, fill=white!#2!blue(munsell)},
stresAI/.style={color=burgundy, fill=white!#2!blue(munsell)},
stresB/.style={color=carnelian, fill=white!#2!cadmiumred},
stresAB/.style={color=egyptianblue, top color=white!#2!cadmiumred, bottom color=white!#2!blue(munsell)},
stresBA/.style={color=egyptianblue, fill=white!#2!cadmiumred},
stresBB/.style={color=carnelian, fill=white!#2!cadmiumred},
stresBC/.style={color=chocolate(web), fill=white!#2!cadmiumred},
stresBD/.style={color=darkgreen, bottom color=white!#2!green(pigment), top color=white!#2!cadmiumred},
stresBE/.style={color=onyx, fill=white!#2!cadmiumred},
stresBF/.style={color=darkbrown, fill=white!#2!cadmiumred},
stresBG/.style={color=darkcyan, fill=white!#2!cadmiumred},
stresBH/.style={color=byzantium, fill=white!#2!cadmiumred},
stresBI/.style={color=burgundy, fill=white!#2!cadmiumred},
stresC/.style={color=chocolate(web), fill=white!#2!chromeyellow},
stresCA/.style={color=egyptianblue, fill=white!#2!chromeyellow},
stresCB/.style={color=carnelian, fill=white!#2!chromeyellow},
stresCC/.style={color=chocolate(web), fill=white!#2!chromeyellow},
stresCD/.style={color=darkgreen, fill=white!#2!chromeyellow},
stresCE/.style={color=onyx, fill=white!#2!chromeyellow},
stresCF/.style={color=darkbrown, fill=white!#2!chromeyellow},
stresCG/.style={color=darkcyan, fill=white!#2!chromeyellow},
stresCH/.style={color=byzantium, fill=white!#2!chromeyellow},
stresCI/.style={color=burgundy, fill=white!#2!chromeyellow},
stresD/.style={color=darkgreen, fill=white!#2!green(pigment)},
stresDA/.style={color=egyptianblue, fill=white!#2!green(pigment)},
stresDB/.style={color=carnelian, fill=white!#2!green(pigment)},
stresDC/.style={color=chocolate(web), fill=white!#2!green(pigment)},
stresDD/.style={color=darkgreen, fill=white!#2!green(pigment)},
stresDE/.style={color=onyx, fill=white!#2!green(pigment)},
stresDF/.style={color=darkbrown, fill=white!#2!green(pigment)},
stresDG/.style={color=darkcyan, fill=white!#2!green(pigment)},
stresDH/.style={color=byzantium, fill=white!#2!green(pigment)},
stresDI/.style={color=burgundy, fill=white!#2!green(pigment)},
stresE/.style={color=onyx, fill=white!#2!platinum},
stresEA/.style={color=egyptianblue, fill=white!#2!platinum},
stresEB/.style={color=carnelian, fill=white!#2!platinum},
stresEC/.style={color=chocolate(web), fill=white!#2!platinum},
stresED/.style={color=darkgreen, fill=white!#2!platinum},
stresEE/.style={color=onyx, fill=white!#2!platinum},
stresEF/.style={color=darkbrown, fill=white!#2!platinum},
stresEG/.style={color=darkcyan, fill=white!#2!platinum},
stresEH/.style={color=byzantium, fill=white!#2!platinum},
stresEI/.style={color=burgundy, fill=white!#2!platinum},
stresF/.style={color=darkbrown, fill=white!#2!cinnamon},
stresFA/.style={color=egyptianblue, fill=white!#2!cinnamon},
stresFB/.style={color=carnelian, fill=white!#2!cinnamon},
stresFC/.style={color=chocolate(web), fill=white!#2!cinnamon},
stresFD/.style={color=darkgreen, fill=white!#2!cinnamon},
stresFE/.style={color=onyx, fill=white!#2!cinnamon},
stresFF/.style={color=darkbrown, fill=white!#2!cinnamon},
stresFG/.style={color=darkcyan, fill=white!#2!cinnamon},
stresFH/.style={color=byzantium, fill=white!#2!cinnamon},
stresFI/.style={color=burgundy, fill=white!#2!cinnamon},
stresG/.style={color=darkcyan, fill=white!#2!verdigris},
stresGA/.style={color=egyptianblue, fill=white!#2!verdigris},
stresGB/.style={color=carnelian, fill=white!#2!verdigris},
stresGC/.style={color=chocolate(web), fill=white!#2!verdigris},
stresGD/.style={color=darkgreen, fill=white!#2!verdigris},
stresGE/.style={color=onyx, fill=white!#2!verdigris},
stresGF/.style={color=darkbrown, fill=white!#2!verdigris},
stresGG/.style={color=darkcyan, fill=white!#2!verdigris},
stresGH/.style={color=byzantium, fill=white!#2!verdigris},
stresGI/.style={color=burgundy, fill=white!#2!verdigris},
stresH/.style={color=byzantium, fill=white!#2!darkpastelpurple},
stresHA/.style={color=egyptianblue, fill=white!#2!darkpastelpurple},
stresHB/.style={color=carnelian, fill=white!#2!darkpastelpurple},
stresHC/.style={color=chocolate(web), fill=white!#2!darkpastelpurple},
stresHD/.style={color=darkgreen, fill=white!#2!darkpastelpurple},
stresHE/.style={color=onyx, fill=white!#2!darkpastelpurple},
stresHF/.style={color=darkbrown, fill=white!#2!darkpastelpurple},
stresHG/.style={color=darkcyan, fill=white!#2!darkpastelpurple},
stresHH/.style={color=byzantium, fill=white!#2!darkpastelpurple},
stresHI/.style={color=burgundy, fill=white!#2!darkpastelpurple},
stresI/.style={color=burgundy, fill=white!#2!brinkpink},
stresIA/.style={color=egyptianblue, fill=white!#2!brinkpink},
stresIB/.style={color=carnelian, fill=white!#2!brinkpink},
stresIC/.style={color=chocolate(web), fill=white!#2!brinkpink},
stresID/.style={color=darkgreen, fill=white!#2!brinkpink},
stresIE/.style={color=onyx, fill=white!#2!brinkpink},
stresIF/.style={color=darkbrown, fill=white!#2!brinkpink},
stresIG/.style={color=darkcyan, fill=white!#2!brinkpink},
stresIH/.style={color=byzantium, fill=white!#2!brinkpink},
stresII/.style={color=burgundy, fill=white!#2!brinkpink},
			stres/.style={color=black, fill=deepchampagne},
extstA/.style={color=egyptianblue, fill=white!#2!blue(munsell)!50},
extstAA/.style={color=egyptianblue, fill=white!#2!blue(munsell)!50},
extstAB/.style={color=carnelian, fill=white!#2!blue(munsell)!50},
extstAC/.style={color=chocolate(web), fill=white!#2!blue(munsell)!50},
extstAD/.style={color=darkgreen, fill=white!#2!blue(munsell)!50},
extstAE/.style={color=onyx, fill=white!#2!blue(munsell)!50},
extstAF/.style={color=darkbrown, fill=white!#2!blue(munsell)!50},
extstAG/.style={color=darkcyan, fill=white!#2!blue(munsell)!50},
extstAH/.style={color=byzantium, fill=white!#2!blue(munsell)!50},
extstAI/.style={color=burgundy, fill=white!#2!blue(munsell)!50},
extstB/.style={color=carnelian, fill=white!#2!cadmiumred!50},
extstBA/.style={color=egyptianblue, fill=white!#2!cadmiumred!50},
extstBB/.style={color=carnelian, fill=white!#2!cadmiumred!50},
extstBC/.style={color=chocolate(web), fill=white!#2!cadmiumred!50},
extstBD/.style={color=darkgreen, fill=white!#2!cadmiumred!50},
extstBE/.style={color=onyx, fill=white!#2!cadmiumred!50},
extstBF/.style={color=darkbrown, fill=white!#2!cadmiumred!50},
extstBG/.style={color=darkcyan, fill=white!#2!cadmiumred!50},
extstBH/.style={color=byzantium, fill=white!#2!cadmiumred!50},
extstBI/.style={color=burgundy, fill=white!#2!cadmiumred!50},
extstC/.style={color=chocolate(web), fill=white!#2!chromeyellow!50},
extstCA/.style={color=egyptianblue, fill=white!#2!chromeyellow!50},
extstCB/.style={color=carnelian, fill=white!#2!chromeyellow!50},
extstCC/.style={color=chocolate(web), fill=white!#2!chromeyellow!50},
extstCD/.style={color=darkgreen, fill=white!#2!chromeyellow!50},
extstCE/.style={color=onyx, fill=white!#2!chromeyellow!50},
extstCF/.style={color=darkbrown, fill=white!#2!chromeyellow!50},
extstCG/.style={color=darkcyan, fill=white!#2!chromeyellow!50},
extstCH/.style={color=byzantium, fill=white!#2!chromeyellow!50},
extstCI/.style={color=burgundy, fill=white!#2!chromeyellow!50},
extstD/.style={color=darkgreen, fill=white!#2!green(pigment)!50},
extstDA/.style={color=egyptianblue, fill=white!#2!green(pigment)!50},
extstDB/.style={color=carnelian, fill=white!#2!green(pigment)!50},
extstDC/.style={color=chocolate(web), fill=white!#2!green(pigment)!50},
extstDD/.style={color=darkgreen, fill=white!#2!green(pigment)!50},
extstDE/.style={color=onyx, fill=white!#2!green(pigment)!50},
extstDF/.style={color=darkbrown, fill=white!#2!green(pigment)!50},
extstDG/.style={color=darkcyan, fill=white!#2!green(pigment)!50},
extstDH/.style={color=byzantium, fill=white!#2!green(pigment)!50},
extstDI/.style={color=burgundy, fill=white!#2!green(pigment)!50},
extstE/.style={color=onyx, fill=white!#2!platinum!50},
extstEA/.style={color=egyptianblue, fill=white!#2!platinum!50},
extstEB/.style={color=carnelian, fill=white!#2!platinum!50},
extstEC/.style={color=chocolate(web), fill=white!#2!platinum!50},
extstED/.style={color=darkgreen, fill=white!#2!platinum!50},
extstEE/.style={color=onyx, fill=white!#2!platinum!50},
extstEF/.style={color=darkbrown, fill=white!#2!platinum!50},
extstEG/.style={color=darkcyan, fill=white!#2!platinum!50},
extstEH/.style={color=byzantium, fill=white!#2!platinum!50},
extstEI/.style={color=burgundy, fill=white!#2!platinum!50},
extstF/.style={color=darkbrown, fill=white!#2!cinnamon!50},
extstFA/.style={color=egyptianblue, fill=white!#2!cinnamon!50},
extstFB/.style={color=carnelian, fill=white!#2!cinnamon!50},
extstFC/.style={color=chocolate(web), fill=white!#2!cinnamon!50},
extstFD/.style={color=darkgreen, fill=white!#2!cinnamon!50},
extstFE/.style={color=onyx, fill=white!#2!cinnamon!50},
extstFF/.style={color=darkbrown, fill=white!#2!cinnamon!50},
extstFG/.style={color=darkcyan, fill=white!#2!cinnamon!50},
extstFH/.style={color=byzantium, fill=white!#2!cinnamon!50},
extstFI/.style={color=burgundy, fill=white!#2!cinnamon!50},
extstG/.style={color=darkcyan, fill=white!#2!verdigris!50},
extstGA/.style={color=egyptianblue, fill=white!#2!verdigris!50},
extstGB/.style={color=carnelian, fill=white!#2!verdigris!50},
extstGC/.style={color=chocolate(web), fill=white!#2!verdigris!50},
extstGD/.style={color=darkgreen, fill=white!#2!verdigris!50},
extstGE/.style={color=onyx, fill=white!#2!verdigris!50},
extstGF/.style={color=darkbrown, fill=white!#2!verdigris!50},
extstGG/.style={color=darkcyan, fill=white!#2!verdigris!50},
extstGH/.style={color=byzantium, fill=white!#2!verdigris!50},
extstGI/.style={color=burgundy, fill=white!#2!verdigris!50},
extstH/.style={color=byzantium, fill=white!#2!darkpastelpurple!50},
extstHA/.style={color=egyptianblue, fill=white!#2!darkpastelpurple!50},
extstHB/.style={color=carnelian, fill=white!#2!darkpastelpurple!50},
extstHC/.style={color=chocolate(web), fill=white!#2!darkpastelpurple!50},
extstHD/.style={color=darkgreen, fill=white!#2!darkpastelpurple!50},
extstHE/.style={color=onyx, fill=white!#2!darkpastelpurple!50},
extstHF/.style={color=darkbrown, fill=white!#2!darkpastelpurple!50},
extstHG/.style={color=darkcyan, fill=white!#2!darkpastelpurple!50},
extstHH/.style={color=byzantium, fill=white!#2!darkpastelpurple!50},
extstHI/.style={color=burgundy, fill=white!#2!darkpastelpurple!50},
extstI/.style={color=burgundy, fill=white!#2!brinkpink!50},
extstIA/.style={color=egyptianblue, fill=white!#2!brinkpink!50},
extstIB/.style={color=carnelian, fill=white!#2!brinkpink!50},
extstIC/.style={color=chocolate(web), fill=white!#2!brinkpink!50},
extstID/.style={color=darkgreen, fill=white!#2!brinkpink!50},
extstIE/.style={color=onyx, fill=white!#2!brinkpink!50},
extstIF/.style={color=darkbrown, fill=white!#2!brinkpink!50},
extstIG/.style={color=darkcyan, fill=white!#2!brinkpink!50},
extstIH/.style={color=byzantium, fill=white!#2!brinkpink!50},
extstII/.style={color=burgundy, fill=white!#2!brinkpink!50},
			extst/.style={color=black, fill=deepchampagne!50},
			decision/.style={very thick, draw=carnelian},
			backdecision/.style={very thick, draw=carnelian, densely dashed},
			back/.style={densely dashed},
			rback/.style={bend right=5, densely dashed},
			ub/.style={very thick, draw=carnelian, dashed},
			lb/.style={very thick, draw=darkgreen, dashed},
			profA/.style={color=egyptianblue},
			profB/.style={color=carnelian},
			profC/.style={color=chocolate(web)},
			profD/.style={color=darkgreen},
			profE/.style={color=onyx},
			profF/.style={color=darkbrown},
			profG/.style={color=darkcyan},
			profH/.style={color=byzantium},
			profI/.style={color=burgundy},
			pruA/.style={pattern color=egyptianblue},
			pruB/.style={pattern color=carnelian},
			pruC/.style={pattern color=chocolate(web)},
			pruD/.style={pattern color=darkgreen},
			pruE/.style={pattern color=onyx},
			pruF/.style={pattern color=darkbrown},
			pruG/.style={pattern color=darkcyan},
			pruH/.style={pattern color=byzantium},
			pruI/.style={pattern color=burgundy},
    ]
		\def\scalefactor{#1}
		\setlength{\topsep}{0pt}
		\setlength{\partopsep}{0pt}
}{%
  \end{tikzpicture}
}
\newenvironment{colorschedfigure}[1]{%
  \begin{colorfadedschedfigure}{#1}{40}
}{%
  \end{colorfadedschedfigure}
}
\makeatother



\newcommand\AllInterval[6]{ %1:est, %2:lst, %3:ect, %4:lct, %5:row, %6:demand
	  \draw[thick,color=blue!50!black] (#1+\tasksep*1.75,-#5-\tasksep/3) -- (#1+\tasksep*.75,-#5-\tasksep/3) -- (#1+\tasksep*.75,-#5-#6+\tasksep/3) -- (#1+\tasksep*1.75,-#5-#6+\tasksep/4);
	  \draw[thick,color=blue!50!black] (#2-\tasksep*1.75,-#5-\tasksep/3) -- (#2-\tasksep*.75,-#5-\tasksep/3) -- (#2-\tasksep*.75,-#5-#6+\tasksep/3) -- (#2-\tasksep*1.75,-#5-#6+\tasksep/4);
		\draw[very thin, latex-latex, densely dashed,color=blue!50!black] (#1+\tasksep,-#6/2-#5+\tasksep) -- (#2-\tasksep,-#6/2-#5+\tasksep);


		\draw[thick,color=red!50!black] (#3+\tasksep*1.75,-#5-\tasksep/3) -- (#3+\tasksep*.75,-#5-\tasksep/3) -- (#3+\tasksep*.75,-#5-#6+\tasksep/3) -- (#3+\tasksep*1.75,-#5-#6+\tasksep/4);
	  \draw[thick,color=red!50!black] (#4-\tasksep*1.75,-#5-\tasksep/3) -- (#4-\tasksep*.75,-#5-\tasksep/3) -- (#4-\tasksep*.75,-#5-#6+\tasksep/3) -- (#4-\tasksep*1.75,-#5-#6+\tasksep/4);
		\draw[very thin, latex-latex, densely dashed, color=red!50!black] (#3+\tasksep,-#6/2-#5-\tasksep) -- (#4-\tasksep,-#6/2-#5-\tasksep);
}

% \newcommand\RightInterval[4]{ %1:task pos, %2:interval pos, %3:row, %4:demand
% 	  \draw[thick] (#2-\tasksep*1.25,-#3-\tasksep/3) -- (#2,-#3-\tasksep/3) -- (#2,-#3-#4+\tasksep/3) -- (#2-\tasksep*1.25,-#3-#4+\tasksep/4);
% 		\draw[very thin, -latex, densely dashed] (#1,-#4/2-#3) -- (#2-\tasksep/3,-#4/2-#3);
% }


\newcommand\LeftInterval[4]{ %1:task pos, %2:interval pos, %3:row, %4:demand
	  \draw[thick] (#2+\tasksep*1.25,-#3-\tasksep/3) -- (#2,-#3-\tasksep/3) -- (#2,-#3-#4+\tasksep/3) -- (#2+\tasksep*1.25,-#3-#4+\tasksep/4);
		\draw[very thin, -latex, densely dashed] (#1,-#4/2-#3) -- (#2+\tasksep/3,-#4/2-#3);
}

% \newcommand\PrunedLeftInterval[5]{ %1:task pos, %2:interval pos, %3:row, %4:demand
% 	  \draw[thick] (#2+\tasksep*1.25,-#3-\tasksep/3) -- (#2+\tasksep/3,-#3-\tasksep/3) -- (#2+\tasksep/3,-#3-#4+\tasksep/3) -- (#2+\tasksep*1.25,-#3-#4+\tasksep/4);
% 		\draw[prof#5, pattern=north east lines] (#2-\tasksep/3,-#3-#4+\tasksep/3) rectangle (#1,-#3-\tasksep/3);
% 		% \draw[very thin, -latex, densely dashed] (#1,-#4/2-#3) -- (#2+\tasksep/3,-#4/2-#3);
% }

\newcommand\Intertask[4]{ %1:task1 pos, %2:task2 pos, %3:row, %4:dem
		\draw[latex-latex, color=black!70, densely dashed] (#1+\tasksep/3,-#4/2-#3) -- (#2-\tasksep/3,-#4/2-#3);
}

\newcommand\RightInterval[4]{ %1:task pos, %2:interval pos, %3:row, %4:demand
	  \draw[thick] (#2-\tasksep*1.25,-#3-\tasksep/3) -- (#2,-#3-\tasksep/3) -- (#2,-#3-#4+\tasksep/3) -- (#2-\tasksep*1.25,-#3-#4+\tasksep/4);
		\draw[very thin, -latex, densely dashed] (#1,-#4/2-#3) -- (#2-\tasksep/3,-#4/2-#3);
}

\newcommand\RightBracket[3]{ %1:interval pos, %2:row, %3:demand
	  \draw[thick] (#1-\tasksep*1.25,-#2-\tasksep/3) -- (#1,-#2-\tasksep/3) -- (#1,-#2-#3+\tasksep/3) -- (#1-\tasksep*1.25,-#2-#3+\tasksep/4);
}

\newcommand\LeftBracket[3]{ %1:interval pos, %2:row, %3:demand
	  \draw[thick] (#1+\tasksep*1.25,-#2-\tasksep/3) -- (#1,-#2-\tasksep/3) -- (#1,-#2-#3+\tasksep/3) -- (#1+\tasksep*1.25,-#2-#3+\tasksep/4);
}

\newcommand\Precedence[3]{ 
\ExtractXYCoordinate{t#1.east}
\ExtractABCoordinate{t#2.west}
\ifdim\dimexpr\ACoord-3pt>\XCoord
	\draw[-latex, shorten >=1mm,shorten <=1mm,#3] (t#1.east) -- (t#2.west);%
\else
	\ifdim\YCoord>\BCoord
		 \draw[-latex, shorten >=1mm,shorten <=1mm,#3] (t#1.south) -- (t#2.west);%
	\else%
		 \ifdim\YCoord<\BCoord
		 		\draw[-latex, shorten >=1mm,shorten <=1mm,#3] (t#1.north) -- (t#2.west);%
		 \fi
	\fi
\fi
}

\newcommand\BackPrecedence[3]{ 
	\ExtractXYCoordinate{t#1}
	\ExtractABCoordinate{t#2}
	\ifdim\YCoord>\BCoord
		\ExtractXYCoordinate{t#1.south}
		\ExtractABCoordinate{t#2.north}
		\ifdim\dimexpr\YCoord-3pt>\BCoord
			\path[] (t#1.south) edge [bend left=5, -latex, shorten >=1mm,shorten <=1mm,#3] (t#2.north);%
		\else
		 	\path[] (t#1) edge [bend left=5, -latex, shorten >=1mm,shorten <=1mm,#3] (t#2.north);%
		\fi
	\else%
		\ExtractXYCoordinate{t#1.north}
		\ExtractABCoordinate{t#2.south}
		\ifdim\dimexpr\YCoord+3pt<\BCoord
			\path[] (t#1.north) edge [bend left=5, -latex, shorten >=1mm,shorten <=1mm,#3] (t#2.south);%
		\else
		 	\path[] (t#1) edge [bend left=5, -latex, shorten >=1mm,shorten <=1mm,#3] (t#2.south);%
		\fi
	\fi
}

%1:est, 2:lst, 3:ect, 4:lct, 5:duration, 6:demand, 7:row, 8:color, 9:label
\newcommand\PreemptiveVariableTask[9]{
\PhantomExtensibleTask{#1/2+#4/2-#5/2}{#5}{#5}{#6}{#7}{#8}{#9}
\AllInterval{#1}{#2}{#3}{#4}{#7}{#6}
}


%1:est, 2:lst, 3:ect, 4:lct, 5:duration, 6:demand, 7:row, 8:color, 9:label
\newcommand\PreemptiveAutoVariableTask[9]{
\PhantomExtensibleTask{#1/2+#4/2-#5/2}{#5}{#5}{#6}{#7}{#8}{#9}
\AllInterval{#1}{#4-#5}{#1+#5}{#4}{#7}{#6}

}


\newcommand\PrunedPreemptiveVariableTaskOver[9]{%
%1:est, 2:lst, 3:ect, 4:lct, 5:duration, 6:demand, 7:row, 8:color, 9:label, 10:pruned est, 11: pruned lct
\PreemptiveVariableTask{#1}{#2}{#3}{#4}{#5}{#6}{#7}{#8}{#9}

		\ifthenelse{\equal{#2}{#4-#5}}{}{
		\draw[thick, latex-, shorten >=3pt, shorten <=1pt] (#2,-#7) to[bend left] (#4-#5,-#7);
		}

		\fill[pattern=north east lines, draw=black!50] (#4-#5+\tasksep,-#7-#6+\tasksep) rectangle (#2-\tasksep,-#7-\tasksep);


		\ifthenelse{\equal{#3}{#1+#5}}{}{
		\draw[thick, latex-, shorten >=3pt, shorten <=1pt] (#3,-#7) to[bend right] (#1+#5,-#7);
		}

		\fill[pattern=north east lines, draw=black!50] (#3-\tasksep,-#7-#6+\tasksep) rectangle (#1+#5+\tasksep,-#7-\tasksep);
	
}

\newcommand\Pruning[4]{%
\fill[pattern=north east lines] (#1,-#4-#3+2*\tasksep) rectangle (#2,-#4-2*\tasksep);
}%

\newcommand\PruningOver[4]{%
%1:est, 2:lst, 3:ect, 4:lct, 5:duration, 6:demand, 7:row, 8:color, 9:label, 10:pruned est, 11: pruned lct
% \PreemptiveVariableTask{#1}{#2}{#3}{#4}

		\ifthenelse{\equal{#1}{#2}}{}{
		\draw[thick, -latex, shorten >=1pt, shorten <=1pt] (#1,-#4) to[bend left] (#2,-#4);

		\Pruning{#1}{#2}{#3}{#4}
		}
		
		}%

		\newcommand\PruningUnder[4]{%
%1:est, 2:lst, 3:ect, 4:lct, 5:duration, 6:demand, 7:row, 8:color, 9:label, 10:pruned est, 11: pruned lct
% \PreemptiveVariableTask{#1}{#2}{#3}{#4}

		\ifthenelse{\equal{#1}{#2}}{}{
		\draw[thick, -latex, shorten >=1pt, shorten <=1pt] (#2,-#4-#3) to[bend left] (#1,-#4-#3);

		\Pruning{#1}{#2}{#3}{#4}
		% \fill[pattern=north east lines] (#1,-#4-#3+2*\tasksep) rectangle (#2,-#4-2*\tasksep);
		}
		
		}%

% \newcommand\RightPruning[4]{%
% 		\ifthenelse{\equal{#1}{#2}}{}{
% 		\draw[thick, latex-, shorten >=3pt, shorten <=1pt] (#3,-#7) to[bend right] (#1+#5,-#7);
% 		}

% 		\fill[pattern=north east lines, draw=black!50] (#3-\tasksep,-#7-#6+\tasksep) rectangle (#1+#5+\tasksep,-#7-\tasksep);
	
% }


% \newcommand\PrunedPreemptiveVariableTaskUnder[9]{%
% %1:est, 2:lst, 3:ect, 4:lct, 5:duration, 6:demand, 7:row, 8:color, 9:label, 10:pruned est, 11: pruned lct
% \PreemptiveVariableTask{#1}{#2}{#3}{#4}{#5}{#6}{#7}{#8}{#9}

% 		\ifthenelse{\equal{#2}{#4-#5}}{}{
% 		\draw[thick, latex-, shorten >=3pt, shorten <=1pt] (#2,-#7-#6) to[bend right] (#4-#5,-#7-#6);
% 		}

% 		\ifthenelse{\equal{#3}{#1+#5}}{}{
% 		\draw[thick, latex-, shorten >=3pt, shorten <=1pt] (#3,-#7-#6) to[bend left] (#1+#5,-#7-#6);
% 		}

% 		% \ifthenelse{\equal{#1}{#10}}{}{
% 		% \draw[thick, opacity=.2, \getExtStyle{#8}, rounded corners] (#10,-#7-#6+\tasksep) rectangle (#1,-\tasksep-#7);
% 		% }
	
% }



\newcommand\VariableTask[7]{
\ExtensibleVariableTask{#1}{#2}{#3}{#3}{#4}{#5}{#6}{#7}
}
\newcommand\ExtensibleVariableTask[8]{ %1:est, 2:lct, 3:min duration, 4:max duration, 5:demand, 6:row, 7:color, 8:label
	\ExtensibleTask{#1/2+#2/2-#4/2}{#3}{#4}{#5}{#6}{#7}{#8}
	\LeftInterval{#1/2+#2/2-#4/2}{#1}{#6}{#5};
	\RightInterval{#1/2+#2/2+#4/2}{#2}{#6}{#5};
}
\newcommand\UnlabeledExtensibleVariableTask[8]{ %1:est, 2:lct, 3:min duration, 4:max duration, 5:demand, 6:row, 7:color, 8:label
	\UnlabeledTask{#1/2+#2/2-#4/2}{#3}{#4}{#5}{#6}{#7}{#8}
	\LeftInterval{#1/2+#2/2-#4/2}{#1}{#6}{#5};
	\RightInterval{#1/2+#2/2+#4/2}{#2}{#6}{#5};
}
\newcommand\PhantomExtensibleVariableTask[8]{ %1:est, 2:lct, 3:min duration, 4:max duration, 5:demand, 6:row, 7:color, 8:label
	\PhantomExtensibleTask{#1/2+#2/2-#4/2}{#3}{#4}{#5}{#6}{#7}{#8}
	\LeftInterval{#1/2+#2/2-#4/2}{#1}{#6}{#5};
	\RightInterval{#1/2+#2/2+#4/2}{#2}{#6}{#5};
}

\newcommand\LeftVariableTask[7]{ %1:est, 2:lct, 3: duration, 4:demand, 5:row, 6:color, 7:label
\LeftExtensibleVariableTask{#1}{#2}{#3}{#3}{#4}{#5}{#6}{#7}
}
\newcommand\LeftExtensibleVariableTask[8]{ %1:est, 2:lct, 3:min duration, 4:max duration, 5:demand, 6:row, 7:color, 8:label
	\ExtensibleTask{#1}{#3}{#4}{#5}{#6}{#7}{#8}
	\LeftBracket{#1}{#6}{#5};
	\RightInterval{#1+#4}{#2}{#6}{#5};
}
\newcommand\LeftPhantomExtensibleVariableTask[8]{ %1:est, 2:lct, 3:min duration, 4:max duration, 5:demand, 6:row, 7:color, 8:label
	\PhantomExtensibleTask{#1}{#3}{#4}{#5}{#6}{#7}{#8}
	\LeftBracket{#1}{#6}{#5};
	\RightInterval{#1+#4}{#2}{#6}{#5};
}

\newcommand\RightVariableTask[7]{
\RightExtensibleVariableTask{#1}{#2}{#3}{#3}{#4}{#5}{#6}{#7}
}
\newcommand\RightExtensibleVariableTask[8]{ %1:est, 2:lct, 3:min duration, 4:max duration, 5:demand, 6:row, 7:color, 8:label
	\ExtensibleTask{#2-#4}{#3}{#4}{#5}{#6}{#7}{#8}
	\LeftInterval{#2-#4}{#1}{#6}{#5};
	\RightBracket{#2}{#6}{#5};
}
\newcommand\RightPhantomExtensibleVariableTask[8]{ %1:est, 2:lct, 3:min duration, 4:max duration, 5:demand, 6:row, 7:color, 8:label
	\PhantomExtensibleTask{#2-#4}{#3}{#4}{#5}{#6}{#7}{#8}
	\LeftInterval{#2-#4}{#1}{#6}{#5};
	\RightBracket{#2}{#6}{#5};
}


\newcommand\NormalTask[6]{ %1:start, 2:duration, 3:demand, 4:row, 5:color, 6:label 
		\LeftVariableTask{#1}{#1+#2}{#2}{#3}{#4}{#5}{#6}
}

% \newcommand\GroundTask[6]{ %1:start, 2:duration, 3:demand, 4:row, 5:color, 6:label
%     \draw[thick, \getStyle{#5}] (#1+\tasksep*2/3,-#4-#3+\tasksep*2/3) rectangle (#1+#2-\tasksep*2/3,-\tasksep*2/3-#4);
%     \node[minimum width=#2*.48cm] (t#6) at (#1+#2/2,-#4-#3/2) {{\scriptsize #6}};
% }
%
% \newcommand\UnlabeledGroundTask[6]{ %1:start, 2:duration, 3:demand, 4:row, 5:color, 6:label
%     \draw[thick, \getStyle{#5}] (#1+\tasksep*2/3,-#4-#3+\tasksep*2/3) rectangle (#1+#2-\tasksep*2/3,-\tasksep*2/3-#4);
%     \node[minimum width=#2*.48cm] (t) at (#1+#2/2,-#4-#3/2) {{\scriptsize #6}};
% }

\newcommand\GroundTask[7][]{ % 1:print label (optional) 2:start, 3:duration, 4:demand, 5:row, 6:color, 7:label 
    \draw[thick, \getStyle{#6}, rounded corners] (#2+\tasksep/3,-#5-#4+\tasksep) rectangle (#2+#3-\tasksep/3,-#5-\tasksep);
    \node[minimum width=#3*.48cm] (t#7) at (#2+#3/2,-#5-#4/2) {{\scriptsize #1}};
}

% \newcommand\UnlabeledGroundTask[6]{ %1:start, 2:duration, 3:demand, 4:row, 5:color, 6:label 
%     \draw[thick, \getStyle{#5}, rounded corners] (#1,-#4-#3) rectangle (#1+#2,-#4);
%     \node[minimum width=#2*.48cm] (t) at (#1+#2/2,-#4-#3/2) {{\scriptsize #6}};
% }


% \newcommand\MandatoryPartTask[7]{ %1:start, 2:min duration, 3:max duration, 4:demand, 5:row, 6:color, 7:label
% 		\draw[thick, \getExtStyle{#6}] (#1+\tasksep,-#5-#4+\tasksep*2) rectangle (#1+#3-\tasksep,-\tasksep*2-#5);
% 		\draw[thick, \getStyle{#6}] (#1+#3/2-#2/2+\tasksep,-#5-#4+\tasksep*2) rectangle (#1+#3/2+#2/2-\tasksep,-\tasksep*2-#5);
%     \node[minimum width=#3*\scalefactor cm, minimum height=#4*\scalefactor cm] (t#7) at (#1+#3/2,-#5-#4/2) {{\tiny #7}};
% }

\newcommand\MandatoryPartTask[7]{ %1:start, 2:min duration, 3:max duration, 4:demand, 5:row, 6:color, 7:label 
		\draw[thick, \getExtStyle{#6}, rounded corners] (#1,-#5-#4+\tasksep) rectangle (#1+#3,-\tasksep-#5);
		\draw[thick, \getStyle{#6}] (#1+#3/2-#2/2,-#5-#4+\tasksep) rectangle (#1+#3/2+#2/2,-\tasksep-#5);
    \node[minimum width=#3*\scalefactor cm, minimum height=#4*\scalefactor cm] (t#7) at (#1+#3/2,-#5-#4/2) {{\tiny #7}};
}

% \newcommand\PhantomExtensibleTask[7]{ %1:start, 2:min duration, 3:max duration, 4:demand, 5:row, 6:color, 7:label
% 		% \draw[thick, opacity=.2, \getExtStyle{#6}] (#1+\tasksep,-#5-#4+\tasksep*2) rectangle (#1+#3-\tasksep,-\tasksep*2-#5);
% 		% \draw[thick, opacity=.2, \getStyle{#6}] (#1+#3/2-#2/2-\tasksep,-#5-#4+\tasksep*2) rectangle (#1+#3/2+#2/2-\tasksep,-\tasksep*2-#5);
% 		%     \node[] (t#7) at (#1+#3/2,-#5-#4/2) {{\scriptsize #7}};
% 		\draw[thick, opacity=.3, \getStyle{#6}] (#1+\tasksep*1.5,-#5-#4+\tasksep*2) rectangle (#1+#2-\tasksep*1.5,-\tasksep*2-#5);
% 		\draw[thick, opacity=.2, \getExtStyle{#6}] (#1+#2-\tasksep*1.5,-#5-#4+\tasksep*2) rectangle (#1+#3-\tasksep*1.5,-\tasksep*2-#5);
% 		\node[minimum width=#3*\scalefactor cm, minimum height=#4*\scalefactor cm,opacity=.3] (t#7) at (#1+#2/2,-#5-#4/2) {{\tiny #7}};
% }
%
% \newcommand\PrunedExtensibleTask[7]{
% 		\draw[thick, opacity=.3, pru#6, \getStyle{#6}, pattern=north east lines] (#1+\tasksep*1.5,-#5-#4+\tasksep*2) rectangle (#1+#2-\tasksep*1.5,-\tasksep*2-#5);
% 		\draw[thick, opacity=.2, \getExtStyle{#6}] (#1+#2-\tasksep*1.5,-#5-#4+\tasksep*2) rectangle (#1+#3-\tasksep*1.5,-\tasksep*2-#5);
% 		\node[minimum width=#3*\scalefactor cm, minimum height=#4*\scalefactor cm,opacity=.3] (t#7) at (#1+#2/2,-#5-#4/2) {{\tiny #7}};
% }
%
% \newcommand\ExtensibleTask[7]{ %1:start, 2:min duration, 3:max duration, 4:demand, 5:row, 6:color, 7:label
%     \draw[thick, \getExtStyle{#6}] (#1+\tasksep*1.5,-#5-#4+\tasksep*2) rectangle (#1+#3-\tasksep*1.5,-\tasksep*2-#5);
% 		\draw[thick, \getStyle{#6}] (#1+\tasksep*1.5,-#5-#4+\tasksep*2) rectangle (#1+#2-\tasksep*1.5,-\tasksep*2-#5);
%     \node[minimum width=#3*\scalefactor cm, minimum height=#4*\scalefactor cm] (t#7) at (#1+#3/2,-#5-#4/2) {{\tiny #7}};
% }
%
% \newcommand\UnlabeledTask[7]{ %1:start, 2:min duration, 3:max duration, 4:demand, 5:row, 6:color, 7:label
%     \draw[thick, \getExtStyle{#6}] (#1+\tasksep*1.5,-#5-#4+\tasksep*2) rectangle (#1+#3-\tasksep*1.5,-\tasksep*2-#5);
% 		\draw[thick, \getStyle{#6}] (#1+\tasksep*1.5,-#5-#4+\tasksep*2) rectangle (#1+#2-\tasksep*1.5,-\tasksep*2-#5);
%     \node[minimum width=#3*\scalefactor cm, minimum height=#4*\scalefactor cm] (t) at (#1+#3/2,-#5-#4/2) {{\tiny #7}};
% }

\newcommand\PhantomExtensibleTask[7]{ %1:start, 2:min duration, 3:max duration, 4:demand, 5:row, 6:color, 7:label 
		% \draw[thick, opacity=.2, \getExtStyle{#6}] (#1+\tasksep,-#5-#4+\tasksep*2) rectangle (#1+#3-\tasksep,-\tasksep*2-#5);
		% \draw[thick, opacity=.2, \getStyle{#6}] (#1+#3/2-#2/2-\tasksep,-#5-#4+\tasksep*2) rectangle (#1+#3/2+#2/2-\tasksep,-\tasksep*2-#5);
		%     \node[] (t#7) at (#1+#3/2,-#5-#4/2) {{\scriptsize #7}};
		\draw[thick, opacity=.3, \getStyle{#6}, rounded corners] (#1,-#5-#4+\tasksep) rectangle (#1+#2,-\tasksep-#5);

		\ifthenelse{\equal{#2}{#3}}{}{
		\draw[thick, opacity=.2, \getExtStyle{#6}, rounded corners] (#1+#2,-#5-#4+\tasksep) rectangle (#1+#3,-\tasksep-#5);
		}
		\node[minimum width=#3*\scalefactor cm, minimum height=#4*\scalefactor cm,opacity=.3] (t#7) at (#1+#2/2,-#5-#4/2) {{\tiny #7}};
}

\newcommand\PrunedExtensibleTask[7]{ 
		\draw[thick, opacity=.3, pru#6, \getStyle{#6}, pattern=north east lines, rounded corners] (#1,-#5-#4+\tasksep) rectangle (#1+#2,-\tasksep-#5);

		\ifthenelse{\equal{#2}{#3}}{}{
		\draw[thick, opacity=.2, \getExtStyle{#6}, rounded corners] (#1+#2,-#5-#4+\tasksep) rectangle (#1+#3,-\tasksep-#5);
		}
		
		\node[minimum width=#3*\scalefactor cm, minimum height=#4*\scalefactor cm,opacity=.3] (t#7) at (#1+#2/2,-#5-#4/2) {{\tiny #7}};
}

\newcommand\ExtensibleTask[7]{ %1:start, 2:min duration, 3:max duration, 4:demand, 5:row, 6:color, 7:label
    \draw[thick, \getExtStyle{#6}, rounded corners] (#1,-#5-#4+\tasksep) rectangle (#1+#3,-\tasksep-#5);
		\draw[thick, \getStyle{#6}, rounded corners] (#1,-#5-#4+\tasksep) rectangle (#1+#2,-\tasksep-#5);
    \node[minimum width=#3*\scalefactor cm, minimum height=#4*\scalefactor cm] (t#7) at (#1+#3/2,-#5-#4/2) {{\tiny #7}};
}

\newcommand\UnlabeledTask[7]{ %1:start, 2:min duration, 3:max duration, 4:demand, 5:row, 6:color, 7:label
    \draw[thick, \getExtStyle{#6}, rounded corners] (#1,-#5-#4+\tasksep) rectangle (#1+#3,-\tasksep-#5);
		\draw[thick, \getStyle{#6}, rounded corners] (#1,-#5-#4+\tasksep) rectangle (#1+#2,-\tasksep-#5);
    \node[minimum width=#3*\scalefactor cm, minimum height=#4*\scalefactor cm] (t) at (#1+#3/2,-#5-#4/2) {{\tiny #7}};
}





\newcommand\PrintRow[1] {
	\setcounter{x}{0}
	\foreach \ijob in {#1} {
		\addtocounter{x}{1}
		\node[] at (-.5,-.5-\value{x}) {$J_{\arabic{x}}$};
		\setcounter{\getJob{\ijob}}{0}
		\setcounter{\getRow{\ijob}}{\value{x}}
	}
}

\newcommand\PrintJobs[1] {
	\setcounter{x}{0}
	\foreach \ijob in {#1} {
		\addtocounter{x}{1}
		\node[] at (-.5,-.5-\value{x}) {$J_{\ijob}$};
		\setcounter{\getJob{\ijob}}{0}
		\setcounter{\getRow{\ijob}}{\value{x}}
	}
}

\newcommand\PrintGrid[2] {
	\draw[densely dotted, color=black!90] (0, 0) grid (#1, -#2);
}
% \newcommand\PrintTics[2] {
% 	\foreach \x in {#1} {
% 		\node[font=\tiny] at (\x*#2, .3) {\x};
% 	}
% }
\newcommand\PrintTics[2] {
	\foreach \x in {#1} {
		\node[font=\tiny] at (\x, .3) {\roundandprint{\x / #2}};
	}
}


\newcommand\Jobshop[3] { %1: numjobs, 2:list of machines' tasks
	\foreach \imach in {#1} {
		\setcounter{\getResource{\imach}}{0}
	}
	\PrintJobs{#2}
	
	  \foreach \imach/\tdur/\ijob in #3 {
			\ifnum\value{\getResource{\imach}}<\value{\getJob{\ijob}}
					\GroundTask[\ijob:\tdur]{\value{\getJob{\ijob}}}{\tdur}{1}{\value{\getRow{\ijob}}}{\imach}{\ijob\imach}
					\addtocounter{\getJob{\ijob}}{\tdur}
					\setcounter{\getResource{\imach}}{\value{\getJob{\ijob}}}
	 			\else
					\GroundTask[\ijob:\tdur]{\value{\getResource{\imach}}}{\tdur}{1}{\value{\getRow{\ijob}}}{\imach}{\ijob\imach}
					\addtocounter{\getResource{\imach}}{\tdur}
					\setcounter{\getJob{\ijob}}{\value{\getResource{\imach}}}
			\fi
	}
}

\newcommand\Profile[3] {
	\setcounter{px}{0}
	\setcounter{py}{0}
	\setcounter{x}{0}
	\setcounter{y}{0}

	\foreach \X/\Y in #1 {
		\fill[color=bostonuniversityred, opacity=.3] (\arabic{x}, -#3) rectangle (\X, \arabic{y}-#3);
		\draw[ultra thick, color=bostonuniversityred] (\arabic{px}, \arabic{py}-#3) -- (\arabic{x}, \arabic{py}-#3) -- (\arabic{x}, \arabic{y}-#3) -- (\X, \arabic{y}-#3) -- (\X,\Y-#3);
		
		\setcounter{px}{\value{x}}
		\setcounter{py}{\value{y}}
		\setcounter{x}{\X}
		\setcounter{y}{\Y}
	}
	
	\draw[dashed, thick, color=bostonuniversityred] (0, #2-#3) -- (\arabic{x}, #2-#3);
	
}


\tikzset{edgestyle/.style={thick, shorten >=1pt, shorten <=1pt, <-}} 
\tikzset{vertexstyle/.style={shape=circle,draw,inner sep=1pt}}
\tikzset{labelstyle/.style={inner sep=1.2pt, fill=white, font=\tiny}}


% \newcommand\CProfile[3] {
% 	\foreach \Xf/\Yf/\Xs/\Ys in #1 {
% 		\fill[color=bostonuniversityred, opacity=.3] (\Xf, -#3) rectangle (\Xs, \Yf-#3);
% 		\draw[ultra thick, color=bostonuniversityred] (\Xf, \Yf-#3) -- (\Xs, \Yf-#3) -- (\Xs, \Yf-#3) -- (\Xs,\Ys-#3);
% 		\draw[dashed, thick, color=bostonuniversityred] (\Xf, #2-#3) -- (\Xs, #2-#3);
% 	}
% }

\newcommand\CProfile[5] {
	\foreach \Xf/\Yf/\Xs/\Ys in #1 {
		\fill[prof#4, opacity=.3] (\Xf, -#3) rectangle (\Xs, \Yf-#3);
		\draw[ultra thick, prof#4] (\Xf, \Yf-#3) -- (\Xs, \Yf-#3) -- (\Xs, \Yf-#3) -- (\Xs,\Ys-#3);
	}
	\draw[dashed, thick, prof#4] (0, #2-#3) -- (#5, #2-#3);
}


\newcommand\memph[1]{\textcolor{carnelian}{#1}}





\newcommand{\sptext}[2] {%
			{$dur:{#1}$}%
			  \nodepart{second}{$est:{#2}$}%
			  % \nodepart{third}{$est = {#3}$}%
}


\newcommand{\ttnode}[3] {%
	node[punkt] [rectangle split, rectangle split, rectangle split parts=2,text ragged, active on=<#3->] {%
		\sptext{#1}{#2}
	}%
}

\newcommand{\thetaroot}[3] {%
	\node[punkt] [rectangle split, rectangle split, rectangle split parts=2,text ragged, active on=<#3->] {%
		\sptext{#1}{#2}
	}%
}


  \tikzset{
    inactive/.style={color=red!15!white},
    active on/.style={alt={#1{}{inactive}}},
    alt/.code args={<#1>#2#3}{%
      \alt<#1>{\pgfkeysalso{#2}}{\pgfkeysalso{#3}} % \pgfkeysalso doesn't change the path
    },
  }

  \tikzset{
    invisible/.style={opacity=0},
    visible on/.style={alt={#1{}{invisible}}},
    alt/.code args={<#1>#2#3}{%
      \alt<#1>{\pgfkeysalso{#2}}{\pgfkeysalso{#3}} % \pgfkeysalso doesn't change the path
    },
  }
	
  \tikzset{
    highlighted/.style={draw=red, color=red!80!black},
    highlighted on/.style={alt={#1{highlighted}{}}},
    alt/.code args={<#1>#2#3}{%
      \alt<#1>{\pgfkeysalso{#2}}{\pgfkeysalso{#3}} % \pgfkeysalso doesn't change the path
    },
  }


\newenvironment{downtwolvltree}{%
\begin{tikzpicture}[%
    grow=down,%
    level 1/.style={sibling distance=3cm, level distance=1.6cm},%
    level 2/.style={sibling distance=1.5cm, level distance=1.6cm},%
		kant/.style={font=\scriptsize},%
    edge from parent/.style={very thick,draw=red!40!black!60,%
        shorten >=5pt, shorten <=5pt},%
    edge from parent path={(\tikzparentnode.south) -- (\tikzchildnode.north)},%
    punkt/.style={text ragged, inner sep=1mm, font=\scriptsize, rectangle, rounded corners, shade, top color=white,%
    bottom color=red!50!black!20, draw=red!40!black!60, very thick }%
    ]%
}{%
  \end{tikzpicture}
}


\newenvironment{righttwolvltree}{%
\begin{tikzpicture}[%
    grow=right,
    level 1/.style={sibling distance=3.2cm, level distance=2.2cm},
    level 2/.style={sibling distance=1.6cm, level distance=1.8cm},
		kant/.style={text width=2cm, text centered, sloped},
    edge from parent/.style={very thick,draw=blue!40!black!60,
        shorten >=5pt, shorten <=5pt},
    edge from parent path={(\tikzparentnode.east) -- (\tikzchildnode.west)},
    punkt/.style={text ragged, inner sep=1mm, font=\scriptsize, rectangle, rounded corners, shade, top color=white,
    bottom color=blue!50!black!20, draw=blue!40!black!60, very thick }%
    ]%
}{%
  \end{tikzpicture}
}



\newenvironment{downthreelvltree}{%
\begin{tikzpicture}[%
    scale=.8,%
    grow=down,%
    level 1/.style={sibling distance=4.5cm, level distance=2cm},%
    level 2/.style={sibling distance=3cm, level distance=2cm},%
		level 3/.style={sibling distance=1.5cm, level distance=2cm},%
		kant/.style={font=\scriptsize},%
    edge from parent/.style={very thick,draw=red!40!black!60,%
        shorten >=3pt, shorten <=3pt},%
    edge from parent path={(\tikzparentnode.south) -- (\tikzchildnode.north)},%
    punkt/.style={text ragged, inner sep=1mm, font=\scriptsize, rectangle, rounded corners, shade, top color=white,%
    bottom color=red!50!black!20, draw=red!40!black!60, very thick }%
    ]%
}{%
  \end{tikzpicture}
}



\def\thetree{
\thetaroot{All tasks} {\only<2>{6}\only<3>{11}\only<4>{16}}{\only<2>{8}\only<3>{11}\only<4>{16}}
    child {
				\ttnode{$A ~\&~ B$} {4}{14}
				child {
					\ttnode{ $A$} {4}{14}
				}
				child {
					\ttnode{ $B$} {4}{3}
				}
    }
    child {
				\ttnode{ $C ~\&~ D$} {6}{\only<1>{8}\only<2>{9}\only<3->{10}}
				child[visible on=<4->, highlighted on=<4>] {
					\ttnode{ $C$} {4}{14}
				}
				child[visible on=<2->, highlighted on=<2-3>] {
					\ttnode{ $D$} {4}{12}
				}
    };
}





\def\acsp{\ensuremath{\mathcal{N}}}
\def\atup{\ensuremath{\tau}}
\def\otup{\ensuremath{\sigma}}
% \def\tuplels{\langle}
% \def\tuplers{\rangle}
\def\tuplels{(}
\def\tuplers{)}
\newcommand{\pair}[2]{\ensuremath{\tuplels{#1},{#2}\tuplers}}
\newcommand{\triple}[3]{\ensuremath{\tuplels{#1},{#2},{#3}\tuplers}}
\newcommand{\tuple}[2]{\ensuremath{\tuplels{#1},\ldots,{#2}\tuplers}}
\newcommand{\elt}[2]{\ensuremath{{#1}_{#2}}}


\newcommand{\project}[2]{\ensuremath{{#1}_{#2}}}

\newcommand\class[1]{\ensuremath{{\cal {#1}}}}
\def\anet{\ensuremath{\class{N}}}


\newcommand{\todo}[1]{{\color{red!50!black}\textbf{[TODO: #1]}}}

\newcommand{\mytodo}[2]{\textcolor{red!50!black}{\textbf{#1: #2}}}
\newcommand{\cgtodo}[1]{\mytodo{CG}{#1}}
\newcommand{\ehtodo}[1]{\mytodo{EH}{#1}}
% \renewcommand{\mytodo}[2]{} % Uncomment this line to disable todos


\newcommand{\constraint}[1]{\text{\textsc{#1}}}


\newcommand{\CSP}{\classname{CSP}}
\newcommand{\vars}{\ensuremath{{\mathbf{x}}}\xspace}
\newcommand{\cons}{\ensuremath{{\mathbf{c}}}\xspace}
\newcommand{\activations}{\ensuremath{{\mathbf{a}}}\xspace}
\newcommand{\thestarts}{\ensuremath{{\mathbf{s}}}\xspace}
\newcommand{\theends}{\ensuremath{{\mathbf{e}}}\xspace}


\newcommand{\var}[1][]{\ensuremath{x_{{#1}}}}
\newcommand{\ovar}[1][]{\ensuremath{y_{{#1}}}}
\newcommand{\val}[1][]{\ensuremath{v_{{#1}}}}



\newcommand{\dom}[1][]{\ensuremath{{\cal D}\ifthenelse{\equal{#1}{}}{}{({#1})}}\xspace}
\newcommand{\cdom}[1][]{\ensuremath{{D}\ifthenelse{\equal{#1}{}}{}{({#1})}}\xspace}


% \newcommand{\cons}{\ensuremath{{\cal C}}\xspace}
\newcommand{\con}[1][]{\ensuremath{c_{{#1}}}}
\newcommand{\scope}[1]{\ensuremath{S_{#1}}}
\newcommand{\rel}[1]{\ensuremath{P_{#1}}}

\newcommand{\sol}[1][]{\ensuremath{\sigma\ifthenelse{\equal{#1}{}}{}{({#1})}}}
\newcommand{\solp}[2][]{\ensuremath{\sigma_{#2}\ifthenelse{\equal{#1}{}}{}{({#1})}}}


\newcommand{\nooverlap}[1][]{\constraint{NoOverlap\ifthenelse{\equal{#1}{}}{}{(\ensuremath{#1})}}\xspace}
\newcommand{\pnooverlap}[1][]{\constraint{Preemptive\allowbreak NoOverlap\ifthenelse{\equal{#1}{}}{}{(\ensuremath{#1})}}\xspace}
% \newcommand{\nooverlap}[1][]{\constraint{PreemptiveNoOverlap(\ensuremath{#1})}\xspace}
\def\alldiff{\constraint{AllDifferent}\xspace}
\def\alldiffprec{\constraint{AllDiffPrec}\xspace}
\def\gcc{\constraint{GlobalCardinality}\xspace}
% \def\pno{\constraint{PreemptiveExclusive}\xspace}
% \def\pno{\constraint{PreemptiveDisjunctive}\xspace}
\def\pno{\constraint{Preemptive\allowbreak NoOverlap}\xspace}
\def\noc{\constraint{NoOverlap}\xspace}
\def\pnoa{\constraint{Preemptive\allowbreak NoOverlap}\xspace}
\def\noo{\constraint{NoOverload}\xspace}

\def\timepoint{\tau}
\def\horizon{{\ensuremath{\mathcal H}}}

\newcommand{\ub}[1]{\ensuremath{\max({#1})}}
\newcommand{\lb}[1]{\ensuremath{\min({#1})}}


\newcommand{\xstart}[1]{\ensuremath{s_{#1}}\xspace}
\newcommand{\xend}[1]{\ensuremath{e_{#1}}\xspace}
\newcommand{\xproc}[1]{\ensuremath{x_{#1}}\xspace}
\newcommand{\xact}[1]{\ensuremath{a_{#1}}\xspace}


\newcommand{\ointerval}[2]{\ensuremath{[{#1},{#2})}}
\newcommand{\cinterval}[2]{\ensuremath{[{#1},{#2}]}}


% \newcommand{\totalenergy}{\ensuremath{N}}

% \newcommand{\alltasks}{\ensuremath{{\mathcal T}}\xspace}
\newcommand{\sometasks}{\ensuremath{\Omega}\xspace}

\newcommand{\myemph}[1]{\textcolor{bostonuniversityred}{\textit{#1}}}

\SetKwFunction{pathmax}{pathmax}
\SetKwFunction{pathset}{pathset}


\renewcommand<>\cellcolor[1]{\only#2{\beameroriginal\cellcolor{#1}}}


\AtBeginSection[]{
  \begin{frame}
  \vfill
  \centering
  \begin{beamercolorbox}[sep=8pt,center,shadow=true,rounded=true]{title}
    \usebeamerfont{title}\insertsectionhead\par%
  \end{beamercolorbox}
  \vfill
  \end{frame}
}


	\begin{document}



\maketitle

\begin{frame}{Overview}
\tableofcontents
\end{frame}


\section{Conclusion}


\begin{frame}
\frametitle{Conclusion}

%% Summary of experimental results

\vfill

\begin{footnotesize}
\begin{tabular}{lrrrrrr}
\toprule
\multirow{2}{*}{Instance}&  \multicolumn{3}{c}{CP Optimizer} & \multicolumn{3}{c}{Tempo}\\
\cmidrule(rr){2-4}\cmidrule(rr){5-7}
% & \multicolumn{1}{c}{\%Opt.} 
& \multicolumn{1}{c}{Objective} & \multicolumn{1}{c}{\#Branches} & \multicolumn{1}{c}{CPU (s)} & 
% \multicolumn{1}{c}{\%Opt.} & 
\multicolumn{1}{c}{Objective} & \multicolumn{1}{c}{\#Branches} & \multicolumn{1}{c}{CPU (s)} \\
\midrule

\texttt{j7-per0-0} & 1048.00 & \cellcolor{red!50}<2->{24625154} & \cellcolor{red!50}<2->{476.39} & 1048.00 & \cellcolor{green!50}<2->{4221305} & \cellcolor{green!50}<2->{261.11}\\
\texttt{j8-per0-1} & 1039.00 & \cellcolor{red!50}<2->{486860201} & \cellcolor{red!50}<2->{6177.08} & 1039.00 & \cellcolor{green!50}<2->{12323946} & \cellcolor{green!50}<2->{933.30}\\

% \texttt{j7-per0-0} & 100 & 1048.00 & 24625154 & \cellcolor{red!50}<4->{476.39} & 100 & 1048.00 & 4221305 & \cellcolor{green!50}<4->{261.11}\\
% \texttt{j8-per0-1} & \cellcolor{red!50}<2->{85} & \cellcolor{red!50}<3->{1039.45} & 262592977 & \cellcolor{red!50}<4->{4289.58} & \cellcolor{green!50}<2->{100} & \cellcolor{green!50}<3->{1039.00} & 12323946 & \cellcolor{green!50}<4->{933.29}\\
\bottomrule
\end{tabular}
\end{footnotesize}

\vfill

\uncover<2->{
\begin{itemize}
  \item CP Optimizer requires more than 3h to prove optimality on 3 out of 20 runs

%   \vfill\uncover<3->{
%   \item CP Optimizer fails to find the optimal solution on 3 out of 20 runs
% }

  \vfill
  \item up to 7 times slower (explores up to 40 times more nodes)
\end{itemize}
}

\vfill

\end{frame}


\begin{frame}
\frametitle{Jobshop with Time Lags}

%% Summary of experimental results

\vfill

\begin{footnotesize}
\begin{tabular}{lrrrrrr}
\toprule
\multirow{2}{*}{Instance}&  \multicolumn{3}{c}{CP Optimizer} & \multicolumn{3}{c}{Tempo$^*$}\\
\cmidrule(rr){2-4}\cmidrule(rr){5-7}
% & \multicolumn{1}{c}{\%Opt.} 
& \multicolumn{1}{c}{Objective} & \multicolumn{1}{c}{\#Branches} & \multicolumn{1}{c}{CPU Time} 
% & \multicolumn{1}{c}{\%Opt.} 
& \multicolumn{1}{c}{Objective} & \multicolumn{1}{c}{\#Branches} & \multicolumn{1}{c}{CPU Time} \\
\midrule

\texttt{la25\_0\_0} & 1906.00 & 7272098 & \cellcolor{red!50}838.46 & 1906.00 & 189442 & \cellcolor{green!50}{26.31}\\
\texttt{la25\_0\_0,5} & 1284.00 & 88457381 & \cellcolor{red!50}6150.42 & 1284.00 & 1610738 & \cellcolor{green!50}{272.01}\\
\texttt{la25\_0\_1} & 1087.00 & 31424922 & \cellcolor{red!50}2109.69 & 1087.00 & 6098393 & \cellcolor{green!50}{1173.10}\\
\texttt{la25\_0\_3} & 992.00 & 887841 & \cellcolor{red!50}40.56 & 992.00 & 53859 & \cellcolor{green!50}{6.53}\\
\texttt{la25\_0\_10} & 977.00 & 467144 & \cellcolor{red!50}19.36 & 977.00 & 68138 & \cellcolor{green!50}{9.51}\\
\bottomrule
\end{tabular}
\end{footnotesize}

\vfill

\begin{itemize}
  \item Similar results on Jobshop with time lags

  \vfill\pause
  \item But I have to \emph{cheat} with the initial upper bound
\end{itemize}

\vfill

\end{frame}


\begin{frame}
\frametitle{Jobshop}

%% Summary of experimental results

\vfill

\begin{footnotesize}
\begin{tabular}{lrrrrrr}
\toprule
\multirow{2}{*}{Instance}&  \multicolumn{3}{c}{CP Optimizer} & \multicolumn{3}{c}{Tempo}\\
\cmidrule(rr){2-4}\cmidrule(rr){5-7}
% & \multicolumn{1}{c}{\%Opt.} 
& \multicolumn{1}{c}{Objective} & \multicolumn{1}{c}{\#Branches} & \multicolumn{1}{c}{CPU Time} 
% & \multicolumn{1}{c}{\%Opt.} 
& \multicolumn{1}{c}{Objective} & \multicolumn{1}{c}{\#Branches} & \multicolumn{1}{c}{CPU Time} \\
\midrule

\texttt{ft20} & 1165.00 & 40568 & \cellcolor{green!50}0.73 & 1165.00 & 622493 & \cellcolor{red!50}66.48\\
\texttt{la25} & 977.00 & 443022 & \cellcolor{red!50}17.95 & 977.00 & 72039 & \cellcolor{green!50}10.12\\
\texttt{orb03} & 1005.00 & 239964 & \cellcolor{green!50}8.26 & 1005.00 & 808686 & \cellcolor{red!50}74.13\\
\texttt{swv16} & 2924.00 & 5766 & \cellcolor{green!50}0.12 & 2924.00 & 981377 & \cellcolor{red!50}225.94\\
\bottomrule
\end{tabular}
\end{footnotesize}

\vfill

\begin{itemize}
  \item Not as good on jobshop

  \vfill\pause
  \item yet

  \vfill\pause
  \item But also, CP Optimizer:
  \vfill
  \begin{itemize}
    \item \memph{Can handle time transitions, optional tasks, cumulative resources, reservoirs, etc.}
  \end{itemize}
\end{itemize}

\vfill

\end{frame}


% \end{document}

\begin{frame}
\frametitle{How does it work?}

% \begin{itemize}
%   \item Hybrid CP/SAT/SMT
% \end{itemize}

\begin{myblock}{Variables}
\begin{itemize}
  \item \memph{Numeric} variables (bounds, generic type)

  \vfill
  \item \memph{Boolean} variables

  \vfill
  \begin{itemize}
     \item Some Boolean variables can have a semantic (in difference logic: \memph{$y - x \leq k$} with $x$ and $y$ numeric variables)
   \end{itemize} 
\end{itemize}
\end{myblock}

\begin{myblock}{Global Constraints}
\begin{itemize}
  \item {Edge-Finding} 

  \vfill
  \item \memph{Precedence Reasoning} 
\end{itemize}
\end{myblock}

\begin{myblock}{Clause Learning}
\begin{itemize}
  \item Literals in the clause can be

  \vfill
  \begin{itemize}
    \item Pure Booleans; Difference logic literals ($y - x \leq k$); or Bound literals ($x \leq k$)
  \end{itemize}
\end{itemize}
\end{myblock}

\end{frame}


\section{Search}

\begin{frame}
\frametitle{Mistral (back to 2008)}

\pause
\begin{itemize}
  \item Mistral happenned to be very good on the Open Shop instances of the solver competition (D. Grimes)

  \vfill\pause
  \begin{itemize}
     \item No sophisticated propagation, no dedicated heuristic

     \vfill
     \item \memph{Binary Disjunction} $b_{ij} = \texttt{true} \implies e_i \leq s_j$ and $b_{ij} = \texttt{false} \implies e_j \leq s_i$
   \end{itemize} 

   \vfill\pause
   \begin{itemize}
     \item The \memph{weighted degree} heuristic works very well

     \vfill
     \begin{itemize}
       \item Increment the \memph{weight} of the variables involved in a \memph{failed} constraint ($b_{ij},s_i,s_j$)

       \vfill
       \item select the variable with minimum ratio \emph{domain size} over \emph{weight}
     \end{itemize}

      \vfill\pause
      \item Branching on the Boolean variables $b_{ij}$ is a good idea

   \end{itemize}

   \vfill\pause
   \item Paper ``Closing the Open-Shop'' with Diarmuid Grimes and Arnaud Malapert
\end{itemize}

\end{frame}


\begin{frame}
\frametitle{Edge Branching}

\begin{colorschedfigure}{.5}

\uncover<1>{
\VariableTask{0}{13}{5}{1}{1}{A}{}
\VariableTask{5}{16}{3}{1}{2}{B}{}
\VariableTask{8}{20}{4}{1}{3}{C}{}
\VariableTask{12}{25}{5}{1}{4}{D}{}
}
\uncover<2>{
\VariableTask{0}{12}{5}{1}{1}{A}{}
\VariableTask{5}{15}{3}{1}{2}{B}{}
\VariableTask{8}{20}{4}{1}{3}{C}{}
\VariableTask{12}{25}{5}{1}{4}{D}{}
\fill[pattern=north east lines] (12,-1) rectangle (13,-2);
\fill[pattern=north east lines] (15,-2) rectangle (16,-3);
}
\uncover<3>{
\VariableTask{0}{13}{5}{1}{1}{A}{}
\VariableTask{7}{16}{3}{1}{2}{B}{}
\VariableTask{10}{20}{4}{1}{3}{C}{}
\VariableTask{14}{25}{5}{1}{4}{D}{}
\fill[pattern=north east lines] (5,-2) rectangle (7,-3);
\fill[pattern=north east lines] (8,-3) rectangle (10,-4);
\fill[pattern=north east lines] (12,-4) rectangle (14,-5);
}

\uncover<1>{
\VariableTask{0}{15}{3}{1}{5}{D}{}
\VariableTask{3}{19}{4}{1}{6}{B}{}
\VariableTask{7}{21}{2}{1}{7}{A}{}
\VariableTask{9}{25}{4}{1}{8}{C}{}
}
\uncover<2>{
\VariableTask{0}{15}{3}{1}{5}{D}{}
\VariableTask{8}{19}{4}{1}{6}{B}{}
\VariableTask{12}{21}{2}{1}{7}{A}{}
\VariableTask{14}{25}{4}{1}{8}{C}{}
\fill[pattern=north east lines] (3,-6) rectangle (8,-7);
\fill[pattern=north east lines] (7,-7) rectangle (12,-8);
\fill[pattern=north east lines] (9,-8) rectangle (14,-9);
}
\uncover<3>{
\VariableTask{0}{12}{3}{1}{5}{D}{}
\VariableTask{3}{16}{4}{1}{6}{B}{}
\VariableTask{7}{21}{2}{1}{7}{A}{}
\VariableTask{9}{25}{4}{1}{8}{C}{}
\fill[pattern=north east lines] (12,-5) rectangle (15,-6);
\fill[pattern=north east lines] (16,-6) rectangle (19,-7);
}

\VariableTask{0}{13}{3}{1}{9}{B}{}
\VariableTask{3}{16}{3}{1}{10}{D}{}
\VariableTask{6}{22}{6}{1}{11}{A}{}
\VariableTask{12}{25}{3}{1}{12}{C}{}

\uncover<2>{
  \draw[color=black, ->, ultra thick, shorten >=3pt, shorten <=3pt] (11.5, -6) -- (10.5, -3);
}
\uncover<3>{
  \draw[color=black, ->, ultra thick, shorten >=3pt, shorten <=3pt] (10.5, -6) -- (11.5, -3);
}
\end{colorschedfigure}

\end{frame}




\begin{frame}
\frametitle{Search Tree}

\vfill

\begin{myblock}{Branching on Boolean disjunctions}

\begin{itemize}
  \item Often triggers some pruning

  \vfill\pause
  \item \emph{Tends to constrain both branches}
\end{itemize}

\end{myblock}

\vfill\pause

\begin{myblock}{Branching on start times}

\begin{itemize}
  \item The size of the tree depends on the chosen precision 

  \vfill\pause

% \begin{itemize}
%   \item Dedicated heuristics to mitigate that
% \end{itemize}

% \vfill
  \item Might make the tree \memph{unbalanced}:

  \vfill
  \begin{itemize}
    \item The constraint $s_i = t$ is much stronger than  $s_i \neq t$
  \end{itemize}

  \vfill\pause
  \item Dedicated heuristics to mitigate that

\end{itemize}

\end{myblock}

\vfill

\end{frame}






\begin{frame}
\frametitle{Difference Logic, Temporal Graph and Bounds Consistency}

\begin{columns}

\column{.48\textwidth}

\begin{myblock}{Difference Logic}

System of inequations of the form
\[
\memph{y - x \leq k \textrm{~(with k a constant)}}
\]


% \begin{itemize}
%   \item \memph{Theory} in SMT
%   \vfill
%   \begin{itemize}
%     \item Compute the consistent closure 

%     \vfill
%     \item Explain the closure
%   \end{itemize}
% \end{itemize}

\end{myblock}

\column{.48\textwidth}

\begin{myblock}{Temporal Graph}

\begin{itemize}
  \item Vertices: temporal variables

  \vfill
  \item Edges: maximum delay

\vfill
\begin{itemize}
  \item label \memph{$k$} on edge \memph{$(x \rightarrow y)$}: $x$ is at most $k$ after $y$
\end{itemize}
  
\end{itemize}

\end{myblock}

\end{columns}

\vfill

\begin{itemize}
  \item Closure of the difference logic system $\equiv$ Transitive closure of the graph

  \vfill
  \item Satisfiable if and only if there is no negative cycle 
\end{itemize}

% \vspace{-1cm}



\end{frame}


\begin{frame}
\frametitle{Precedence Graph}

  \begin{center}
 % \resizebox{10cm}{4.5cm}{%
    \begin{colorschedfigure}{.6}
      \uncover<1>{
        \node[vertexstyle] (x0) at (0.000000,4.000000) {o};
\node[vertexstyle,extstA] (x2) at (2.500000,4.000000) {$s_{A}$};
\node[vertexstyle,extstA] (x3) at (5.000000,4.000000) {$e_{A}$};
\node[vertexstyle,extstA] (x4) at (7.500000,3.000000) {$s_{B}$};
\node[vertexstyle,extstA] (x6) at (7.500000,6.000000) {$s_{C}$};
\node[vertexstyle,extstA] (x5) at (10.000000,3.000000) {$e_{B}$};
\node[vertexstyle,extstA] (x7) at (10.000000,6.000000) {$e_{C}$};
\node[vertexstyle,extstB] (x8) at (12.500000,2.000000) {$s_{D}$};
\node[vertexstyle,extstB] (x10) at (12.500000,4.000000) {$s_{E}$};
\node[vertexstyle,extstB] (x12) at (12.500000,6.000000) {$s_{F}$};
\node[vertexstyle,extstB] (x9) at (15.000000,2.000000) {$e_{D}$};
\node[vertexstyle,extstB] (x11) at (15.000000,4.000000) {$e_{E}$};
\node[vertexstyle,extstB] (x13) at (15.000000,6.000000) {$e_{F}$};
\node[vertexstyle] (x1) at (17.500000,4.000000) {m};
\path (x2) edge[edgestyle, bend left] node[labelstyle] {\textcolor{coolblack}{$-4$}} (x3);
\path (x3) edge[edgestyle, bend left] node[labelstyle] {\textcolor{coolblack}{$6$}} (x2);
\path (x3) edge[edgestyle] node[labelstyle] {\textcolor{coolblack}{$0$}} (x4);
\path (x3) edge[edgestyle] node[labelstyle] {\textcolor{coolblack}{$0$}} (x6);
\path (x4) edge[edgestyle, bend left] node[labelstyle] {\textcolor{coolblack}{$-3$}} (x5);
\path (x5) edge[edgestyle, bend left] node[labelstyle] {\textcolor{coolblack}{$3$}} (x4);
\path (x5) edge[edgestyle] node[labelstyle] {\textcolor{coolblack}{$0$}} (x8);
\path (x5) edge[edgestyle] node[labelstyle] {\textcolor{coolblack}{$0$}} (x10);
\path (x6) edge[edgestyle, bend left] node[labelstyle] {\textcolor{coolblack}{$-5$}} (x7);
\path (x7) edge[edgestyle, bend left] node[labelstyle] {\textcolor{coolblack}{$5$}} (x6);
\path (x7) edge[edgestyle] node[labelstyle] {\textcolor{coolblack}{$0$}} (x12);
\path (x8) edge[edgestyle, bend left] node[labelstyle] {\textcolor{coolblack}{$-4$}} (x9);
\path (x9) edge[edgestyle, bend left] node[labelstyle] {\textcolor{coolblack}{$6$}} (x8);
\path (x10) edge[edgestyle, bend left] node[labelstyle] {\textcolor{coolblack}{$-3$}} (x11);
\path (x11) edge[edgestyle, bend left] node[labelstyle] {\textcolor{coolblack}{$7$}} (x10);
\path (x12) edge[edgestyle, bend left] node[labelstyle] {\textcolor{coolblack}{$-5$}} (x13);
\path (x13) edge[edgestyle, bend left] node[labelstyle] {\textcolor{coolblack}{$5$}} (x12);
\path (x0) edge[edgestyle] node[labelstyle] {\textcolor{coolblack}{$0$}} (x2);
\path (x9) edge[edgestyle] node[labelstyle] {\textcolor{coolblack}{$0$}} (x1);
\path (x11) edge[edgestyle] node[labelstyle] {\textcolor{coolblack}{$0$}} (x1);
\path (x13) edge[edgestyle] node[labelstyle] {\textcolor{coolblack}{$0$}} (x1);
\path (x1) edge[edgestyle, bend right=60] node[labelstyle] {\textcolor{coolblack}{$20$}} (x0);

      }
      \uncover<2->{
        \node[vertexstyle] (x0) at (0.000000,4.000000) {o};
\node[vertexstyle,extstA] (x2) at (2.500000,4.000000) {$s_{A}$};
\node[vertexstyle,extstA] (x3) at (5.000000,4.000000) {$e_{A}$};
\node[vertexstyle,extstA] (x4) at (7.500000,3.000000) {$s_{B}$};
\node[vertexstyle,extstA] (x6) at (7.500000,6.000000) {$s_{C}$};
\node[vertexstyle,extstA] (x5) at (10.000000,3.000000) {$e_{B}$};
\node[vertexstyle,extstA] (x7) at (10.000000,6.000000) {$e_{C}$};
\node[vertexstyle,extstB] (x8) at (12.500000,2.000000) {$s_{D}$};
\node[vertexstyle,extstB] (x10) at (12.500000,4.000000) {$s_{E}$};
\node[vertexstyle,extstB] (x12) at (12.500000,6.000000) {$s_{F}$};
\node[vertexstyle,extstB] (x9) at (15.000000,2.000000) {$e_{D}$};
\node[vertexstyle,extstB] (x11) at (15.000000,4.000000) {$e_{E}$};
\node[vertexstyle,extstB] (x13) at (15.000000,6.000000) {$e_{F}$};
\node[vertexstyle] (x1) at (17.500000,4.000000) {m};
\path (x2) edge[edgestyle, bend left, color=bostonuniversityred] node[labelstyle] {\textcolor{bostonuniversityred}{$-4$}} (x3);
\path (x3) edge[edgestyle, bend left] node[labelstyle] {\textcolor{coolblack}{$6$}} (x2);
\path (x3) edge[edgestyle, color=bostonuniversityred] node[labelstyle] {\textcolor{bostonuniversityred}{$0$}} (x4);
\path (x3) edge[edgestyle] node[labelstyle] {\textcolor{coolblack}{$0$}} (x6);
\path (x4) edge[edgestyle, bend left, color=bostonuniversityred] node[labelstyle] {\textcolor{bostonuniversityred}{$-3$}} (x5);
\path (x5) edge[edgestyle, bend left] node[labelstyle] {\textcolor{coolblack}{$3$}} (x4);
\path (x5) edge[edgestyle, color=bostonuniversityred] node[labelstyle] {\textcolor{bostonuniversityred}{$0$}} (x8);
\path (x5) edge[edgestyle] node[labelstyle] {\textcolor{coolblack}{$0$}} (x10);
\path (x6) edge[edgestyle, bend left] node[labelstyle] {\textcolor{coolblack}{$-5$}} (x7);
\path (x7) edge[edgestyle, bend left] node[labelstyle] {\textcolor{coolblack}{$5$}} (x6);
\path (x7) edge[edgestyle] node[labelstyle] {\textcolor{coolblack}{$0$}} (x12);
\path (x8) edge[edgestyle, bend left] node[labelstyle] {\textcolor{coolblack}{$-4$}} (x9);
\path (x9) edge[edgestyle, bend left] node[labelstyle] {\textcolor{coolblack}{$6$}} (x8);
\path (x10) edge[edgestyle, bend left] node[labelstyle] {\textcolor{coolblack}{$-3$}} (x11);
\path (x11) edge[edgestyle, bend left] node[labelstyle] {\textcolor{coolblack}{$7$}} (x10);
\path (x12) edge[edgestyle, bend left] node[labelstyle] {\textcolor{coolblack}{$-5$}} (x13);
\path (x13) edge[edgestyle, bend left] node[labelstyle] {\textcolor{coolblack}{$5$}} (x12);
\path (x0) edge[edgestyle, color=bostonuniversityred] node[labelstyle] {\textcolor{bostonuniversityred}{$0$}} (x2);
\path (x9) edge[edgestyle] node[labelstyle] {\textcolor{coolblack}{$0$}} (x1);
\path (x11) edge[edgestyle] node[labelstyle] {\textcolor{coolblack}{$0$}} (x1);
\path (x13) edge[edgestyle] node[labelstyle] {\textcolor{coolblack}{$0$}} (x1);
\path (x1) edge[edgestyle, bend right=60] node[labelstyle] {\textcolor{coolblack}{$20$}} (x0);

      }
      \uncover<3->{
        \node[vertexstyle] (x0) at (0.000000,4.000000) {o};
\node[vertexstyle,extstA] (x2) at (2.500000,4.000000) {$s_{A}$};
\node[vertexstyle,extstA] (x3) at (5.000000,4.000000) {$e_{A}$};
\node[vertexstyle,extstA] (x4) at (7.500000,3.000000) {$s_{B}$};
\node[vertexstyle,extstA] (x6) at (7.500000,6.000000) {$s_{C}$};
\node[vertexstyle,extstA] (x5) at (10.000000,3.000000) {$e_{B}$};
\node[vertexstyle,extstA] (x7) at (10.000000,6.000000) {$e_{C}$};
\node[vertexstyle,extstB] (x8) at (12.500000,2.000000) {$s_{D}$};
\node[vertexstyle,extstB] (x10) at (12.500000,4.000000) {$s_{E}$};
\node[vertexstyle,extstB] (x12) at (12.500000,6.000000) {$s_{F}$};
\node[vertexstyle,extstB] (x9) at (15.000000,2.000000) {$e_{D}$};
\node[vertexstyle,extstB] (x11) at (15.000000,4.000000) {$e_{E}$};
\node[vertexstyle,extstB] (x13) at (15.000000,6.000000) {$e_{F}$};
\node[vertexstyle] (x1) at (17.500000,4.000000) {m};
\path (x2) edge[edgestyle, bend left] node[labelstyle] {\textcolor{coolblack}{$-4$}} (x3);
\path (x3) edge[edgestyle, bend left] node[labelstyle] {\textcolor{coolblack}{$6$}} (x2);
\path (x3) edge[edgestyle] node[labelstyle] {\textcolor{coolblack}{$0$}} (x4);
\path (x3) edge[edgestyle, color=bostonuniversityred] node[labelstyle] {\textcolor{bostonuniversityred}{$0$}} (x6);
\path (x4) edge[edgestyle, bend left] node[labelstyle] {\textcolor{coolblack}{$-3$}} (x5);
\path (x5) edge[edgestyle, bend left] node[labelstyle] {\textcolor{coolblack}{$3$}} (x4);
\path (x5) edge[edgestyle] node[labelstyle] {\textcolor{coolblack}{$0$}} (x8);
\path (x5) edge[edgestyle] node[labelstyle] {\textcolor{coolblack}{$0$}} (x10);
\path (x6) edge[edgestyle, bend left, color=bostonuniversityred] node[labelstyle] {\textcolor{bostonuniversityred}{$-5$}} (x7);
\path (x7) edge[edgestyle, bend left] node[labelstyle] {\textcolor{coolblack}{$5$}} (x6);
\path (x7) edge[edgestyle, color=bostonuniversityred] node[labelstyle] {\textcolor{bostonuniversityred}{$0$}} (x12);
\path (x8) edge[edgestyle, bend left] node[labelstyle] {\textcolor{coolblack}{$-4$}} (x9);
\path (x9) edge[edgestyle, bend left] node[labelstyle] {\textcolor{coolblack}{$6$}} (x8);
\path (x10) edge[edgestyle, bend left] node[labelstyle] {\textcolor{coolblack}{$-3$}} (x11);
\path (x11) edge[edgestyle, bend left] node[labelstyle] {\textcolor{coolblack}{$7$}} (x10);
\path (x12) edge[edgestyle, bend left, color=bostonuniversityred] node[labelstyle] {\textcolor{bostonuniversityred}{$-5$}} (x13);
\path (x13) edge[edgestyle, bend left] node[labelstyle] {\textcolor{coolblack}{$5$}} (x12);
\path (x0) edge[edgestyle] node[labelstyle] {\textcolor{coolblack}{$0$}} (x2);
\path (x9) edge[edgestyle] node[labelstyle] {\textcolor{coolblack}{$0$}} (x1);
\path (x11) edge[edgestyle] node[labelstyle] {\textcolor{coolblack}{$0$}} (x1);
\path (x13) edge[edgestyle, color=bostonuniversityred] node[labelstyle] {\textcolor{bostonuniversityred}{$0$}} (x1);
\path (x1) edge[edgestyle, bend right=60, color=bostonuniversityred] node[labelstyle] {\textcolor{bostonuniversityred}{$20$}} (x0);

      }
    \end{colorschedfigure}
    % }
  \end{center}

  \begin{itemize}
    \item Lower bound of a temporal variable $x$: (opposite of the) \memph{shortest path} from $x$ to $0$

\vfill
    \item Upper bound of a temporal variable $x$: \memph{shortest path} from $0$ to $x$
  \end{itemize}

\end{frame}



\begin{frame}
\frametitle{Difference Logic with Full Transitivity}

\vfill
\begin{itemize}
  \item Why not trying to make \memph{all the deductions} (full transitive closure)
  \vfill
  \begin{itemize}
    \item A lot of information can be deduced, but can it be used to propagate?

    \vfill
    \item Potential gains for clause learning
  \end{itemize}


  \vfill
  \begin{itemize}
    \item Computing the transitive closure with Floyd-Warshall is too costly

    \vfill
    \begin{itemize}
      \item Incremental closure; Merge vertices in null-cycles: \memph{not sufficient}
    \end{itemize}

    \vfill
    \item No really promising ideas on how to use the transitive closure to propagate

    \vfill
    \item Implementation of learning was very complex
  \end{itemize}

\end{itemize}

\vfill
\end{frame}

\begin{frame}
\frametitle{Chosen Representation}

\vfill

  \begin{center}
    \begin{colorschedfigure}{.6}
        \node[vertexstyle] (x0) at (0.000000,4.000000) {o};
\node[vertexstyle,extstA] (x2) at (2.500000,4.000000) {$s_{A}$};
\node[vertexstyle,extstA] (x3) at (5.000000,4.000000) {$e_{A}$};
\node[vertexstyle,extstA] (x4) at (7.500000,3.000000) {$s_{B}$};
\node[vertexstyle,extstA] (x6) at (7.500000,6.000000) {$s_{C}$};
\node[vertexstyle,extstA] (x5) at (10.000000,3.000000) {$e_{B}$};
\node[vertexstyle,extstA] (x7) at (10.000000,6.000000) {$e_{C}$};
\node[vertexstyle,extstB] (x8) at (12.500000,2.000000) {$s_{D}$};
\node[vertexstyle,extstB] (x10) at (12.500000,4.000000) {$s_{E}$};
\node[vertexstyle,extstB] (x12) at (12.500000,6.000000) {$s_{F}$};
\node[vertexstyle,extstB] (x9) at (15.000000,2.000000) {$e_{D}$};
\node[vertexstyle,extstB] (x11) at (15.000000,4.000000) {$e_{E}$};
\node[vertexstyle,extstB] (x13) at (15.000000,6.000000) {$e_{F}$};
\node[vertexstyle] (x1) at (17.500000,4.000000) {m};
\path (x2) edge[edgestyle, bend left] node[labelstyle] {\textcolor{coolblack}{$-4$}} (x3);
\path (x3) edge[edgestyle, bend left] node[labelstyle] {\textcolor{coolblack}{$6$}} (x2);
\path (x3) edge[edgestyle] node[labelstyle] {\textcolor{coolblack}{$0$}} (x4);
\path (x3) edge[edgestyle] node[labelstyle] {\textcolor{coolblack}{$0$}} (x6);
\path (x4) edge[edgestyle, bend left] node[labelstyle] {\textcolor{coolblack}{$-3$}} (x5);
\path (x5) edge[edgestyle, bend left] node[labelstyle] {\textcolor{coolblack}{$3$}} (x4);
\path (x5) edge[edgestyle] node[labelstyle] {\textcolor{coolblack}{$0$}} (x8);
\path (x5) edge[edgestyle] node[labelstyle] {\textcolor{coolblack}{$0$}} (x10);
\path (x6) edge[edgestyle, bend left] node[labelstyle] {\textcolor{coolblack}{$-5$}} (x7);
\path (x7) edge[edgestyle, bend left] node[labelstyle] {\textcolor{coolblack}{$5$}} (x6);
\path (x7) edge[edgestyle] node[labelstyle] {\textcolor{coolblack}{$0$}} (x12);
\path (x8) edge[edgestyle, bend left] node[labelstyle] {\textcolor{coolblack}{$-4$}} (x9);
\path (x9) edge[edgestyle, bend left] node[labelstyle] {\textcolor{coolblack}{$6$}} (x8);
\path (x10) edge[edgestyle, bend left] node[labelstyle] {\textcolor{coolblack}{$-3$}} (x11);
\path (x11) edge[edgestyle, bend left] node[labelstyle] {\textcolor{coolblack}{$7$}} (x10);
\path (x12) edge[edgestyle, bend left] node[labelstyle] {\textcolor{coolblack}{$-5$}} (x13);
\path (x13) edge[edgestyle, bend left] node[labelstyle] {\textcolor{coolblack}{$5$}} (x12);
\path (x0) edge[edgestyle] node[labelstyle] {\textcolor{coolblack}{$0$}} (x2);
\path (x9) edge[edgestyle] node[labelstyle] {\textcolor{coolblack}{$0$}} (x1);
\path (x11) edge[edgestyle] node[labelstyle] {\textcolor{coolblack}{$0$}} (x1);
\path (x13) edge[edgestyle] node[labelstyle] {\textcolor{coolblack}{$0$}} (x1);
\path (x1) edge[edgestyle, bend right=60] node[labelstyle] {\textcolor{coolblack}{$20$}} (x0);

\uncover<2->{
        \path (x5) edge[edgestyle, bend left=15, densely dashed, shorten <=3pt, shorten >=3pt] node[labelstyle, near end] {\textcolor{coolblack}{$0$}} (x6);
        \path (x7) edge[edgestyle, bend left=15, densely dashed, shorten <=3pt, shorten >=3pt] node[labelstyle, near start] {\textcolor{coolblack}{$0$}} (x4);
        }
    \end{colorschedfigure}
  \end{center}
\vfill

\uncover<2->{
\begin{itemize}
  \item $b_{ij} \implies e_i \leq s_j$; $\neg b_{ij} \implies e_j \leq s_i$ (as in Mistral)

  \vfill\uncover<3->{
  \item The only difference between updating the bounds using Bellman-Ford in this graph, or Arc Consistency in the constraint graph is that negative cycles can be detected earlier
  }
\end{itemize}
}

\vfill
\end{frame}




\section{Clause Learning}

%------------------------------------------------------------------------------%
% File:        learn.tex
%
% Description: 
%
% Created:     27 Jun 2013.
%
% Author:      Joao Marques-Silva (jpms).
%------------------------------------------------------------------------------%
\def\aformula{\ensuremath{\varphi}}

\begin{frame}[fragile]
\frametitle{Implication Graph}

\begin{tikzpicture}[shorten >=1pt,node distance=2.0cm,on
      grid,>=stealth', lines/.style={line
        width=1pt,color=coolblack},thick, color=coolblack,
      text=coolblack] %[auto]

\node (L)  {Level}; %[drop shadow] 
\node (D)  [right= 1cm of L] {Dec.};    % 2cm and 4cm
\node (I)  [right= of D] {Unit Prop.};     % 2cm and 4cm
% \node (C1) [right= 3cm of I] {$(\bar{z} \lor \bar{x} \lor a)$};
% \node (C2) [right= 2cm of C1] {$(\bar{z} \lor b)$};
% \node (C3) [right= 1.7cm of C2] {$(\bar{a} \lor \bar{b})$};
% \node (C4) [right= 1.7cm of C3] {$(\bar{y} \lor \bar{c})$};

\node (L0) [visible on=<2->, below= 0.75cm of L]  {$0$};
\node (L1) [visible on=<3->, below= 0.75cm of L0] {$1$};
\node (L2) [visible on=<4->, below= 1cm of L1] {$2$};
\node (L3) [visible on=<5->, below= 1cm of L2] {$3$};
\node (L4) [visible on=<6->, below= 1cm of L3] {$4$};
\node (L5) [visible on=<7->, below= 1.25cm of L4] {$5$};

\node (D0) [visible on=<2->, right= 1.25cm of L0] {$\emptyset$};
\node [visible on=<3->] (a)  [right= 1.25cm of L1] {$a$};
\node [visible on=<4->] (b)  [right= 1.25cm of L2] {$b$};
\node [visible on=<5->] (c)  [right= 1.25cm of L3] {$c$};
\node [visible on=<6->] (d)  [right= 1.25cm of L4] {$d$};
\node [visible on=<7->] (e)  [right= 1.25cm of L5] {$e$};


\node (nf) [visible on=<4->, right= of b]  {$\bar{f}$};
\path[->,line width=1pt,color=coolblack,visible on=<4->]
  (b)  edge  node [above] {} (nf)
  ;

\node (g) [visible on=<4->, right= of nf]  {$g$};
\path[->,line width=1pt,color=coolblack,visible on=<4->]
  (nf)  edge node [above] {} (g)
  ;
\path[->,line width=1pt,color=coolblack,visible on=<4->]
(a)  edge [bend left=30] node [above] {} (g)
;


\node (nh) [visible on=<6->, right= of d]  {$\bar{h}$};
\path[->,line width=1pt,color=coolblack,visible on=<6->]
  (d)  edge node [above] {} (nh)
  ;
\path[->,line width=1pt,color=coolblack,visible on=<6->]
(b)  edge [bend left=30] node [above] {} (nh)
(c)  edge [bend left=30] node [above] {} (nh)
;

\node (i) [visible on=<6->, right= of nh]  {$i$};
\path[->,line width=1pt,color=coolblack,visible on=<6->]
  (nh)  edge node [above] {} (i)
  (g)  edge node [above] {} (i)
  ;

  \node (j) [visible on=<7->, above right= .5cm and 2cm of e]  {$j$};
  \node (k) [visible on=<7->, below right= .5cm and 2cm of e]  {$k$};


\path[->,line width=1pt,color=coolblack,visible on=<7->]
  (e)  edge node [above] {} (j)
  (e)  edge node [above] {} (k)
  ;

  \node (l) [visible on=<7->, below= 1.25cm of i]  {$l$};
  \path[->,line width=1pt,color=coolblack,visible on=<7->]
  (j)  edge node [above] {} (l)
  (k)  edge node [above] {} (l)
  (g) edge [bend left=30] node [above] {} (l)
  ;

    \node (FAIL) [visible on=<7->, right=  of l]  {$\bot$};
  \path[->,line width=1pt,color=coolblack,visible on=<7->]
  (l)  edge node [above] {} (FAIL)
  (g) edge [bend left=30] node [above] {} (FAIL)
  ;
%

\node (formula) [right = 10cm of L2] {
  $\begin{aligned}
\aformula = &\textcolor<4>{bostonuniversityred}{(\bar{b} \lor \bar{f})} & \land \\
&\textcolor<4>{bostonuniversityred}{(\bar{a} \lor f \lor g)} & \land \\
&\textcolor<6>{bostonuniversityred}{(\bar{d} \lor \bar{c} \lor \bar{b} \lor \bar{h})} & \land \\
&\textcolor<6>{bostonuniversityred}{(h \lor \bar{g} \lor i)} & \land \\
&\textcolor<7>{bostonuniversityred}{(\bar{e} \lor j)} & \land \\
&\textcolor<7>{bostonuniversityred}{(\bar{e} \lor k)} & \land \\
&\textcolor<7>{bostonuniversityred}{(\bar{j} \lor \bar{k} \lor \bar{g} \lor l)} &  \land \\
&\textcolor<7>{bostonuniversityred}{(\bar{l} \lor \bar{g})} & 
  \end{aligned}$
  };

  \node [visible on=<8->, below = 0cm of formula.south] (cut1) {\memph{$(a \land b \land c \land d \land e) \implies \bot$}%
  %and hence $\aformula \lmod (\bar{a} \lor \bar{b} \lor \bar{c} \lor \bar{d} \lor \bar{e})$
  };
\node [visible on=<8-9>, fill=bittersweet] at (a) {$a$};
\node [visible on=<8-9>, fill=bittersweet] at (b) {$b$};
\node [visible on=<8>, fill=bittersweet] at (c) {$c$};
\node [visible on=<8>, fill=bittersweet] at (d) {$d$};
\node [visible on=<8-9>, fill=bittersweet] at (e) {$e$};


  \node [visible on=<9->, below = 0.5cm of cut1] (cut2) {\memph{$(a \land b \land e) \implies \bot$}%
  %and hence $\aformula \lmod (\bar{a} \lor \bar{b} \lor \bar{e})$
  };

  \node [visible on=<10->, below = 0.5cm of cut2] (cut3) {\memph{$(g \land j \land k) \implies \bot$}% 
  %and hence $\aformula \lmod (\bar{g} \lor \bar{e})$
  };
  \node [visible on=<10->, fill=bittersweet] at (g) {$g$};
  \node [visible on=<10>, fill=bittersweet] at (j) {$j$};
  \node [visible on=<10>, fill=bittersweet] at (k) {$k$};


    \node [visible on=<11->, below = 0.5cm of cut3] (cut4) {\memph{$(l \land g) \implies \bot$}% 
  %and hence $\aformula \lmod (\bar{g} \lor \bar{e})$
  };
  \node [visible on=<11->, fill=bittersweet] at (l) {$l$};




% \node (C1) [right = 6cm of I] {$(\bar{b} \lor \bar{f}) \land$} ;
% \node (C2) [below = of C1] {$(\bar{a} \lor f \lor g) \land$} ;
% \node (C3) [below = of C2] {$(\bar{d} \lor \bar{c} \lor \bar{b} \lor \bar{h}) \land$} ;
% \node (C4) [below = of C3] {$(h \lor \bar{g} \lor i) \land$} ;
% \node (C5) [below = of C4] {$(\bar{e} \lor j) \land$} ;
% \node (C6) [below = of C5] {$(\bar{e} \lor k) \land$} ;
% \node (C7) [below = of C6] {$(\bar{j} \lor \bar{k} \lor \bar{g})$} ;


% %
% \node (a)  [visible on=<5->, right= of z]  {$a$};
% \node (nc)  [visible on=<4->, right= 2.5cm of y]  {$\bar{c}$};
% \node (b)  [visible on=<5->, below= 1.25cm of a]  {$b$};
% \node (k)  [visible on=<5->, right= of a]  {$\cnode$};

% \path[->,line width=1pt,color=coolblack,visible on=<5->]
%   (x)  edge [bend left=30] node [above] {} (a)
%   ;

% \path[->,line width=1pt,color=coolblack,visible on=<4->]
%   (y)  edge node [above] {} (nc)
% ;

% \path[->,line width=1pt,color=coolblack,visible on=<5>]
%   (z)  edge node [above] {} (a)
%        edge node [above] {} (b)
%   (a)  edge node [above] {} (k)
%   (b)  edge node [above] {} (k)
% ;

% \path[->,line width=2pt,color=darkred,visible on=<6->]
%   (z)  edge node [above] {} (a)
%        edge node [above] {} (b)
%   (a)  edge node [above] {} (k)
%   (b)  edge node [above] {} (k)
% ;
% %\begin{pgfonlayer}{background}
% %  \path[-,selected edge,visible on=<3->]
% %  (z)  edge node [above] {} (a)
% %       edge node [above] {} (b)
% %  (a)  edge node [above] {} (k)
% %  (b)  edge node [above] {} (k)
% %  %(x)  edge [bend left=30] node [above] {} (a)
% %  ;
% %  %\path[selected edge] (x.center) -- (a.center);
% %\end{pgfonlayer}
\end{tikzpicture}


\end{frame}


\begin{frame}
\frametitle{Explanation}

\vfill

\begin{itemize}
  \item New bound \memph{$x \leq k$} via Bellman-Ford: we store $y$ the predecessor of $x$ in the shortest path from $0$

  \vfill

  \begin{itemize}
    \item \memph{$x \leq k$} is explained by \memph{$x - y \leq k'$} and $y \leq k-k'$
  \end{itemize}

  \vfill

  \item Truth value for $b_{ij} \implies e_i \leq s_j$; $\neg b_{ij} \implies e_j \leq s_i$ via binary disjunction:

\vfill

  \begin{itemize}
    \item $b_{ij}$ is explained by $e_j > s_i$ ($e_j \geq k$ and $s_i \leq k'$)
  \end{itemize}

\end{itemize}

\vfill

\begin{myblock}{Aries -- Arthur Bit-Monnod's solver}
\begin{itemize}
  \item Aries is pretty much Mistral + clause learning

  \vfill
  \item During his thesis, Mohamed wrote a version of Mistral with the same ideas

  \vfill
  \begin{itemize}
    \item Key difference, there was no literal for bounds ($x \leq k$)
  \end{itemize}
\end{itemize}

\end{myblock}

\vfill


\end{frame}


\section{Precedence Reasoning}

\begin{frame}
  \frametitle{Propagation: Precedences \& Disjunctive Resource}

\vfill 

\begin{itemize}
  \item Precedences are added during search (by decisions, propagation of edge-finding, learnt clauses and binary disjunctions). \uncover<2->{\memph{How can we take advantage of those?}}
\end{itemize}

\vfill

\begin{columns}

\column{.6\textwidth}

\begin{center}

\begin{tikzpicture}

\node[shape=circle, thick, draw=coolblack] (a) {{\scriptsize $a$}};
\node[shape=circle, thick, draw=coolblack, above right=1cm and 1.5cm of a] (b) {{\scriptsize $b$}};
\node[shape=circle, thick, draw=coolblack, right=1.5cm of a] (c) {{\scriptsize $c$}};
\node[shape=circle, thick, draw=coolblack, below right=1cm and 1.5cm of a] (d) {{\scriptsize $d$}};
\node[shape=circle, thick, draw=coolblack, above left=.5cm and 1.5cm of a] (e) {{\scriptsize $e$}};
\node[shape=circle, thick, draw=coolblack, above left=.5cm and 1.5cm of e] (f) {{\scriptsize $f$}};
\node[shape=circle, thick, draw=coolblack, below left=.5cm and 1.5cm of e] (g) {{\scriptsize $g$}};
\node[shape=circle, thick, draw=coolblack, below right=.5cm and 1.5cm of c] (h) {{\scriptsize $h$}};
\node[shape=circle, thick, draw=coolblack, above right=.5cm and 1.5cm of c] (i) {{\scriptsize $i$}};

\uncover<3>{
\node[shape=circle, thick, draw=coolblack, fill=red!50] at (e) {{\scriptsize $e$}};
\node[shape=circle, thick, draw=bostonuniversityred] at (f) {{\scriptsize $f$}};
\node[shape=circle, thick, draw=bostonuniversityred] at (g) {{\scriptsize $g$}};
}


\uncover<4>{
\node[shape=circle, thick, draw=bostonuniversityred] at (a) {{\scriptsize $a$}};
\node[shape=circle, thick, draw=bostonuniversityred] at (b) {{\scriptsize $b$}};
\node[shape=circle, thick, draw=bostonuniversityred] at (c) {{\scriptsize $c$}};
\node[shape=circle, thick, draw=bostonuniversityred] at (d) {{\scriptsize $d$}};
\node[shape=circle, thick, draw=bostonuniversityred] at (e) {{\scriptsize $e$}};
\node[shape=circle, thick, draw=coolblack, fill=red!50] at (g) {{\scriptsize $g$}};
\node[shape=circle, thick, draw=bostonuniversityred] at (h) {{\scriptsize $h$}};
\node[shape=circle, thick, draw=bostonuniversityred] at (i) {{\scriptsize $i$}};
}

\path (a) edge[-latex, very thick, shorten <=3pt, shorten >=3pt] (b);
\path (a) edge[-latex, very thick, shorten <=3pt, shorten >=3pt] (c);
\path (a) edge[-latex, very thick, shorten <=3pt, shorten >=3pt] (d);
\path (f) edge[-latex, very thick, shorten <=3pt, shorten >=3pt] (e);
\path (g) edge[-latex, very thick, shorten <=3pt, shorten >=3pt] (e);
\path (e) edge[-latex, very thick, shorten <=3pt, shorten >=3pt] (a);
\path (c) edge[-latex, very thick, shorten <=3pt, shorten >=3pt] (h);
\path (c) edge[-latex, very thick, shorten <=3pt, shorten >=3pt] (i);


\end{tikzpicture}

\end{center}

\column{.4\textwidth}

\vbox to .4\textheight{%
\uncover<3->{
\begin{itemize}
  \item If tasks $a,\ldots,i$ all share the same disjunctive resource, then:
  \vfill
  \begin{itemize}
    \item $s_e \geq \min(s_f,s_g) + p_f + p_g$
    \vfill\uncover<4->{
    \item $e_g \leq \max(e_b,e_i,e_h,e_d) - (p_i + p_h + p_c + p_b + p_d + p_a + p_e + p_g)$
    }
  \end{itemize}
\end{itemize}
}
}

\end{columns}

\vfill

\end{frame}



\begin{frame}
  \frametitle{First Step: (Incrementally) Computing the Transitive Closure}

\vfill 
\begin{center}

\begin{tikzpicture}

\node[shape=circle, thick, draw=coolblack] (a) {{\scriptsize $a$}};
\node[shape=circle, thick, draw=coolblack, above right=1cm and 1.5cm of a] (b) {{\scriptsize $b$}};
\node[shape=circle, thick, draw=coolblack, right=1.5cm of a] (c) {{\scriptsize $c$}};
\node[shape=circle, thick, draw=coolblack, below right=1cm and 1.5cm of a] (d) {{\scriptsize $d$}};
\node[shape=circle, thick, draw=coolblack, above left=.5cm and 1.5cm of a] (e) {{\scriptsize $e$}};
\node[shape=circle, thick, draw=coolblack, above left=.5cm and 1.5cm of e] (f) {{\scriptsize $f$}};
\node[shape=circle, thick, draw=coolblack, below left=.5cm and 1.5cm of e] (g) {{\scriptsize $g$}};
\node[shape=circle, thick, draw=coolblack, below right=.5cm and 1.5cm of c] (h) {{\scriptsize $h$}};
\node[shape=circle, thick, draw=coolblack, above right=.5cm and 1.5cm of c] (i) {{\scriptsize $i$}};

\uncover<2->{
\path (a) edge[-latex, very thick, shorten <=3pt, shorten >=3pt] (b);
}
\uncover<3->{
\path (a) edge[-latex, very thick, shorten <=3pt, shorten >=3pt] (c);
}
\uncover<4->{
\path (a) edge[-latex, very thick, shorten <=3pt, shorten >=3pt] (d);
}
\uncover<5->{
\path (e) edge[-latex, very thick, shorten <=3pt, shorten >=3pt] (a);
}
\uncover<6->{
\path (e) edge[-latex, densely dashed, very thick, shorten <=3pt, shorten >=3pt] (b);
\path (e) edge[-latex, densely dashed, very thick, shorten <=3pt, shorten >=3pt, bend left=10] (c);
\path (e) edge[-latex, densely dashed, very thick, shorten <=3pt, shorten >=3pt, bend right] (d);
}
\uncover<7->{
\path (f) edge[-latex, very thick, shorten <=3pt, shorten >=3pt] (e);
\path (f) edge[-latex, densely dashed, very thick, shorten <=3pt, shorten >=3pt, bend right] (a);
\path (f) edge[-latex, densely dashed, very thick, shorten <=3pt, shorten >=3pt, bend left] (b);
\path (f) edge[-latex, densely dashed, very thick, shorten <=3pt, shorten >=3pt, bend left=10] (c);
\path (f) edge[-latex, densely dashed, very thick, shorten <=3pt, shorten >=3pt, bend right] (d);
}
% \path (g) edge[-latex, very thick, shorten <=3pt, shorten >=3pt] (e);
% \path (c) edge[-latex, very thick, shorten <=3pt, shorten >=3pt] (h);
% \path (c) edge[-latex, very thick, shorten <=3pt, shorten >=3pt] (i);

\end{tikzpicture}

\end{center}

\vfill

\end{frame}


\begin{frame}
  \frametitle{Second Step: Computing the (Lower) Bounds}

\vfill 

\begin{columns}

\column{.45\textwidth}

\begin{center}

\begin{tikzpicture}

% \node[shape=circle, thick, draw=coolblack, above left=.5cm and 1.5cm of e] (f) {{\scriptsize 0}};
% \node[shape=circle, thick, draw=coolblack, below left=.5cm and 1.5cm of e] (g) {{\scriptsize 0}};
\node[shape=circle, thick, draw=coolblack, above right=.5cm and 1.5cm of a] (e) {{\scriptsize $e:0$}};
\node[shape=circle, thick, draw=coolblack] (a) {{\scriptsize $a:0$}};
\node[shape=circle, thick, draw=coolblack, above left=1cm and 1.5cm of a] (b) {{\scriptsize $b:0$}};
\node[shape=circle, thick, draw=coolblack, left=1.5cm of a] (c) {{\scriptsize $c:0$}};
\node[shape=circle, thick, draw=coolblack, below left=1cm and 1.5cm of a] (d) {{\scriptsize $d:0$}};
% \node[shape=circle, thick, draw=coolblack, above=1cm of e] (h) {{\scriptsize 0}};
\node[shape=circle, thick, draw=coolblack, above right=.5cm and 1.5cm of b] (f) {{\scriptsize $f:0$}};

% \uncover<4->{
% \node[shape=circle, thick, draw=coolblack, fill=red!50] at (f) {{\scriptsize 0}};
% }
% \uncover<5->{
% \node[shape=circle, thick, draw=coolblack, fill=red!50] at (g) {{\scriptsize 0}};
% }
\uncover<4>{
\node[shape=circle, thick, draw=coolblack, fill=red!50] at (e) {{\scriptsize $e:0$}};
}
\uncover<5>{
\node[shape=circle, thick, draw=coolblack, fill=red!50] at (f) {{\scriptsize $f:0$}};
\node[shape=circle, thick, draw=bostonuniversityred, fill=white] at (e) {{\scriptsize $e:5$}};
% \node[shape=circle, thick, draw=bostonuniversityred, fill=white] at (f) {{\scriptsize 5}};
% \node[shape=circle, thick, draw=bostonuniversityred, fill=white] at (g) {{\scriptsize 5}};
}
\uncover<6>{
\node[shape=circle, thick, draw=coolblack, fill=red!50] at (a) {{\scriptsize $a:0$}};
\node[shape=circle, thick, draw=bostonuniversityred, fill=white] at (e) {{\scriptsize $e:8$}};
% \node[shape=circle, thick, draw=bostonuniversityred, fill=white] at (g) {{\scriptsize 5}};
}
\uncover<7>{
\node[shape=circle, thick, draw=coolblack, fill=red!50] at (b) {{\scriptsize $b:0$}};
\node[shape=circle, thick, draw=bostonuniversityred, fill=white] at (f) {{\scriptsize $f:10$}};
\node[shape=circle, thick, draw=bostonuniversityred, fill=white] at (e) {{\scriptsize $e:18$}};
\node[shape=circle, thick, draw=bostonuniversityred, fill=white] at (a) {{\scriptsize $a:10$}};
}
\uncover<8>{
\node[shape=circle, thick, draw=coolblack, fill=red!50] at (c) {{\scriptsize $c:0$}};
\node[shape=circle, thick, draw=bostonuniversityred, fill=white] at (e) {{\scriptsize $e:22$}};
\node[shape=circle, thick, draw=bostonuniversityred, fill=white] at (a) {{\scriptsize $a:14$}};
}
\uncover<9>{
\node[shape=circle, thick, draw=coolblack, fill=red!50] at (c) {{\scriptsize $c:0$}};
\node[shape=circle, thick, draw=bostonuniversityred, fill=white] at (e) {{\scriptsize $e:22$}};
\node[shape=circle, thick, draw=bostonuniversityred, fill=white] at (a) {{\scriptsize $a:14$}};
}
\uncover<10>{
\node[shape=circle, thick, draw=coolblack, fill=red!50] at (d) {{\scriptsize $d:0$}};
\node[shape=circle, thick, draw=bostonuniversityred, fill=white] at (e) {{\scriptsize $e:29$}};
\node[shape=circle, thick, draw=bostonuniversityred, fill=white] at (a) {{\scriptsize $a:21$}};
}

% \node[shape=circle, thick, draw=coolblack, fill=white] at (e) {{\scriptsize 5}};
% \node[shape=circle, thick, draw=coolblack, fill=white] at (a) {{\scriptsize 5}};
% \node[shape=circle, thick, draw=coolblack, fill=white] at (b) {{\scriptsize 5}};
% \node[shape=circle, thick, draw=coolblack, fill=white] at (c) {{\scriptsize 5}};
% \node[shape=circle, thick, draw=coolblack, fill=white] at (d) {{\scriptsize 5}};
% }


\path (a) edge[latex-, very thick, shorten <=3pt, shorten >=3pt] (b);
\path (a) edge[latex-, very thick, shorten <=3pt, shorten >=3pt] (c);
\path (a) edge[latex-, very thick, shorten <=3pt, shorten >=3pt] (d);
\path (e) edge[latex-, very thick, shorten <=3pt, shorten >=3pt] (a);
\path (e) edge[latex-, densely dashed, very thick, shorten <=3pt, shorten >=3pt, bend right=10] (b);
\path (e) edge[latex-, densely dashed, very thick, shorten <=3pt, shorten >=3pt, bend right=10] (c);
\path (e) edge[latex-, densely dashed, very thick, shorten <=3pt, shorten >=3pt, bend left] (d);
\path (f) edge[latex-, very thick, shorten <=3pt, shorten >=3pt] (b);
\path (e) edge[latex-, very thick, shorten <=3pt, shorten >=3pt] (f);


\end{tikzpicture}

\end{center}

\column{.55\textwidth}

\vbox to .6\textheight{%
\uncover<2->{
\begin{itemize}
\item Assume $p_a=3, p_b=10, p_c=4, p_d=7, p_e=5, p_f=5$

  \vfill
\item Initialise $offset[i]$ to $0$ for every task $t_i$
\vfill\uncover<3->{

  \item Order the tasks by decreasing earliest start time (necessarily a topological order because of propagation) 

  \vfill

  \item For each task $t_i$ in this order, and for each arc $(t_i,t_j)$ increment $offset[j]$ by $p_i$ and set \memph{$s_j \geq s_i + offset[j]$}

\vfill\uncover<5->{
  \begin{itemize}
    \item $s_e \geq \min(s_f)+5$;  \uncover<6->{\memph{$s_e \geq \min(s_a)+8$}; \uncover<7->{\memph{$s_e \geq \min(s_b)+18$}; \uncover<8->{\memph{$s_e \geq \min(s_c)+22$}; \uncover<9->{\memph{$s_e \geq \min(s_d)+29$};}}}}

    \vfill\uncover<7->{
    \item $s_a \geq \min(s_b)+10$; \uncover<8->{\memph{$s_a \geq \min(s_c)+14$}; \uncover<9->{\memph{$s_d \geq \min(s_c)+21$};}}}

    \vfill\uncover<7->{
    \item $s_f \geq \min(s_b)+10$;
    }
  \end{itemize}
}
}
\end{itemize}
}
}

\end{columns}

\vfill

\end{frame}




\begin{frame}
  \frametitle{Explanation}

\vfill 

\begin{columns}

\column{.45\textwidth}

\begin{center}

\begin{tikzpicture}

\node[shape=circle, thick, draw=coolblack, above right=.5cm and 1.5cm of a] (e) {{\scriptsize $e:0$}};
\node[shape=circle, thick, draw=coolblack] (a) {{\scriptsize $a:0$}};
\node[shape=circle, thick, draw=coolblack, above left=1cm and 1.5cm of a] (b) {{\scriptsize $b:0$}};
\node[shape=circle, thick, draw=coolblack, left=1.5cm of a] (c) {{\scriptsize $c:0$}};
\node[shape=circle, thick, draw=coolblack, below left=1cm and 1.5cm of a] (d) {{\scriptsize $d:0$}};
\node[shape=circle, thick, draw=coolblack, above right=.5cm and 1.5cm of b] (f) {{\scriptsize $f:0$}};


\node[shape=circle, thick, draw=coolblack, fill=red!50] at (b) {{\scriptsize $b:0$}};
\node[shape=circle, thick, draw=bostonuniversityred, fill=white] at (f) {{\scriptsize $f:10$}};
\node[shape=circle, thick, draw=bostonuniversityred, fill=white] at (e) {{\scriptsize $e:18$}};
\node[shape=circle, thick, draw=bostonuniversityred, fill=white] at (a) {{\scriptsize $a:10$}};



\path (a) edge[latex-, very thick, shorten <=3pt, shorten >=3pt] (b);
\path (a) edge[latex-, very thick, shorten <=3pt, shorten >=3pt] (c);
\path (a) edge[latex-, very thick, shorten <=3pt, shorten >=3pt] (d);
\path (e) edge[latex-, very thick, shorten <=3pt, shorten >=3pt] (a);
\path (e) edge[latex-, densely dashed, very thick, shorten <=3pt, shorten >=3pt, bend right=10] (b);
\path (e) edge[latex-, densely dashed, very thick, shorten <=3pt, shorten >=3pt, bend right=10] (c);
\path (e) edge[latex-, densely dashed, very thick, shorten <=3pt, shorten >=3pt, bend left] (d);
\path (f) edge[latex-, very thick, shorten <=3pt, shorten >=3pt] (b);
\path (e) edge[latex-, very thick, shorten <=3pt, shorten >=3pt] (f);


\end{tikzpicture}

\end{center}

\column{.55\textwidth}

\vbox to .6\textheight{%
\begin{itemize}
\item The bound \memph{$s_e \geq \min(s_b)+18$} is explained by 
\vfill
\begin{itemize}
  \item $b_{fe}$ ($f$ comes before $e$); 

  \vfill
  \item $b_{ae}$ ($a$ comes before $e$); 

  \vfill
  \item $b_{be}$ ($b$ comes before $e$); 

  \vfill
  \item $s_{b} \geq \min(s_b)$; $s_{a} \geq \min(s_b)$; and $s_{f} \geq \min(s_b)$; 
\end{itemize}
\vfill
\item The transitive edge \memph{$b_{be}$} is explained by $b_{ba}$ and $b_{ae}$

\end{itemize}
}

\end{columns}

\vfill

\end{frame}



\section{Edge Finding}

\begin{frame}
  \frametitle{Propagation: The \emph{Edge-Finding} Rule}

\vfill  

\begin{itemize}
  \item Make deductions w.r.t. \nooverlap  and project them on the variables' bounds
\end{itemize}

\vfill

\begin{columns}

\column{.35\textwidth}

\vbox to .5\textheight{%

\uncover<2->{
\begin{itemize}
  \item Suppose that a task in $\Omega$ is not last

\vfill
\begin{itemize}
  \item Make a hypothetical overload test
  \end{itemize}

\vfill\uncover<3->{
  \item Proof by contradiction
  \vfill
  \begin{itemize}
    \item That task \myemph{must be last} in $\Omega$
  \end{itemize}
  }

\vfill\uncover<4->{
  \item Symmetrical backward pass
  }

  \vfill\uncover<4->{
  \item Algorithm in $O(n \log n)$ \citation{Vil\`{i}m 04}
  }
  \end{itemize}
}
}%

\column{.65\textwidth}


\begin{center}
\begin{colorschedfigure}{.7}
\PrintGrid{11}{4}
\node[] (es) at (2.3, 2) {{\scriptsize \begin{tabular}{c}\textcolor{blue!50!black}{earliest}\\\textcolor{blue!50!black}{start}\end{tabular}}};
\node[] (ls) at (7.7, 2) {{\scriptsize \begin{tabular}{c}\textcolor{blue!50!black}{latest}\\\textcolor{blue!50!black}{start}\end{tabular}}};
\node[] (ee) at (5.3, 2) {{\scriptsize \begin{tabular}{c}\textcolor{red!50!black}{earliest}\\\textcolor{red!50!black}{end}\end{tabular}}};
\node[] (le) at (10.7, 2) {{\scriptsize \begin{tabular}{c}\textcolor{red!50!black}{latest}\\\textcolor{red!50!black}{end}\end{tabular}}};
\draw[->, shorten >=3pt,color=blue!50!black] (ls) -- (7.5,0);
\Pruning{10}{11}{1}{0}
\Pruning{6}{11}{2}{1}
\Pruning{8}{11}{1}{3}
\Pruning{0}{2}{1}{0}
\Pruning{0}{3}{1}{1}
\Pruning{0}{2}{1}{2}


\only<1-2>{
\draw[->, shorten >=3pt,color=red!50!black] (ee) -- (5.5,0);
\draw[->, shorten >=3pt,color=blue!50!black] (es) -- (2.5,0);
}

\only<3->{
\draw[->, shorten >=3pt, color=red!50!black] (ee.south east) to[bend left=10] (9.5,0);
\draw[->, shorten >=3pt,color=blue!50!black] (es.south east) to[bend left=10] (6.5,0);
}



\draw[->, shorten >=3pt,color=red!50!black] (le) -- (10.5,0);

\uncover<1-2>{
\PreemptiveVariableTask{2}{8}{5}{11}{3}{1}{0}{A}{}
}
\uncover<1-3>{
\PreemptiveVariableTask{0}{7}{2}{9}{2}{1}{3}{A}{}
}
\PreemptiveVariableTask{2}{5}{4}{7}{2}{1}{2}{A}{}
\PreemptiveVariableTask{3}{5}{5}{7}{2}{1}{1}{A}{}

\uncover<2>{
\draw[color=black, thick, rounded corners] (-\tasksep,\tasksep) rectangle (11+\tasksep,-1-\tasksep);
  % \PruningUnder{8}{10}{4}{0}
  \Pruning{8}{10}{1}{0}
}

\uncover<3->{
\PreemptiveVariableTask{6}{8}{9}{11}{3}{1}{0}{A}{}
\PruningOver{2}{6}{1}{0}
}

\uncover<4->{
% \GroundTask[]{0}{3}{1}{3}{A}{}
\PreemptiveVariableTask{0}{3}{1}{4}{2}{1}{3}{A}{}
\PruningUnder{4}{8}{1}{3}
}


\end{colorschedfigure}

\end{center}

\end{columns}

\vfill  
      
\end{frame}


\begin{frame}
\frametitle{Overload Checking -- Theta Tree}
\begin{itemize}
  \item Order the tasks by non-decreasing \memph{due date} to compute \memph{$\leftcut{{\cal T}}{j}$} for all \memph{$j \in {\cal T}$}
  \item Order the tasks by non-decreasing \memph{release date} to compute \memph{$\ect{\leftcut{{\cal T}}{j}}$}
\end{itemize}

\begin{myblock}{Solutions}
  \begin{itemize}
    \item Theta tree \citation{Vil\'{i}m et al. 04}
    \begin{itemize}
      \item Explore nested sets of tasks in any order (here non-decreasing due dates)
      \item Incrementally compute a property (here \memph{$\ect{\leftcut{{\cal T}}{j}}$}) requiring another order
    \end{itemize}
  \end{itemize}
\end{myblock}
\end{frame}


\begin{frame}
\frametitle{Theta Tree}
\begin{columns}

\column{.5\textwidth}
  \begin{colorschedfigure}{.3}
    \uncover<1>{
\PrintTics{0,5,...,26}{1.0000}
\PrintGrid{25}{6}
\LeftExtensibleVariableTask{0.000000}{18.750000}{6.250000}{6.250000}{1}{0}{C}{A}
\LeftExtensibleVariableTask{5.000000}{12.500000}{5.000000}{5.000000}{1}{1}{C}{B}
\LeftExtensibleVariableTask{2.500000}{15.000000}{3.750000}{3.750000}{1}{2}{C}{C}
\LeftExtensibleVariableTask{8.750000}{17.500000}{5.000000}{5.000000}{1}{3}{C}{D}
\LeftExtensibleVariableTask{3.750000}{25.000000}{7.500000}{7.500000}{1}{4}{C}{E}
\LeftExtensibleVariableTask{12.500000}{20.000000}{3.750000}{3.750000}{1}{5}{C}{F}
}
\uncover<2>{
\PrintTics{0,5,...,26}{1.0000}
\PrintGrid{25}{6}
\LeftExtensibleVariableTask{0.000000}{18.750000}{6.250000}{6.250000}{1}{0}{C}{A}
\LeftExtensibleVariableTask{2.500000}{15.000000}{3.750000}{3.750000}{1}{1}{C}{C}
\LeftExtensibleVariableTask{3.750000}{25.000000}{7.500000}{7.500000}{1}{2}{C}{E}
\LeftExtensibleVariableTask{5.000000}{12.500000}{5.000000}{5.000000}{1}{3}{C}{B}
\LeftExtensibleVariableTask{8.750000}{17.500000}{5.000000}{5.000000}{1}{4}{C}{D}
\LeftExtensibleVariableTask{12.500000}{20.000000}{3.750000}{3.750000}{1}{5}{C}{F}
}
\uncover<3->{
\PrintTics{0,5,...,26}{1.0000}
\PrintGrid{25}{6}
}
\uncover<3->{\LeftExtensibleVariableTask{5.000000}{12.500000}{5.000000}{5.000000}{1}{0}{C}{B}}
\uncover<4->{\LeftExtensibleVariableTask{2.500000}{15.000000}{3.750000}{3.750000}{1}{1}{C}{C}}
\uncover<5->{\LeftExtensibleVariableTask{8.750000}{17.500000}{5.000000}{5.000000}{1}{2}{C}{D}}
\uncover<6->{\LeftExtensibleVariableTask{0.000000}{18.750000}{6.250000}{6.250000}{1}{3}{C}{A}}
\uncover<7->{\LeftExtensibleVariableTask{12.500000}{20.000000}{3.750000}{3.750000}{1}{4}{C}{F}}
\uncover<8->{\LeftExtensibleVariableTask{3.750000}{25.000000}{7.500000}{7.500000}{1}{5}{C}{E}}

  \end{colorschedfigure}

\uncover<10->{
\begin{itemize}
  \item Explanation $s_A \geq 0 \land s_B \geq 0 \land \ldots \land s_F \geq 0$ and $e_A \leq 25 \land e_B \leq 25 \land \ldots \land e_F \leq 25$
  \vfill
  \uncover<11->{
  \item There can be ``holes''
  }
\end{itemize}
}

\uncover<11->{
   \begin{colorschedfigure}{.3}
    
\PrintTics{0,5,...,26}{1.0000}
\PrintGrid{25}{6}
\LeftExtensibleVariableTask{1.500000}{12.500000}{4.000000}{4.000000}{1}{0}{C}{B}
\LeftExtensibleVariableTask{1.500000}{15.000000}{2.0000}{2.0000}{1}{1}{C}{C}
\LeftExtensibleVariableTask{13.0000}{22.00000}{3.000000}{3.000000}{1}{2}{C}{D}
\LeftExtensibleVariableTask{0.000000}{19.0000}{3.0000}{3.0000}{1}{3}{C}{A}
\LeftExtensibleVariableTask{15.00000}{23.000000}{4.0000}{4.0000}{1}{4}{C}{F}
\LeftExtensibleVariableTask{12.0000}{25.000000}{5.00000}{5.00000}{1}{5}{C}{E}
  \end{colorschedfigure}
  }

\column{.5\textwidth}
  \uncover<2->{
  \begin{downthreelvltree}
    \thetaroot{\only<3>{\textcolor{red!85!black}{4}}\only<4>{\textcolor{red!85!black}{7}}\only<5>{\textcolor{red!85!black}{11}}\only<6>{\textcolor{red!85!black}{16}}\only<7>{\textcolor{red!85!black}{19}}\only<8>{\textcolor{red!85!black}{25}}}{\only<3>{\textcolor{red!85!black}{8}}\only<4>{\textcolor{red!85!black}{9}}\only<5>{\textcolor{red!85!black}{13}}\only<6>{\textcolor{red!85!black}{16}}\only<7>{\textcolor{red!85!black}{19}}\only<8>{\textcolor{red!85!black}{25}}}{3}
child{
\ttnode{\only<3>{\textcolor{red!85!black}{4}}\only<4>{\textcolor{red!85!black}{7}}\only<5>{7}\only<6>{\textcolor{red!85!black}{12}}\only<7>{12}\only<8>{\textcolor{red!85!black}{18}}}{\only<3>{\textcolor{red!85!black}{8}}\only<4>{\textcolor{red!85!black}{9}}\only<5>{9}\only<6>{\textcolor{red!85!black}{12}}\only<7>{12}\only<8>{\textcolor{red!85!black}{18}}}{3}
child{
\ttnode{\only<4>{\textcolor{red!85!black}{3}}\only<5>{3}\only<6>{\textcolor{red!85!black}{8}}\only<7>{8}\only<8>{8}}{\only<4>{\textcolor{red!85!black}{5}}\only<5>{5}\only<6>{\textcolor{red!85!black}{8}}\only<7>{8}\only<8>{8}}{4}
child{
\ttnode{\only<6>{\textcolor{red!85!black}{5}}\only<7>{5}\only<8>{5}}{\only<6>{\textcolor{red!85!black}{5}}\only<7>{5}\only<8>{5}}{6}
edge from parent
       node[kant, left, yshift=1mm, pos=.6] {A}
}
child{
\ttnode{\only<4>{\textcolor{red!85!black}{3}}\only<5>{3}\only<6>{3}\only<7>{3}\only<8>{3}}{\only<4>{\textcolor{red!85!black}{5}}\only<5>{5}\only<6>{5}\only<7>{5}\only<8>{5}}{4}
edge from parent
       node[kant, right, yshift=1mm, pos=.6] {C}
}
}
child{
\ttnode{\only<3>{\textcolor{red!85!black}{4}}\only<4>{4}\only<5>{4}\only<6>{4}\only<7>{4}\only<8>{\textcolor{red!85!black}{10}}}{\only<3>{\textcolor{red!85!black}{8}}\only<4>{8}\only<5>{8}\only<6>{8}\only<7>{8}\only<8>{\textcolor{red!85!black}{13}}}{3}
child{
\ttnode{\only<8>{\textcolor{red!85!black}{6}}}{\only<8>{\textcolor{red!85!black}{9}}}{8}
edge from parent
       node[kant, left, yshift=1mm, pos=.6] {E}
}
child{
\ttnode{\only<3>{\textcolor{red!85!black}{4}}\only<4>{4}\only<5>{4}\only<6>{4}\only<7>{4}\only<8>{4}}{\only<3>{\textcolor{red!85!black}{8}}\only<4>{8}\only<5>{8}\only<6>{8}\only<7>{8}\only<8>{8}}{3}
edge from parent
       node[kant, right, yshift=1mm, pos=.6] {B}
}
}
}
child{
\ttnode{\only<5>{\textcolor{red!85!black}{4}}\only<6>{4}\only<7>{\textcolor{red!85!black}{7}}\only<8>{7}}{\only<5>{\textcolor{red!85!black}{11}}\only<6>{11}\only<7>{\textcolor{red!85!black}{14}}\only<8>{14}}{5}
child{
\ttnode{\only<5>{\textcolor{red!85!black}{4}}\only<6>{4}\only<7>{\textcolor{red!85!black}{7}}\only<8>{7}}{\only<5>{\textcolor{red!85!black}{11}}\only<6>{11}\only<7>{\textcolor{red!85!black}{14}}\only<8>{14}}{5}
child{
\ttnode{\only<5>{\textcolor{red!85!black}{4}}\only<6>{4}\only<7>{4}\only<8>{4}}{\only<5>{\textcolor{red!85!black}{11}}\only<6>{11}\only<7>{11}\only<8>{11}}{5}
edge from parent
       node[kant, left, yshift=1mm, pos=.6] {D}
}
child{
\ttnode{\only<7>{\textcolor{red!85!black}{3}}\only<8>{3}}{\only<7>{\textcolor{red!85!black}{13}}\only<8>{13}}{7}
edge from parent
       node[kant, right, yshift=1mm, pos=.6] {F}
}
}
};

  \end{downthreelvltree}
  }

\end{columns}
\end{frame}



\begin{frame}
  \frametitle{Propagation: The \emph{Edge-Finding} Rule}

\vfill 


\vfill  
      
\end{frame}



\section{Some Experiments}

\begin{frame}[fragile]
\frametitle{Open Shop}
\vfill
\begin{itemize}
	\item Small (64 tasks) but extremely hard instance
\end{itemize}
\vfill
\begin{center}
\cactus{Objective}{CPU time}{CPOptimizer, Tempo}{{{ 
(1077.4500000000044, 1.0) [a] 
(1077.0500000000043, 1.01) [a] 
(1076.7000000000044, 1.11) [a] 
(1076.5500000000043, 1.14) [a] 
(1076.0000000000043, 1.17) [a] 
(1075.9000000000044, 1.18) [a] 
(1075.3500000000045, 1.22) [a] 
(1075.3000000000045, 1.26) [a] 
(1075.0500000000045, 1.27) [a] 
(1074.9500000000046, 1.31) [a] 
(1074.8000000000045, 1.32) [a] 
(1074.5500000000045, 1.37) [a] 
(1074.4500000000046, 1.49) [a] 
(1074.3500000000047, 1.56) [a] 
(1074.3000000000047, 1.58) [a] 
(1073.9500000000048, 1.59) [a] 
(1073.7000000000048, 1.61) [a] 
(1073.350000000005, 1.65) [a] 
(1072.600000000005, 1.67) [a] 
(1072.500000000005, 1.68) [a] 
(1072.400000000005, 1.72) [a] 
(1072.3500000000051, 1.8) [a] 
(1072.3000000000052, 1.81) [a] 
(1071.7500000000052, 1.83) [a] 
(1071.5000000000052, 1.86) [a] 
(1070.7000000000053, 1.88) [a] 
(1070.6500000000053, 2.0) [a] 
(1070.4000000000053, 2.08) [a] 
(1070.0000000000052, 2.11) [a] 
(1069.8500000000051, 2.13) [a] 
(1069.5000000000052, 2.24) [a] 
(1068.8500000000051, 2.27) [a] 
(1068.8000000000052, 2.4) [a] 
(1068.6500000000053, 2.42) [a] 
(1068.4500000000053, 2.48) [a] 
(1067.900000000005, 2.5) [a] 
(1067.6000000000051, 2.55) [a] 
(1067.5000000000052, 2.61) [a] 
(1067.0000000000055, 2.67) [a] 
(1066.9500000000055, 2.73) [a] 
(1066.9000000000055, 2.75) [a] 
(1066.8500000000056, 2.84) [a] 
(1066.5000000000057, 2.91) [a] 
(1066.4000000000058, 2.93) [a] 
(1066.000000000006, 3.06) [a] 
(1065.450000000006, 3.18) [a] 
(1065.300000000006, 3.27) [a] 
(1065.2000000000062, 3.3) [a] 
(1064.8000000000063, 3.33) [a] 
(1064.6500000000062, 3.34) [a] 
(1064.2500000000061, 3.54) [a] 
(1064.100000000006, 3.68) [a] 
(1064.0000000000061, 3.76) [a] 
(1063.850000000006, 3.89) [a] 
(1063.600000000006, 3.95) [a] 
(1062.900000000006, 4.06) [a] 
(1062.750000000006, 4.09) [a] 
(1062.650000000006, 4.13) [a] 
(1062.600000000006, 4.15) [a] 
(1062.550000000006, 4.16) [a] 
(1062.400000000006, 4.25) [a] 
(1062.100000000006, 4.36) [a] 
(1061.800000000006, 4.44) [a] 
(1061.7500000000061, 4.45) [a] 
(1061.6500000000062, 4.47) [a] 
(1061.6000000000063, 4.64) [a] 
(1061.5000000000064, 4.7) [a] 
(1061.3500000000063, 4.79) [a] 
(1061.3000000000063, 4.9) [a] 
(1061.0000000000064, 4.93) [a] 
(1060.9500000000064, 5.5) [a] 
(1060.9000000000065, 5.52) [a] 
(1060.7000000000064, 5.79) [a] 
(1060.6500000000065, 5.9) [a] 
(1060.6000000000065, 5.92) [a] 
(1060.5000000000066, 5.99) [a] 
(1060.3000000000065, 6.01) [a] 
(1060.2000000000066, 6.06) [a] 
(1060.0500000000065, 6.13) [a] 
(1060.0000000000066, 6.15) [a] 
(1059.7000000000066, 6.31) [a] 
(1059.6500000000067, 6.56) [a] 
(1059.5500000000068, 6.63) [a] 
(1059.2000000000069, 6.65) [a] 
(1058.650000000007, 6.87) [a] 
(1058.550000000007, 6.88) [a] 
(1058.350000000007, 7.18) [a] 
(1058.250000000007, 7.41) [a] 
(1058.1500000000071, 7.58) [a] 
(1058.0500000000072, 7.63) [a] 
(1057.9500000000073, 7.65) [a] 
(1057.9000000000074, 7.75) [a] 
(1057.8500000000074, 8.47) [a] 
(1057.4500000000073, 9.02) [a] 
(1057.3000000000072, 9.08) [a] 
(1057.0000000000073, 9.51) [a] 
(1056.9500000000073, 9.56) [a] 
(1056.8000000000072, 9.58) [a] 
(1056.7500000000073, 9.65) [a] 
(1056.7000000000073, 9.74) [a] 
(1056.6000000000074, 9.75) [a] 
(1056.5500000000075, 11.04) [a] 
(1056.5000000000075, 13.54) [a] 
(1056.3500000000074, 13.58) [a] 
(1056.2500000000075, 13.61) [a] 
(1056.1500000000076, 13.63) [a] 
(1056.1000000000076, 14.13) [a] 
(1056.0000000000077, 14.14) [a] 
(1055.9500000000078, 14.32) [a] 
(1055.9000000000078, 14.4) [a] 
(1055.800000000008, 14.48) [a] 
(1055.6500000000078, 14.5) [a] 
(1055.6000000000079, 14.61) [a] 
(1055.500000000008, 14.63) [a] 
(1055.400000000008, 14.96) [a] 
(1055.250000000008, 15.43) [a] 
(1055.050000000008, 16.44) [a] 
(1054.8500000000079, 16.7) [a] 
(1054.800000000008, 17.08) [a] 
(1054.750000000008, 17.38) [a] 
(1054.550000000008, 17.39) [a] 
(1054.500000000008, 17.4) [a] 
(1054.400000000008, 17.58) [a] 
(1054.250000000008, 17.59) [a] 
(1054.150000000008, 19.43) [a] 
(1054.100000000008, 19.5) [a] 
(1054.0500000000081, 19.53) [a] 
(1054.0000000000082, 19.56) [a] 
(1053.8500000000083, 20.22) [a] 
(1053.8000000000084, 20.97) [a] 
(1053.7500000000084, 22.2) [a] 
(1053.7000000000085, 22.52) [a] 
(1053.6000000000085, 22.75) [a] 
(1053.4500000000085, 22.77) [a] 
(1053.3500000000085, 25.33) [a] 
(1053.1500000000087, 28.74) [a] 
(1052.9000000000087, 31.75) [a] 
(1052.7500000000086, 32.33) [a] 
(1052.5000000000086, 34.06) [a] 
(1052.4000000000087, 35.22) [a] 
(1052.3000000000088, 48.06) [a] 
(1052.1000000000088, 48.89) [a] 
(1051.9000000000087, 51.06) [a] 
(1051.7500000000086, 54.63) [a] 
(1051.6500000000087, 55.49) [a] 
(1051.6000000000088, 55.97) [a] 
(1051.4000000000087, 57.71) [a] 
(1051.2500000000086, 57.78) [a] 
(1050.9500000000087, 59.42) [a] 
(1050.9000000000087, 63.02) [a] 
(1050.8000000000088, 63.14) [a] 
(1050.7500000000089, 75.61) [a] 
(1050.3500000000088, 77.15) [a] 
(1050.200000000009, 94.95) [a] 
(1050.150000000009, 95.79) [a] 
(1049.700000000009, 96.64) [a] 
(1049.600000000009, 96.77) [a] 
(1049.550000000009, 106.97) [a] 
(1049.4500000000091, 107.36) [a] 
(1049.4000000000092, 109.11) [a] 
(1049.2000000000091, 111.94) [a] 
(1048.9000000000092, 142.25) [a] 
(1048.750000000009, 158.58) [a] 
(1048.6500000000092, 160.08) [a] 
(1048.5500000000093, 201.32) [a] 
(1048.4000000000094, 209.2) [a] 
(1048.3500000000095, 211.68) [a] 
(1048.3000000000095, 211.84) [a] 
(1048.1500000000096, 212.47) [a] 
(1048.0000000000095, 220.67) [a] 
(1047.9500000000096, 224.69) [a] 
(1047.9000000000096, 224.92) [a] 
(1047.8000000000097, 225.29) [a] 
(1047.7500000000098, 225.36) [a] 
(1047.6500000000099, 228.63) [a] 
(1047.60000000001, 228.93) [a] 
(1047.50000000001, 229.07) [a] 
(1047.30000000001, 244.24) [a] 
(1047.10000000001, 244.7) [a] 
(1047.00000000001, 244.72) [a] 
(1046.95000000001, 244.75) [a] 
(1046.75000000001, 260.11) [a] 
(1046.70000000001, 260.2) [a] 
(1046.45000000001, 273.79) [a] 
(1046.3500000000101, 274.74) [a] 
(1046.20000000001, 276.29) [a] 
(1045.90000000001, 276.36) [a] 
(1045.70000000001, 303.18) [a] 
(1045.55000000001, 303.83) [a] 
(1045.45000000001, 320.6) [a] 
(1045.3500000000101, 367.79) [a] 
(1045.3000000000102, 466.24) [a] 
(1045.2500000000102, 494.28) [a] 
(1045.2000000000103, 494.3) [a] 
(1045.0500000000104, 494.51) [a] 
(1045.0000000000105, 535.08) [a] 
(1044.8500000000104, 536.55) [a] 
(1044.7500000000105, 554.05) [a] 
(1044.4500000000105, 580.4) [a] 
(1044.4000000000106, 582.85) [a] 
(1044.3500000000106, 612.42) [a] 
(1044.3000000000106, 644.75) [a] 
(1044.2500000000107, 644.92) [a] 
(1044.2000000000107, 648.17) [a] 
(1044.1500000000108, 718.36) [a] 
(1044.1000000000108, 718.56) [a] 
(1044.0500000000109, 718.58) [a] 
(1044.000000000011, 746.06) [a] 
(1043.950000000011, 746.65) [a] 
(1043.900000000011, 776.25) [a] 
(1043.800000000011, 816.03) [a] 
(1043.7500000000111, 816.04) [a] 
(1043.7000000000112, 852.55) [a] 
(1043.6000000000113, 855.39) [a] 
(1043.5500000000113, 862.56) [a] 
(1043.3500000000113, 896.59) [a] 
(1043.3000000000113, 897.75) [a] 
(1043.2500000000114, 1009.97) [a] 
(1043.2000000000114, 1128.58) [a] 
(1043.1500000000115, 1304.92) [a] 
(1043.0500000000116, 1304.93) [a] 
(1042.9500000000116, 1639.03) [a] 
(1042.8000000000116, 1659.05) [a] 
(1042.7000000000116, 1956.67) [a] 
(1042.6500000000117, 2012.93) [a] 
(1042.6000000000117, 2148.33) [a] 
(1042.5000000000118, 2148.4) [a] 
(1042.4500000000119, 2168.95) [a] 
(1042.400000000012, 2169.2) [a] 
(1042.350000000012, 2231.75) [a] 
(1042.300000000012, 2231.82) [a] 
(1042.250000000012, 2231.83) [a] 
(1042.1500000000121, 2388.04) [a] 
(1041.9000000000121, 2465.56) [a] 
(1041.750000000012, 2518.41) [a] 
(1041.600000000012, 2522.7) [a] 
(1041.4500000000119, 3322.81) [a] 
(1041.400000000012, 3528.92) [a] 
(1041.350000000012, 3528.93) [a] 
(1041.250000000012, 3574.26) [a] 
(1041.100000000012, 3579.24) [a] 
(1041.050000000012, 3586.86) [a] 
(1041.000000000012, 3586.88) [a] 
(1040.9000000000121, 3659.86) [a] 
(1040.750000000012, 3700.83) [a] 
(1040.600000000012, 4024.9) [a] 
(1040.500000000012, 4703.38) [a] 
(1040.450000000012, 4748.4) [a] 
(1040.3500000000122, 5087.33) [a] 
(1040.200000000012, 5292.83) [a] 
(1040.1000000000122, 6269.38) [a] 
(1040.0500000000122, 6325.92) [a] 
(1039.9000000000121, 6556.05) [a] 
(1039.750000000012, 7021.54) [a] 
(1039.6500000000121, 7354.48) [a] 
(1039.6000000000122, 7355.88) [a] 
(1039.450000000012, 8743.7) [a] 
(1039.3500000000122, 9351.24) [a] 
(1039.3000000000122, 9351.5) [a] 
(1039.2000000000123, 9437.38) [a] 
(1039.1500000000124, 9521.06) [a] 
(1039.0000000000123, 10258.14) [a] 
},{
(1068.8000000000682, 1.003) [b] 
(1068.7500000000682, 1.007) [b] 
(1068.7000000000683, 1.015) [b] 
(1068.5000000000684, 1.018) [b] 
(1068.4500000000685, 1.02) [b] 
(1068.3500000000686, 1.03) [b] 
(1068.3000000000686, 1.031) [b] 
(1068.2000000000687, 1.032) [b] 
(1068.1500000000688, 1.038) [b] 
(1068.0500000000688, 1.045) [b] 
(1067.9000000000688, 1.049) [b] 
(1067.8500000000688, 1.052) [b] 
(1067.8000000000688, 1.056) [b] 
(1067.6500000000688, 1.06) [b] 
(1067.6000000000688, 1.078) [b] 
(1067.5500000000688, 1.084) [b] 
(1067.500000000069, 1.086) [b] 
(1067.450000000069, 1.091) [b] 
(1067.250000000069, 1.105) [b] 
(1067.150000000069, 1.115) [b] 
(1067.100000000069, 1.117) [b] 
(1067.050000000069, 1.13) [b] 
(1067.0000000000691, 1.136) [b] 
(1066.9500000000692, 1.138) [b] 
(1066.9000000000692, 1.146) [b] 
(1066.8500000000693, 1.155) [b] 
(1066.7500000000693, 1.157) [b] 
(1066.6500000000694, 1.16) [b] 
(1066.6000000000695, 1.162) [b] 
(1066.5500000000695, 1.183) [b] 
(1066.5000000000696, 1.206) [b] 
(1066.3500000000695, 1.207) [b] 
(1066.3000000000695, 1.208) [b] 
(1066.2500000000696, 1.219) [b] 
(1066.2000000000696, 1.223) [b] 
(1066.1500000000697, 1.225) [b] 
(1065.8500000000697, 1.229) [b] 
(1065.8000000000698, 1.243) [b] 
(1065.7000000000698, 1.266) [b] 
(1065.60000000007, 1.268) [b] 
(1065.50000000007, 1.271) [b] 
(1065.45000000007, 1.272) [b] 
(1065.30000000007, 1.309) [b] 
(1065.20000000007, 1.315) [b] 
(1065.1500000000701, 1.32) [b] 
(1065.1000000000702, 1.33) [b] 
(1065.0000000000703, 1.338) [b] 
(1064.9500000000703, 1.34) [b] 
(1064.9000000000703, 1.35) [b] 
(1064.8500000000704, 1.366) [b] 
(1064.8000000000704, 1.389) [b] 
(1064.7500000000705, 1.438) [b] 
(1064.4000000000706, 1.471) [b] 
(1064.3000000000707, 1.474) [b] 
(1064.1500000000708, 1.487) [b] 
(1064.1000000000708, 1.492) [b] 
(1064.050000000071, 1.495) [b] 
(1064.000000000071, 1.502) [b] 
(1063.950000000071, 1.512) [b] 
(1063.900000000071, 1.524) [b] 
(1063.850000000071, 1.527) [b] 
(1063.7500000000712, 1.538) [b] 
(1063.6500000000713, 1.551) [b] 
(1063.5500000000713, 1.556) [b] 
(1063.5000000000714, 1.557) [b] 
(1063.4500000000714, 1.567) [b] 
(1063.4000000000715, 1.576) [b] 
(1063.3500000000715, 1.591) [b] 
(1063.3000000000716, 1.601) [b] 
(1063.2500000000716, 1.602) [b] 
(1063.1000000000718, 1.603) [b] 
(1063.0000000000719, 1.608) [b] 
(1062.950000000072, 1.611) [b] 
(1062.900000000072, 1.616) [b] 
(1062.850000000072, 1.638) [b] 
(1062.750000000072, 1.652) [b] 
(1062.7000000000721, 1.665) [b] 
(1062.6000000000722, 1.673) [b] 
(1062.5500000000723, 1.677) [b] 
(1062.4500000000724, 1.698) [b] 
(1062.3500000000724, 1.723) [b] 
(1062.2500000000725, 1.726) [b] 
(1062.1500000000726, 1.74) [b] 
(1062.1000000000727, 1.741) [b] 
(1061.9500000000726, 1.744) [b] 
(1061.9000000000726, 1.753) [b] 
(1061.8500000000727, 1.763) [b] 
(1061.8000000000727, 1.769) [b] 
(1061.7500000000728, 1.781) [b] 
(1061.7000000000728, 1.786) [b] 
(1061.6500000000729, 1.788) [b] 
(1061.5000000000728, 1.795) [b] 
(1061.4500000000728, 1.797) [b] 
(1061.350000000073, 1.799) [b] 
(1061.1500000000729, 1.801) [b] 
(1060.9000000000729, 1.811) [b] 
(1060.850000000073, 1.842) [b] 
(1060.750000000073, 1.843) [b] 
(1060.700000000073, 1.864) [b] 
(1060.550000000073, 1.892) [b] 
(1060.500000000073, 1.896) [b] 
(1060.450000000073, 1.899) [b] 
(1060.400000000073, 1.951) [b] 
(1060.3500000000731, 1.965) [b] 
(1060.3000000000732, 1.987) [b] 
(1060.2500000000732, 2.032) [b] 
(1060.1000000000731, 2.093) [b] 
(1060.0000000000732, 2.181) [b] 
(1059.9500000000733, 2.186) [b] 
(1059.9000000000733, 2.187) [b] 
(1059.8500000000734, 2.191) [b] 
(1059.8000000000734, 2.194) [b] 
(1059.7000000000735, 2.208) [b] 
(1059.6500000000735, 2.246) [b] 
(1059.5000000000734, 2.26) [b] 
(1059.4500000000735, 2.29) [b] 
(1059.4000000000735, 2.303) [b] 
(1059.3500000000736, 2.362) [b] 
(1059.3000000000736, 2.373) [b] 
(1059.2500000000737, 2.388) [b] 
(1059.2000000000737, 2.395) [b] 
(1059.1000000000738, 2.409) [b] 
(1059.000000000074, 2.477) [b] 
(1058.900000000074, 2.484) [b] 
(1058.800000000074, 2.493) [b] 
(1058.7500000000741, 2.509) [b] 
(1058.7000000000742, 2.522) [b] 
(1058.6500000000742, 2.553) [b] 
(1058.6000000000743, 2.601) [b] 
(1058.5500000000743, 2.648) [b] 
(1058.5000000000744, 2.665) [b] 
(1058.4500000000744, 2.671) [b] 
(1058.3500000000745, 2.686) [b] 
(1058.3000000000745, 2.707) [b] 
(1058.2500000000746, 2.739) [b] 
(1058.0500000000745, 2.765) [b] 
(1057.9000000000744, 2.858) [b] 
(1057.8500000000745, 3.083) [b] 
(1057.7500000000746, 3.111) [b] 
(1057.6500000000747, 3.247) [b] 
(1057.6000000000747, 3.427) [b] 
(1057.5000000000748, 3.498) [b] 
(1057.4500000000749, 3.529) [b] 
(1057.100000000075, 3.549) [b] 
(1057.050000000075, 3.566) [b] 
(1057.000000000075, 3.621) [b] 
(1056.950000000075, 3.672) [b] 
(1056.8500000000752, 3.731) [b] 
(1056.7500000000753, 3.807) [b] 
(1056.7000000000753, 3.907) [b] 
(1056.6500000000754, 4.068) [b] 
(1056.5000000000753, 4.491) [b] 
(1056.2500000000753, 4.513) [b] 
(1056.0500000000752, 4.613) [b] 
(1056.0000000000753, 4.633) [b] 
(1055.9500000000753, 4.658) [b] 
(1055.8500000000754, 4.749) [b] 
(1055.8000000000754, 4.888) [b] 
(1055.7500000000755, 4.953) [b] 
(1055.7000000000755, 4.964) [b] 
(1055.6000000000756, 5.121) [b] 
(1055.5500000000757, 5.188) [b] 
(1055.3000000000757, 5.19) [b] 
(1054.8000000000757, 5.212) [b] 
(1054.7500000000757, 5.262) [b] 
(1054.6000000000756, 5.292) [b] 
(1054.5500000000757, 5.407) [b] 
(1054.4000000000756, 5.409) [b] 
(1054.3500000000756, 5.559) [b] 
(1054.3000000000757, 5.588) [b] 
(1054.2500000000757, 5.631) [b] 
(1054.1000000000756, 5.696) [b] 
(1054.0500000000757, 6.018) [b] 
(1054.0000000000757, 6.02) [b] 
(1053.9000000000758, 6.112) [b] 
(1053.8500000000759, 6.256) [b] 
(1053.750000000076, 6.371) [b] 
(1053.700000000076, 6.442) [b] 
(1053.650000000076, 6.496) [b] 
(1053.5500000000761, 6.548) [b] 
(1053.5000000000762, 6.909) [b] 
(1053.4500000000762, 7.257) [b] 
(1053.4000000000763, 7.273) [b] 
(1053.3500000000763, 7.452) [b] 
(1053.3000000000764, 7.573) [b] 
(1053.2500000000764, 7.708) [b] 
(1053.1500000000765, 7.764) [b] 
(1053.1000000000765, 7.774) [b] 
(1053.0500000000766, 7.913) [b] 
(1053.0000000000766, 8.253) [b] 
(1052.9500000000767, 8.627) [b] 
(1052.9000000000767, 8.905) [b] 
(1052.8500000000768, 8.947) [b] 
(1052.8000000000768, 9.082) [b] 
(1052.7500000000769, 9.314) [b] 
(1052.650000000077, 9.48) [b] 
(1052.5000000000769, 10.3) [b] 
(1052.400000000077, 10.92) [b] 
(1052.350000000077, 10.99) [b] 
(1052.100000000077, 11.31) [b] 
(1052.050000000077, 11.43) [b] 
(1052.000000000077, 11.97) [b] 
(1051.9500000000771, 12.1) [b] 
(1051.8500000000772, 12.34) [b] 
(1051.8000000000773, 12.51) [b] 
(1051.7000000000774, 13.49) [b] 
(1051.6500000000774, 13.86) [b] 
(1051.5000000000773, 14.42) [b] 
(1051.4500000000774, 14.59) [b] 
(1051.4000000000774, 15) [b] 
(1051.3500000000774, 15.03) [b] 
(1051.3000000000775, 15.71) [b] 
(1051.2500000000775, 15.91) [b] 
(1051.2000000000776, 16.14) [b] 
(1051.1500000000776, 16.18) [b] 
(1051.0500000000777, 16.36) [b] 
(1051.0000000000778, 16.81) [b] 
(1050.9000000000779, 17.06) [b] 
(1050.850000000078, 17.31) [b] 
(1050.7000000000778, 17.32) [b] 
(1050.6500000000779, 17.34) [b] 
(1050.600000000078, 17.36) [b] 
(1050.550000000078, 17.4) [b] 
(1050.4000000000779, 17.74) [b] 
(1050.300000000078, 18.46) [b] 
(1050.250000000078, 18.48) [b] 
(1050.150000000078, 18.55) [b] 
(1050.1000000000781, 18.63) [b] 
(1049.950000000078, 18.79) [b] 
(1049.900000000078, 19.72) [b] 
(1049.8500000000781, 19.8) [b] 
(1049.8000000000782, 20.23) [b] 
(1049.7500000000782, 20.56) [b] 
(1049.6500000000783, 20.74) [b] 
(1049.5500000000784, 23.08) [b] 
(1049.5000000000784, 23.19) [b] 
(1049.4500000000785, 23.24) [b] 
(1049.4000000000785, 23.65) [b] 
(1049.3000000000786, 26.63) [b] 
(1049.2000000000787, 27.25) [b] 
(1049.1000000000788, 27.52) [b] 
(1049.0500000000789, 29.77) [b] 
(1049.000000000079, 30.24) [b] 
(1048.950000000079, 32.72) [b] 
(1048.900000000079, 32.81) [b] 
(1048.850000000079, 34.07) [b] 
(1048.800000000079, 35.65) [b] 
(1048.7000000000792, 36.08) [b] 
(1048.6500000000792, 36.83) [b] 
(1048.6000000000793, 37.97) [b] 
(1048.4500000000792, 39.01) [b] 
(1048.3500000000793, 41.2) [b] 
(1048.2500000000794, 41.88) [b] 
(1048.2000000000794, 42.12) [b] 
(1048.1500000000794, 42.37) [b] 
(1048.0500000000795, 45.92) [b] 
(1048.0000000000796, 46.69) [b] 
(1047.9500000000796, 47.28) [b] 
(1047.8000000000795, 47.51) [b] 
(1047.6500000000794, 48.78) [b] 
(1047.5500000000795, 49.24) [b] 
(1047.5000000000796, 54.95) [b] 
(1047.4500000000796, 56.29) [b] 
(1047.4000000000797, 56.61) [b] 
(1047.3500000000797, 56.67) [b] 
(1047.3000000000798, 58.49) [b] 
(1047.2500000000798, 63.71) [b] 
(1047.2000000000799, 63.74) [b] 
(1047.15000000008, 65.43) [b] 
(1047.10000000008, 65.57) [b] 
(1047.05000000008, 66.71) [b] 
(1047.00000000008, 68.32) [b] 
(1046.95000000008, 70.16) [b] 
(1046.8500000000802, 71.5) [b] 
(1046.6500000000801, 72.63) [b] 
(1046.6000000000802, 73.77) [b] 
(1046.5500000000802, 74.5) [b] 
(1046.5000000000803, 76.99) [b] 
(1046.3500000000802, 78.57) [b] 
(1046.3000000000802, 84.11) [b] 
(1046.2000000000803, 84.4) [b] 
(1046.1500000000804, 89.62) [b] 
(1046.1000000000804, 89.96) [b] 
(1046.0500000000804, 94.37) [b] 
(1045.8500000000804, 95.34) [b] 
(1045.8000000000804, 103) [b] 
(1045.7500000000805, 109.3) [b] 
(1045.6000000000804, 114.1) [b] 
(1045.5500000000804, 116.7) [b] 
(1045.4500000000805, 124.4) [b] 
(1045.3500000000806, 130.4) [b] 
(1045.3000000000807, 132.9) [b] 
(1045.2500000000807, 133.6) [b] 
(1045.2000000000808, 135.7) [b] 
(1045.1000000000809, 136) [b] 
(1045.050000000081, 136.8) [b] 
(1044.9000000000808, 137.3) [b] 
(1044.8500000000809, 146.6) [b] 
(1044.800000000081, 148.1) [b] 
(1044.750000000081, 151) [b] 
(1044.700000000081, 151.7) [b] 
(1044.650000000081, 151.8) [b] 
(1044.600000000081, 154.1) [b] 
(1044.5000000000812, 171.9) [b] 
(1044.4500000000812, 175.5) [b] 
(1044.4000000000813, 176.2) [b] 
(1044.3500000000813, 208.4) [b] 
(1044.3000000000814, 214.3) [b] 
(1044.2500000000814, 216.6) [b] 
(1044.2000000000814, 237.9) [b] 
(1044.1500000000815, 243) [b] 
(1044.1000000000815, 243.7) [b] 
(1044.0500000000816, 250.7) [b] 
(1044.0000000000816, 255.8) [b] 
(1043.9500000000817, 263.4) [b] 
(1043.8500000000818, 270.2) [b] 
(1043.8000000000818, 270.8) [b] 
(1043.700000000082, 289) [b] 
(1043.5000000000819, 305.1) [b] 
(1043.3500000000818, 306.4) [b] 
(1043.3000000000818, 308.7) [b] 
(1043.2500000000819, 322) [b] 
(1043.150000000082, 324.4) [b] 
(1043.100000000082, 332.7) [b] 
(1043.050000000082, 334.4) [b] 
(1043.000000000082, 339) [b] 
(1042.9500000000821, 343) [b] 
(1042.9000000000822, 343.3) [b] 
(1042.8500000000822, 347.6) [b] 
(1042.7500000000823, 347.7) [b] 
(1042.6500000000824, 363.8) [b] 
(1042.6000000000824, 366.6) [b] 
(1042.5500000000825, 388.7) [b] 
(1042.4500000000826, 404.1) [b] 
(1042.4000000000826, 425.1) [b] 
(1042.3500000000827, 426) [b] 
(1042.3000000000827, 440) [b] 
(1042.1500000000829, 443.5) [b] 
(1042.0000000000828, 497.5) [b] 
(1041.9500000000828, 503.4) [b] 
(1041.9000000000829, 511.2) [b] 
(1041.850000000083, 544.1) [b] 
(1041.800000000083, 547.9) [b] 
(1041.700000000083, 551.3) [b] 
(1041.650000000083, 556) [b] 
(1041.6000000000831, 633.8) [b] 
(1041.5000000000832, 637.9) [b] 
(1041.4500000000833, 639.2) [b] 
(1041.4000000000833, 654.6) [b] 
(1041.3500000000834, 676.8) [b] 
(1041.2500000000834, 694.8) [b] 
(1041.1500000000835, 702.5) [b] 
(1041.1000000000836, 741.2) [b] 
(1041.0500000000836, 747.4) [b] 
(1041.0000000000837, 782.4) [b] 
(1040.9500000000837, 787.2) [b] 
(1040.8500000000838, 787.4) [b] 
(1040.750000000084, 793.9) [b] 
(1040.700000000084, 800.5) [b] 
(1040.600000000084, 804) [b] 
(1040.5000000000841, 804.3) [b] 
(1040.4500000000842, 816.6) [b] 
(1040.3500000000843, 843.4) [b] 
(1040.3000000000843, 856.9) [b] 
(1040.2500000000844, 861.5) [b] 
(1040.2000000000844, 862.9) [b] 
(1040.1500000000844, 875.5) [b] 
(1040.1000000000845, 880.9) [b] 
(1040.0000000000846, 888.7) [b] 
(1039.9500000000846, 894.6) [b] 
(1039.8500000000847, 910.3) [b] 
(1039.8000000000848, 939.4) [b] 
(1039.7500000000848, 965.9) [b] 
(1039.7000000000849, 988.8) [b] 
(1039.650000000085, 993.3) [b] 
(1039.600000000085, 1007) [b] 
(1039.550000000085, 1153) [b] 
(1039.500000000085, 1188) [b] 
(1039.450000000085, 1194) [b] 
(1039.300000000085, 1291) [b] 
(1039.150000000085, 1407) [b] 
(1039.050000000085, 1491) [b] 
(1039.000000000085, 1594) [b] 
}}}{legend pos=north east}
\end{center}
\vfill
\end{frame}


\begin{frame}[fragile]
\frametitle{Job Shop}
\vfill
\begin{itemize}
	\item Larger (500 tasks) but very easy instance\uncover<2->{: 800000 branches but only \memph{926} fails (!)}

	\vfill\uncover<3->{
	\begin{itemize}
	\item CP Optimizer has an extremely good primal heuristic (LNS?) 
\vfill

\item And it scales much better w.r.t the number of tasks 
	\end{itemize}
	}
\end{itemize}
\vfill
\begin{center}
\cactus{Objective}{CPU time}{CPOptimizer, Tempo}{{{
(2937.7000000000003, 0.1) [a] 
(2928.3, 0.11) [a] 
(2927.8, 0.12) [a] 
(2925.8500000000004, 0.14) [a] 
(2925.3500000000004, 0.15) [a] 
(2924.8500000000004, 0.18) [a] 
(2924.0000000000005, 0.21) [a] 
},{
(3991.9, 0.322133) [b] 
(3975.6, 0.380673) [b] 
(3958.35, 0.3969) [b] 
(3948.0499999999997, 0.398987) [b] 
(3926.1, 0.409036) [b] 
(3909.7999999999997, 0.409405) [b] 
(3906.85, 0.4118) [b] 
(3884.15, 0.415629) [b] 
(3879.15, 0.416432) [b] 
(3872.5, 0.4406) [b] 
(3863.3, 0.441549) [b] 
(3857.05, 0.446874) [b] 
(3833.8, 0.455406) [b] 
(3823.5, 0.503453) [b] 
(3807.75, 0.5113) [b] 
(3796.95, 0.5175) [b] 
(3769.95, 0.548063) [b] 
(3762.7999999999997, 0.5516) [b] 
(3750.95, 0.578902) [b] 
(3743.2999999999997, 0.686476) [b] 
(3743.2499999999995, 0.7808) [b] 
(3742.8499999999995, 0.797684) [b] 
(3730.7499999999995, 0.829762) [b] 
(3720.3999999999996, 0.839394) [b] 
(3718.7499999999995, 0.849) [b] 
(3711.6999999999994, 0.8648) [b] 
(3710.6999999999994, 0.9095) [b] 
(3710.5499999999993, 0.9416) [b] 
(3699.0499999999993, 0.968647) [b] 
(3695.5499999999993, 0.970612) [b] 
(3689.0499999999993, 0.9814) [b] 
(3688.5499999999993, 0.99164) [b] 
(3688.3499999999995, 1.05213) [b] 
(3686.3999999999996, 1.06673) [b] 
(3686.3499999999995, 1.093) [b] 
(3686.1499999999996, 1.101) [b] 
(3685.2499999999995, 1.144) [b] 
(3685.1499999999996, 1.20378) [b] 
(3685.0499999999997, 1.21514) [b] 
(3684.85, 1.24408) [b] 
(3684.1, 1.266) [b] 
(3684.0499999999997, 1.271) [b] 
(3683.1, 1.2937) [b] 
(3680.7999999999997, 1.302) [b] 
(3676.85, 1.366) [b] 
(3673.6, 1.383) [b] 
(3673.5, 1.409) [b] 
(3673.3, 1.442) [b] 
(3672.65, 1.47823) [b] 
(3662.9, 1.479) [b] 
(3662.8, 1.502) [b] 
(3662.05, 1.508) [b] 
(3646.6000000000004, 1.542) [b] 
(3644.05, 1.548) [b] 
(3644.0, 1.68) [b] 
(3643.95, 1.686) [b] 
(3640.7, 1.68978) [b] 
(3640.6, 1.711) [b] 
(3637.4, 1.73) [b] 
(3637.15, 1.82) [b] 
(3637.1, 1.843) [b] 
(3636.7999999999997, 1.844) [b] 
(3625.95, 1.84494) [b] 
(3625.35, 1.86) [b] 
(3624.9, 1.86914) [b] 
(3624.7000000000003, 1.894) [b] 
(3624.6000000000004, 1.905) [b] 
(3624.55, 1.908) [b] 
(3623.2000000000003, 1.984) [b] 
(3617.8500000000004, 1.997) [b] 
(3617.6000000000004, 2.017) [b] 
(3617.05, 2.048) [b] 
(3616.9, 2.09683) [b] 
(3616.75, 2.103) [b] 
(3616.7, 2.137) [b] 
(3616.6499999999996, 2.143) [b] 
(3616.5499999999997, 2.211) [b] 
(3615.95, 2.219) [b] 
(3610.0499999999997, 2.27) [b] 
(3609.1499999999996, 2.306) [b] 
(3602.9999999999995, 2.335) [b] 
(3602.2499999999995, 2.35) [b] 
(3601.9999999999995, 2.366) [b] 
(3599.4499999999994, 2.481) [b] 
(3594.3499999999995, 2.485) [b] 
(3594.0999999999995, 2.489) [b] 
(3593.9999999999995, 2.523) [b] 
(3591.6999999999994, 2.53985) [b] 
(3591.649999999999, 2.55451) [b] 
(3591.599999999999, 2.586) [b] 
(3591.549999999999, 2.587) [b] 
(3590.4999999999986, 2.6) [b] 
(3590.3999999999987, 2.632) [b] 
(3589.9999999999986, 2.686) [b] 
(3587.9499999999985, 2.689) [b] 
(3587.5999999999985, 2.737) [b] 
(3580.4499999999985, 2.75) [b] 
(3580.0499999999984, 2.795) [b] 
(3579.9499999999985, 2.82) [b] 
(3579.4999999999986, 2.84) [b] 
(3579.3499999999985, 2.842) [b] 
(3570.0999999999985, 2.875) [b] 
(3569.9499999999985, 2.89) [b] 
(3569.5999999999985, 2.962) [b] 
(3567.2499999999986, 2.98) [b] 
(3567.1499999999987, 2.986) [b] 
(3567.0999999999985, 3.012) [b] 
(3566.7499999999986, 3.014) [b] 
(3566.6999999999985, 3.052) [b] 
(3566.6499999999983, 3.083) [b] 
(3566.4499999999985, 3.134) [b] 
(3566.3999999999983, 3.18) [b] 
(3566.1499999999983, 3.214) [b] 
(3564.999999999998, 3.22894) [b] 
(3564.8999999999983, 3.329) [b] 
(3558.0499999999984, 3.334) [b] 
(3557.6999999999985, 3.339) [b] 
(3557.3499999999985, 3.354) [b] 
(3557.2999999999984, 3.371) [b] 
(3556.5499999999984, 3.4) [b] 
(3555.7999999999984, 3.464) [b] 
(3554.9499999999985, 3.466) [b] 
(3554.1999999999985, 3.503) [b] 
(3554.0999999999985, 3.53) [b] 
(3553.9999999999986, 3.543) [b] 
(3553.9499999999985, 3.545) [b] 
(3550.4499999999985, 3.573) [b] 
(3549.5499999999984, 3.575) [b] 
(3549.499999999998, 3.604) [b] 
(3549.3999999999983, 3.68) [b] 
(3549.1999999999985, 3.693) [b] 
(3548.9999999999986, 3.713) [b] 
(3548.8499999999985, 3.886) [b] 
(3548.7999999999984, 3.89) [b] 
(3548.5999999999985, 3.895) [b] 
(3548.4499999999985, 3.9) [b] 
(3548.2499999999986, 3.907) [b] 
(3547.9999999999986, 3.92) [b] 
(3547.8999999999987, 3.944) [b] 
(3547.799999999999, 3.949) [b] 
(3547.599999999999, 4.004) [b] 
(3547.549999999999, 4.011) [b] 
(3547.1499999999987, 4.028) [b] 
(3546.9999999999986, 4.043) [b] 
(3546.4999999999986, 4.102) [b] 
(3546.2499999999986, 4.122) [b] 
(3545.9999999999986, 4.141) [b] 
(3543.9999999999986, 4.211) [b] 
(3543.5999999999985, 4.248) [b] 
(3543.5499999999984, 4.267) [b] 
(3543.4499999999985, 4.298) [b] 
(3543.2499999999986, 4.351) [b] 
(3543.1499999999987, 4.375) [b] 
(3543.0999999999985, 4.379) [b] 
(3542.9499999999985, 4.429) [b] 
(3542.8999999999983, 4.436) [b] 
(3542.749999999998, 4.491) [b] 
(3542.699999999998, 4.494) [b] 
(3542.599999999998, 4.507) [b] 
(3542.549999999998, 4.566) [b] 
(3542.4999999999977, 4.611) [b] 
(3542.4499999999975, 4.665) [b] 
(3542.3999999999974, 4.669) [b] 
(3542.099999999997, 4.688) [b] 
(3541.9999999999973, 4.735) [b] 
(3541.949999999997, 4.758) [b] 
(3541.649999999997, 4.76) [b] 
(3541.449999999997, 4.79) [b] 
(3541.2499999999973, 4.791) [b] 
(3539.449999999997, 4.846) [b] 
(3539.349999999997, 4.847) [b] 
(3539.199999999997, 4.853) [b] 
(3538.9999999999973, 4.885) [b] 
(3538.599999999997, 4.886) [b] 
(3538.4999999999973, 4.935) [b] 
(3538.449999999997, 4.957) [b] 
(3538.399999999997, 4.989) [b] 
(3538.199999999997, 5.008) [b] 
(3538.049999999997, 5.063) [b] 
(3537.949999999997, 5.074) [b] 
(3537.7499999999973, 5.181) [b] 
(3537.599999999997, 5.228) [b] 
(3537.549999999997, 5.28) [b] 
(3537.499999999997, 5.301) [b] 
(3537.4499999999966, 5.308) [b] 
(3537.1999999999966, 5.325) [b] 
(3537.1499999999965, 5.331) [b] 
(3536.7999999999965, 5.34) [b] 
(3536.6999999999966, 5.389) [b] 
(3536.5499999999965, 5.408) [b] 
(3536.1499999999965, 5.41) [b] 
(3535.9999999999964, 5.431) [b] 
(3535.949999999996, 5.437) [b] 
(3535.899999999996, 5.447) [b] 
(3535.849999999996, 5.496) [b] 
(3535.649999999996, 5.503) [b] 
(3535.549999999996, 5.549) [b] 
(3534.999999999996, 5.554) [b] 
(3534.9499999999957, 5.649) [b] 
(3534.849999999996, 5.7) [b] 
(3534.5499999999956, 5.703) [b] 
(3534.4499999999957, 5.762) [b] 
(3534.2999999999956, 5.811) [b] 
(3534.1499999999955, 5.816) [b] 
(3533.4499999999957, 5.838) [b] 
(3533.3999999999955, 5.865) [b] 
(3532.9999999999955, 5.909) [b] 
(3532.8999999999955, 5.964) [b] 
(3532.6999999999957, 6.018) [b] 
(3532.5499999999956, 6.05) [b] 
(3532.2999999999956, 6.054) [b] 
(3532.1999999999957, 6.072) [b] 
(3531.999999999996, 6.079) [b] 
(3531.799999999996, 6.08) [b] 
(3531.749999999996, 6.092) [b] 
(3531.049999999996, 6.094) [b] 
(3530.649999999996, 6.149) [b] 
(3530.499999999996, 6.172) [b] 
(3529.899999999996, 6.19) [b] 
(3529.849999999996, 6.249) [b] 
(3529.7999999999956, 6.302) [b] 
(3525.4999999999955, 6.412) [b] 
(3525.4499999999953, 6.413) [b] 
(3525.1999999999953, 6.432) [b] 
(3525.149999999995, 6.433) [b] 
(3525.099999999995, 6.464) [b] 
(3525.0499999999947, 6.505) [b] 
(3524.8999999999946, 6.569) [b] 
(3524.7499999999945, 6.575) [b] 
(3524.6499999999946, 6.582) [b] 
(3524.5999999999945, 6.595) [b] 
(3524.5499999999943, 6.627) [b] 
(3524.499999999994, 6.635) [b] 
(3524.399999999994, 6.66) [b] 
(3524.099999999994, 6.799) [b] 
(3523.899999999994, 6.822) [b] 
(3523.849999999994, 6.866) [b] 
(3523.799999999994, 6.879) [b] 
(3523.7499999999936, 6.925) [b] 
(3521.7499999999936, 6.967) [b] 
(3521.299999999994, 6.97) [b] 
(3514.349999999994, 6.986) [b] 
(3514.199999999994, 7.019) [b] 
(3513.849999999994, 7.05) [b] 
(3513.799999999994, 7.053) [b] 
(3513.7499999999936, 7.096) [b] 
(3513.6999999999935, 7.127) [b] 
(3513.5499999999934, 7.13) [b] 
(3513.3499999999935, 7.156) [b] 
(3513.1499999999937, 7.24) [b] 
(3512.949999999994, 7.258) [b] 
(3512.8999999999937, 7.267) [b] 
(3512.3999999999937, 7.273) [b] 
(3511.7499999999936, 7.29) [b] 
(3511.6499999999937, 7.339) [b] 
(3511.3999999999937, 7.357) [b] 
(3511.049999999994, 7.425) [b] 
(3510.9999999999936, 7.429) [b] 
(3510.8999999999937, 7.499) [b] 
(3510.699999999994, 7.518) [b] 
(3510.599999999994, 7.645) [b] 
(3509.949999999994, 7.647) [b] 
(3509.8999999999937, 7.649) [b] 
(3509.8499999999935, 7.663) [b] 
(3509.7999999999934, 7.681) [b] 
(3509.749999999993, 7.705) [b] 
(3509.699999999993, 7.835) [b] 
(3509.649999999993, 7.845) [b] 
(3509.399999999993, 7.849) [b] 
(3509.2499999999927, 7.85) [b] 
(3508.7499999999927, 7.858) [b] 
(3508.5999999999926, 7.86) [b] 
(3508.3499999999926, 7.87) [b] 
(3507.649999999993, 7.871) [b] 
(3507.549999999993, 7.876) [b] 
(3507.349999999993, 7.916) [b] 
(3507.099999999993, 8.031) [b] 
(3507.049999999993, 8.038) [b] 
(3506.949999999993, 8.085) [b] 
(3506.899999999993, 8.172) [b] 
(3506.7499999999927, 8.23) [b] 
(3506.3499999999926, 8.26) [b] 
(3506.2499999999927, 8.262) [b] 
(3505.9999999999927, 8.297) [b] 
(3505.9499999999925, 8.31) [b] 
(3505.8999999999924, 8.353) [b] 
(3505.7999999999925, 8.361) [b] 
(3505.6499999999924, 8.406) [b] 
(3505.4499999999925, 8.407) [b] 
(3505.2499999999927, 8.421) [b] 
(3505.049999999993, 8.49) [b] 
(3504.799999999993, 8.501) [b] 
(3504.549999999993, 8.572) [b] 
(3504.299999999993, 8.592) [b] 
(3503.799999999993, 8.615) [b] 
(3503.7499999999927, 8.65) [b] 
(3503.649999999993, 8.659) [b] 
(3503.5999999999926, 8.67) [b] 
(3503.4499999999925, 8.678) [b] 
(3503.2499999999927, 8.753) [b] 
(3503.1999999999925, 8.783) [b] 
(3503.1499999999924, 8.788) [b] 
(3502.9999999999923, 8.807) [b] 
(3502.6499999999924, 8.813) [b] 
(3502.1999999999925, 8.909) [b] 
(3502.1499999999924, 8.94) [b] 
(3501.8999999999924, 8.973) [b] 
(3501.849999999992, 8.99) [b] 
(3501.599999999992, 9.001) [b] 
(3501.3999999999924, 9.022) [b] 
(3501.1999999999925, 9.038) [b] 
(3501.1499999999924, 9.074) [b] 
(3501.099999999992, 9.118) [b] 
(3500.849999999992, 9.134) [b] 
(3500.7499999999923, 9.209) [b] 
(3500.699999999992, 9.219) [b] 
(3500.4999999999923, 9.307) [b] 
(3500.449999999992, 9.323) [b] 
(3500.299999999992, 9.345) [b] 
(3500.199999999992, 9.38) [b] 
(3500.149999999992, 9.405) [b] 
(3500.049999999992, 9.407) [b] 
(3499.849999999992, 9.416) [b] 
(3499.699999999992, 9.418) [b] 
(3499.599999999992, 9.427) [b] 
(3499.549999999992, 9.508) [b] 
(3499.399999999992, 9.519) [b] 
(3499.199999999992, 9.612) [b] 
(3498.849999999992, 9.638) [b] 
(3498.699999999992, 9.664) [b] 
(3498.649999999992, 9.696) [b] 
(3498.499999999992, 9.697) [b] 
(3498.399999999992, 9.719) [b] 
(3498.299999999992, 9.776) [b] 
(3498.049999999992, 9.782) [b] 
(3497.599999999992, 9.807) [b] 
(3497.549999999992, 9.863) [b] 
(3497.449999999992, 9.883) [b] 
(3497.2499999999923, 9.9) [b] 
(3497.199999999992, 10.01) [b] 
(3497.099999999992, 10.02) [b] 
(3496.849999999992, 10.07) [b] 
(3496.799999999992, 10.11) [b] 
(3496.499999999992, 10.13) [b] 
(3496.399999999992, 10.17) [b] 
(3496.3499999999917, 10.21) [b] 
(3496.2999999999915, 10.23) [b] 
(3496.2499999999914, 10.25) [b] 
(3496.199999999991, 10.26) [b] 
(3495.949999999991, 10.3) [b] 
(3495.5999999999913, 10.33) [b] 
(3495.4999999999914, 10.37) [b] 
(3494.9999999999914, 10.39) [b] 
(3494.949999999991, 10.4) [b] 
(3494.549999999991, 10.41) [b] 
(3494.249999999991, 10.43) [b] 
(3494.1499999999905, 10.44) [b] 
(3494.0999999999904, 10.6) [b] 
(3493.6999999999903, 10.66) [b] 
(3493.5999999999904, 10.7) [b] 
(3493.49999999999, 10.71) [b] 
(3493.44999999999, 10.77) [b] 
(3492.94999999999, 10.79) [b] 
(3492.8999999999896, 10.82) [b] 
(3492.7499999999895, 10.83) [b] 
(3492.6999999999894, 10.86) [b] 
(3492.4999999999895, 10.89) [b] 
(3492.4499999999894, 10.93) [b] 
(3492.3499999999894, 10.97) [b] 
(3491.899999999989, 10.99) [b] 
(3491.849999999989, 11.01) [b] 
(3491.699999999989, 11.02) [b] 
(3491.599999999989, 11.06) [b] 
(3491.349999999989, 11.1) [b] 
(3491.099999999989, 11.19) [b] 
(3490.749999999989, 11.23) [b] 
(3490.599999999989, 11.29) [b] 
(3490.549999999989, 11.32) [b] 
(3490.2499999999886, 11.35) [b] 
(3490.1999999999884, 11.41) [b] 
(3489.7999999999884, 11.46) [b] 
(3489.749999999988, 11.48) [b] 
(3489.5499999999884, 11.49) [b] 
(3489.3999999999883, 11.54) [b] 
(3489.1999999999884, 11.55) [b] 
(3489.1499999999883, 11.56) [b] 
(3488.3999999999883, 11.68) [b] 
(3488.2999999999884, 11.72) [b] 
(3488.1999999999884, 11.77) [b] 
(3488.0499999999884, 11.86) [b] 
(3487.999999999988, 11.89) [b] 
(3487.949999999988, 11.9) [b] 
(3486.549999999988, 11.92) [b] 
(3486.299999999988, 11.93) [b] 
(3486.2499999999877, 11.94) [b] 
(3486.1999999999875, 11.98) [b] 
(3485.9499999999875, 11.99) [b] 
(3485.8999999999874, 12) [b] 
(3485.6499999999874, 12.04) [b] 
(3485.599999999987, 12.06) [b] 
(3485.4999999999873, 12.09) [b] 
(3485.449999999987, 12.13) [b] 
(3485.299999999987, 12.23) [b] 
(3485.249999999987, 12.29) [b] 
(3484.899999999987, 12.31) [b] 
(3484.8499999999867, 12.39) [b] 
(3484.7999999999865, 12.44) [b] 
(3484.4499999999866, 12.45) [b] 
(3484.249999999987, 12.48) [b] 
(3484.0999999999867, 12.5) [b] 
(3483.6999999999866, 12.55) [b] 
(3483.4499999999866, 12.56) [b] 
(3483.2999999999865, 12.57) [b] 
(3483.2499999999864, 12.6) [b] 
(3482.699999999986, 12.67) [b] 
(3482.649999999986, 12.68) [b] 
(3482.3499999999863, 12.74) [b] 
(3482.299999999986, 12.82) [b] 
(3482.049999999986, 12.85) [b] 
(3481.999999999986, 12.92) [b] 
(3481.9499999999857, 13.02) [b] 
(3481.599999999986, 13.06) [b] 
(3481.4499999999857, 13.09) [b] 
(3481.249999999986, 13.13) [b] 
(3480.5499999999856, 13.15) [b] 
(3480.4999999999854, 13.16) [b] 
(3480.0999999999854, 13.26) [b] 
(3479.8499999999854, 13.29) [b] 
(3479.799999999985, 13.37) [b] 
(3479.5999999999854, 13.38) [b] 
(3479.1499999999855, 13.43) [b] 
(3478.9499999999857, 13.45) [b] 
(3478.099999999986, 13.48) [b] 
(3478.0499999999856, 13.5) [b] 
(3477.9999999999854, 13.53) [b] 
(3477.9499999999853, 13.55) [b] 
(3477.4999999999854, 13.61) [b] 
(3477.2999999999856, 13.62) [b] 
(3477.149999999985, 13.68) [b] 
(3476.999999999985, 13.7) [b] 
(3476.949999999985, 13.78) [b] 
(3476.8999999999846, 13.88) [b] 
(3476.7999999999847, 13.92) [b] 
(3476.449999999985, 13.97) [b] 
(3476.099999999985, 13.99) [b] 
(3475.9999999999845, 14) [b] 
(3475.9499999999844, 14.02) [b] 
(3475.849999999984, 14.09) [b] 
(3475.699999999984, 14.1) [b] 
(3475.599999999984, 14.15) [b] 
(3475.449999999984, 14.19) [b] 
(3474.999999999984, 14.2) [b] 
(3474.749999999984, 14.32) [b] 
(3474.249999999984, 14.41) [b] 
(3474.199999999984, 14.43) [b] 
(3474.049999999984, 14.44) [b] 
(3473.9999999999836, 14.49) [b] 
(3473.9499999999834, 14.5) [b] 
(3473.8499999999835, 14.58) [b] 
(3473.7499999999836, 14.59) [b] 
(3473.6999999999834, 14.6) [b] 
(3473.5999999999835, 14.61) [b] 
(3473.5499999999834, 14.63) [b] 
(3473.4499999999834, 14.67) [b] 
(3473.0499999999834, 14.68) [b] 
(3472.999999999983, 14.75) [b] 
(3472.949999999983, 14.76) [b] 
(3472.749999999983, 14.77) [b] 
(3472.599999999983, 14.8) [b] 
(3472.549999999983, 14.9) [b] 
(3472.4999999999827, 14.95) [b] 
(3472.4499999999825, 14.99) [b] 
(3472.2999999999824, 15.03) [b] 
(3471.6499999999824, 15.04) [b] 
(3471.599999999982, 15.09) [b] 
(3471.549999999982, 15.1) [b] 
(3471.499999999982, 15.15) [b] 
(3471.299999999982, 15.16) [b] 
(3471.199999999982, 15.17) [b] 
(3470.899999999982, 15.23) [b] 
(3470.8499999999817, 15.28) [b] 
(3470.7999999999815, 15.32) [b] 
(3470.4999999999814, 15.34) [b] 
(3470.3999999999814, 15.37) [b] 
(3470.3499999999813, 15.38) [b] 
(3469.9999999999814, 15.39) [b] 
(3469.1999999999816, 15.42) [b] 
(3469.0499999999815, 15.54) [b] 
(3468.8999999999814, 15.59) [b] 
(3468.8499999999813, 15.64) [b] 
(3468.799999999981, 15.68) [b] 
(3468.699999999981, 15.71) [b] 
(3468.649999999981, 15.75) [b] 
(3468.599999999981, 15.78) [b] 
(3468.3499999999804, 15.9) [b] 
(3468.2499999999804, 15.91) [b] 
(3468.1999999999803, 15.92) [b] 
(3468.14999999998, 15.95) [b] 
(3467.74999999998, 15.98) [b] 
(3467.5499999999797, 16.04) [b] 
(3467.2499999999795, 16.05) [b] 
(3467.0999999999794, 16.07) [b] 
(3466.9499999999794, 16.1) [b] 
(3466.849999999979, 16.16) [b] 
(3466.749999999979, 16.23) [b] 
(3466.499999999979, 16.38) [b] 
(3466.449999999979, 16.43) [b] 
(3466.3999999999787, 16.44) [b] 
(3466.3499999999785, 16.51) [b] 
(3466.2999999999784, 16.53) [b] 
(3465.999999999978, 16.54) [b] 
(3465.949999999978, 16.56) [b] 
(3465.799999999978, 16.62) [b] 
(3465.7499999999777, 16.64) [b] 
(3465.199999999978, 16.69) [b] 
(3464.899999999978, 16.7) [b] 
(3464.7499999999777, 16.71) [b] 
(3464.4999999999777, 16.77) [b] 
(3464.399999999978, 16.78) [b] 
(3464.3499999999776, 16.8) [b] 
(3464.0499999999774, 16.82) [b] 
(3463.8999999999774, 16.83) [b] 
(3463.849999999977, 16.98) [b] 
(3463.6499999999774, 17.02) [b] 
(3463.599999999977, 17.03) [b] 
(3463.4999999999773, 17.05) [b] 
(3463.3999999999774, 17.06) [b] 
(3463.2999999999774, 17.16) [b] 
(3463.2499999999773, 17.18) [b] 
(3463.1499999999774, 17.26) [b] 
(3462.9499999999775, 17.31) [b] 
(3462.7999999999774, 17.32) [b] 
(3462.6499999999774, 17.35) [b] 
(3462.5499999999774, 17.37) [b] 
(3462.3999999999774, 17.43) [b] 
(3461.8999999999774, 17.47) [b] 
(3461.7999999999774, 17.48) [b] 
(3461.7499999999773, 17.52) [b] 
(3461.549999999977, 17.54) [b] 
(3461.3499999999767, 17.55) [b] 
(3461.249999999977, 17.6) [b] 
(3461.149999999977, 17.63) [b] 
(3461.0999999999767, 17.66) [b] 
(3461.0499999999765, 17.74) [b] 
(3460.9999999999764, 17.76) [b] 
(3460.949999999976, 17.9) [b] 
(3460.8499999999763, 17.91) [b] 
(3460.799999999976, 17.94) [b] 
(3460.699999999976, 17.96) [b] 
(3460.649999999976, 17.98) [b] 
(3460.4499999999757, 17.99) [b] 
(3460.3999999999755, 18.03) [b] 
(3460.1999999999757, 18.05) [b] 
(3459.9499999999757, 18.07) [b] 
(3459.549999999976, 18.15) [b] 
(3459.399999999976, 18.19) [b] 
(3459.299999999976, 18.2) [b] 
(3459.0999999999763, 18.24) [b] 
(3458.9999999999764, 18.27) [b] 
(3458.8999999999764, 18.28) [b] 
(3458.8499999999763, 18.29) [b] 
(3458.799999999976, 18.34) [b] 
(3458.749999999976, 18.42) [b] 
(3458.6999999999757, 18.5) [b] 
(3458.6499999999755, 18.55) [b] 
(3458.5999999999754, 18.59) [b] 
(3458.299999999975, 18.62) [b] 
(3458.249999999975, 18.63) [b] 
(3458.149999999975, 18.65) [b] 
(3458.099999999975, 18.68) [b] 
(3457.899999999975, 18.71) [b] 
(3457.7999999999747, 18.75) [b] 
(3457.699999999975, 18.78) [b] 
(3457.5499999999747, 18.8) [b] 
(3457.449999999975, 18.82) [b] 
(3457.3999999999746, 18.9) [b] 
(3457.3499999999744, 18.92) [b] 
(3457.1499999999746, 18.94) [b] 
(3457.0999999999744, 18.97) [b] 
(3457.0499999999743, 18.99) [b] 
(3456.999999999974, 19.04) [b] 
(3456.899999999974, 19.09) [b] 
(3456.6999999999744, 19.1) [b] 
(3456.649999999974, 19.11) [b] 
(3456.599999999974, 19.17) [b] 
(3456.499999999974, 19.34) [b] 
(3456.399999999974, 19.36) [b] 
(3455.8499999999744, 19.38) [b] 
(3455.4999999999745, 19.41) [b] 
(3455.3499999999744, 19.45) [b] 
(3455.2499999999745, 19.46) [b] 
(3455.1999999999744, 19.48) [b] 
(3455.0499999999743, 19.55) [b] 
(3454.899999999974, 19.58) [b] 
(3454.6999999999744, 19.61) [b] 
(3454.649999999974, 19.64) [b] 
(3454.599999999974, 19.67) [b] 
(3454.549999999974, 19.71) [b] 
(3454.449999999974, 19.72) [b] 
(3454.3999999999737, 19.79) [b] 
(3454.3499999999735, 19.86) [b] 
(3454.2499999999736, 19.94) [b] 
(3454.1999999999734, 19.96) [b] 
(3454.0999999999735, 19.97) [b] 
(3453.9999999999736, 19.99) [b] 
(3453.549999999974, 20.01) [b] 
(3453.2499999999736, 20.02) [b] 
(3453.049999999974, 20.05) [b] 
(3452.949999999974, 20.06) [b] 
(3452.8999999999737, 20.09) [b] 
(3452.6499999999737, 20.13) [b] 
(3452.4499999999734, 20.15) [b] 
(3452.3999999999733, 20.16) [b] 
(3452.349999999973, 20.27) [b] 
(3452.299999999973, 20.33) [b] 
(3452.199999999973, 20.35) [b] 
(3452.049999999973, 20.38) [b] 
(3451.9999999999727, 20.43) [b] 
(3451.799999999973, 20.47) [b] 
(3451.699999999973, 20.55) [b] 
(3451.649999999973, 20.56) [b] 
(3451.5999999999726, 20.57) [b] 
(3451.3499999999726, 20.58) [b] 
(3451.1999999999725, 20.67) [b] 
(3451.1499999999724, 20.68) [b] 
(3451.099999999972, 20.7) [b] 
(3451.049999999972, 20.75) [b] 
(3449.899999999972, 20.76) [b] 
(3449.649999999972, 20.78) [b] 
(3449.3499999999717, 20.81) [b] 
(3449.0999999999717, 20.86) [b] 
(3449.0499999999715, 20.92) [b] 
(3448.9999999999714, 20.95) [b] 
(3448.949999999971, 21.03) [b] 
(3448.549999999971, 21.07) [b] 
(3448.499999999971, 21.11) [b] 
(3448.399999999971, 21.14) [b] 
(3448.349999999971, 21.17) [b] 
(3448.1999999999707, 21.18) [b] 
(3447.999999999971, 21.19) [b] 
(3447.9499999999707, 21.21) [b] 
(3447.8999999999705, 21.24) [b] 
(3447.79999999997, 21.26) [b] 
(3447.54999999997, 21.35) [b] 
(3447.24999999997, 21.39) [b] 
(3447.19999999997, 21.48) [b] 
(3447.09999999997, 21.49) [b] 
(3446.59999999997, 21.51) [b] 
(3446.49999999997, 21.57) [b] 
(3446.34999999997, 21.59) [b] 
(3446.2999999999697, 21.61) [b] 
(3446.2499999999695, 21.62) [b] 
(3446.0999999999694, 21.63) [b] 
(3445.9499999999694, 21.69) [b] 
(3445.899999999969, 21.71) [b] 
(3445.849999999969, 21.81) [b] 
(3445.699999999969, 21.82) [b] 
(3445.6499999999687, 21.84) [b] 
(3445.549999999969, 21.88) [b] 
(3445.4999999999686, 21.96) [b] 
(3445.3499999999685, 22) [b] 
(3445.1999999999684, 22.02) [b] 
(3445.0499999999683, 22.03) [b] 
(3444.9499999999684, 22.11) [b] 
(3444.8999999999683, 22.15) [b] 
(3444.749999999968, 22.16) [b] 
(3444.499999999968, 22.17) [b] 
(3444.449999999968, 22.18) [b] 
(3444.2499999999677, 22.2) [b] 
(3444.1999999999675, 22.26) [b] 
(3444.1499999999673, 22.35) [b] 
(3443.6999999999675, 22.36) [b] 
(3443.0999999999676, 22.37) [b] 
(3442.899999999968, 22.39) [b] 
(3442.8499999999676, 22.41) [b] 
(3442.7999999999674, 22.54) [b] 
(3442.6999999999675, 22.61) [b] 
(3442.6499999999673, 22.64) [b] 
(3442.349999999967, 22.66) [b] 
(3442.299999999967, 22.67) [b] 
(3442.199999999967, 22.7) [b] 
(3442.049999999967, 22.72) [b] 
(3441.999999999967, 22.74) [b] 
(3441.9499999999666, 22.77) [b] 
(3441.8499999999667, 22.88) [b] 
(3441.2999999999665, 23.01) [b] 
(3441.2499999999663, 23.04) [b] 
(3441.199999999966, 23.07) [b] 
(3441.099999999966, 23.08) [b] 
(3440.999999999966, 23.11) [b] 
(3440.9499999999657, 23.13) [b] 
(3440.7999999999656, 23.14) [b] 
(3440.6499999999655, 23.15) [b] 
(3440.5999999999653, 23.17) [b] 
(3440.549999999965, 23.18) [b] 
(3440.499999999965, 23.19) [b] 
(3440.299999999965, 23.22) [b] 
(3440.249999999965, 23.29) [b] 
(3440.099999999965, 23.32) [b] 
(3439.999999999965, 23.33) [b] 
(3439.799999999965, 23.39) [b] 
(3439.749999999965, 23.4) [b] 
(3439.699999999965, 23.42) [b] 
(3439.6499999999646, 23.54) [b] 
(3439.4999999999645, 23.59) [b] 
(3439.3999999999646, 23.66) [b] 
(3439.249999999964, 23.67) [b] 
(3439.149999999964, 23.71) [b] 
(3439.0499999999643, 23.73) [b] 
(3438.999999999964, 23.78) [b] 
(3438.949999999964, 23.79) [b] 
(3438.8999999999637, 23.83) [b] 
(3438.799999999964, 23.85) [b] 
(3438.7499999999636, 23.91) [b] 
(3438.6999999999634, 23.93) [b] 
(3438.5999999999635, 23.95) [b] 
(3438.5499999999633, 23.99) [b] 
(3438.499999999963, 24.01) [b] 
(3438.449999999963, 24.03) [b] 
(3438.399999999963, 24.08) [b] 
(3438.199999999963, 24.15) [b] 
(3438.049999999963, 24.17) [b] 
(3437.899999999963, 24.19) [b] 
(3437.8499999999626, 24.23) [b] 
(3437.7499999999627, 24.28) [b] 
(3437.649999999963, 24.29) [b] 
(3437.5999999999626, 24.33) [b] 
(3437.5499999999624, 24.34) [b] 
(3437.3999999999623, 24.35) [b] 
(3437.2999999999624, 24.39) [b] 
(3437.0999999999626, 24.4) [b] 
(3436.9999999999627, 24.47) [b] 
(3436.899999999963, 24.51) [b] 
(3436.799999999963, 24.55) [b] 
(3436.599999999963, 24.6) [b] 
(3436.549999999963, 24.65) [b] 
(3436.4499999999625, 24.68) [b] 
(3436.2999999999624, 24.69) [b] 
(3436.1999999999625, 24.72) [b] 
(3436.1499999999623, 24.76) [b] 
(3435.9999999999623, 24.79) [b] 
(3435.699999999962, 24.8) [b] 
(3435.549999999962, 24.84) [b] 
(3435.449999999962, 24.85) [b] 
(3435.349999999962, 24.86) [b] 
(3435.299999999962, 24.88) [b] 
(3435.249999999962, 24.9) [b] 
(3435.0999999999617, 24.97) [b] 
(3434.9499999999616, 25.02) [b] 
(3434.7999999999615, 25.05) [b] 
(3434.7499999999613, 25.08) [b] 
(3434.699999999961, 25.18) [b] 
(3434.4999999999613, 25.32) [b] 
(3434.449999999961, 25.34) [b] 
(3434.249999999961, 25.35) [b] 
(3434.149999999961, 25.36) [b] 
(3433.999999999961, 25.41) [b] 
(3433.9499999999607, 25.42) [b] 
(3433.7999999999606, 25.44) [b] 
(3433.6999999999607, 25.48) [b] 
(3433.6499999999605, 25.52) [b] 
(3433.4999999999604, 25.54) [b] 
(3433.29999999996, 25.56) [b] 
(3433.1999999999603, 25.6) [b] 
(3433.04999999996, 25.61) [b] 
(3432.99999999996, 25.62) [b] 
(3432.89999999996, 25.66) [b] 
(3432.84999999996, 25.75) [b] 
(3432.74999999996, 25.83) [b] 
(3432.64999999996, 25.84) [b] 
(3432.59999999996, 25.88) [b] 
(3432.49999999996, 25.91) [b] 
(3432.24999999996, 25.95) [b] 
(3432.0499999999597, 25.97) [b] 
(3431.9999999999595, 26.05) [b] 
(3431.8999999999596, 26.16) [b] 
(3431.5499999999593, 26.18) [b] 
(3431.2999999999593, 26.22) [b] 
(3431.249999999959, 26.28) [b] 
(3431.099999999959, 26.32) [b] 
(3430.999999999959, 26.34) [b] 
(3430.949999999959, 26.35) [b] 
(3430.549999999959, 26.36) [b] 
(3430.099999999959, 26.37) [b] 
(3430.049999999959, 26.38) [b] 
(3429.9999999999586, 26.44) [b] 
(3429.9499999999584, 26.45) [b] 
(3429.8499999999585, 26.56) [b] 
(3429.7499999999586, 26.61) [b] 
(3429.5999999999585, 26.63) [b] 
(3429.5499999999583, 26.65) [b] 
(3429.3999999999583, 26.67) [b] 
(3429.249999999958, 26.69) [b] 
(3428.949999999958, 26.78) [b] 
(3428.899999999958, 26.8) [b] 
(3427.7499999999577, 26.85) [b] 
(3427.4999999999577, 26.86) [b] 
(3427.399999999958, 26.88) [b] 
(3427.3499999999576, 27.02) [b] 
(3426.9999999999573, 27.05) [b] 
(3426.799999999957, 27.06) [b] 
(3426.6999999999566, 27.09) [b] 
(3426.5499999999565, 27.12) [b] 
(3426.3999999999564, 27.2) [b] 
(3426.3499999999563, 27.21) [b] 
(3425.799999999956, 27.23) [b] 
(3425.749999999956, 27.26) [b] 
(3425.6999999999557, 27.31) [b] 
(3425.6499999999555, 27.33) [b] 
(3425.3999999999555, 27.37) [b] 
(3425.3499999999553, 27.41) [b] 
(3424.8499999999553, 27.42) [b] 
(3424.6499999999555, 27.45) [b] 
(3424.5499999999556, 27.53) [b] 
(3424.4999999999554, 27.58) [b] 
(3424.3999999999555, 27.62) [b] 
(3424.3499999999553, 27.64) [b] 
(3424.299999999955, 27.8) [b] 
(3424.0999999999553, 27.81) [b] 
(3424.049999999955, 27.83) [b] 
(3423.899999999955, 27.84) [b] 
(3423.649999999955, 27.86) [b] 
(3423.599999999955, 27.87) [b] 
(3423.399999999955, 27.93) [b] 
(3422.749999999955, 27.94) [b] 
(3422.599999999955, 27.95) [b] 
(3422.5499999999547, 27.98) [b] 
(3422.449999999955, 28.02) [b] 
(3422.3999999999546, 28.03) [b] 
(3421.949999999955, 28.07) [b] 
(3421.849999999955, 28.1) [b] 
(3421.749999999955, 28.11) [b] 
(3421.499999999955, 28.13) [b] 
(3421.449999999955, 28.14) [b] 
(3421.349999999955, 28.2) [b] 
(3421.2999999999547, 28.23) [b] 
(3421.2499999999545, 28.26) [b] 
(3421.1999999999543, 28.35) [b] 
(3421.149999999954, 28.39) [b] 
(3420.999999999954, 28.53) [b] 
(3420.899999999954, 28.54) [b] 
(3420.799999999954, 28.6) [b] 
(3420.699999999954, 28.62) [b] 
(3420.499999999954, 28.63) [b] 
(3420.1499999999537, 28.67) [b] 
(3419.699999999954, 28.73) [b] 
(3419.6499999999537, 28.74) [b] 
(3419.2499999999536, 28.76) [b] 
(3419.0499999999533, 28.8) [b] 
(3418.8999999999533, 28.81) [b] 
(3418.749999999953, 28.87) [b] 
(3418.6499999999533, 28.91) [b] 
(3418.599999999953, 28.95) [b] 
(3418.349999999953, 29.08) [b] 
(3418.299999999953, 29.15) [b] 
(3418.2499999999527, 29.16) [b] 
(3418.0999999999526, 29.18) [b] 
(3418.0499999999524, 29.19) [b] 
(3417.8999999999523, 29.21) [b] 
(3417.849999999952, 29.26) [b] 
(3417.7499999999523, 29.31) [b] 
(3417.699999999952, 29.32) [b] 
(3417.649999999952, 29.33) [b] 
(3417.5999999999517, 29.34) [b] 
(3417.5499999999515, 29.36) [b] 
(3417.0499999999515, 29.43) [b] 
(3416.9999999999513, 29.45) [b] 
(3416.8999999999514, 29.46) [b] 
(3416.8499999999513, 29.49) [b] 
(3416.7499999999513, 29.5) [b] 
(3416.699999999951, 29.55) [b] 
(3416.649999999951, 29.6) [b] 
(3416.449999999951, 29.63) [b] 
(3416.399999999951, 29.73) [b] 
(3416.349999999951, 29.74) [b] 
(3416.2999999999506, 29.75) [b] 
(3416.2499999999504, 29.79) [b] 
(3415.9999999999504, 29.8) [b] 
(3415.8499999999503, 29.81) [b] 
(3415.6499999999505, 29.82) [b] 
(3415.3499999999503, 29.88) [b] 
(3415.29999999995, 29.93) [b] 
(3415.14999999995, 29.96) [b] 
(3415.09999999995, 30.02) [b] 
(3415.0499999999497, 30.04) [b] 
(3414.84999999995, 30.08) [b] 
(3414.7499999999495, 30.12) [b] 
(3414.5999999999494, 30.17) [b] 
(3414.4499999999493, 30.29) [b] 
(3414.2999999999493, 30.33) [b] 
(3414.149999999949, 30.35) [b] 
(3413.899999999949, 30.38) [b] 
(3413.7499999999486, 30.39) [b] 
(3413.5999999999485, 30.47) [b] 
(3413.499999999948, 30.49) [b] 
(3413.449999999948, 30.5) [b] 
(3413.349999999948, 30.51) [b] 
(3413.199999999948, 30.52) [b] 
(3413.149999999948, 30.64) [b] 
(3412.8499999999476, 30.68) [b] 
(3412.6999999999475, 30.69) [b] 
(3412.5499999999474, 30.77) [b] 
(3412.4999999999472, 30.83) [b] 
(3412.349999999947, 30.85) [b] 
(3412.199999999947, 30.86) [b] 
(3412.149999999947, 30.87) [b] 
(3411.899999999947, 30.88) [b] 
(3411.8499999999467, 30.9) [b] 
(3411.7499999999463, 30.94) [b] 
(3411.6499999999464, 30.95) [b] 
(3411.5999999999462, 31) [b] 
(3411.449999999946, 31.08) [b] 
(3411.399999999946, 31.1) [b] 
(3411.299999999946, 31.14) [b] 
(3411.249999999946, 31.17) [b] 
(3411.1999999999457, 31.18) [b] 
(3411.1499999999455, 31.24) [b] 
(3411.0999999999453, 31.27) [b] 
(3411.049999999945, 31.29) [b] 
(3410.849999999945, 31.33) [b] 
(3410.7999999999447, 31.34) [b] 
(3410.6499999999446, 31.4) [b] 
(3410.4999999999445, 31.42) [b] 
(3410.4499999999443, 31.43) [b] 
(3410.2999999999442, 31.49) [b] 
(3410.199999999944, 31.51) [b] 
(3410.1499999999437, 31.53) [b] 
(3410.0999999999435, 31.61) [b] 
(3410.0499999999433, 31.62) [b] 
(3409.6999999999434, 31.67) [b] 
(3409.499999999943, 31.7) [b] 
(3409.449999999943, 31.71) [b] 
(3409.399999999943, 31.8) [b] 
(3409.2999999999424, 31.88) [b] 
(3409.1999999999425, 31.91) [b] 
(3409.1499999999423, 31.92) [b] 
(3408.8999999999423, 31.97) [b] 
(3408.7999999999424, 31.98) [b] 
(3408.6999999999425, 32) [b] 
(3408.6499999999423, 32.07) [b] 
(3408.599999999942, 32.08) [b] 
(3408.549999999942, 32.14) [b] 
(3407.0999999999417, 32.17) [b] 
(3407.0499999999415, 32.21) [b] 
(3406.7499999999413, 32.23) [b] 
(3406.699999999941, 32.25) [b] 
(3406.5999999999412, 32.29) [b] 
(3406.549999999941, 32.31) [b] 
(3406.449999999941, 32.43) [b] 
(3406.399999999941, 32.45) [b] 
(3406.249999999941, 32.48) [b] 
(3406.1999999999407, 32.5) [b] 
(3406.1499999999405, 32.51) [b] 
(3406.0499999999406, 32.53) [b] 
(3405.9999999999404, 32.54) [b] 
(3405.6999999999402, 32.57) [b] 
(3405.64999999994, 32.6) [b] 
(3405.59999999994, 32.61) [b] 
(3405.2999999999397, 32.73) [b] 
(3404.6499999999396, 32.86) [b] 
(3404.5499999999397, 32.87) [b] 
(3404.4999999999395, 32.88) [b] 
(3404.3999999999396, 32.89) [b] 
(3404.2499999999395, 32.92) [b] 
(3404.1499999999396, 32.93) [b] 
(3403.8999999999396, 33) [b] 
(3403.7999999999397, 33.02) [b] 
(3403.6499999999396, 33.03) [b] 
(3403.3999999999396, 33.05) [b] 
(3403.2499999999395, 33.06) [b] 
(3403.1499999999396, 33.13) [b] 
(3403.0499999999397, 33.16) [b] 
(3402.8999999999396, 33.28) [b] 
(3402.5499999999397, 33.36) [b] 
(3402.4999999999395, 33.37) [b] 
(3402.4499999999393, 33.4) [b] 
(3402.2999999999392, 33.49) [b] 
(3402.199999999939, 33.52) [b] 
(3401.8999999999387, 33.56) [b] 
(3401.799999999939, 33.58) [b] 
(3401.699999999939, 33.59) [b] 
(3401.6499999999387, 33.6) [b] 
(3401.5499999999383, 33.63) [b] 
(3401.4499999999384, 33.65) [b] 
(3401.3999999999382, 33.69) [b] 
(3401.349999999938, 33.79) [b] 
(3401.249999999938, 33.84) [b] 
(3400.949999999938, 33.87) [b] 
(3400.899999999938, 33.88) [b] 
(3400.8499999999376, 33.92) [b] 
(3400.7999999999374, 33.96) [b] 
(3400.4999999999377, 34.02) [b] 
(3400.399999999938, 34.04) [b] 
(3400.199999999938, 34.05) [b] 
(3400.149999999938, 34.06) [b] 
(3400.0999999999376, 34.08) [b] 
(3399.9999999999377, 34.09) [b] 
(3399.9499999999375, 34.15) [b] 
(3399.8999999999373, 34.17) [b] 
(3399.7999999999374, 34.18) [b] 
(3399.6999999999375, 34.21) [b] 
(3399.6499999999373, 34.23) [b] 
(3399.599999999937, 34.29) [b] 
(3399.549999999937, 34.36) [b] 
(3399.449999999937, 34.38) [b] 
(3399.399999999937, 34.39) [b] 
(3399.3499999999367, 34.4) [b] 
(3399.2999999999365, 34.52) [b] 
(3399.2499999999363, 34.55) [b] 
(3399.199999999936, 34.59) [b] 
(3399.049999999936, 34.6) [b] 
(3398.899999999936, 34.62) [b] 
(3398.799999999936, 34.63) [b] 
(3398.699999999936, 34.67) [b] 
(3398.549999999936, 34.72) [b] 
(3398.449999999936, 34.76) [b] 
(3398.299999999936, 34.77) [b] 
(3398.199999999936, 34.78) [b] 
(3398.149999999936, 34.79) [b] 
(3398.099999999936, 34.86) [b] 
(3398.0499999999356, 34.87) [b] 
(3397.9999999999354, 34.91) [b] 
(3397.8999999999355, 34.97) [b] 
(3397.7499999999354, 35.01) [b] 
(3397.5999999999353, 35.02) [b] 
(3397.4499999999352, 35.07) [b] 
(3397.0999999999353, 35.09) [b] 
(3397.049999999935, 35.1) [b] 
(3396.999999999935, 35.11) [b] 
(3396.849999999935, 35.17) [b] 
(3396.7999999999347, 35.19) [b] 
(3396.7499999999345, 35.33) [b] 
(3396.6499999999346, 35.37) [b] 
(3396.5999999999344, 35.39) [b] 
(3396.5499999999342, 35.4) [b] 
(3396.499999999934, 35.46) [b] 
(3396.449999999934, 35.49) [b] 
(3396.349999999934, 35.54) [b] 
(3396.099999999934, 35.58) [b] 
(3395.8999999999337, 35.59) [b] 
(3395.8499999999335, 35.63) [b] 
(3395.7999999999333, 35.68) [b] 
(3395.6499999999332, 35.7) [b] 
(3395.2999999999333, 35.71) [b] 
(3395.0999999999335, 35.72) [b] 
(3394.9499999999334, 35.74) [b] 
(3394.8999999999332, 35.78) [b] 
(3394.849999999933, 35.81) [b] 
(3394.799999999933, 35.85) [b] 
(3394.649999999933, 35.88) [b] 
(3394.4999999999327, 35.91) [b] 
(3394.4499999999325, 35.97) [b] 
(3394.1499999999323, 36.03) [b] 
(3393.9999999999322, 36.04) [b] 
(3393.7499999999322, 36.09) [b] 
(3393.699999999932, 36.18) [b] 
(3393.649999999932, 36.21) [b] 
(3393.5999999999317, 36.23) [b] 
(3393.0499999999315, 36.25) [b] 
(3392.9499999999316, 36.31) [b] 
(3392.8499999999317, 36.32) [b] 
(3392.7999999999315, 36.37) [b] 
(3392.7499999999313, 36.39) [b] 
(3392.4999999999313, 36.4) [b] 
(3392.2999999999315, 36.41) [b] 
(3392.1999999999316, 36.45) [b] 
(3392.1499999999314, 36.48) [b] 
(3391.9999999999313, 36.51) [b] 
(3391.8499999999312, 36.52) [b] 
(3391.799999999931, 36.55) [b] 
(3391.6999999999307, 36.58) [b] 
(3391.6499999999305, 36.64) [b] 
(3391.5499999999306, 36.65) [b] 
(3391.4999999999304, 36.72) [b] 
(3391.3999999999305, 36.74) [b] 
(3391.3499999999303, 36.77) [b] 
(3391.2499999999304, 36.8) [b] 
(3390.8499999999303, 36.92) [b] 
(3390.6999999999302, 36.93) [b] 
(3390.54999999993, 36.97) [b] 
(3390.49999999993, 37.03) [b] 
(3390.1499999999296, 37.05) [b] 
(3390.0999999999294, 37.09) [b] 
(3390.0499999999292, 37.1) [b] 
(3389.999999999929, 37.13) [b] 
(3389.849999999929, 37.16) [b] 
(3389.699999999929, 37.17) [b] 
(3389.6499999999287, 37.23) [b] 
(3389.549999999929, 37.25) [b] 
(3389.4999999999286, 37.27) [b] 
(3389.4499999999284, 37.28) [b] 
(3389.3999999999282, 37.3) [b] 
(3389.2999999999283, 37.31) [b] 
(3389.0499999999283, 37.42) [b] 
(3388.9499999999284, 37.45) [b] 
(3388.8999999999282, 37.53) [b] 
(3388.749999999928, 37.56) [b] 
(3388.599999999928, 37.59) [b] 
(3388.499999999928, 37.63) [b] 
(3388.449999999928, 37.67) [b] 
(3388.399999999928, 37.7) [b] 
(3388.099999999928, 37.79) [b] 
(3388.049999999928, 37.84) [b] 
(3387.799999999928, 37.85) [b] 
(3387.7499999999277, 37.87) [b] 
(3387.6999999999275, 37.92) [b] 
(3387.5499999999274, 37.93) [b] 
(3387.4999999999272, 37.94) [b] 
(3387.2499999999272, 37.96) [b] 
(3387.099999999927, 37.98) [b] 
(3387.049999999927, 38.02) [b] 
(3386.849999999927, 38.04) [b] 
(3386.799999999927, 38.05) [b] 
(3386.749999999927, 38.06) [b] 
(3386.6999999999266, 38.07) [b] 
(3386.5999999999262, 38.13) [b] 
(3386.549999999926, 38.16) [b] 
(3386.499999999926, 38.29) [b] 
(3386.4499999999257, 38.31) [b] 
(3386.2999999999256, 38.35) [b] 
(3386.2499999999254, 38.37) [b] 
(3386.0999999999253, 38.4) [b] 
(3386.049999999925, 38.45) [b] 
(3385.999999999925, 38.46) [b] 
(3385.949999999925, 38.52) [b] 
(3385.699999999925, 38.56) [b] 
(3385.5499999999247, 38.61) [b] 
(3385.3999999999246, 38.62) [b] 
(3385.199999999925, 38.64) [b] 
(3385.1499999999246, 38.66) [b] 
(3384.9999999999245, 38.7) [b] 
(3384.9499999999243, 38.74) [b] 
(3384.8499999999244, 38.77) [b] 
(3384.7999999999242, 38.8) [b] 
(3384.749999999924, 38.85) [b] 
(3384.699999999924, 38.91) [b] 
(3384.6499999999237, 38.93) [b] 
(3384.449999999924, 38.94) [b] 
(3384.3999999999237, 39.01) [b] 
(3384.3499999999235, 39.02) [b] 
(3384.2999999999233, 39.06) [b] 
(3384.0999999999235, 39.09) [b] 
(3383.7999999999233, 39.14) [b] 
(3383.5499999999233, 39.18) [b] 
(3383.0499999999233, 39.19) [b] 
(3382.9499999999234, 39.21) [b] 
(3382.8999999999232, 39.22) [b] 
(3382.849999999923, 39.23) [b] 
(3382.6499999999232, 39.29) [b] 
(3382.499999999923, 39.32) [b] 
(3382.3999999999232, 39.39) [b] 
(3382.349999999923, 39.41) [b] 
(3382.199999999923, 39.46) [b] 
(3382.049999999923, 39.55) [b] 
(3381.8499999999226, 39.57) [b] 
(3381.6999999999225, 39.58) [b] 
(3381.2999999999224, 39.74) [b] 
(3381.0999999999226, 39.76) [b] 
(3380.899999999923, 39.78) [b] 
(3380.8499999999226, 39.81) [b] 
(3380.7499999999227, 39.85) [b] 
(3380.649999999923, 39.86) [b] 
(3380.4999999999227, 39.88) [b] 
(3380.4499999999225, 39.89) [b] 
(3380.2499999999222, 39.9) [b] 
(3380.199999999922, 39.99) [b] 
(3380.149999999922, 40.02) [b] 
(3380.049999999922, 40.08) [b] 
(3379.899999999922, 40.1) [b] 
(3379.799999999922, 40.15) [b] 
(3379.699999999922, 40.2) [b] 
(3379.649999999922, 40.21) [b] 
(3379.5999999999217, 40.26) [b] 
(3379.399999999922, 40.28) [b] 
(3379.3499999999217, 40.31) [b] 
(3379.2999999999215, 40.33) [b] 
(3379.1999999999216, 40.41) [b] 
(3379.1499999999214, 40.44) [b] 
(3379.0499999999215, 40.46) [b] 
(3378.949999999921, 40.53) [b] 
(3378.899999999921, 40.55) [b] 
(3378.799999999921, 40.57) [b] 
(3378.749999999921, 40.58) [b] 
(3378.649999999921, 40.59) [b] 
(3378.099999999921, 40.65) [b] 
(3378.0499999999206, 40.67) [b] 
(3377.9999999999204, 40.73) [b] 
(3377.6999999999207, 40.75) [b] 
(3377.5499999999206, 40.9) [b] 
(3377.4999999999204, 40.92) [b] 
(3377.29999999992, 40.93) [b] 
(3377.19999999992, 40.94) [b] 
(3377.1499999999196, 40.97) [b] 
(3377.0999999999194, 41.1) [b] 
(3377.0499999999192, 41.12) [b] 
(3376.9499999999193, 41.15) [b] 
(3376.899999999919, 41.24) [b] 
(3376.849999999919, 41.25) [b] 
(3376.799999999919, 41.27) [b] 
(3376.5999999999185, 41.28) [b] 
(3376.5499999999183, 41.29) [b] 
(3376.3999999999182, 41.31) [b] 
(3376.299999999918, 41.33) [b] 
(3376.149999999918, 41.36) [b] 
(3375.9999999999177, 41.42) [b] 
(3375.9499999999175, 41.44) [b] 
(3375.7999999999174, 41.46) [b] 
(3375.7499999999172, 41.52) [b] 
(3375.699999999917, 41.54) [b] 
(3375.549999999917, 41.58) [b] 
(3375.499999999917, 41.61) [b] 
(3375.299999999917, 41.62) [b] 
(3375.249999999917, 41.64) [b] 
(3375.1999999999166, 41.69) [b] 
(3375.0999999999167, 41.74) [b] 
(3374.999999999917, 41.76) [b] 
(3374.9499999999166, 41.83) [b] 
(3374.6499999999164, 41.87) [b] 
(3374.5999999999162, 41.88) [b] 
(3374.549999999916, 41.92) [b] 
(3374.399999999916, 41.99) [b] 
(3374.349999999916, 42.01) [b] 
(3374.1999999999157, 42.03) [b] 
(3374.1499999999155, 42.07) [b] 
(3374.0999999999153, 42.09) [b] 
(3374.049999999915, 42.11) [b] 
(3373.899999999915, 42.12) [b] 
(3373.849999999915, 42.14) [b] 
(3373.599999999915, 42.24) [b] 
(3373.499999999915, 42.28) [b] 
(3373.349999999915, 42.29) [b] 
(3373.2999999999147, 42.34) [b] 
(3373.1499999999146, 42.35) [b] 
(3373.0499999999147, 42.37) [b] 
(3372.9499999999143, 42.45) [b] 
(3372.899999999914, 42.47) [b] 
(3372.849999999914, 42.49) [b] 
(3372.799999999914, 42.5) [b] 
(3372.549999999914, 42.53) [b] 
(3372.4999999999136, 42.6) [b] 
(3372.4499999999134, 42.61) [b] 
(3372.3499999999135, 42.69) [b] 
(3372.2499999999136, 42.73) [b] 
(3372.0999999999135, 42.75) [b] 
(3371.9999999999136, 42.77) [b] 
(3371.9499999999134, 42.78) [b] 
(3371.8999999999132, 42.79) [b] 
(3371.849999999913, 42.8) [b] 
(3371.799999999913, 42.81) [b] 
(3371.7499999999127, 42.92) [b] 
(3371.649999999913, 42.98) [b] 
(3371.4999999999127, 43.01) [b] 
(3371.4499999999125, 43.03) [b] 
(3371.3499999999126, 43.05) [b] 
(3371.2499999999127, 43.06) [b] 
(3371.1499999999123, 43.09) [b] 
(3371.0499999999124, 43.12) [b] 
(3370.7999999999124, 43.13) [b] 
(3370.6999999999125, 43.16) [b] 
(3370.6499999999123, 43.17) [b] 
(3370.599999999912, 43.22) [b] 
(3370.349999999912, 43.26) [b] 
(3370.199999999912, 43.37) [b] 
(3370.049999999912, 43.38) [b] 
(3369.999999999912, 43.42) [b] 
(3369.6999999999116, 43.43) [b] 
(3369.5999999999117, 43.52) [b] 
(3369.5499999999115, 43.55) [b] 
(3369.4999999999113, 43.62) [b] 
(3369.449999999911, 43.68) [b] 
(3369.2499999999113, 43.69) [b] 
(3369.0499999999115, 43.71) [b] 
(3368.9499999999116, 43.72) [b] 
(3368.8499999999117, 43.77) [b] 
(3368.7999999999115, 43.78) [b] 
(3368.7499999999113, 43.79) [b] 
(3368.6499999999114, 43.84) [b] 
(3368.449999999911, 43.85) [b] 
(3368.399999999911, 43.86) [b] 
(3368.349999999911, 43.89) [b] 
(3368.2999999999106, 43.9) [b] 
(3368.1999999999107, 43.91) [b] 
(3368.1499999999105, 43.99) [b] 
(3368.0999999999103, 44.02) [b] 
(3368.04999999991, 44.08) [b] 
(3367.49999999991, 44.1) [b] 
(3367.24999999991, 44.13) [b] 
(3367.19999999991, 44.21) [b] 
(3367.1499999999096, 44.3) [b] 
(3366.94999999991, 44.34) [b] 
(3366.84999999991, 44.38) [b] 
(3366.74999999991, 44.4) [b] 
(3366.5499999999097, 44.45) [b] 
(3366.3999999999096, 44.46) [b] 
(3366.2999999999097, 44.48) [b] 
(3366.2499999999095, 44.49) [b] 
(3366.1999999999093, 44.54) [b] 
(3366.149999999909, 44.57) [b] 
(3366.099999999909, 44.62) [b] 
(3366.049999999909, 44.63) [b] 
(3365.9999999999086, 44.64) [b] 
(3365.9499999999084, 44.65) [b] 
(3365.8999999999082, 44.68) [b] 
(3365.749999999908, 44.69) [b] 
(3365.699999999908, 44.73) [b] 
(3365.649999999908, 44.74) [b] 
(3365.5999999999076, 44.77) [b] 
(3365.4999999999077, 44.82) [b] 
(3365.4499999999075, 44.84) [b] 
(3365.2499999999077, 44.92) [b] 
(3365.1999999999075, 44.97) [b] 
(3365.1499999999073, 45) [b] 
(3365.099999999907, 45.02) [b] 
(3364.9999999999072, 45.03) [b] 
(3364.949999999907, 45.15) [b] 
(3364.799999999907, 45.17) [b] 
(3364.749999999907, 45.19) [b] 
(3364.499999999907, 45.2) [b] 
(3364.4499999999066, 45.23) [b] 
(3364.1499999999064, 45.26) [b] 
(3364.0999999999062, 45.28) [b] 
(3364.049999999906, 45.29) [b] 
(3363.999999999906, 45.33) [b] 
(3363.849999999906, 45.4) [b] 
(3363.7999999999056, 45.41) [b] 
(3363.7499999999054, 45.47) [b] 
(3363.6499999999055, 45.55) [b] 
(3363.5999999999053, 45.58) [b] 
(3363.4999999999054, 45.64) [b] 
(3363.4499999999052, 45.73) [b] 
(3363.1499999999055, 45.75) [b] 
(3363.0499999999056, 45.78) [b] 
(3362.8999999999055, 45.8) [b] 
(3362.8499999999053, 45.83) [b] 
(3362.7499999999054, 45.84) [b] 
(3362.6499999999055, 45.85) [b] 
(3362.4999999999054, 45.89) [b] 
(3362.4499999999052, 45.91) [b] 
(3362.399999999905, 45.92) [b] 
(3362.349999999905, 45.93) [b] 
(3362.2999999999047, 46.05) [b] 
(3362.199999999905, 46.07) [b] 
(3362.1499999999046, 46.09) [b] 
(3362.0999999999044, 46.18) [b] 
(3362.0499999999042, 46.19) [b] 
(3361.999999999904, 46.2) [b] 
(3361.949999999904, 46.23) [b] 
(3361.849999999904, 46.24) [b] 
(3361.799999999904, 46.25) [b] 
(3361.6999999999034, 46.3) [b] 
(3361.3999999999032, 46.31) [b] 
(3361.2999999999033, 46.43) [b] 
(3361.1499999999032, 46.46) [b] 
(3360.5999999999035, 46.55) [b] 
(3360.3999999999037, 46.56) [b] 
(3360.199999999904, 46.61) [b] 
(3360.1499999999037, 46.62) [b] 
(3360.0999999999035, 46.68) [b] 
(3359.8999999999032, 46.72) [b] 
(3359.7999999999033, 46.75) [b] 
(3359.6999999999034, 46.79) [b] 
(3359.2499999999036, 46.82) [b] 
(3359.1999999999034, 46.83) [b] 
(3359.0999999999035, 46.84) [b] 
(3359.0499999999033, 46.89) [b] 
(3358.9499999999034, 46.93) [b] 
(3358.8999999999032, 46.99) [b] 
(3358.849999999903, 47) [b] 
(3358.799999999903, 47.04) [b] 
(3358.7499999999027, 47.05) [b] 
(3358.6999999999025, 47.07) [b] 
(3358.6499999999023, 47.1) [b] 
(3358.5499999999024, 47.16) [b] 
(3358.4999999999022, 47.2) [b] 
(3358.449999999902, 47.21) [b] 
(3358.399999999902, 47.25) [b] 
(3358.299999999902, 47.26) [b] 
(3358.249999999902, 47.27) [b] 
(3358.0499999999015, 47.37) [b] 
(3357.9999999999013, 47.38) [b] 
(3357.949999999901, 47.39) [b] 
(3357.8499999999012, 47.4) [b] 
(3357.699999999901, 47.44) [b] 
(3357.449999999901, 47.53) [b] 
(3357.399999999901, 47.56) [b] 
(3357.049999999901, 47.57) [b] 
(3356.899999999901, 47.69) [b] 
(3356.7999999999006, 47.73) [b] 
(3356.7499999999004, 47.75) [b] 
(3356.6499999999005, 47.78) [b] 
(3356.4999999999004, 47.82) [b] 
(3356.4499999999002, 47.87) [b] 
(3356.3999999999, 47.88) [b] 
(3356.2499999999, 47.89) [b] 
(3356.1499999999, 47.91) [b] 
(3356.0999999999, 47.92) [b] 
(3356.0499999998997, 47.94) [b] 
(3355.9999999998995, 47.96) [b] 
(3355.9499999998993, 47.98) [b] 
(3355.8499999998994, 48.09) [b] 
(3355.7999999998992, 48.12) [b] 
(3355.699999999899, 48.15) [b] 
(3355.599999999899, 48.19) [b] 
(3355.549999999899, 48.22) [b] 
(3355.4999999998986, 48.26) [b] 
(3355.4499999998984, 48.36) [b] 
(3355.3499999998985, 48.38) [b] 
(3355.2999999998983, 48.39) [b] 
(3355.249999999898, 48.43) [b] 
(3355.199999999898, 48.46) [b] 
(3355.049999999898, 48.47) [b] 
(3354.899999999898, 48.48) [b] 
(3354.7499999998977, 48.49) [b] 
(3354.5999999998976, 48.51) [b] 
(3354.5499999998974, 48.58) [b] 
(3354.249999999897, 48.67) [b] 
(3354.199999999897, 48.68) [b] 
(3354.149999999897, 48.69) [b] 
(3354.049999999897, 48.79) [b] 
(3353.949999999897, 48.81) [b] 
(3353.749999999897, 48.83) [b] 
(3353.699999999897, 48.85) [b] 
(3353.649999999897, 48.91) [b] 
(3353.5499999998965, 48.93) [b] 
(3353.4999999998963, 48.97) [b] 
(3353.449999999896, 49.08) [b] 
(3353.3499999998958, 49.1) [b] 
(3353.2999999998956, 49.16) [b] 
(3353.1999999998957, 49.2) [b] 
(3353.1499999998955, 49.23) [b] 
(3353.0999999998953, 49.26) [b] 
(3353.049999999895, 49.29) [b] 
(3352.999999999895, 49.32) [b] 
(3352.849999999895, 49.33) [b] 
(3352.749999999895, 49.36) [b] 
(3352.6999999998948, 49.37) [b] 
(3352.6499999998946, 49.38) [b] 
(3352.5499999998947, 49.39) [b] 
(3352.4999999998945, 49.45) [b] 
(3352.399999999894, 49.46) [b] 
(3352.349999999894, 49.48) [b] 
(3352.149999999894, 49.54) [b] 
(3351.999999999894, 49.59) [b] 
(3351.849999999894, 49.62) [b] 
(3351.7999999998938, 49.7) [b] 
(3351.4999999998936, 49.81) [b] 
(3351.4499999998934, 49.86) [b] 
(3351.399999999893, 49.89) [b] 
(3351.349999999893, 49.93) [b] 
(3351.249999999893, 49.96) [b] 
(3351.1499999998928, 49.97) [b] 
(3350.7999999998924, 50.03) [b] 
(3350.6999999998925, 50.05) [b] 
(3350.6499999998923, 50.06) [b] 
(3350.499999999892, 50.07) [b] 
(3350.3999999998923, 50.08) [b] 
(3349.9499999998925, 50.17) [b] 
(3349.8499999998926, 50.18) [b] 
(3349.6999999998925, 50.21) [b] 
(3349.5999999998926, 50.28) [b] 
(3349.4499999998925, 50.35) [b] 
(3349.3499999998926, 50.41) [b] 
(3349.2999999998924, 50.45) [b] 
(3349.249999999892, 50.46) [b] 
(3349.1499999998923, 50.48) [b] 
(3349.099999999892, 50.5) [b] 
(3348.999999999892, 50.52) [b] 
(3348.8999999998923, 50.53) [b] 
(3348.449999999892, 50.56) [b] 
(3348.399999999892, 50.58) [b] 
(3348.3499999998917, 50.66) [b] 
(3348.2499999998918, 50.69) [b] 
(3348.149999999892, 50.73) [b] 
(3348.0999999998917, 50.79) [b] 
(3348.0499999998915, 50.82) [b] 
(3347.9999999998913, 50.83) [b] 
(3347.949999999891, 50.85) [b] 
(3347.899999999891, 50.93) [b] 
(3347.8499999998908, 50.99) [b] 
(3347.7999999998906, 51.01) [b] 
(3347.7499999998904, 51.02) [b] 
(3347.6499999998905, 51.03) [b] 
(3347.5999999998903, 51.05) [b] 
(3347.44999999989, 51.06) [b] 
(3347.34999999989, 51.08) [b] 
(3347.14999999989, 51.11) [b] 
(3347.09999999989, 51.12) [b] 
(3346.7999999998897, 51.21) [b] 
(3346.7499999998895, 51.25) [b] 
(3346.6499999998896, 51.43) [b] 
(3346.5999999998894, 51.45) [b] 
(3346.549999999889, 51.46) [b] 
(3346.399999999889, 51.48) [b] 
(3346.149999999889, 51.52) [b] 
(3346.099999999889, 51.54) [b] 
(3346.0499999998888, 51.57) [b] 
(3345.9999999998886, 51.58) [b] 
(3345.7999999998888, 51.61) [b] 
(3345.699999999889, 51.64) [b] 
(3345.4999999998886, 51.66) [b] 
(3345.3999999998887, 51.7) [b] 
(3345.3499999998885, 51.77) [b] 
(3345.2999999998883, 51.78) [b] 
(3344.999999999888, 51.8) [b] 
(3344.849999999888, 51.85) [b] 
(3344.799999999888, 51.95) [b] 
(3344.699999999888, 52.03) [b] 
(3344.599999999888, 52.05) [b] 
(3344.449999999888, 52.06) [b] 
(3344.3999999998878, 52.07) [b] 
(3344.3499999998876, 52.09) [b] 
(3344.2999999998874, 52.12) [b] 
(3344.249999999887, 52.13) [b] 
(3344.0499999998874, 52.14) [b] 
(3343.999999999887, 52.15) [b] 
(3343.949999999887, 52.22) [b] 
(3343.899999999887, 52.24) [b] 
(3343.7499999998868, 52.26) [b] 
(3343.6999999998866, 52.28) [b] 
(3343.5999999998867, 52.33) [b] 
(3343.4999999998868, 52.4) [b] 
(3343.4499999998866, 52.43) [b] 
(3343.3999999998864, 52.47) [b] 
(3343.349999999886, 52.48) [b] 
(3343.2499999998863, 52.55) [b] 
(3343.199999999886, 52.59) [b] 
(3342.9999999998863, 52.61) [b] 
(3342.949999999886, 52.62) [b] 
(3342.799999999886, 52.68) [b] 
(3342.649999999886, 52.7) [b] 
(3342.499999999886, 52.8) [b] 
(3342.4499999998857, 52.82) [b] 
(3342.2999999998856, 52.83) [b] 
(3342.1999999998857, 52.94) [b] 
(3342.1499999998855, 52.95) [b] 
(3341.9999999998854, 52.97) [b] 
(3341.8999999998855, 53) [b] 
(3341.8499999998853, 53.02) [b] 
(3341.799999999885, 53.03) [b] 
(3341.699999999885, 53.1) [b] 
(3341.5999999998853, 53.13) [b] 
(3341.399999999885, 53.19) [b] 
(3341.299999999885, 53.21) [b] 
(3340.649999999885, 53.22) [b] 
(3340.499999999885, 53.25) [b] 
(3340.1999999998848, 53.26) [b] 
(3340.1499999998846, 53.38) [b] 
(3340.0999999998844, 53.41) [b] 
(3339.8499999998844, 53.49) [b] 
(3339.7499999998845, 53.51) [b] 
(3339.4999999998845, 53.54) [b] 
(3339.4499999998843, 53.62) [b] 
(3339.3499999998844, 53.63) [b] 
(3339.299999999884, 53.64) [b] 
(3339.049999999884, 53.65) [b] 
(3338.999999999884, 53.69) [b] 
(3338.949999999884, 53.71) [b] 
(3338.8499999998835, 53.72) [b] 
(3338.7499999998836, 53.75) [b] 
(3338.6499999998837, 53.76) [b] 
(3338.5999999998835, 53.77) [b] 
(3338.5499999998833, 53.8) [b] 
(3338.499999999883, 53.85) [b] 
(3338.399999999883, 53.88) [b] 
(3338.349999999883, 53.95) [b] 
(3338.299999999883, 54.02) [b] 
(3338.199999999883, 54.03) [b] 
(3338.099999999883, 54.12) [b] 
(3337.999999999883, 54.16) [b] 
(3337.849999999883, 54.21) [b] 
(3337.799999999883, 54.25) [b] 
(3337.549999999883, 54.27) [b] 
(3337.4999999998827, 54.28) [b] 
(3337.4499999998825, 54.3) [b] 
(3337.2499999998827, 54.33) [b] 
(3337.0999999998826, 54.35) [b] 
(3336.9999999998827, 54.36) [b] 
(3336.9499999998825, 54.37) [b] 
(3336.8999999998823, 54.4) [b] 
(3336.849999999882, 54.46) [b] 
(3336.749999999882, 54.47) [b] 
(3336.6499999998823, 54.54) [b] 
(3336.599999999882, 54.63) [b] 
(3336.549999999882, 54.64) [b] 
(3336.4999999998818, 54.66) [b] 
(3336.4499999998816, 54.69) [b] 
(3336.3999999998814, 54.72) [b] 
(3336.299999999881, 54.73) [b] 
(3336.199999999881, 54.79) [b] 
(3336.049999999881, 54.82) [b] 
(3335.999999999881, 54.85) [b] 
(3335.899999999881, 54.88) [b] 
(3335.8499999998808, 54.91) [b] 
(3335.749999999881, 54.93) [b] 
(3335.6999999998807, 54.98) [b] 
(3335.6499999998805, 54.99) [b] 
(3335.5499999998806, 55.03) [b] 
(3335.4499999998807, 55.05) [b] 
(3335.3999999998805, 55.14) [b] 
(3335.3499999998803, 55.16) [b] 
(3335.09999999988, 55.24) [b] 
(3334.9499999998798, 55.29) [b] 
(3334.8999999998796, 55.35) [b] 
(3334.7999999998797, 55.4) [b] 
(3334.6499999998796, 55.41) [b] 
(3334.4499999998793, 55.44) [b] 
(3334.349999999879, 55.49) [b] 
(3334.2999999998788, 55.5) [b] 
(3334.0499999998788, 55.51) [b] 
(3333.849999999879, 55.55) [b] 
(3333.749999999879, 55.61) [b] 
(3333.649999999879, 55.62) [b] 
(3333.599999999879, 55.67) [b] 
(3333.5499999998788, 55.7) [b] 
(3333.449999999879, 55.73) [b] 
(3333.3499999998785, 55.83) [b] 
(3333.2999999998783, 55.84) [b] 
(3333.249999999878, 55.85) [b] 
(3333.1499999998778, 55.87) [b] 
(3333.0999999998776, 55.88) [b] 
(3332.8999999998778, 56) [b] 
(3332.799999999878, 56.03) [b] 
(3332.699999999878, 56.07) [b] 
(3332.599999999878, 56.12) [b] 
(3332.549999999878, 56.18) [b] 
(3332.449999999878, 56.22) [b] 
(3332.3999999998778, 56.26) [b] 
(3332.3499999998776, 56.27) [b] 
(3332.1999999998775, 56.3) [b] 
(3332.1499999998773, 56.36) [b] 
(3331.9999999998768, 56.37) [b] 
(3331.9499999998766, 56.39) [b] 
(3331.8999999998764, 56.4) [b] 
(3331.849999999876, 56.41) [b] 
(3331.799999999876, 56.46) [b] 
(3331.749999999876, 56.49) [b] 
(3331.649999999876, 56.55) [b] 
(3331.549999999876, 56.58) [b] 
(3331.399999999876, 56.59) [b] 
(3331.249999999876, 56.63) [b] 
(3330.899999999876, 56.67) [b] 
(3330.8499999998758, 56.73) [b] 
(3330.749999999876, 56.74) [b] 
(3330.6999999998757, 56.8) [b] 
(3330.5499999998756, 56.85) [b] 
(3330.4999999998754, 56.86) [b] 
(3330.449999999875, 56.89) [b] 
(3330.399999999875, 56.93) [b] 
(3330.299999999875, 56.95) [b] 
(3330.149999999875, 56.97) [b] 
(3330.099999999875, 56.99) [b] 
(3330.0499999998747, 57) [b] 
(3329.9999999998745, 57.06) [b] 
(3329.8999999998746, 57.07) [b] 
(3329.7999999998747, 57.09) [b] 
(3329.6999999998748, 57.1) [b] 
(3329.6499999998746, 57.12) [b] 
(3329.5999999998744, 57.16) [b] 
(3329.2499999998745, 57.23) [b] 
(3329.1999999998743, 57.26) [b] 
(3329.049999999874, 57.35) [b] 
(3328.999999999874, 57.38) [b] 
(3328.899999999874, 57.39) [b] 
(3328.649999999874, 57.43) [b] 
(3328.299999999874, 57.44) [b] 
(3328.249999999874, 57.47) [b] 
(3328.199999999874, 57.48) [b] 
(3328.1499999998737, 57.49) [b] 
(3328.0499999998738, 57.53) [b] 
(3327.9999999998736, 57.54) [b] 
(3327.9499999998734, 57.55) [b] 
(3327.899999999873, 57.56) [b] 
(3327.849999999873, 57.64) [b] 
(3327.699999999873, 57.65) [b] 
(3327.449999999873, 57.66) [b] 
(3327.3999999998728, 57.75) [b] 
(3327.3499999998726, 57.77) [b] 
(3327.1999999998725, 57.83) [b] 
(3327.1499999998723, 57.86) [b] 
(3326.999999999872, 57.89) [b] 
(3326.949999999872, 57.93) [b] 
(3326.899999999872, 57.98) [b] 
(3326.799999999872, 58) [b] 
(3326.7499999998718, 58.04) [b] 
(3326.6999999998716, 58.06) [b] 
(3326.5999999998717, 58.08) [b] 
(3326.4499999998716, 58.12) [b] 
(3326.3999999998714, 58.13) [b] 
(3326.349999999871, 58.14) [b] 
(3326.199999999871, 58.19) [b] 
(3326.149999999871, 58.22) [b] 
(3325.999999999871, 58.24) [b] 
(3325.899999999871, 58.34) [b] 
(3325.799999999871, 58.4) [b] 
(3325.749999999871, 58.44) [b] 
(3325.6499999998705, 58.48) [b] 
(3325.5499999998706, 58.52) [b] 
(3325.4999999998704, 58.53) [b] 
(3325.44999999987, 58.55) [b] 
(3325.39999999987, 58.57) [b] 
(3325.34999999987, 58.58) [b] 
(3325.1999999998698, 58.59) [b] 
(3325.09999999987, 58.6) [b] 
(3325.0499999998697, 58.66) [b] 
(3324.9999999998695, 58.67) [b] 
(3324.8499999998694, 58.72) [b] 
(3324.799999999869, 58.82) [b] 
(3324.4499999998693, 58.91) [b] 
(3324.299999999869, 58.93) [b] 
(3324.1999999998693, 58.96) [b] 
(3324.0999999998694, 58.97) [b] 
(3324.049999999869, 58.99) [b] 
(3323.849999999869, 59.03) [b] 
(3323.7999999998688, 59.04) [b] 
(3323.6999999998684, 59.08) [b] 
(3323.649999999868, 59.09) [b] 
(3323.599999999868, 59.1) [b] 
(3323.549999999868, 59.14) [b] 
(3323.4999999998677, 59.16) [b] 
(3323.4499999998675, 59.21) [b] 
(3323.3999999998673, 59.25) [b] 
(3323.1499999998673, 59.28) [b] 
(3323.099999999867, 59.31) [b] 
(3322.999999999867, 59.35) [b] 
(3322.7999999998674, 59.4) [b] 
(3322.6999999998675, 59.44) [b] 
(3322.5499999998674, 59.48) [b] 
(3322.3999999998673, 59.55) [b] 
(3322.249999999867, 59.61) [b] 
(3322.1499999998673, 59.62) [b] 
(3321.9499999998675, 59.63) [b] 
(3321.8499999998676, 59.68) [b] 
(3321.7999999998674, 59.74) [b] 
(3321.6499999998673, 59.75) [b] 
(3321.599999999867, 59.76) [b] 
(3321.549999999867, 59.8) [b] 
(3321.4999999998668, 59.82) [b] 
(3321.4499999998666, 59.89) [b] 
(3321.2999999998665, 59.91) [b] 
(3321.0499999998665, 59.93) [b] 
(3320.949999999866, 59.99) [b] 
(3320.899999999866, 60.03) [b] 
(3320.8499999998658, 60.06) [b] 
(3320.749999999866, 60.07) [b] 
(3320.6999999998657, 60.17) [b] 
(3320.5499999998656, 60.19) [b] 
(3320.4999999998654, 60.2) [b] 
(3320.399999999865, 60.25) [b] 
(3320.199999999865, 60.35) [b] 
(3320.049999999865, 60.37) [b] 
(3319.999999999865, 60.38) [b] 
(3319.849999999865, 60.41) [b] 
(3319.7999999998647, 60.42) [b] 
(3319.6499999998646, 60.45) [b] 
(3319.4499999998648, 60.46) [b] 
(3319.249999999865, 60.49) [b] 
(3319.1999999998648, 60.52) [b] 
(3319.0499999998647, 60.55) [b] 
(3318.7999999998647, 60.66) [b] 
(3318.6499999998646, 60.67) [b] 
(3318.5999999998644, 60.7) [b] 
(3318.4999999998645, 60.73) [b] 
(3318.399999999864, 60.74) [b] 
(3318.349999999864, 60.79) [b] 
(3318.2499999998636, 60.86) [b] 
(3318.1999999998634, 60.87) [b] 
(3318.0499999998633, 60.89) [b] 
(3317.999999999863, 60.95) [b] 
(3317.949999999863, 60.98) [b] 
(3317.849999999863, 61.02) [b] 
(3317.699999999863, 61.08) [b] 
(3317.6499999998628, 61.11) [b] 
(3317.549999999863, 61.18) [b] 
(3317.4999999998627, 61.19) [b] 
(3317.3999999998628, 61.25) [b] 
(3317.1499999998628, 61.27) [b] 
(3317.0999999998626, 61.28) [b] 
(3316.9999999998627, 61.29) [b] 
(3316.8499999998626, 61.35) [b] 
(3316.7999999998624, 61.37) [b] 
(3316.749999999862, 61.42) [b] 
(3316.699999999862, 61.5) [b] 
(3316.649999999862, 61.51) [b] 
(3316.4499999998616, 61.53) [b] 
(3316.3999999998614, 61.54) [b] 
(3316.349999999861, 61.55) [b] 
(3316.1499999998614, 61.59) [b] 
(3315.9999999998613, 61.6) [b] 
(3315.849999999861, 61.65) [b] 
(3315.799999999861, 61.66) [b] 
(3315.749999999861, 61.76) [b] 
(3315.649999999861, 61.79) [b] 
(3315.5999999998608, 61.82) [b] 
(3315.5499999998606, 61.84) [b] 
(3315.4999999998604, 61.85) [b] 
(3315.39999999986, 61.92) [b] 
(3315.29999999986, 61.93) [b] 
(3315.24999999986, 61.94) [b] 
(3315.1499999998596, 62.08) [b] 
(3314.9499999998593, 62.09) [b] 
(3314.899999999859, 62.12) [b] 
(3314.749999999859, 62.17) [b] 
(3314.6499999998587, 62.18) [b] 
(3314.5999999998585, 62.19) [b] 
(3314.5499999998583, 62.27) [b] 
(3314.499999999858, 62.29) [b] 
(3314.399999999858, 62.33) [b] 
(3314.2999999998583, 62.36) [b] 
(3314.1999999998584, 62.37) [b] 
(3314.0999999998585, 62.4) [b] 
(3314.0499999998583, 62.42) [b] 
(3313.999999999858, 62.45) [b] 
(3313.899999999858, 62.48) [b] 
(3313.849999999858, 62.52) [b] 
(3313.799999999858, 62.55) [b] 
(3313.7499999998577, 62.63) [b] 
(3313.6999999998575, 62.65) [b] 
(3313.5499999998574, 62.67) [b] 
(3313.449999999857, 62.72) [b] 
(3313.399999999857, 62.73) [b] 
(3313.2499999998568, 62.76) [b] 
(3313.049999999857, 62.77) [b] 
(3312.899999999857, 62.8) [b] 
(3312.7499999998568, 62.81) [b] 
(3312.6999999998566, 62.96) [b] 
(3312.6499999998564, 62.97) [b] 
(3312.599999999856, 62.98) [b] 
(3312.499999999856, 62.99) [b] 
(3312.399999999856, 63) [b] 
(3312.299999999856, 63.02) [b] 
(3312.249999999856, 63.14) [b] 
(3312.149999999856, 63.15) [b] 
(3312.0999999998558, 63.22) [b] 
(3312.0499999998556, 63.25) [b] 
(3311.8999999998555, 63.26) [b] 
(3311.5999999998558, 63.31) [b] 
(3311.5499999998556, 63.35) [b] 
(3311.4999999998554, 63.37) [b] 
(3311.449999999855, 63.4) [b] 
(3311.3499999998553, 63.42) [b] 
(3311.299999999855, 63.45) [b] 
(3311.149999999855, 63.48) [b] 
(3310.999999999855, 63.51) [b] 
(3310.849999999855, 63.62) [b] 
(3310.7999999998547, 63.63) [b] 
(3310.7499999998545, 63.66) [b] 
(3310.649999999854, 63.73) [b] 
(3310.499999999854, 63.74) [b] 
(3310.449999999854, 63.85) [b] 
(3310.3999999998537, 63.86) [b] 
(3310.2499999998536, 63.87) [b] 
(3309.8499999998535, 63.91) [b] 
(3309.6999999998534, 63.93) [b] 
(3309.5499999998533, 63.97) [b] 
(3309.499999999853, 64.03) [b] 
(3309.349999999853, 64.06) [b] 
(3309.299999999853, 64.09) [b] 
(3309.2499999998527, 64.11) [b] 
(3309.049999999853, 64.13) [b] 
(3308.8499999998526, 64.21) [b] 
(3308.7999999998524, 64.23) [b] 
(3308.699999999852, 64.25) [b] 
(3308.599999999852, 64.28) [b] 
(3308.499999999852, 64.37) [b] 
(3308.3999999998523, 64.43) [b] 
(3308.2999999998524, 64.46) [b] 
(3308.199999999852, 64.5) [b] 
(3308.0999999998517, 64.52) [b] 
(3307.9999999998513, 64.55) [b] 
(3307.949999999851, 64.58) [b] 
(3307.8499999998508, 64.59) [b] 
(3307.6999999998507, 64.63) [b] 
(3307.6499999998505, 64.71) [b] 
(3307.5999999998503, 64.74) [b] 
(3307.54999999985, 64.81) [b] 
(3307.49999999985, 64.82) [b] 
(3307.04999999985, 64.84) [b] 
(3306.8499999998503, 64.94) [b] 
(3306.74999999985, 65.08) [b] 
(3306.6999999998498, 65.14) [b] 
(3306.59999999985, 65.15) [b] 
(3306.5499999998497, 65.16) [b] 
(3306.4999999998495, 65.2) [b] 
(3306.3999999998496, 65.22) [b] 
(3306.2999999998497, 65.25) [b] 
(3306.2499999998495, 65.28) [b] 
(3306.1999999998493, 65.29) [b] 
(3305.9999999998495, 65.34) [b] 
(3305.8999999998496, 65.35) [b] 
(3305.8499999998494, 65.36) [b] 
(3305.799999999849, 65.4) [b] 
(3305.749999999849, 65.42) [b] 
(3305.349999999849, 65.45) [b] 
(3305.2999999998488, 65.46) [b] 
(3305.2499999998486, 65.5) [b] 
(3305.1999999998484, 65.51) [b] 
(3305.149999999848, 65.52) [b] 
(3305.099999999848, 65.61) [b] 
(3305.049999999848, 65.65) [b] 
(3304.9999999998477, 65.66) [b] 
(3304.9499999998475, 65.7) [b] 
(3304.6499999998473, 65.73) [b] 
(3304.549999999847, 65.77) [b] 
(3304.4999999998468, 65.81) [b] 
(3304.4499999998466, 65.82) [b] 
(3304.3999999998464, 65.9) [b] 
(3304.2499999998463, 65.96) [b] 
(3304.199999999846, 66.03) [b] 
(3304.0999999998457, 66.04) [b] 
(3304.0499999998456, 66.06) [b] 
(3303.9999999998454, 66.1) [b] 
(3303.949999999845, 66.12) [b] 
(3303.899999999845, 66.13) [b] 
(3303.849999999845, 66.17) [b] 
(3303.5499999998447, 66.29) [b] 
(3303.4999999998445, 66.33) [b] 
(3303.4499999998443, 66.38) [b] 
(3303.2999999998437, 66.39) [b] 
(3303.1499999998437, 66.41) [b] 
(3303.0999999998435, 66.42) [b] 
(3303.0499999998433, 66.45) [b] 
(3302.999999999843, 66.47) [b] 
(3302.899999999843, 66.52) [b] 
(3302.849999999843, 66.56) [b] 
(3302.799999999843, 66.59) [b] 
(3302.699999999843, 66.67) [b] 
(3302.449999999843, 66.68) [b] 
(3302.3999999998427, 66.74) [b] 
(3302.299999999843, 66.76) [b] 
(3302.2499999998427, 66.78) [b] 
(3302.0999999998426, 66.8) [b] 
(3302.0499999998424, 66.85) [b] 
(3301.999999999842, 66.95) [b] 
(3301.7999999998424, 66.98) [b] 
(3301.6999999998425, 66.99) [b] 
(3301.5999999998426, 67.02) [b] 
(3301.4499999998425, 67.06) [b] 
(3301.1499999998423, 67.08) [b] 
(3301.049999999842, 67.1) [b] 
(3300.9999999998417, 67.11) [b] 
(3300.8499999998417, 67.16) [b] 
(3300.7499999998413, 67.18) [b] 
(3300.549999999841, 67.33) [b] 
(3300.399999999841, 67.34) [b] 
(3300.3499999998407, 67.42) [b] 
(3300.1999999998407, 67.49) [b] 
(3300.1499999998405, 67.5) [b] 
(3300.0999999998403, 67.53) [b] 
(3299.9999999998404, 67.54) [b] 
(3299.8999999998405, 67.55) [b] 
(3299.8499999998403, 67.61) [b] 
(3299.79999999984, 67.64) [b] 
(3299.5499999998397, 67.66) [b] 
(3299.4499999998397, 67.7) [b] 
(3299.3999999998396, 67.72) [b] 
(3299.3499999998394, 67.74) [b] 
(3299.2499999998395, 67.82) [b] 
(3299.0499999998397, 67.88) [b] 
(3298.9499999998397, 67.9) [b] 
(3298.8999999998396, 67.94) [b] 
(3298.7999999998397, 67.95) [b] 
(3298.7499999998395, 68.01) [b] 
(3298.6999999998393, 68.03) [b] 
(3298.649999999839, 68.06) [b] 
(3298.499999999839, 68.07) [b] 
(3298.399999999839, 68.08) [b] 
(3298.349999999839, 68.09) [b] 
(3298.2999999998387, 68.1) [b] 
(3298.1499999998387, 68.16) [b] 
(3297.5499999998387, 68.21) [b] 
(3297.449999999839, 68.27) [b] 
(3297.3499999998385, 68.31) [b] 
(3297.2999999998383, 68.33) [b] 
(3297.249999999838, 68.42) [b] 
(3297.149999999838, 68.43) [b] 
(3297.099999999838, 68.46) [b] 
(3296.999999999838, 68.5) [b] 
(3296.949999999838, 68.51) [b] 
(3296.8999999998377, 68.53) [b] 
(3296.8499999998376, 68.58) [b] 
(3296.6999999998375, 68.61) [b] 
(3296.6499999998373, 68.64) [b] 
(3296.5499999998374, 68.65) [b] 
(3296.3499999998376, 68.68) [b] 
(3296.1499999998377, 68.71) [b] 
(3296.0999999998376, 68.74) [b] 
(3296.0499999998374, 68.76) [b] 
(3295.8999999998373, 68.79) [b] 
(3295.849999999837, 68.8) [b] 
(3295.699999999837, 68.85) [b] 
(3295.649999999837, 68.9) [b] 
(3295.449999999837, 68.92) [b] 
(3295.399999999837, 68.93) [b] 
(3295.2999999998365, 68.94) [b] 
(3295.1999999998366, 68.99) [b] 
(3295.0999999998367, 69.04) [b] 
(3295.0499999998365, 69.06) [b] 
(3294.949999999836, 69.08) [b] 
(3294.899999999836, 69.2) [b] 
(3294.799999999836, 69.24) [b] 
(3294.749999999836, 69.27) [b] 
(3294.6999999998357, 69.29) [b] 
(3294.5999999998357, 69.33) [b] 
(3294.5499999998356, 69.34) [b] 
(3294.3999999998355, 69.35) [b] 
(3294.1999999998357, 69.4) [b] 
(3294.1499999998355, 69.41) [b] 
(3293.9999999998354, 69.43) [b] 
(3293.8999999998355, 69.52) [b] 
(3293.8499999998353, 69.55) [b] 
(3293.749999999835, 69.59) [b] 
(3293.6999999998347, 69.62) [b] 
(3293.6499999998346, 69.64) [b] 
(3293.5999999998344, 69.77) [b] 
(3293.349999999834, 69.79) [b] 
(3293.2999999998337, 69.81) [b] 
(3293.199999999834, 69.83) [b] 
(3293.099999999834, 69.88) [b] 
(3293.0499999998337, 69.92) [b] 
(3292.949999999834, 70) [b] 
(3292.8999999998337, 70.04) [b] 
(3292.8499999998335, 70.07) [b] 
(3292.6999999998334, 70.11) [b] 
(3292.649999999833, 70.15) [b] 
(3292.499999999833, 70.16) [b] 
(3292.449999999833, 70.17) [b] 
(3292.3999999998327, 70.18) [b] 
(3292.2499999998327, 70.19) [b] 
(3292.0499999998324, 70.23) [b] 
(3291.999999999832, 70.26) [b] 
(3291.949999999832, 70.31) [b] 
(3291.899999999832, 70.33) [b] 
(3291.799999999832, 70.34) [b] 
(3291.7499999998317, 70.39) [b] 
(3291.6999999998316, 70.43) [b] 
(3291.6499999998314, 70.46) [b] 
(3291.599999999831, 70.56) [b] 
(3291.4999999998313, 70.57) [b] 
(3291.349999999831, 70.64) [b] 
(3291.049999999831, 70.66) [b] 
(3290.949999999831, 70.72) [b] 
(3290.5999999998307, 70.73) [b] 
(3290.499999999831, 70.76) [b] 
(3290.4499999998307, 70.78) [b] 
(3290.2999999998306, 70.81) [b] 
(3290.2499999998304, 70.84) [b] 
(3290.14999999983, 70.88) [b] 
(3289.99999999983, 70.91) [b] 
(3289.9499999998297, 70.93) [b] 
(3289.8499999998294, 70.94) [b] 
(3289.799999999829, 71.01) [b] 
(3289.749999999829, 71.02) [b] 
(3289.699999999829, 71.15) [b] 
(3289.6499999998287, 71.2) [b] 
(3289.5999999998285, 71.24) [b] 
(3289.5499999998283, 71.27) [b] 
(3288.899999999828, 71.28) [b] 
(3288.849999999828, 71.31) [b] 
(3288.749999999828, 71.32) [b] 
(3288.699999999828, 71.36) [b] 
(3288.449999999828, 71.37) [b] 
(3288.3999999998277, 71.46) [b] 
(3288.3499999998276, 71.47) [b] 
(3288.2499999998277, 71.56) [b] 
(3288.1999999998275, 71.57) [b] 
(3288.1499999998273, 71.62) [b] 
(3288.099999999827, 71.64) [b] 
(3287.849999999827, 71.65) [b] 
(3287.799999999827, 71.7) [b] 
(3287.7499999998267, 71.73) [b] 
(3287.6999999998266, 71.79) [b] 
(3287.6499999998264, 71.84) [b] 
(3287.549999999826, 71.86) [b] 
(3287.499999999826, 71.9) [b] 
(3287.4499999998256, 71.91) [b] 
(3287.3499999998257, 71.96) [b] 
(3287.1999999998256, 71.97) [b] 
(3287.1499999998255, 71.99) [b] 
(3287.049999999825, 72) [b] 
(3286.999999999825, 72.02) [b] 
(3286.549999999825, 72.18) [b] 
(3286.499999999825, 72.23) [b] 
(3286.4499999998247, 72.24) [b] 
(3286.3999999998246, 72.3) [b] 
(3286.299999999824, 72.31) [b] 
(3286.199999999824, 72.38) [b] 
(3285.9999999998236, 72.39) [b] 
(3285.9499999998234, 72.42) [b] 
(3285.899999999823, 72.43) [b] 
(3285.849999999823, 72.5) [b] 
(3285.799999999823, 72.51) [b] 
(3285.4999999998226, 72.53) [b] 
(3285.3499999998226, 72.65) [b] 
(3285.2999999998224, 72.67) [b] 
(3285.249999999822, 72.68) [b] 
(3285.199999999822, 72.71) [b] 
(3285.149999999822, 72.8) [b] 
(3284.949999999822, 72.83) [b] 
(3284.899999999822, 72.86) [b] 
(3284.799999999822, 72.87) [b] 
(3284.7499999998217, 72.89) [b] 
(3284.4499999998216, 72.9) [b] 
(3284.3999999998214, 72.92) [b] 
(3284.299999999821, 72.99) [b] 
(3284.199999999821, 73.04) [b] 
(3284.149999999821, 73.06) [b] 
(3284.0999999998207, 73.07) [b] 
(3283.8499999998207, 73.1) [b] 
(3283.6499999998205, 73.13) [b] 
(3283.5499999998206, 73.31) [b] 
(3283.3499999998207, 73.33) [b] 
(3283.1999999998206, 73.34) [b] 
(3283.1499999998205, 73.35) [b] 
(3283.0999999998203, 73.46) [b] 
(3283.04999999982, 73.47) [b] 
(3282.99999999982, 73.48) [b] 
(3282.89999999982, 73.5) [b] 
(3282.79999999982, 73.53) [b] 
(3282.69999999982, 73.54) [b] 
(3282.64999999982, 73.56) [b] 
(3282.54999999982, 73.58) [b] 
(3282.44999999982, 73.63) [b] 
(3282.1499999998205, 73.65) [b] 
(3281.6499999998205, 73.66) [b] 
(3281.3499999998203, 73.69) [b] 
(3281.29999999982, 73.76) [b] 
(3281.24999999982, 73.79) [b] 
(3281.1999999998197, 73.87) [b] 
(3281.09999999982, 73.91) [b] 
(3280.8499999998194, 73.99) [b] 
(3280.799999999819, 74.02) [b] 
(3280.549999999819, 74.11) [b] 
(3280.399999999819, 74.12) [b] 
(3280.349999999819, 74.17) [b] 
(3280.249999999819, 74.22) [b] 
(3280.199999999819, 74.23) [b] 
(3280.099999999819, 74.24) [b] 
(3280.0499999998187, 74.27) [b] 
(3279.8999999998186, 74.34) [b] 
(3279.8499999998185, 74.35) [b] 
(3279.7999999998183, 74.36) [b] 
(3279.6999999998184, 74.43) [b] 
(3279.5999999998185, 74.44) [b] 
(3279.4499999998184, 74.52) [b] 
(3279.2999999998183, 74.54) [b] 
(3279.249999999818, 74.59) [b] 
(3279.099999999818, 74.6) [b] 
(3278.949999999818, 74.61) [b] 
(3278.8999999998177, 74.7) [b] 
(3278.7999999998174, 74.73) [b] 
(3278.749999999817, 74.75) [b] 
(3278.6499999998173, 74.79) [b] 
(3278.549999999817, 74.81) [b] 
(3278.4999999998167, 74.82) [b] 
(3278.4499999998166, 74.89) [b] 
(3278.3999999998164, 74.93) [b] 
(3278.2999999998165, 74.95) [b] 
(3278.2499999998163, 75) [b] 
(3278.0499999998165, 75.01) [b] 
(3277.9499999998166, 75.02) [b] 
(3277.8499999998166, 75.04) [b] 
(3277.5999999998166, 75.11) [b] 
(3277.5499999998165, 75.16) [b] 
(3277.4999999998163, 75.2) [b] 
(3277.1999999998156, 75.21) [b] 
(3277.1499999998155, 75.33) [b] 
(3277.0999999998153, 75.34) [b] 
(3276.799999999815, 75.36) [b] 
(3276.749999999815, 75.49) [b] 
(3276.599999999815, 75.51) [b] 
(3276.5499999998146, 75.52) [b] 
(3276.4999999998145, 75.53) [b] 
(3276.4499999998143, 75.6) [b] 
(3276.3499999998144, 75.66) [b] 
(3276.049999999814, 75.69) [b] 
(3275.899999999814, 75.7) [b] 
(3275.849999999814, 75.74) [b] 
(3275.7999999998137, 75.8) [b] 
(3275.5999999998135, 75.85) [b] 
(3275.4499999998134, 75.86) [b] 
(3275.2999999998133, 75.89) [b] 
(3275.249999999813, 75.93) [b] 
(3275.149999999813, 75.94) [b] 
(3275.099999999813, 75.96) [b] 
(3274.949999999813, 76.03) [b] 
(3274.549999999813, 76.04) [b] 
(3274.4999999998126, 76.05) [b] 
(3274.4499999998125, 76.15) [b] 
(3274.3499999998126, 76.19) [b] 
(3273.9999999998126, 76.25) [b] 
(3273.7499999998126, 76.27) [b] 
(3273.6999999998125, 76.28) [b] 
(3273.6499999998123, 76.29) [b] 
(3273.599999999812, 76.31) [b] 
(3273.549999999812, 76.35) [b] 
(3273.4999999998117, 76.4) [b] 
(3273.1999999998116, 76.41) [b] 
(3273.1499999998114, 76.42) [b] 
(3273.099999999811, 76.47) [b] 
(3273.049999999811, 76.5) [b] 
(3272.999999999811, 76.51) [b] 
(3272.749999999811, 76.52) [b] 
(3272.549999999811, 76.58) [b] 
(3272.499999999811, 76.67) [b] 
(3272.4499999998106, 76.71) [b] 
(3272.3999999998105, 76.73) [b] 
(3272.2499999998104, 76.77) [b] 
(3272.19999999981, 76.81) [b] 
(3272.04999999981, 76.85) [b] 
(3271.99999999981, 76.89) [b] 
(3271.9499999998097, 77) [b] 
(3271.8499999998094, 77.01) [b] 
(3271.799999999809, 77.05) [b] 
(3271.6999999998093, 77.06) [b] 
(3271.649999999809, 77.07) [b] 
(3271.599999999809, 77.09) [b] 
(3271.5499999998087, 77.1) [b] 
(3271.3999999998086, 77.12) [b] 
(3271.2499999998086, 77.16) [b] 
(3271.149999999808, 77.21) [b] 
(3271.099999999808, 77.24) [b] 
(3270.999999999808, 77.27) [b] 
(3270.949999999808, 77.33) [b] 
(3270.749999999808, 77.37) [b] 
(3270.499999999808, 77.39) [b] 
(3270.449999999808, 77.46) [b] 
(3270.3499999998076, 77.53) [b] 
(3270.1999999998075, 77.55) [b] 
(3270.0499999998074, 77.6) [b] 
(3269.9499999998075, 77.62) [b] 
(3269.8499999998076, 77.74) [b] 
(3269.7499999998076, 77.78) [b] 
(3269.499999999807, 77.82) [b] 
(3269.3999999998073, 77.83) [b] 
(3269.2999999998074, 77.84) [b] 
(3269.199999999807, 77.87) [b] 
(3269.099999999807, 77.88) [b] 
(3268.999999999807, 77.89) [b] 
(3268.949999999807, 78.09) [b] 
(3268.899999999807, 78.12) [b] 
(3268.6999999998066, 78.14) [b] 
(3268.4499999998066, 78.15) [b] 
(3268.3999999998064, 78.17) [b] 
(3268.349999999806, 78.2) [b] 
(3268.299999999806, 78.22) [b] 
(3268.199999999806, 78.25) [b] 
(3268.149999999806, 78.34) [b] 
(3268.0999999998057, 78.35) [b] 
(3267.999999999806, 78.4) [b] 
(3267.899999999806, 78.41) [b] 
(3267.699999999806, 78.42) [b] 
(3267.649999999806, 78.48) [b] 
(3267.5499999998056, 78.49) [b] 
(3267.3999999998055, 78.51) [b] 
(3267.3499999998053, 78.59) [b] 
(3267.299999999805, 78.68) [b] 
(3267.199999999805, 78.7) [b] 
(3266.9999999998054, 78.71) [b] 
(3266.949999999805, 78.74) [b] 
(3266.7499999998054, 78.75) [b] 
(3266.699999999805, 78.79) [b] 
(3266.649999999805, 78.82) [b] 
(3266.599999999805, 78.84) [b] 
(3266.5499999998046, 78.87) [b] 
(3266.4499999998043, 78.92) [b] 
(3266.399999999804, 78.93) [b] 
(3266.0499999998037, 78.96) [b] 
(3265.9499999998034, 79.02) [b] 
(3265.899999999803, 79.06) [b] 
(3265.849999999803, 79.13) [b] 
(3265.749999999803, 79.2) [b] 
(3265.699999999803, 79.22) [b] 
(3265.6499999998027, 79.24) [b] 
(3265.5999999998025, 79.26) [b] 
(3265.4999999998026, 79.27) [b] 
(3265.3999999998023, 79.3) [b] 
(3265.2999999998024, 79.34) [b] 
(3265.0499999998024, 79.4) [b] 
(3264.999999999802, 79.41) [b] 
(3264.7999999998024, 79.43) [b] 
(3264.6499999998023, 79.48) [b] 
(3264.5499999998024, 79.51) [b] 
(3264.499999999802, 79.54) [b] 
(3264.3999999998023, 79.57) [b] 
(3264.249999999802, 79.59) [b] 
(3264.0499999998024, 79.6) [b] 
(3263.999999999802, 79.66) [b] 
(3263.949999999802, 79.67) [b] 
(3263.899999999802, 79.68) [b] 
(3263.8499999998016, 79.74) [b] 
(3263.7499999998017, 79.79) [b] 
(3263.6999999998015, 79.85) [b] 
(3263.549999999801, 79.87) [b] 
(3263.499999999801, 79.88) [b] 
(3263.4499999998006, 79.91) [b] 
(3263.3999999998005, 79.94) [b] 
(3263.2499999998004, 79.97) [b] 
(3263.1999999998, 80.04) [b] 
(3263.1499999998, 80.05) [b] 
(3263.0999999998, 80.09) [b] 
(3262.9999999998, 80.14) [b] 
(3262.8499999998, 80.18) [b] 
(3262.7999999997996, 80.2) [b] 
(3262.6999999997993, 80.21) [b] 
(3262.5999999997994, 80.3) [b] 
(3262.549999999799, 80.33) [b] 
(3262.499999999799, 80.36) [b] 
(3262.349999999799, 80.42) [b] 
(3262.199999999799, 80.43) [b] 
(3261.849999999799, 80.45) [b] 
(3261.7999999997987, 80.48) [b] 
(3261.699999999799, 80.54) [b] 
(3261.6499999997986, 80.58) [b] 
(3261.5499999997987, 80.59) [b] 
(3261.4999999997985, 80.6) [b] 
(3261.4499999997984, 80.63) [b] 
(3261.3499999997985, 80.65) [b] 
(3261.2999999997983, 80.66) [b] 
(3261.249999999798, 80.67) [b] 
(3260.999999999798, 80.79) [b] 
(3260.949999999798, 80.83) [b] 
(3260.8999999997977, 80.87) [b] 
(3260.8499999997975, 80.91) [b] 
(3260.7999999997974, 80.92) [b] 
(3260.749999999797, 80.93) [b] 
(3260.599999999797, 80.94) [b] 
(3260.4999999997967, 80.98) [b] 
(3260.2999999997965, 81.01) [b] 
(3260.2499999997963, 81.04) [b] 
(3260.1499999997964, 81.06) [b] 
(3259.2499999997963, 81.16) [b] 
(3259.199999999796, 81.18) [b] 
(3259.0999999997957, 81.23) [b] 
(3259.0499999997955, 81.35) [b] 
(3258.9999999997954, 81.38) [b] 
(3258.949999999795, 81.41) [b] 
(3258.899999999795, 81.43) [b] 
(3258.799999999795, 81.45) [b] 
(3258.749999999795, 81.46) [b] 
(3258.599999999795, 81.48) [b] 
(3258.5499999997946, 81.49) [b] 
(3258.4999999997945, 81.5) [b] 
(3258.399999999794, 81.55) [b] 
(3258.349999999794, 81.59) [b] 
(3258.099999999794, 81.6) [b] 
(3257.999999999794, 81.61) [b] 
(3257.949999999794, 81.68) [b] 
(3257.8499999997935, 81.75) [b] 
(3257.6499999997936, 81.81) [b] 
(3257.5499999997937, 81.83) [b] 
(3257.2999999997937, 81.95) [b] 
(3257.2499999997935, 81.96) [b] 
(3257.1499999997936, 81.99) [b] 
(3257.0999999997935, 82) [b] 
(3257.0499999997933, 82.04) [b] 
(3256.999999999793, 82.05) [b] 
(3256.949999999793, 82.08) [b] 
(3256.8999999997927, 82.1) [b] 
(3256.8499999997925, 82.11) [b] 
(3256.7499999997926, 82.16) [b] 
(3256.5999999997925, 82.2) [b] 
(3256.5499999997924, 82.22) [b] 
(3256.499999999792, 82.23) [b] 
(3256.449999999792, 82.26) [b] 
(3256.349999999792, 82.29) [b] 
(3256.199999999792, 82.31) [b] 
(3256.149999999792, 82.37) [b] 
(3256.0999999997916, 82.42) [b] 
(3256.0499999997915, 82.43) [b] 
(3255.8999999997914, 82.46) [b] 
(3255.849999999791, 82.49) [b] 
(3255.7499999997913, 82.5) [b] 
(3255.599999999791, 82.54) [b] 
(3255.549999999791, 82.6) [b] 
(3255.449999999791, 82.61) [b] 
(3255.399999999791, 82.7) [b] 
(3255.299999999791, 82.72) [b] 
(3255.199999999791, 82.78) [b] 
(3255.099999999791, 82.8) [b] 
(3254.949999999791, 82.81) [b] 
(3254.799999999791, 82.86) [b] 
(3254.749999999791, 82.9) [b] 
(3254.6999999997906, 82.92) [b] 
(3254.6499999997905, 82.97) [b] 
(3254.4999999997904, 82.98) [b] 
(3254.3999999997905, 83) [b] 
(3254.3499999997903, 83.01) [b] 
(3254.29999999979, 83.06) [b] 
(3254.24999999979, 83.07) [b] 
(3254.09999999979, 83.08) [b] 
(3253.9499999997897, 83.11) [b] 
(3253.8999999997895, 83.13) [b] 
(3253.8499999997894, 83.17) [b] 
(3253.799999999789, 83.21) [b] 
(3253.649999999789, 83.27) [b] 
(3253.599999999789, 83.3) [b] 
(3253.499999999789, 83.32) [b] 
(3253.449999999789, 83.35) [b] 
(3253.3999999997886, 83.36) [b] 
(3253.3499999997885, 83.37) [b] 
(3253.2999999997883, 83.38) [b] 
(3253.1999999997884, 83.42) [b] 
(3253.149999999788, 83.44) [b] 
(3253.099999999788, 83.45) [b] 
(3252.899999999788, 83.49) [b] 
(3252.7999999997883, 83.51) [b] 
(3252.6999999997884, 83.53) [b] 
(3252.5999999997885, 83.61) [b] 
(3252.5499999997883, 83.65) [b] 
(3252.399999999788, 83.68) [b] 
(3252.349999999788, 83.71) [b] 
(3252.299999999788, 83.73) [b] 
(3252.2499999997876, 83.75) [b] 
(3252.049999999788, 83.81) [b] 
(3251.9999999997876, 83.85) [b] 
(3251.9499999997875, 83.88) [b] 
(3251.8999999997873, 83.92) [b] 
(3251.849999999787, 83.94) [b] 
(3251.799999999787, 83.95) [b] 
(3251.7499999997867, 83.97) [b] 
(3251.6999999997865, 83.98) [b] 
(3251.5999999997866, 84.01) [b] 
(3251.399999999787, 84.03) [b] 
(3251.3499999997866, 84.04) [b] 
(3251.0999999997866, 84.06) [b] 
(3250.9999999997863, 84.07) [b] 
(3250.949999999786, 84.08) [b] 
(3250.899999999786, 84.1) [b] 
(3250.8499999997857, 84.14) [b] 
(3250.7999999997855, 84.28) [b] 
(3250.5999999997857, 84.29) [b] 
(3250.5499999997855, 84.3) [b] 
(3250.4499999997856, 84.36) [b] 
(3250.3999999997855, 84.44) [b] 
(3250.2999999997855, 84.46) [b] 
(3250.0999999997857, 84.48) [b] 
(3249.999999999786, 84.49) [b] 
(3249.8499999997857, 84.57) [b] 
(3249.7999999997855, 84.58) [b] 
(3249.7499999997854, 84.6) [b] 
(3249.6499999997855, 84.61) [b] 
(3249.5999999997853, 84.62) [b] 
(3249.399999999785, 84.64) [b] 
(3249.349999999785, 84.65) [b] 
(3249.249999999785, 84.74) [b] 
(3249.1999999997847, 84.77) [b] 
(3249.1499999997845, 84.81) [b] 
(3248.9999999997844, 84.86) [b] 
(3248.8999999997845, 84.88) [b] 
(3248.7999999997846, 84.9) [b] 
(3248.7499999997844, 84.92) [b] 
(3248.6999999997843, 84.97) [b] 
(3248.649999999784, 85) [b] 
(3248.599999999784, 85.02) [b] 
(3248.5499999997837, 85.05) [b] 
(3248.4999999997835, 85.08) [b] 
(3248.4499999997834, 85.15) [b] 
(3248.149999999783, 85.17) [b] 
(3248.099999999783, 85.18) [b] 
(3247.899999999783, 85.25) [b] 
(3247.7999999997833, 85.26) [b] 
(3247.749999999783, 85.29) [b] 
(3247.649999999783, 85.3) [b] 
(3247.599999999783, 85.33) [b] 
(3247.549999999783, 85.36) [b] 
(3247.3999999997827, 85.47) [b] 
(3247.199999999783, 85.52) [b] 
(3247.1499999997827, 85.54) [b] 
(3247.0999999997825, 85.55) [b] 
(3247.0499999997824, 85.57) [b] 
(3246.9499999997824, 85.59) [b] 
(3246.8499999997825, 85.63) [b] 
(3246.6999999997824, 85.65) [b] 
(3246.6499999997823, 85.66) [b] 
(3246.5499999997824, 85.7) [b] 
(3246.499999999782, 85.71) [b] 
(3246.449999999782, 85.74) [b] 
(3246.399999999782, 85.8) [b] 
(3246.3499999997816, 85.84) [b] 
(3246.2999999997814, 85.87) [b] 
(3246.0499999997814, 85.89) [b] 
(3245.9999999997813, 85.91) [b] 
(3245.8999999997814, 85.97) [b] 
(3245.849999999781, 85.99) [b] 
(3245.549999999781, 86) [b] 
(3245.499999999781, 86.06) [b] 
(3245.299999999781, 86.08) [b] 
(3245.249999999781, 86.11) [b] 
(3245.1999999997806, 86.19) [b] 
(3245.1499999997804, 86.2) [b] 
(3245.0999999997803, 86.21) [b] 
(3245.04999999978, 86.23) [b] 
(3244.99999999978, 86.27) [b] 
(3244.9499999997797, 86.29) [b] 
(3244.6499999997795, 86.36) [b] 
(3244.3999999997795, 86.38) [b] 
(3244.3499999997794, 86.47) [b] 
(3244.299999999779, 86.49) [b] 
(3244.1999999997793, 86.52) [b] 
(3244.149999999779, 86.55) [b] 
(3244.0499999997787, 86.57) [b] 
(3243.9999999997785, 86.58) [b] 
(3243.9499999997784, 86.6) [b] 
(3243.899999999778, 86.62) [b] 
(3243.649999999778, 86.63) [b] 
(3243.599999999778, 86.67) [b] 
(3243.549999999778, 86.71) [b] 
(3243.3499999997775, 86.75) [b] 
(3243.2999999997774, 86.76) [b] 
(3243.249999999777, 86.81) [b] 
(3243.1499999997773, 86.85) [b] 
(3242.999999999777, 86.86) [b] 
(3242.849999999777, 86.96) [b] 
(3242.699999999777, 86.99) [b] 
(3242.649999999777, 87.05) [b] 
(3242.549999999777, 87.11) [b] 
(3242.449999999777, 87.13) [b] 
(3242.399999999777, 87.14) [b] 
(3242.3499999997766, 87.17) [b] 
(3242.2999999997764, 87.18) [b] 
(3242.2499999997763, 87.19) [b] 
(3242.199999999776, 87.21) [b] 
(3241.949999999776, 87.22) [b] 
(3241.899999999776, 87.26) [b] 
(3241.799999999776, 87.3) [b] 
(3241.699999999776, 87.31) [b] 
(3241.599999999776, 87.35) [b] 
(3241.549999999776, 87.36) [b] 
(3241.499999999776, 87.44) [b] 
(3241.4499999997756, 87.46) [b] 
(3241.3999999997754, 87.47) [b] 
(3241.3499999997753, 87.48) [b] 
(3241.299999999775, 87.5) [b] 
(3241.249999999775, 87.56) [b] 
(3240.999999999775, 87.64) [b] 
(3240.9499999997747, 87.65) [b] 
(3240.849999999775, 87.67) [b] 
(3240.7999999997746, 87.71) [b] 
(3240.7499999997744, 87.72) [b] 
(3240.6499999997745, 87.73) [b] 
(3240.5999999997744, 87.74) [b] 
(3240.4999999997744, 87.76) [b] 
(3240.3999999997745, 87.81) [b] 
(3240.3499999997744, 87.82) [b] 
(3240.299999999774, 87.83) [b] 
(3240.249999999774, 87.93) [b] 
(3240.199999999774, 87.95) [b] 
(3240.1499999997736, 87.97) [b] 
(3240.0999999997734, 88.03) [b] 
(3240.0499999997733, 88.07) [b] 
(3239.999999999773, 88.09) [b] 
(3239.949999999773, 88.1) [b] 
(3239.8999999997727, 88.12) [b] 
(3239.799999999773, 88.14) [b] 
(3239.7499999997726, 88.17) [b] 
(3239.6999999997724, 88.22) [b] 
(3239.6499999997723, 88.24) [b] 
(3239.599999999772, 88.26) [b] 
(3239.549999999772, 88.28) [b] 
(3239.4999999997717, 88.3) [b] 
(3239.3999999997714, 88.33) [b] 
(3239.2499999997713, 88.35) [b] 
(3239.199999999771, 88.39) [b] 
(3239.149999999771, 88.43) [b] 
(3238.8499999997707, 88.44) [b] 
(3238.749999999771, 88.46) [b] 
(3238.649999999771, 88.56) [b] 
(3238.5999999997707, 88.57) [b] 
(3238.5499999997705, 88.6) [b] 
(3238.4999999997704, 88.69) [b] 
(3238.39999999977, 88.72) [b] 
(3238.34999999977, 88.75) [b] 
(3238.24999999977, 88.77) [b] 
(3238.1999999997697, 88.82) [b] 
(3238.0499999997696, 88.83) [b] 
(3237.9499999997697, 88.85) [b] 
(3237.8999999997695, 88.88) [b] 
(3237.7499999997694, 88.9) [b] 
(3237.6999999997693, 88.94) [b] 
(3237.649999999769, 88.95) [b] 
(3237.599999999769, 88.98) [b] 
(3237.5499999997687, 89) [b] 
(3237.4999999997685, 89.02) [b] 
(3237.4499999997684, 89.11) [b] 
(3237.2999999997683, 89.14) [b] 
(3237.1999999997684, 89.19) [b] 
(3237.149999999768, 89.25) [b] 
(3237.099999999768, 89.27) [b] 
(3237.049999999768, 89.28) [b] 
(3236.9999999997676, 89.31) [b] 
(3236.8999999997677, 89.32) [b] 
(3236.8499999997675, 89.34) [b] 
(3236.7999999997674, 89.38) [b] 
(3236.6999999997674, 89.43) [b] 
(3236.5499999997674, 89.45) [b] 
(3236.499999999767, 89.46) [b] 
(3236.3999999997673, 89.49) [b] 
(3236.349999999767, 89.51) [b] 
(3236.1499999997673, 89.52) [b] 
(3235.999999999767, 89.53) [b] 
(3235.949999999767, 89.56) [b] 
(3235.849999999767, 89.58) [b] 
(3235.799999999767, 89.59) [b] 
(3235.649999999767, 89.69) [b] 
(3235.5999999997666, 89.7) [b] 
(3235.4999999997667, 89.72) [b] 
(3235.3999999997664, 89.74) [b] 
(3235.349999999766, 89.79) [b] 
(3235.299999999766, 89.84) [b] 
(3235.249999999766, 89.88) [b] 
(3235.149999999766, 89.9) [b] 
(3234.9499999997656, 89.99) [b] 
(3234.8499999997657, 90.03) [b] 
(3234.7999999997655, 90.05) [b] 
(3234.7499999997654, 90.15) [b] 
(3234.699999999765, 90.16) [b] 
(3234.399999999765, 90.18) [b] 
(3234.349999999765, 90.19) [b] 
(3234.1999999997647, 90.22) [b] 
(3234.1499999997645, 90.24) [b] 
(3233.799999999764, 90.25) [b] 
(3233.749999999764, 90.27) [b] 
(3233.599999999764, 90.31) [b] 
(3233.5499999997637, 90.39) [b] 
(3233.3999999997636, 90.4) [b] 
(3233.3499999997634, 90.46) [b] 
(3233.2999999997633, 90.5) [b] 
(3233.149999999763, 90.55) [b] 
(3233.099999999763, 90.67) [b] 
(3232.899999999763, 90.68) [b] 
(3232.849999999763, 90.7) [b] 
(3232.799999999763, 90.73) [b] 
(3232.4999999997626, 90.8) [b] 
(3232.3999999997623, 90.86) [b] 
(3232.349999999762, 90.87) [b] 
(3232.299999999762, 90.88) [b] 
(3232.2499999997617, 90.89) [b] 
(3232.1999999997615, 90.9) [b] 
(3232.0999999997616, 90.91) [b] 
(3232.0499999997614, 90.92) [b] 
(3231.9499999997615, 91.04) [b] 
(3231.8999999997613, 91.08) [b] 
(3231.799999999761, 91.14) [b] 
(3231.699999999761, 91.15) [b] 
(3231.549999999761, 91.17) [b] 
(3231.499999999761, 91.23) [b] 
(3231.3499999997607, 91.24) [b] 
(3231.249999999761, 91.27) [b] 
(3231.149999999761, 91.31) [b] 
(3231.0999999997607, 91.4) [b] 
(3230.999999999761, 91.43) [b] 
(3230.9499999997606, 91.44) [b] 
(3230.8999999997604, 91.45) [b] 
(3230.5999999997603, 91.46) [b] 
(3230.44999999976, 91.55) [b] 
(3230.3499999997603, 91.56) [b] 
(3230.29999999976, 91.58) [b] 
(3230.24999999976, 91.61) [b] 
(3229.59999999976, 91.66) [b] 
(3229.5499999997596, 91.7) [b] 
(3229.4999999997594, 91.75) [b] 
(3229.4499999997593, 91.77) [b] 
(3229.3499999997593, 91.79) [b] 
(3229.149999999759, 91.88) [b] 
(3229.099999999759, 91.89) [b] 
(3229.0499999997587, 91.92) [b] 
(3228.9999999997585, 91.93) [b] 
(3228.5499999997587, 91.95) [b] 
(3228.3999999997586, 91.98) [b] 
(3228.3499999997584, 92) [b] 
(3228.2499999997585, 92.02) [b] 
(3228.1499999997586, 92.04) [b] 
(3227.9999999997585, 92.09) [b] 
(3227.9499999997583, 92.11) [b] 
(3227.7999999997583, 92.19) [b] 
(3227.6999999997583, 92.22) [b] 
(3227.649999999758, 92.24) [b] 
(3227.549999999758, 92.29) [b] 
(3227.449999999758, 92.3) [b] 
(3227.3999999997577, 92.32) [b] 
(3227.299999999758, 92.4) [b] 
(3227.2499999997576, 92.42) [b] 
(3227.1999999997574, 92.46) [b] 
(3226.9999999997576, 92.48) [b] 
(3226.8499999997575, 92.55) [b] 
(3226.7999999997573, 92.57) [b] 
(3226.749999999757, 92.59) [b] 
(3226.6499999997573, 92.63) [b] 
(3226.499999999757, 92.66) [b] 
(3226.399999999757, 92.67) [b] 
(3226.299999999757, 92.69) [b] 
(3226.2499999997567, 92.73) [b] 
(3226.149999999757, 92.76) [b] 
(3226.0999999997566, 92.8) [b] 
(3226.0499999997564, 92.81) [b] 
(3225.9999999997563, 92.82) [b] 
(3225.949999999756, 92.84) [b] 
(3225.899999999756, 92.85) [b] 
(3225.799999999756, 92.94) [b] 
(3225.749999999756, 92.96) [b] 
(3225.649999999756, 92.99) [b] 
(3225.549999999756, 93) [b] 
(3225.4499999997556, 93.03) [b] 
(3225.2999999997555, 93.05) [b] 
(3225.1999999997556, 93.12) [b] 
(3224.999999999756, 93.18) [b] 
(3224.899999999756, 93.31) [b] 
(3224.8499999997557, 93.32) [b] 
(3224.6999999997556, 93.33) [b] 
(3224.5499999997555, 93.35) [b] 
(3224.4499999997556, 93.37) [b] 
(3224.3999999997554, 93.38) [b] 
(3224.299999999755, 93.39) [b] 
(3224.0999999997553, 93.48) [b] 
(3223.999999999755, 93.51) [b] 
(3223.8999999997545, 93.53) [b] 
(3223.8499999997543, 93.64) [b] 
(3223.799999999754, 93.65) [b] 
(3223.749999999754, 93.67) [b] 
(3223.699999999754, 93.71) [b] 
(3223.5999999997534, 93.73) [b] 
(3223.5499999997533, 93.83) [b] 
(3223.499999999753, 93.89) [b] 
(3223.449999999753, 93.91) [b] 
(3223.299999999753, 93.92) [b] 
(3223.2499999997526, 93.95) [b] 
(3222.8499999997525, 93.96) [b] 
(3222.7999999997523, 93.97) [b] 
(3222.6499999997523, 94.03) [b] 
(3222.5499999997523, 94.04) [b] 
(3222.499999999752, 94.05) [b] 
(3222.3999999997523, 94.07) [b] 
(3222.349999999752, 94.09) [b] 
(3222.299999999752, 94.17) [b] 
(3222.149999999752, 94.21) [b] 
(3222.0499999997514, 94.22) [b] 
(3221.9999999997513, 94.28) [b] 
(3221.949999999751, 94.3) [b] 
(3221.899999999751, 94.31) [b] 
(3221.749999999751, 94.32) [b] 
(3221.6999999997506, 94.35) [b] 
(3221.6499999997504, 94.38) [b] 
(3221.5999999997503, 94.39) [b] 
(3221.54999999975, 94.44) [b] 
(3221.49999999975, 94.49) [b] 
(3221.39999999975, 94.51) [b] 
(3221.29999999975, 94.54) [b] 
(3221.24999999975, 94.58) [b] 
(3221.1999999997497, 94.63) [b] 
(3221.1499999997495, 94.67) [b] 
(3221.0999999997493, 94.72) [b] 
(3220.9499999997493, 94.74) [b] 
(3220.899999999749, 94.76) [b] 
(3220.849999999749, 94.79) [b] 
(3220.7999999997487, 94.8) [b] 
(3220.449999999749, 94.83) [b] 
(3220.3999999997486, 94.9) [b] 
(3220.3499999997484, 94.93) [b] 
(3220.2999999997483, 94.98) [b] 
(3220.249999999748, 94.99) [b] 
(3219.999999999748, 95) [b] 
(3219.649999999748, 95.03) [b] 
(3219.599999999748, 95.04) [b] 
(3219.549999999748, 95.06) [b] 
(3219.4999999997476, 95.12) [b] 
(3219.3999999997473, 95.14) [b] 
(3219.2999999997473, 95.21) [b] 
(3219.249999999747, 95.25) [b] 
(3219.199999999747, 95.26) [b] 
(3219.149999999747, 95.3) [b] 
(3219.0999999997466, 95.31) [b] 
(3219.0499999997464, 95.4) [b] 
(3218.9999999997463, 95.42) [b] 
(3218.949999999746, 95.43) [b] 
(3218.899999999746, 95.46) [b] 
(3218.6999999997456, 95.51) [b] 
(3218.6499999997454, 95.53) [b] 
(3218.5999999997453, 95.54) [b] 
(3218.4999999997453, 95.57) [b] 
(3218.449999999745, 95.66) [b] 
(3218.3499999997453, 95.67) [b] 
(3218.2499999997453, 95.72) [b] 
(3218.199999999745, 95.77) [b] 
(3218.099999999745, 95.78) [b] 
(3218.0499999997446, 95.85) [b] 
(3217.8999999997445, 95.87) [b] 
(3217.7999999997446, 95.9) [b] 
(3217.7499999997444, 95.91) [b] 
(3217.6999999997443, 95.94) [b] 
(3217.649999999744, 95.97) [b] 
(3217.549999999744, 95.98) [b] 
(3217.3499999997443, 95.99) [b] 
(3217.299999999744, 96.03) [b] 
(3217.249999999744, 96.06) [b] 
(3217.199999999744, 96.09) [b] 
(3217.1499999997436, 96.22) [b] 
(3217.0999999997434, 96.23) [b] 
(3217.0499999997432, 96.26) [b] 
(3216.899999999743, 96.29) [b] 
(3216.7999999997432, 96.34) [b] 
(3216.749999999743, 96.38) [b] 
(3216.699999999743, 96.42) [b] 
(3216.6499999997427, 96.48) [b] 
(3216.5999999997425, 96.49) [b] 
(3216.5499999997423, 96.5) [b] 
(3216.349999999742, 96.54) [b] 
(3216.299999999742, 96.55) [b] 
(3216.2499999997417, 96.6) [b] 
(3216.1499999997413, 96.67) [b] 
(3216.099999999741, 96.71) [b] 
(3215.8999999997413, 96.76) [b] 
(3215.7999999997414, 96.86) [b] 
(3215.649999999741, 96.88) [b] 
(3215.5999999997407, 96.9) [b] 
(3215.5499999997405, 96.92) [b] 
(3215.4999999997403, 96.97) [b] 
(3215.44999999974, 97.04) [b] 
(3215.39999999974, 97.1) [b] 
(3215.34999999974, 97.13) [b] 
(3214.9999999997394, 97.15) [b] 
(3214.8999999997395, 97.16) [b] 
(3214.8499999997393, 97.18) [b] 
(3214.7499999997394, 97.23) [b] 
(3214.4999999997394, 97.24) [b] 
(3214.3999999997395, 97.25) [b] 
(3214.2999999997396, 97.27) [b] 
(3214.2499999997394, 97.32) [b] 
(3214.1999999997392, 97.33) [b] 
(3214.149999999739, 97.39) [b] 
(3214.049999999739, 97.41) [b] 
(3213.999999999739, 97.5) [b] 
(3213.849999999739, 97.51) [b] 
(3213.7499999997385, 97.57) [b] 
(3213.6999999997383, 97.66) [b] 
(3213.5999999997384, 97.67) [b] 
(3213.5499999997382, 97.69) [b] 
(3213.499999999738, 97.73) [b] 
(3213.399999999738, 97.74) [b] 
(3213.349999999738, 97.75) [b] 
(3213.299999999738, 97.8) [b] 
(3213.2499999997376, 97.81) [b] 
(3213.1999999997374, 97.82) [b] 
(3213.1499999997372, 97.86) [b] 
(3212.999999999737, 97.88) [b] 
(3212.949999999737, 97.91) [b] 
(3212.849999999737, 97.94) [b] 
(3212.799999999737, 97.96) [b] 
(3212.7499999997367, 97.99) [b] 
(3212.549999999737, 98.02) [b] 
(3212.4999999997367, 98.08) [b] 
(3212.4499999997365, 98.09) [b] 
(3212.3999999997363, 98.11) [b] 
(3212.2499999997362, 98.12) [b] 
(3212.199999999736, 98.22) [b] 
(3212.099999999736, 98.24) [b] 
(3212.049999999736, 98.31) [b] 
(3211.899999999736, 98.34) [b] 
(3211.799999999736, 98.37) [b] 
(3211.749999999736, 98.4) [b] 
(3211.6999999997356, 98.41) [b] 
(3211.6499999997354, 98.42) [b] 
(3211.4999999997353, 98.43) [b] 
(3211.3499999997352, 98.45) [b] 
(3211.299999999735, 98.48) [b] 
(3211.199999999735, 98.61) [b] 
(3211.149999999735, 98.63) [b] 
(3211.049999999735, 98.71) [b] 
(3210.949999999735, 98.73) [b] 
(3210.7999999997346, 98.74) [b] 
(3210.7499999997344, 98.75) [b] 
(3210.649999999734, 98.78) [b] 
(3210.599999999734, 98.82) [b] 
(3210.349999999734, 98.85) [b] 
(3210.249999999734, 98.93) [b] 
(3210.199999999734, 98.96) [b] 
(3210.1499999997336, 99) [b] 
(3209.9999999997335, 99.03) [b] 
(3209.9499999997333, 99.06) [b] 
(3209.899999999733, 99.08) [b] 
(3209.749999999733, 99.12) [b] 
(3209.599999999733, 99.15) [b] 
(3209.399999999733, 99.16) [b] 
(3209.349999999733, 99.2) [b] 
(3209.299999999733, 99.21) [b] 
(3209.2499999997326, 99.25) [b] 
(3209.1999999997324, 99.32) [b] 
(3209.0499999997323, 99.34) [b] 
(3208.999999999732, 99.37) [b] 
(3208.8999999997322, 99.42) [b] 
(3208.849999999732, 99.43) [b] 
(3208.799999999732, 99.46) [b] 
(3208.699999999732, 99.49) [b] 
(3208.649999999732, 99.52) [b] 
(3208.549999999732, 99.58) [b] 
(3208.399999999732, 99.61) [b] 
(3208.2499999997317, 99.65) [b] 
(3208.149999999732, 99.66) [b] 
(3207.949999999732, 99.68) [b] 
(3207.899999999732, 99.69) [b] 
(3207.6999999997315, 99.73) [b] 
(3207.6499999997313, 99.78) [b] 
(3207.5499999997314, 99.79) [b] 
(3207.4999999997312, 99.85) [b] 
(3207.3999999997313, 99.92) [b] 
(3207.349999999731, 99.93) [b] 
(3207.199999999731, 99.99) [b] 
(3207.149999999731, 100.1) [b] 
(3206.8499999997307, 100.2) [b] 
(3206.4999999997303, 100.3) [b] 
(3206.1999999997297, 100.4) [b] 
(3205.899999999729, 100.5) [b] 
(3205.6499999997286, 100.6) [b] 
(3205.3499999997284, 100.7) [b] 
(3205.199999999728, 100.8) [b] 
(3204.9999999997276, 100.9) [b] 
(3204.299999999727, 101.1) [b] 
(3203.899999999726, 101.2) [b] 
(3203.749999999726, 101.3) [b] 
(3203.5999999997257, 101.4) [b] 
(3203.3999999997254, 101.5) [b] 
(3203.299999999725, 101.6) [b] 
(3202.8499999997243, 101.7) [b] 
(3202.749999999724, 101.8) [b] 
(3202.4999999997235, 101.9) [b] 
(3202.149999999723, 102) [b] 
(3201.949999999723, 102.1) [b] 
(3201.6499999997222, 102.2) [b] 
(3201.4999999997217, 102.3) [b] 
(3201.2499999997212, 102.4) [b] 
(3201.199999999721, 102.5) [b] 
(3201.049999999721, 102.6) [b] 
(3200.44999999972, 102.7) [b] 
(3200.1499999997204, 102.8) [b] 
(3199.89999999972, 102.9) [b] 
(3199.4999999997194, 103) [b] 
(3199.4499999997192, 103.1) [b] 
(3199.0499999997187, 103.2) [b] 
(3198.8499999997184, 103.3) [b] 
(3198.4999999997176, 103.4) [b] 
(3198.2999999997173, 103.5) [b] 
(3198.049999999717, 103.6) [b] 
(3197.2499999997167, 103.7) [b] 
(3197.0499999997164, 103.8) [b] 
(3196.9999999997162, 103.9) [b] 
(3196.7499999997162, 104) [b] 
(3196.6499999997163, 104.1) [b] 
(3196.1999999997156, 104.2) [b] 
(3195.8499999997152, 104.3) [b] 
(3195.3999999997145, 104.4) [b] 
(3195.3499999997143, 104.5) [b] 
(3195.199999999714, 104.6) [b] 
(3194.9999999997135, 104.7) [b] 
(3194.749999999713, 104.8) [b] 
(3194.449999999713, 104.9) [b] 
(3194.3499999997125, 105) [b] 
(3194.199999999712, 105.1) [b] 
(3193.4499999997115, 105.2) [b] 
(3193.2499999997112, 105.3) [b] 
(3193.149999999711, 105.4) [b] 
(3192.7999999997105, 105.5) [b] 
(3192.64999999971, 105.6) [b] 
(3192.4499999997097, 105.7) [b] 
(3191.899999999709, 105.8) [b] 
(3191.6499999997086, 105.9) [b] 
(3191.449999999709, 106) [b] 
(3191.0499999997082, 106.1) [b] 
(3190.6999999997074, 106.2) [b] 
(3190.3499999997075, 106.3) [b] 
(3190.1499999997072, 106.4) [b] 
(3189.9999999997067, 106.6) [b] 
(3189.6999999997065, 106.7) [b] 
(3189.0999999997057, 106.8) [b] 
(3188.6499999997054, 106.9) [b] 
(3188.4999999997053, 107) [b] 
(3188.449999999705, 107.1) [b] 
(3188.249999999705, 107.2) [b] 
(3187.9499999997042, 107.3) [b] 
(3187.5999999997034, 107.4) [b] 
(3187.249999999703, 107.5) [b] 
(3187.0999999997025, 107.6) [b] 
(3186.849999999702, 107.7) [b] 
(3186.6499999997022, 107.8) [b] 
(3186.599999999702, 107.9) [b] 
(3186.249999999701, 108) [b] 
(3185.9499999997006, 108.1) [b] 
(3185.3499999997002, 108.2) [b] 
(3185.2999999997, 108.3) [b] 
(3185.1499999996995, 108.4) [b] 
(3184.9499999996992, 108.5) [b] 
(3184.5999999996984, 108.6) [b] 
(3184.0499999996982, 108.7) [b] 
(3183.999999999698, 108.8) [b] 
(3183.7999999996973, 108.9) [b] 
(3183.649999999697, 109) [b] 
(3183.199999999696, 109.1) [b] 
(3182.9999999996953, 109.2) [b] 
(3182.7499999996953, 109.3) [b] 
(3182.4999999996953, 109.4) [b] 
(3182.149999999695, 109.5) [b] 
(3181.4999999996944, 109.7) [b] 
(3180.999999999694, 109.8) [b] 
(3180.7999999996937, 109.9) [b] 
(3180.499999999693, 110) [b] 
(3180.349999999693, 110.1) [b] 
(3180.2499999996926, 110.2) [b] 
(3179.8999999996927, 110.3) [b] 
(3179.399999999692, 110.4) [b] 
(3179.2499999996917, 110.5) [b] 
(3178.999999999691, 110.6) [b] 
(3178.899999999691, 110.7) [b] 
(3178.7999999996905, 110.8) [b] 
(3178.54999999969, 110.9) [b] 
(3178.1999999996897, 111) [b] 
(3178.049999999689, 111.1) [b] 
(3177.8499999996884, 111.2) [b] 
(3177.549999999688, 111.3) [b] 
(3177.2999999996878, 111.4) [b] 
(3177.0499999996873, 111.5) [b] 
(3176.749999999687, 111.6) [b] 
(3175.9499999996865, 111.7) [b] 
(3175.3999999996863, 111.9) [b] 
(3175.1999999996856, 112) [b] 
(3174.699999999685, 112.1) [b] 
(3174.1999999996847, 112.2) [b] 
(3174.1499999996845, 112.3) [b] 
(3173.899999999684, 112.5) [b] 
(3173.6999999996833, 112.6) [b] 
(3173.2999999996828, 112.7) [b] 
(3173.149999999682, 112.8) [b] 
(3172.799999999682, 112.9) [b] 
(3172.5999999996816, 113) [b] 
(3172.2999999996814, 113.1) [b] 
(3172.199999999681, 113.2) [b] 
(3172.049999999681, 113.3) [b] 
(3171.84999999968, 113.4) [b] 
(3171.4999999996794, 113.5) [b] 
(3171.299999999679, 113.6) [b] 
(3170.6999999996783, 113.7) [b] 
(3170.4999999996785, 113.8) [b] 
(3170.4499999996783, 113.9) [b] 
(3169.9999999996776, 114) [b] 
(3169.7999999996773, 114.1) [b] 
(3169.5999999996775, 114.2) [b] 
(3169.2499999996767, 114.3) [b] 
(3168.9999999996767, 114.4) [b] 
(3168.649999999676, 114.5) [b] 
(3168.4499999996756, 114.6) [b] 
(3167.849999999675, 114.7) [b] 
(3167.6999999996747, 114.8) [b] 
(3167.2999999996746, 114.9) [b] 
(3167.099999999674, 115) [b] 
(3166.849999999674, 115.1) [b] 
(3166.599999999674, 115.2) [b] 
(3166.049999999673, 115.3) [b] 
(3165.7499999996726, 115.4) [b] 
(3165.4499999996724, 115.5) [b] 
(3165.149999999672, 115.6) [b] 
(3164.799999999672, 115.7) [b] 
(3164.6999999996715, 115.8) [b] 
(3164.1499999996713, 115.9) [b] 
(3163.849999999671, 116) [b] 
(3163.3499999996707, 116.1) [b] 
(3163.1499999996704, 116.2) [b] 
(3162.6999999996697, 116.3) [b] 
(3162.5999999996698, 116.4) [b] 
(3162.2499999996694, 116.5) [b] 
(3161.8999999996695, 116.6) [b] 
(3161.549999999669, 116.7) [b] 
(3161.349999999669, 116.8) [b] 
(3160.999999999669, 116.9) [b] 
(3160.4999999996685, 117) [b] 
(3160.249999999668, 117.1) [b] 
(3160.099999999668, 117.2) [b] 
(3159.5999999996675, 117.3) [b] 
(3159.2999999996678, 117.4) [b] 
(3159.0999999996675, 117.5) [b] 
(3158.899999999667, 117.6) [b] 
(3158.4999999996658, 117.7) [b] 
(3158.2999999996655, 117.8) [b] 
(3158.099999999665, 117.9) [b] 
(3157.699999999664, 118) [b] 
(3157.549999999664, 118.1) [b] 
(3157.3499999996634, 118.2) [b] 
(3157.1999999996633, 118.3) [b] 
(3157.0499999996628, 118.4) [b] 
(3156.7999999996623, 118.5) [b] 
(3156.549999999662, 118.6) [b] 
(3155.799999999662, 118.7) [b] 
(3155.6499999996618, 118.8) [b] 
(3155.449999999661, 118.9) [b] 
(3155.149999999661, 119) [b] 
(3154.849999999661, 119.1) [b] 
(3154.649999999661, 119.2) [b] 
(3154.24999999966, 119.3) [b] 
(3154.0999999996593, 119.4) [b] 
(3153.749999999659, 119.5) [b] 
(3153.6499999996586, 119.6) [b] 
(3153.3499999996584, 119.7) [b] 
(3153.149999999658, 119.8) [b] 
(3152.9499999996574, 119.9) [b] 
(3152.349999999657, 120) [b] 
(3152.099999999656, 120.1) [b] 
(3151.9999999996558, 120.3) [b] 
(3151.6999999996556, 120.4) [b] 
(3151.299999999655, 120.5) [b] 
(3150.9499999996547, 120.6) [b] 
(3150.8999999996545, 120.7) [b] 
(3150.599999999654, 120.8) [b] 
(3150.3999999996536, 120.9) [b] 
(3149.749999999653, 121) [b] 
(3149.599999999653, 121.1) [b] 
(3149.249999999652, 121.2) [b] 
(3149.1499999996518, 121.3) [b] 
(3148.9499999996515, 121.4) [b] 
(3148.7999999996514, 121.5) [b] 
(3148.4499999996506, 121.6) [b] 
(3148.24999999965, 121.7) [b] 
(3147.899999999649, 121.8) [b] 
(3147.799999999649, 121.9) [b] 
(3147.349999999649, 122) [b] 
(3147.2999999996487, 122.1) [b] 
(3147.0999999996484, 122.2) [b] 
(3146.949999999648, 122.3) [b] 
(3146.5999999996475, 122.4) [b] 
(3146.149999999647, 122.5) [b] 
(3146.099999999647, 122.6) [b] 
(3145.749999999646, 122.7) [b] 
(3145.549999999646, 122.8) [b] 
(3144.9999999996453, 123) [b] 
(3144.7999999996446, 123.1) [b] 
(3144.699999999644, 123.2) [b] 
(3144.5499999996437, 123.3) [b] 
(3144.299999999643, 123.4) [b] 
(3143.949999999643, 123.5) [b] 
(3143.7999999996423, 123.6) [b] 
(3143.6499999996418, 123.7) [b] 
(3143.399999999641, 123.8) [b] 
(3143.1499999996404, 123.9) [b] 
(3142.8999999996404, 124) [b] 
(3142.6499999996404, 124.1) [b] 
(3142.54999999964, 124.2) [b] 
(3142.3999999996395, 124.3) [b] 
(3141.9999999996394, 124.4) [b] 
(3141.749999999639, 124.5) [b] 
(3141.599999999639, 124.6) [b] 
(3141.299999999638, 124.7) [b] 
(3141.149999999638, 124.8) [b] 
(3140.8999999996377, 124.9) [b] 
(3140.7499999996376, 125) [b] 
(3140.3999999996367, 125.1) [b] 
(3140.2499999996367, 125.2) [b] 
(3139.949999999636, 125.3) [b] 
(3139.5999999996357, 125.4) [b] 
(3139.2999999996355, 125.5) [b] 
(3138.949999999635, 125.7) [b] 
(3138.749999999635, 125.8) [b] 
(3138.6499999996345, 125.9) [b] 
(3138.4499999996337, 126) [b] 
(3138.2999999996337, 126.1) [b] 
(3137.9999999996335, 126.2) [b] 
(3137.849999999633, 126.3) [b] 
(3137.599999999633, 126.4) [b] 
(3137.3999999996327, 126.5) [b] 
(3137.2999999996323, 126.6) [b] 
(3137.149999999632, 126.7) [b] 
(3136.5999999996316, 126.9) [b] 
(3136.4499999996315, 127) [b] 
(3136.2499999996307, 127.1) [b] 
(3136.1999999996306, 127.2) [b] 
(3135.9499999996306, 127.3) [b] 
(3135.59999999963, 127.4) [b] 
(3135.54999999963, 127.5) [b] 
(3135.14999999963, 127.6) [b] 
(3134.9499999996297, 127.7) [b] 
(3134.7499999996294, 127.8) [b] 
(3134.549999999629, 127.9) [b] 
(3134.3999999996286, 128) [b] 
(3134.149999999628, 128.1) [b] 
(3133.949999999628, 128.2) [b] 
(3133.8499999996275, 128.3) [b] 
(3133.649999999627, 128.4) [b] 
(3133.249999999627, 128.5) [b] 
(3132.8499999996266, 128.6) [b] 
(3132.699999999626, 128.7) [b] 
(3132.649999999626, 128.8) [b] 
(3132.4999999996253, 128.9) [b] 
(3132.449999999625, 129) [b] 
(3132.0999999996247, 129.1) [b] 
(3131.549999999624, 129.2) [b] 
(3131.1999999996237, 129.3) [b] 
(3130.8999999996236, 129.4) [b] 
(3130.649999999623, 129.5) [b] 
(3130.449999999623, 129.6) [b] 
(3130.1999999996224, 129.7) [b] 
(3130.149999999622, 129.8) [b] 
(3129.849999999622, 129.9) [b] 
(3129.799999999622, 130) [b] 
(3129.2999999996214, 130.1) [b] 
(3128.9999999996217, 130.2) [b] 
(3128.4999999996217, 130.3) [b] 
(3128.249999999621, 130.4) [b] 
(3128.149999999621, 130.5) [b] 
(3128.0499999996205, 130.6) [b] 
(3127.7499999996203, 130.7) [b] 
(3127.54999999962, 130.8) [b] 
(3127.1499999996195, 130.9) [b] 
(3126.649999999619, 131.1) [b] 
(3126.5499999996186, 131.2) [b] 
(3126.349999999618, 131.3) [b] 
(3126.2999999996177, 131.4) [b] 
(3125.9999999996176, 131.5) [b] 
(3125.899999999617, 131.6) [b] 
(3125.699999999617, 131.7) [b] 
(3125.249999999616, 131.8) [b] 
(3125.049999999616, 132) [b] 
(3124.6999999996156, 132.1) [b] 
(3124.4499999996156, 132.2) [b] 
(3123.9999999996153, 132.3) [b] 
(3123.849999999615, 132.4) [b] 
(3123.6499999996154, 132.5) [b] 
(3123.549999999615, 132.6) [b] 
(3123.299999999615, 132.7) [b] 
(3122.949999999615, 132.8) [b] 
(3122.549999999614, 132.9) [b] 
(3122.4499999996137, 133) [b] 
(3122.3499999996134, 133.1) [b] 
(3122.299999999613, 133.2) [b] 
(3122.049999999613, 133.3) [b] 
(3121.8499999996125, 133.4) [b] 
(3121.699999999612, 133.5) [b] 
(3121.299999999612, 133.6) [b] 
(3121.049999999612, 133.7) [b] 
(3120.9999999996116, 133.8) [b] 
(3120.5999999996106, 133.9) [b] 
(3120.4499999996106, 134) [b] 
(3120.3999999996104, 134.1) [b] 
(3120.24999999961, 134.2) [b] 
(3119.8499999996093, 134.3) [b] 
(3119.599999999609, 134.4) [b] 
(3119.4499999996083, 134.5) [b] 
(3119.3499999996084, 134.6) [b] 
(3119.199999999608, 134.7) [b] 
(3119.0499999996073, 134.8) [b] 
(3118.6999999996065, 134.9) [b] 
(3118.299999999606, 135.1) [b] 
(3118.0999999996056, 135.2) [b] 
(3117.7499999996053, 135.3) [b] 
(3117.599999999605, 135.4) [b] 
(3117.4499999996046, 135.5) [b] 
(3117.3499999996043, 135.6) [b] 
(3117.1999999996037, 135.7) [b] 
(3116.9999999996035, 135.8) [b] 
(3116.699999999603, 135.9) [b] 
(3116.6499999996026, 136) [b] 
(3116.3499999996025, 136.1) [b] 
(3115.999999999602, 136.2) [b] 
(3115.799999999602, 136.3) [b] 
(3115.6499999996013, 136.4) [b] 
(3115.5499999996014, 136.5) [b] 
(3115.399999999601, 136.6) [b] 
(3115.1999999996005, 136.7) [b] 
(3114.9499999996, 136.8) [b] 
(3114.6499999996, 136.9) [b] 
(3114.5499999995995, 137) [b] 
(3113.9999999995994, 137.2) [b] 
(3113.7999999995986, 137.3) [b] 
(3113.499999999598, 137.4) [b] 
(3113.449999999598, 137.5) [b] 
(3113.1499999995976, 137.6) [b] 
(3113.0499999995973, 137.7) [b] 
(3112.899999999597, 137.8) [b] 
(3112.6499999995967, 137.9) [b] 
(3112.1499999995963, 138) [b] 
(3111.999999999596, 138.1) [b] 
(3111.8499999995956, 138.2) [b] 
(3111.7999999995955, 138.3) [b] 
(3111.599999999595, 138.4) [b] 
(3111.3999999995945, 138.5) [b] 
(3111.1999999995937, 138.6) [b] 
(3110.8499999995934, 138.7) [b] 
(3110.799999999593, 138.9) [b] 
(3110.749999999593, 139) [b] 
(3110.349999999593, 139.1) [b] 
(3109.999999999592, 139.2) [b] 
(3109.8999999995917, 139.3) [b] 
(3109.699999999591, 139.4) [b] 
(3109.549999999591, 139.5) [b] 
(3109.3999999995904, 139.6) [b] 
(3109.19999999959, 139.7) [b] 
(3108.7999999995895, 139.8) [b] 
(3108.649999999589, 140) [b] 
(3108.5499999995886, 140.1) [b] 
(3108.299999999588, 140.2) [b] 
(3108.0499999995877, 140.3) [b] 
(3107.849999999587, 140.4) [b] 
(3107.7499999995866, 140.5) [b] 
(3107.6499999995863, 140.6) [b] 
(3107.549999999586, 140.7) [b] 
(3107.4499999995855, 140.8) [b] 
(3107.2499999995853, 140.9) [b] 
(3106.9499999995846, 141) [b] 
(3106.8999999995845, 141.1) [b] 
(3106.599999999584, 141.2) [b] 
(3106.4499999995837, 141.3) [b] 
(3106.3999999995835, 141.4) [b] 
(3105.949999999583, 141.5) [b] 
(3105.8999999995826, 141.6) [b] 
(3105.649999999582, 141.7) [b] 
(3105.3499999995815, 141.8) [b] 
(3105.2499999995816, 141.9) [b] 
(3104.949999999581, 142) [b] 
(3104.749999999581, 142.1) [b] 
(3104.649999999581, 142.2) [b] 
(3104.4999999995807, 142.3) [b] 
(3104.0499999995805, 142.4) [b] 
(3103.89999999958, 142.5) [b] 
(3103.6499999995794, 142.6) [b] 
(3103.4499999995787, 142.7) [b] 
(3103.2499999995784, 142.8) [b] 
(3102.649999999578, 142.9) [b] 
(3102.549999999578, 143) [b] 
(3102.449999999578, 143.1) [b] 
(3102.1999999995774, 143.2) [b] 
(3102.099999999577, 143.3) [b] 
(3101.8499999995765, 143.4) [b] 
(3101.499999999576, 143.5) [b] 
(3101.449999999576, 143.6) [b] 
(3101.3499999995756, 143.7) [b] 
(3101.099999999575, 143.8) [b] 
(3100.9499999995746, 143.9) [b] 
(3100.7499999995744, 144) [b] 
(3100.599999999574, 144.1) [b] 
(3100.3999999995735, 144.2) [b] 
(3100.299999999573, 144.3) [b] 
(3099.749999999573, 144.4) [b] 
(3099.5999999995724, 144.5) [b] 
(3099.299999999572, 144.6) [b] 
(3099.2499999995716, 144.7) [b] 
(3098.949999999571, 144.8) [b] 
(3098.7499999995703, 144.9) [b] 
(3098.54999999957, 145) [b] 
(3098.39999999957, 145.1) [b] 
(3098.0999999995697, 145.2) [b] 
(3097.9999999995694, 145.3) [b] 
(3097.4499999995687, 145.4) [b] 
(3097.249999999569, 145.6) [b] 
(3097.099999999569, 145.7) [b] 
(3096.5499999995686, 145.8) [b] 
(3096.0499999995686, 145.9) [b] 
(3095.9499999995683, 146) [b] 
(3095.8499999995684, 146.1) [b] 
(3095.499999999568, 146.3) [b] 
(3095.0499999995673, 146.4) [b] 
(3094.6499999995663, 146.5) [b] 
(3094.549999999566, 146.7) [b] 
(3094.3999999995654, 146.8) [b] 
(3094.299999999565, 146.9) [b] 
(3093.549999999564, 147) [b] 
(3093.4499999995637, 147.1) [b] 
(3093.3999999995635, 147.2) [b] 
(3093.199999999563, 147.3) [b] 
(3093.1499999995626, 147.4) [b] 
(3092.949999999562, 147.5) [b] 
(3092.599999999561, 147.6) [b] 
(3092.349999999561, 147.7) [b] 
(3092.299999999561, 147.8) [b] 
(3092.1999999995605, 147.9) [b] 
(3092.04999999956, 148) [b] 
(3091.49999999956, 148.1) [b] 
(3091.2999999995595, 148.2) [b] 
(3091.1499999995594, 148.3) [b] 
(3091.0999999995593, 148.4) [b] 
(3090.9499999995587, 148.5) [b] 
(3090.749999999558, 148.6) [b] 
(3090.4999999995575, 148.7) [b] 
(3090.1999999995573, 148.8) [b] 
(3090.0499999995573, 148.9) [b] 
(3089.8999999995567, 149) [b] 
(3089.599999999556, 149.1) [b] 
(3089.4999999995557, 149.2) [b] 
(3089.2999999995554, 149.3) [b] 
(3089.0999999995556, 149.4) [b] 
(3088.949999999555, 149.5) [b] 
(3088.7999999995545, 149.6) [b] 
(3088.4499999995537, 149.7) [b] 
(3088.3499999995533, 149.8) [b] 
(3088.1999999995533, 149.9) [b] 
(3088.149999999553, 150) [b] 
(3087.8499999995524, 150.1) [b] 
(3087.4999999995516, 150.2) [b] 
(3087.4499999995514, 150.3) [b] 
(3087.3499999995515, 150.4) [b] 
(3087.0499999995513, 150.5) [b] 
(3086.4499999995505, 150.6) [b] 
(3086.09999999955, 150.7) [b] 
(3085.99999999955, 150.8) [b] 
(3085.8499999995497, 150.9) [b] 
(3085.7499999995493, 151) [b] 
(3085.249999999549, 151.1) [b] 
(3085.099999999549, 151.2) [b] 
(3084.5999999995483, 151.3) [b] 
(3084.499999999548, 151.4) [b] 
(3084.249999999548, 151.5) [b] 
(3083.999999999548, 151.6) [b] 
(3083.7499999995475, 151.7) [b] 
(3083.5999999995474, 151.8) [b] 
(3083.449999999547, 151.9) [b] 
(3083.2999999995463, 152) [b] 
(3082.899999999546, 152.1) [b] 
(3082.699999999545, 152.2) [b] 
(3082.449999999545, 152.3) [b] 
(3082.399999999545, 152.4) [b] 
(3082.3499999995447, 152.5) [b] 
(3081.949999999544, 152.6) [b] 
(3081.799999999544, 152.7) [b] 
(3081.4999999995434, 152.8) [b] 
(3081.249999999543, 152.9) [b] 
(3081.199999999543, 153) [b] 
(3081.0999999995424, 153.1) [b] 
(3080.8499999995415, 153.2) [b] 
(3080.6499999995413, 153.3) [b] 
(3080.2499999995407, 153.4) [b] 
(3080.1499999995403, 153.6) [b] 
(3079.7999999995395, 153.7) [b] 
(3079.6999999995396, 153.8) [b] 
(3079.499999999539, 153.9) [b] 
(3079.2999999995386, 154) [b] 
(3078.9499999995382, 154.1) [b] 
(3078.649999999538, 154.2) [b] 
(3078.499999999538, 154.3) [b] 
(3078.299999999538, 154.4) [b] 
(3078.1499999995376, 154.5) [b] 
(3077.999999999537, 154.6) [b] 
(3077.8499999995365, 154.7) [b] 
(3077.649999999536, 154.8) [b] 
(3077.4999999995357, 154.9) [b] 
(3077.349999999535, 155.1) [b] 
(3077.1499999995344, 155.2) [b] 
(3077.049999999534, 155.3) [b] 
(3076.4499999995332, 155.4) [b] 
(3076.149999999533, 155.5) [b] 
(3076.099999999533, 155.6) [b] 
(3075.8999999995326, 155.7) [b] 
(3075.649999999532, 155.8) [b] 
(3075.599999999532, 155.9) [b] 
(3075.349999999531, 156) [b] 
(3075.049999999531, 156.1) [b] 
(3074.8999999995303, 156.2) [b] 
(3074.7499999995302, 156.3) [b] 
(3074.5999999995297, 156.4) [b] 
(3074.4499999995296, 156.5) [b] 
(3074.1499999995294, 156.6) [b] 
(3073.9999999995293, 156.7) [b] 
(3073.749999999529, 156.8) [b] 
(3073.6499999995285, 156.9) [b] 
(3073.449999999528, 157) [b] 
(3072.8499999995274, 157.1) [b] 
(3072.6999999995273, 157.2) [b] 
(3072.549999999527, 157.3) [b] 
(3072.2499999995266, 157.4) [b] 
(3072.0999999995265, 157.5) [b] 
(3071.499999999526, 157.6) [b] 
(3071.099999999526, 157.7) [b] 
(3070.8499999995256, 157.8) [b] 
(3070.7999999995254, 157.9) [b] 
(3070.699999999525, 158) [b] 
(3070.5999999995247, 158.1) [b] 
(3070.3499999995242, 158.2) [b] 
(3070.049999999523, 158.3) [b] 
(3069.949999999523, 158.4) [b] 
(3069.7499999995225, 158.5) [b] 
(3069.5499999995227, 158.6) [b] 
(3069.4499999995223, 158.7) [b] 
(3069.1499999995217, 158.8) [b] 
(3068.799999999521, 158.9) [b] 
(3068.5499999995204, 159) [b] 
(3068.3999999995203, 159.1) [b] 
(3068.29999999952, 159.2) [b] 
(3068.1499999995194, 159.3) [b] 
(3067.9499999995187, 159.4) [b] 
(3067.7999999995186, 159.5) [b] 
(3067.549999999518, 159.6) [b] 
(3067.499999999518, 159.7) [b] 
(3067.0999999995174, 159.8) [b] 
(3066.999999999517, 159.9) [b] 
(3066.699999999517, 160) [b] 
(3066.5499999995163, 160.1) [b] 
(3066.449999999516, 160.2) [b] 
(3066.2999999995154, 160.3) [b] 
(3066.149999999515, 160.4) [b] 
(3066.049999999515, 160.5) [b] 
(3065.499999999514, 160.6) [b] 
(3065.3999999995135, 160.7) [b] 
(3065.149999999513, 160.8) [b] 
(3064.9999999995125, 161) [b] 
(3064.7999999995122, 161.1) [b] 
(3064.4499999995114, 161.2) [b] 
(3064.299999999511, 161.3) [b] 
(3064.1999999995105, 161.4) [b] 
(3063.9999999995102, 161.5) [b] 
(3063.89999999951, 161.6) [b] 
(3063.5999999995092, 161.7) [b] 
(3063.049999999509, 161.8) [b] 
(3062.999999999509, 161.9) [b] 
(3062.749999999509, 162) [b] 
(3062.4499999995082, 162.1) [b] 
(3062.349999999508, 162.2) [b] 
(3062.249999999508, 162.3) [b] 
(3062.049999999508, 162.4) [b] 
(3061.7999999995077, 162.5) [b] 
(3061.449999999507, 162.6) [b] 
(3061.299999999507, 162.7) [b] 
(3061.1499999995062, 162.8) [b] 
(3060.9499999995055, 163) [b] 
(3060.599999999505, 163.1) [b] 
(3060.499999999505, 163.2) [b] 
(3060.299999999504, 163.3) [b] 
(3060.0999999995042, 163.4) [b] 
(3059.949999999504, 163.5) [b] 
(3059.5499999995036, 163.6) [b] 
(3059.1999999995032, 163.7) [b] 
(3059.049999999503, 163.8) [b] 
(3058.8999999995026, 163.9) [b] 
(3058.599999999502, 164) [b] 
(3058.4499999995014, 164.1) [b] 
(3058.1999999995005, 164.2) [b] 
(3058.1499999995003, 164.3) [b] 
(3057.9999999995, 164.4) [b] 
(3057.7999999994995, 164.5) [b] 
(3057.6499999994994, 164.6) [b] 
(3057.449999999499, 164.7) [b] 
(3056.9999999994984, 164.8) [b] 
(3056.5499999994977, 165) [b] 
(3056.3999999994976, 165.1) [b] 
(3056.049999999497, 165.3) [b] 
(3055.9999999994966, 165.4) [b] 
(3055.749999999496, 165.5) [b] 
(3055.399999999496, 165.6) [b] 
(3055.2499999994957, 165.7) [b] 
(3055.099999999495, 165.8) [b] 
(3054.949999999495, 165.9) [b] 
(3054.749999999495, 166) [b] 
(3054.549999999495, 166.1) [b] 
(3054.2999999994945, 166.2) [b] 
(3054.199999999494, 166.3) [b] 
(3053.849999999494, 166.4) [b] 
(3053.6999999994937, 166.5) [b] 
(3053.5999999994933, 166.6) [b] 
(3053.499999999493, 166.7) [b] 
(3053.3999999994926, 166.8) [b] 
(3052.7999999994922, 166.9) [b] 
(3052.699999999492, 167) [b] 
(3052.4999999994916, 167.1) [b] 
(3052.2999999994913, 167.2) [b] 
(3052.0999999994906, 167.3) [b] 
(3052.0499999994904, 167.4) [b] 
(3051.74999999949, 167.5) [b] 
(3051.49999999949, 167.6) [b] 
(3051.3999999994894, 167.7) [b] 
(3051.049999999489, 167.8) [b] 
(3050.899999999489, 167.9) [b] 
(3050.5499999994877, 168) [b] 
(3050.4499999994873, 168.1) [b] 
(3050.349999999487, 168.3) [b] 
(3050.0499999994863, 168.4) [b] 
(3049.749999999485, 168.5) [b] 
(3049.549999999485, 168.6) [b] 
(3049.4999999994848, 168.7) [b] 
(3049.2499999994848, 168.8) [b] 
(3048.899999999484, 168.9) [b] 
(3048.7999999994836, 169) [b] 
(3048.7499999994834, 169.1) [b] 
(3048.499999999483, 169.2) [b] 
(3048.2999999994827, 169.3) [b] 
(3047.8999999994817, 169.4) [b] 
(3047.8499999994815, 169.5) [b] 
(3047.699999999481, 169.6) [b] 
(3047.6499999994808, 169.7) [b] 
(3047.3999999994808, 169.8) [b] 
(3046.8999999994794, 169.9) [b] 
(3046.749999999479, 170.1) [b] 
(3046.6999999994787, 170.2) [b] 
(3046.4999999994784, 170.3) [b] 
(3046.2999999994777, 170.4) [b] 
(3046.099999999477, 170.5) [b] 
(3046.0499999994768, 170.6) [b] 
(3045.7999999994763, 170.7) [b] 
(3045.699999999476, 170.8) [b] 
(3045.449999999475, 170.9) [b] 
(3045.2999999994745, 171) [b] 
(3045.149999999474, 171.1) [b] 
(3045.0999999994738, 171.2) [b] 
(3044.799999999473, 171.3) [b] 
(3044.6499999994726, 171.4) [b] 
(3044.549999999472, 171.5) [b] 
(3044.1499999994717, 171.6) [b] 
(3043.9999999994716, 171.7) [b] 
(3043.899999999471, 171.8) [b] 
(3043.799999999471, 171.9) [b] 
(3043.5499999994704, 172) [b] 
(3043.44999999947, 172.1) [b] 
(3042.9999999994698, 172.2) [b] 
(3042.699999999469, 172.3) [b] 
(3042.649999999469, 172.4) [b] 
(3042.5999999994688, 172.5) [b] 
(3042.4499999994687, 172.6) [b] 
(3042.149999999468, 172.7) [b] 
(3041.6499999994676, 172.8) [b] 
(3041.499999999467, 172.9) [b] 
(3041.349999999467, 173) [b] 
(3041.2999999994668, 173.1) [b] 
(3041.0999999994665, 173.2) [b] 
(3040.8499999994656, 173.3) [b] 
(3040.699999999465, 173.4) [b] 
(3040.499999999465, 173.5) [b] 
(3040.449999999465, 173.6) [b] 
(3040.1499999994644, 173.7) [b] 
(3039.3499999994638, 173.8) [b] 
(3039.0999999994638, 173.9) [b] 
(3038.999999999464, 174.1) [b] 
(3038.649999999463, 174.2) [b] 
(3038.299999999462, 174.3) [b] 
(3038.1499999994617, 174.4) [b] 
(3037.9499999994614, 174.5) [b] 
(3037.899999999461, 174.7) [b] 
(3037.599999999461, 174.8) [b] 
(3037.1999999994605, 174.9) [b] 
(3037.1499999994603, 175) [b] 
(3036.9999999994598, 175.1) [b] 
(3036.849999999459, 175.2) [b] 
(3036.699999999459, 175.3) [b] 
(3036.399999999459, 175.4) [b] 
(3036.1999999994587, 175.5) [b] 
(3036.0999999994583, 175.6) [b] 
(3035.799999999458, 175.7) [b] 
(3035.749999999458, 175.8) [b] 
(3035.4999999994575, 175.9) [b] 
(3035.4499999994573, 176) [b] 
(3035.0499999994568, 176.1) [b] 
(3034.7999999994568, 176.2) [b] 
(3034.649999999456, 176.3) [b] 
(3034.599999999456, 176.4) [b] 
(3034.4499999994555, 176.5) [b] 
(3033.8999999994544, 176.6) [b] 
(3033.6499999994544, 176.7) [b] 
(3033.499999999454, 176.8) [b] 
(3033.4499999994537, 176.9) [b] 
(3033.249999999454, 177) [b] 
(3032.949999999453, 177.1) [b] 
(3032.749999999453, 177.2) [b] 
(3032.4999999994525, 177.3) [b] 
(3032.3499999994524, 177.4) [b] 
(3032.199999999452, 177.5) [b] 
(3032.0999999994515, 177.6) [b] 
(3031.949999999451, 177.7) [b] 
(3031.399999999451, 177.8) [b] 
(3031.299999999451, 177.9) [b] 
(3031.1499999994503, 178) [b] 
(3030.9999999994498, 178.1) [b] 
(3030.849999999449, 178.2) [b] 
(3030.599999999449, 178.3) [b] 
(3030.3499999994488, 178.4) [b] 
(3030.1999999994487, 178.5) [b] 
(3030.049999999448, 178.6) [b] 
(3029.9499999994478, 178.7) [b] 
(3029.499999999447, 178.8) [b] 
(3029.449999999447, 178.9) [b] 
(3029.149999999446, 179) [b] 
(3029.049999999446, 179.1) [b] 
(3028.8999999994453, 179.2) [b] 
(3028.849999999445, 179.3) [b] 
(3028.5499999994445, 179.4) [b] 
(3028.449999999444, 179.5) [b] 
(3028.1999999994437, 179.6) [b] 
(3028.0999999994433, 179.7) [b] 
(3027.999999999443, 179.8) [b] 
(3027.6999999994428, 179.9) [b] 
(3027.4999999994425, 180) [b] 
(3027.1499999994417, 180.1) [b] 
(3026.999999999441, 180.2) [b] 
(3026.949999999441, 180.3) [b] 
(3026.7499999994407, 180.4) [b] 
(3026.6499999994403, 180.5) [b] 
(3026.49999999944, 180.6) [b] 
(3026.1999999994396, 180.7) [b] 
(3025.8999999994385, 180.8) [b] 
(3025.649999999438, 180.9) [b] 
(3025.499999999438, 181) [b] 
(3025.2999999994377, 181.2) [b] 
(3024.8999999994367, 181.3) [b] 
(3024.6499999994367, 181.4) [b] 
(3024.3999999994367, 181.5) [b] 
(3024.3499999994365, 181.6) [b] 
(3024.249999999436, 181.7) [b] 
(3023.8999999994357, 181.8) [b] 
(3023.749999999435, 181.9) [b] 
(3023.649999999435, 182) [b] 
(3023.5499999994345, 182.1) [b] 
(3023.3499999994337, 182.2) [b] 
(3023.199999999433, 182.3) [b] 
(3023.149999999433, 182.4) [b] 
(3022.9499999994323, 182.5) [b] 
(3022.5999999994315, 182.6) [b] 
(3022.5499999994313, 182.7) [b] 
(3022.4499999994314, 182.8) [b] 
(3022.199999999431, 182.9) [b] 
(3021.9999999994307, 183) [b] 
(3021.54999999943, 183.1) [b] 
(3021.4999999994297, 183.2) [b] 
(3021.3999999994294, 183.3) [b] 
(3021.349999999429, 183.4) [b] 
(3021.0499999994286, 183.5) [b] 
(3020.649999999428, 183.6) [b] 
(3020.5499999994277, 183.7) [b] 
(3020.399999999427, 183.8) [b] 
(3020.1999999994264, 184) [b] 
(3019.799999999426, 184.1) [b] 
(3019.7499999994257, 184.2) [b] 
(3019.599999999425, 184.3) [b] 
(3019.4999999994247, 184.4) [b] 
(3019.249999999424, 184.6) [b] 
(3018.9499999994237, 184.7) [b] 
(3018.749999999424, 184.8) [b] 
(3018.549999999423, 184.9) [b] 
(3018.499999999423, 185.1) [b] 
(3018.1499999994226, 185.2) [b] 
(3017.6999999994223, 185.3) [b] 
(3017.4999999994216, 185.4) [b] 
(3017.3499999994215, 185.5) [b] 
(3017.2999999994213, 185.6) [b] 
(3017.049999999421, 185.7) [b] 
(3016.6999999994205, 185.8) [b] 
(3016.4999999994197, 185.9) [b] 
(3016.3999999994194, 186) [b] 
(3016.349999999419, 186.1) [b] 
(3016.299999999419, 186.2) [b] 
(3015.9999999994184, 186.3) [b] 
(3015.699999999418, 186.4) [b] 
(3015.599999999418, 186.5) [b] 
(3015.2999999994177, 186.7) [b] 
(3015.1499999994176, 186.8) [b] 
(3015.049999999417, 186.9) [b] 
(3014.8499999994165, 187) [b] 
(3014.6999999994164, 187.1) [b] 
(3014.499999999416, 187.2) [b] 
(3014.349999999416, 187.3) [b] 
(3014.2499999994156, 187.4) [b] 
(3013.8999999994153, 187.5) [b] 
(3013.7499999994147, 187.6) [b] 
(3013.6499999994144, 187.7) [b] 
(3013.499999999414, 187.8) [b] 
(3013.3999999994135, 187.9) [b] 
(3013.249999999413, 188) [b] 
(3012.9499999994123, 188.1) [b] 
(3012.649999999412, 188.2) [b] 
(3012.4999999994116, 188.3) [b] 
(3012.4499999994114, 188.4) [b] 
(3012.149999999411, 188.5) [b] 
(3012.049999999411, 188.6) [b] 
(3011.79999999941, 188.7) [b] 
(3011.3999999994094, 188.8) [b] 
(3011.349999999409, 188.9) [b] 
(3011.299999999409, 189) [b] 
(3011.1499999994085, 189.1) [b] 
(3010.899999999408, 189.2) [b] 
(3010.6999999994073, 189.3) [b] 
(3010.649999999407, 189.4) [b] 
(3010.4499999994064, 189.5) [b] 
(3010.249999999406, 189.6) [b] 
(3009.9999999994056, 189.7) [b] 
(3009.749999999405, 189.8) [b] 
(3009.549999999405, 189.9) [b] 
(3009.4999999994047, 190) [b] 
(3009.199999999404, 190.1) [b] 
(3009.0499999994036, 190.2) [b] 
(3008.7999999994036, 190.3) [b] 
(3008.2999999994036, 190.4) [b] 
(3008.0499999994026, 190.6) [b] 
(3007.9499999994023, 190.7) [b] 
(3007.899999999402, 190.8) [b] 
(3007.6499999994016, 190.9) [b] 
(3007.5499999994013, 191) [b] 
(3007.2499999994006, 191.1) [b] 
(3007.1999999994005, 191.2) [b] 
(3006.9999999993997, 191.3) [b] 
(3006.7999999993995, 191.5) [b] 
(3006.399999999399, 191.6) [b] 
(3006.2499999993984, 191.7) [b] 
(3005.9999999993984, 191.8) [b] 
(3005.899999999398, 191.9) [b] 
(3005.6999999993977, 192) [b] 
(3005.5999999993974, 192.1) [b] 
(3005.349999999397, 192.2) [b] 
(3005.199999999397, 192.3) [b] 
(3005.0499999993963, 192.4) [b] 
(3004.999999999396, 192.5) [b] 
(3004.449999999396, 192.6) [b] 
(3004.2499999993956, 192.7) [b] 
(3004.0999999993955, 192.8) [b] 
(3004.0499999993954, 192.9) [b] 
(3003.899999999395, 193) [b] 
(3003.5999999993946, 193.1) [b] 
(3003.4499999993945, 193.2) [b] 
(3003.199999999394, 193.3) [b] 
(3003.0999999993937, 193.4) [b] 
(3002.949999999393, 193.5) [b] 
(3002.7999999993926, 193.6) [b] 
(3002.7499999993925, 193.7) [b] 
(3002.649999999392, 193.8) [b] 
(3002.4499999993914, 193.9) [b] 
(3002.1999999993905, 194) [b] 
(3002.0499999993904, 194.1) [b] 
(3001.84999999939, 194.2) [b] 
(3001.7499999993897, 194.3) [b] 
(3001.6499999993894, 194.4) [b] 
(3001.549999999389, 194.5) [b] 
(3001.3999999993885, 194.6) [b] 
(3001.199999999388, 194.7) [b] 
(3001.0499999993876, 194.8) [b] 
(3000.849999999387, 194.9) [b] 
(3000.7999999993867, 195) [b] 
(3000.6499999993866, 195.1) [b] 
(3000.499999999386, 195.2) [b] 
(3000.349999999386, 195.3) [b] 
(3000.0499999993854, 195.4) [b] 
(2999.949999999385, 195.5) [b] 
(2999.799999999385, 195.6) [b] 
(2999.5499999993845, 195.7) [b] 
(2999.4499999993845, 195.8) [b] 
(2999.149999999384, 195.9) [b] 
(2999.0999999993837, 196.1) [b] 
(2999.0499999993835, 196.2) [b] 
(2998.749999999383, 196.3) [b] 
(2998.4499999993827, 196.4) [b] 
(2998.3999999993825, 196.6) [b] 
(2998.3499999993824, 196.7) [b] 
(2998.099999999382, 196.8) [b] 
(2997.8999999993816, 196.9) [b] 
(2997.499999999381, 197) [b] 
(2997.449999999381, 197.1) [b] 
(2997.3499999993805, 197.2) [b] 
(2997.2999999993804, 197.3) [b] 
(2997.0999999993796, 197.4) [b] 
(2997.0499999993795, 197.6) [b] 
(2996.6999999993786, 197.7) [b] 
(2996.5999999993783, 197.8) [b] 
(2996.3999999993775, 197.9) [b] 
(2996.249999999377, 198) [b] 
(2996.199999999377, 198.2) [b] 
(2996.1499999993766, 198.3) [b] 
(2995.8999999993757, 198.4) [b] 
(2995.5999999993746, 198.5) [b] 
(2995.5499999993744, 198.6) [b] 
(2995.399999999374, 198.9) [b] 
(2995.1999999993736, 199) [b] 
(2994.699999999373, 199.1) [b] 
(2994.4999999993734, 199.2) [b] 
(2994.449999999373, 199.3) [b] 
(2994.399999999373, 199.4) [b] 
(2994.1499999993725, 199.5) [b] 
(2994.049999999372, 199.6) [b] 
(2993.949999999372, 199.7) [b] 
(2993.849999999372, 199.8) [b] 
(2993.6499999993716, 199.9) [b] 
(2993.499999999371, 200) [b] 
(2993.299999999371, 200.1) [b] 
(2993.1499999993707, 200.2) [b] 
(2993.0999999993705, 200.3) [b] 
(2992.74999999937, 200.5) [b] 
(2992.2999999993694, 200.6) [b] 
(2992.1999999993695, 200.7) [b] 
(2992.0499999993694, 200.8) [b] 
(2991.9999999993693, 201) [b] 
(2991.899999999369, 201.1) [b] 
(2991.7999999993685, 201.2) [b] 
(2991.3499999993687, 201.3) [b] 
(2991.1999999993686, 201.4) [b] 
(2990.999999999369, 201.5) [b] 
(2990.7999999993685, 201.6) [b] 
(2990.7499999993684, 201.7) [b] 
(2990.6499999993684, 201.8) [b] 
(2990.449999999368, 201.9) [b] 
(2990.399999999368, 202) [b] 
(2990.0999999993683, 202.1) [b] 
(2989.949999999368, 202.2) [b] 
(2989.899999999368, 202.3) [b] 
(2989.849999999368, 202.4) [b] 
(2989.6499999993675, 202.5) [b] 
(2989.5499999993676, 202.6) [b] 
(2989.249999999367, 202.7) [b] 
(2989.199999999367, 202.8) [b] 
(2988.8499999993664, 203.1) [b] 
(2988.749999999366, 203.2) [b] 
(2988.599999999366, 203.3) [b] 
(2988.499999999366, 203.4) [b] 
(2988.3999999993657, 203.5) [b] 
(2988.199999999366, 203.6) [b] 
(2988.0499999993654, 203.7) [b] 
(2987.849999999365, 203.8) [b] 
(2987.799999999365, 203.9) [b] 
(2987.699999999365, 204) [b] 
(2987.5499999993644, 204.1) [b] 
(2987.2999999993644, 204.2) [b] 
(2987.149999999364, 204.3) [b] 
(2987.0999999993637, 204.4) [b] 
(2987.0499999993635, 204.5) [b] 
(2986.949999999363, 204.6) [b] 
(2986.7999999993626, 204.7) [b] 
(2986.7499999993624, 204.8) [b] 
(2986.4999999993615, 204.9) [b] 
(2986.4499999993614, 205.2) [b] 
(2986.299999999361, 205.3) [b] 
(2986.1499999993603, 205.4) [b] 
(2986.04999999936, 205.5) [b] 
(2985.9999999993597, 205.6) [b] 
(2985.9499999993595, 205.7) [b] 
(2985.8999999993594, 205.8) [b] 
(2985.799999999359, 205.9) [b] 
(2985.649999999359, 206) [b] 
(2985.2999999993585, 206.1) [b] 
(2985.149999999358, 206.4) [b] 
(2985.099999999358, 206.5) [b] 
(2984.9499999993573, 206.6) [b] 
(2984.7999999993567, 206.7) [b] 
(2984.599999999357, 206.8) [b] 
(2984.4999999993565, 206.9) [b] 
(2984.399999999356, 207) [b] 
(2984.349999999356, 207.1) [b] 
(2984.199999999356, 207.2) [b] 
(2984.1499999993557, 207.3) [b] 
(2984.0499999993554, 207.4) [b] 
(2983.899999999355, 207.5) [b] 
(2983.8499999993546, 207.6) [b] 
(2983.7499999993543, 207.7) [b] 
(2983.549999999354, 207.8) [b] 
(2983.499999999354, 207.9) [b] 
(2983.3999999993534, 208) [b] 
(2983.2499999993533, 208.1) [b] 
(2983.199999999353, 208.2) [b] 
(2983.099999999353, 208.3) [b] 
(2982.9499999993527, 208.4) [b] 
(2982.6999999993523, 208.5) [b] 
(2982.549999999352, 208.6) [b] 
(2982.499999999352, 208.8) [b] 
(2982.149999999352, 208.9) [b] 
(2982.049999999352, 209) [b] 
(2981.8999999993516, 209.1) [b] 
(2981.7999999993513, 209.2) [b] 
(2981.6999999993513, 209.3) [b] 
(2981.599999999351, 209.4) [b] 
(2981.499999999351, 209.5) [b] 
(2981.449999999351, 209.6) [b] 
(2981.1499999993503, 209.7) [b] 
(2980.94999999935, 209.9) [b] 
(2980.8499999993496, 210) [b] 
(2980.6999999993495, 210.1) [b] 
(2980.5499999993494, 210.2) [b] 
(2980.4999999993493, 210.3) [b] 
(2980.3499999993487, 210.4) [b] 
(2980.2499999993483, 210.5) [b] 
(2980.0999999993483, 210.6) [b] 
(2979.899999999348, 210.8) [b] 
(2979.699999999348, 211) [b] 
(2979.5499999993476, 211.1) [b] 
(2979.4999999993474, 211.2) [b] 
(2979.4499999993473, 211.3) [b] 
(2979.149999999347, 211.5) [b] 
(2979.0499999993467, 211.6) [b] 
(2978.849999999346, 211.7) [b] 
(2978.749999999346, 212) [b] 
(2978.6499999993457, 212.1) [b] 
(2978.5499999993453, 212.2) [b] 
(2978.449999999345, 212.3) [b] 
(2978.249999999345, 212.4) [b] 
(2978.199999999345, 212.5) [b] 
(2978.0999999993446, 212.6) [b] 
(2977.9999999993443, 212.7) [b] 
(2977.899999999344, 212.8) [b] 
(2977.799999999344, 212.9) [b] 
(2977.699999999344, 213) [b] 
(2977.499999999344, 213.1) [b] 
(2977.3999999993434, 213.2) [b] 
(2977.3499999993433, 213.3) [b] 
(2977.2499999993433, 213.4) [b] 
(2977.199999999343, 213.5) [b] 
(2976.999999999343, 213.6) [b] 
(2976.6999999993423, 213.8) [b] 
(2976.599999999342, 213.9) [b] 
(2976.5499999993417, 214) [b] 
(2976.4999999993415, 214.1) [b] 
(2976.4499999993413, 214.2) [b] 
(2976.349999999341, 214.3) [b] 
(2976.2499999993406, 214.4) [b] 
(2976.04999999934, 214.5) [b] 
(2975.9499999993395, 214.6) [b] 
(2975.8499999993396, 214.7) [b] 
(2975.7999999993394, 214.8) [b] 
(2975.7499999993393, 214.9) [b] 
(2975.5999999993387, 215) [b] 
(2975.4999999993383, 215.1) [b] 
(2975.449999999338, 215.2) [b] 
(2975.249999999338, 215.3) [b] 
(2975.099999999338, 215.4) [b] 
(2975.0499999993376, 215.5) [b] 
(2974.9999999993374, 215.6) [b] 
(2974.899999999337, 215.7) [b] 
(2974.699999999337, 215.8) [b] 
(2974.4999999993365, 215.9) [b] 
(2974.4499999993363, 216) [b] 
(2974.2999999993363, 216.1) [b] 
(2974.099999999336, 216.2) [b] 
(2973.9499999993354, 216.4) [b] 
(2973.7999999993353, 216.5) [b] 
(2973.699999999335, 216.6) [b] 
(2973.5999999993346, 216.7) [b] 
(2973.5499999993344, 216.8) [b] 
(2973.399999999334, 216.9) [b] 
(2973.3499999993337, 217) [b] 
(2973.1999999993336, 217.1) [b] 
(2972.9999999993333, 217.2) [b] 
(2972.8999999993334, 217.3) [b] 
(2972.6999999993327, 217.4) [b] 
(2972.6499999993325, 217.6) [b] 
(2972.5999999993323, 217.7) [b] 
(2972.549999999332, 217.8) [b] 
(2972.3999999993316, 217.9) [b] 
(2972.2999999993312, 218) [b] 
(2972.249999999331, 218.1) [b] 
(2971.949999999331, 218.2) [b] 
(2971.8999999993307, 218.4) [b] 
(2971.6499999993302, 218.5) [b] 
(2971.54999999933, 218.6) [b] 
(2971.39999999933, 218.7) [b] 
(2971.29999999933, 218.8) [b] 
(2971.2499999993297, 218.9) [b] 
(2971.099999999329, 219) [b] 
(2970.9499999993286, 219.2) [b] 
(2970.8499999993282, 219.3) [b] 
(2970.5999999993273, 219.5) [b] 
(2970.549999999327, 219.7) [b] 
(2970.299999999327, 219.8) [b] 
(2970.199999999327, 219.9) [b] 
(2970.0499999993267, 220) [b] 
(2969.899999999326, 220.1) [b] 
(2969.799999999326, 220.2) [b] 
(2969.7499999993256, 220.3) [b] 
(2969.599999999325, 220.5) [b] 
(2969.449999999325, 220.6) [b] 
(2969.299999999325, 220.7) [b] 
(2969.1999999993245, 220.8) [b] 
(2969.1499999993243, 220.9) [b] 
(2968.999999999324, 221) [b] 
(2968.699999999323, 221.2) [b] 
(2968.649999999323, 221.3) [b] 
(2968.5499999993226, 221.5) [b] 
(2968.399999999322, 221.6) [b] 
(2968.349999999322, 221.7) [b] 
(2968.1499999993216, 221.8) [b] 
(2968.0999999993214, 221.9) [b] 
(2967.9499999993213, 222) [b] 
(2967.849999999321, 222.2) [b] 
(2967.6499999993207, 222.3) [b] 
(2967.3999999993207, 222.4) [b] 
(2967.24999999932, 222.5) [b] 
(2967.14999999932, 222.8) [b] 
(2966.9499999993195, 222.9) [b] 
(2966.7499999993192, 223) [b] 
(2966.649999999319, 223.1) [b] 
(2966.5499999993185, 223.3) [b] 
(2966.4999999993183, 223.4) [b] 
(2966.399999999318, 223.5) [b] 
(2966.1999999993177, 223.6) [b] 
(2966.1499999993175, 223.7) [b] 
(2966.049999999317, 223.8) [b] 
(2965.899999999317, 223.9) [b] 
(2965.849999999317, 224) [b] 
(2965.6999999993163, 224.1) [b] 
(2965.4999999993165, 224.3) [b] 
(2965.2499999993165, 224.4) [b] 
(2965.149999999316, 224.5) [b] 
(2965.099999999316, 224.6) [b] 
(2965.049999999316, 224.7) [b] 
(2964.9999999993156, 224.8) [b] 
(2964.8999999993152, 224.9) [b] 
(2964.5999999993146, 225.1) [b] 
(2964.5499999993144, 225.2) [b] 
(2964.4999999993142, 225.3) [b] 
(2964.449999999314, 225.4) [b] 
(2964.1999999993136, 225.6) [b] 
(2964.1499999993134, 225.7) [b] 
(2963.799999999313, 225.8) [b] 
(2963.699999999313, 225.9) [b] 
(2963.649999999313, 226) [b] 
(2963.4999999993124, 226.2) [b] 
(2963.299999999312, 226.3) [b] 
(2963.099999999312, 226.4) [b] 
(2962.999999999312, 226.5) [b] 
(2962.949999999312, 226.6) [b] 
(2962.7999999993117, 226.7) [b] 
(2962.6499999993116, 226.8) [b] 
(2962.399999999311, 226.9) [b] 
(2962.349999999311, 227) [b] 
(2962.0999999993105, 227.2) [b] 
(2961.8999999993102, 227.4) [b] 
(2961.84999999931, 227.5) [b] 
(2961.79999999931, 227.6) [b] 
(2961.64999999931, 227.7) [b] 
(2961.4499999993095, 227.8) [b] 
(2961.2999999993094, 227.9) [b] 
(2961.2499999993092, 228) [b] 
(2961.0999999993087, 228.2) [b] 
(2961.0499999993085, 228.3) [b] 
(2960.899999999308, 228.4) [b] 
(2960.849999999308, 228.5) [b] 
(2960.7999999993076, 228.6) [b] 
(2960.7499999993074, 228.7) [b] 
(2960.6499999993075, 228.8) [b] 
(2960.4499999993072, 228.9) [b] 
(2960.299999999307, 229) [b] 
(2960.249999999307, 229.1) [b] 
(2959.899999999307, 229.3) [b] 
(2959.849999999307, 229.4) [b] 
(2959.5999999993064, 229.5) [b] 
(2959.4999999993065, 229.6) [b] 
(2959.399999999306, 229.7) [b] 
(2959.299999999306, 229.9) [b] 
(2959.099999999305, 230) [b] 
(2959.049999999305, 230.1) [b] 
(2958.8499999993046, 230.2) [b] 
(2958.6999999993045, 230.3) [b] 
(2958.549999999304, 230.5) [b] 
(2958.449999999304, 230.7) [b] 
(2958.3499999993037, 230.8) [b] 
(2958.2499999993033, 230.9) [b] 
(2957.899999999303, 231) [b] 
(2957.6999999993027, 231.3) [b] 
(2957.4499999993022, 231.5) [b] 
(2957.399999999302, 231.7) [b] 
(2957.2999999993017, 231.8) [b] 
(2957.2499999993015, 231.9) [b] 
(2957.1999999993013, 232) [b] 
(2956.849999999301, 232.1) [b] 
(2956.799999999301, 232.2) [b] 
(2956.6999999993004, 232.3) [b] 
(2956.5999999993005, 232.4) [b] 
(2956.5499999993003, 232.5) [b] 
(2956.4499999993, 232.6) [b] 
(2956.2999999992994, 232.7) [b] 
(2955.8999999992993, 232.8) [b] 
(2955.699999999299, 233.1) [b] 
(2955.549999999299, 233.2) [b] 
(2955.499999999299, 233.3) [b] 
(2955.299999999298, 233.4) [b] 
(2955.249999999298, 233.6) [b] 
(2955.1499999992975, 233.7) [b] 
(2954.999999999297, 233.8) [b] 
(2954.949999999297, 234) [b] 
(2954.699999999297, 234.2) [b] 
(2954.5999999992964, 234.3) [b] 
(2954.5499999992962, 234.4) [b] 
(2954.499999999296, 234.5) [b] 
(2954.449999999296, 234.6) [b] 
(2954.199999999296, 234.7) [b] 
(2954.049999999296, 234.9) [b] 
(2953.9999999992956, 235) [b] 
(2953.9499999992954, 235.1) [b] 
(2953.749999999295, 235.2) [b] 
(2953.6499999992952, 235.4) [b] 
(2953.549999999295, 235.5) [b] 
(2953.4499999992945, 235.7) [b] 
(2953.349999999294, 235.8) [b] 
(2953.299999999294, 235.9) [b] 
(2953.1999999992936, 236) [b] 
(2953.0999999992932, 236.2) [b] 
(2952.999999999293, 236.3) [b] 
(2952.9499999992927, 236.5) [b] 
(2952.799999999292, 236.6) [b] 
(2952.749999999292, 236.7) [b] 
(2952.6499999992916, 236.8) [b] 
(2952.5999999992914, 237) [b] 
(2952.4999999992915, 237.1) [b] 
(2952.3499999992914, 237.2) [b] 
(2952.099999999291, 237.3) [b] 
(2952.049999999291, 237.7) [b] 
(2951.8999999992902, 237.8) [b] 
(2951.6999999992904, 237.9) [b] 
(2951.6499999992902, 238) [b] 
(2951.59999999929, 238.2) [b] 
(2951.4499999992895, 238.3) [b] 
(2951.3999999992893, 238.6) [b] 
(2951.249999999289, 238.7) [b] 
(2951.1499999992884, 238.8) [b] 
(2951.0999999992882, 238.9) [b] 
(2951.049999999288, 239.1) [b] 
(2950.999999999288, 239.2) [b] 
(2950.8999999992875, 239.3) [b] 
(2950.749999999287, 239.4) [b] 
(2950.699999999287, 239.5) [b] 
(2950.6499999992866, 239.6) [b] 
(2950.5499999992867, 239.8) [b] 
(2950.399999999286, 239.9) [b] 
(2950.299999999286, 240) [b] 
(2950.1999999992854, 240.1) [b] 
(2949.849999999285, 240.5) [b] 
(2949.7499999992847, 240.6) [b] 
(2949.6999999992845, 240.7) [b] 
(2949.5999999992846, 240.8) [b] 
(2949.5499999992844, 240.9) [b] 
(2949.4999999992842, 241) [b] 
(2949.399999999284, 241.1) [b] 
(2949.3499999992837, 241.2) [b] 
(2949.2499999992833, 241.3) [b] 
(2949.049999999283, 241.5) [b] 
(2948.949999999283, 241.6) [b] 
(2948.899999999283, 241.7) [b] 
(2948.8499999992828, 241.8) [b] 
(2948.7999999992826, 241.9) [b] 
(2948.599999999282, 242) [b] 
(2948.5499999992817, 242.1) [b] 
(2948.4999999992815, 242.3) [b] 
(2948.4499999992813, 242.4) [b] 
(2948.2999999992808, 242.5) [b] 
(2948.2499999992806, 242.6) [b] 
(2948.1999999992804, 242.7) [b] 
(2948.14999999928, 242.8) [b] 
(2948.04999999928, 242.9) [b] 
(2947.8999999992793, 243.1) [b] 
(2947.7999999992794, 243.2) [b] 
(2947.699999999279, 243.4) [b] 
(2947.4999999992788, 243.5) [b] 
(2947.4499999992786, 243.6) [b] 
(2947.349999999278, 243.7) [b] 
(2947.249999999278, 243.9) [b] 
(2947.0999999992778, 244.1) [b] 
(2947.0499999992776, 244.2) [b] 
(2946.9999999992774, 244.3) [b] 
(2946.899999999277, 244.4) [b] 
(2946.849999999277, 244.6) [b] 
(2946.6999999992768, 244.7) [b] 
(2946.549999999276, 244.8) [b] 
(2946.449999999276, 245) [b] 
(2946.3999999992757, 245.2) [b] 
(2946.1999999992754, 245.3) [b] 
(2946.149999999275, 245.4) [b] 
(2945.999999999275, 245.6) [b] 
(2945.8999999992748, 245.7) [b] 
(2945.7999999992744, 245.8) [b] 
(2945.699999999274, 245.9) [b] 
(2945.649999999274, 246.1) [b] 
(2945.349999999273, 246.3) [b] 
(2945.299999999273, 246.5) [b] 
(2945.249999999273, 246.7) [b] 
(2945.0999999992723, 246.8) [b] 
(2944.949999999272, 246.9) [b] 
(2944.899999999272, 247) [b] 
(2944.849999999272, 247.2) [b] 
(2944.7499999992715, 247.3) [b] 
(2944.6999999992713, 247.4) [b] 
(2944.599999999271, 247.5) [b] 
(2944.5499999992708, 247.7) [b] 
(2944.2499999992706, 247.8) [b] 
(2944.1999999992704, 248) [b] 
(2944.14999999927, 248.1) [b] 
(2944.09999999927, 248.2) [b] 
(2943.9499999992695, 248.3) [b] 
(2943.8999999992693, 248.4) [b] 
(2943.849999999269, 248.5) [b] 
(2943.799999999269, 248.7) [b] 
(2943.6999999992686, 248.8) [b] 
(2943.6499999992684, 248.9) [b] 
(2943.599999999268, 249) [b] 
(2943.499999999268, 249.1) [b] 
(2943.3499999992678, 249.3) [b] 
(2943.2499999992674, 249.4) [b] 
(2943.0999999992673, 249.5) [b] 
(2943.049999999267, 249.7) [b] 
(2942.9499999992668, 249.8) [b] 
(2942.8999999992666, 249.9) [b] 
(2942.749999999266, 250) [b] 
(2942.699999999266, 250.2) [b] 
(2942.6499999992657, 250.3) [b] 
(2942.5999999992655, 250.4) [b] 
(2942.499999999265, 250.5) [b] 
(2942.449999999265, 250.6) [b] 
(2942.3999999992648, 250.7) [b] 
(2942.299999999265, 250.8) [b] 
(2942.2499999992647, 250.9) [b] 
(2942.1999999992645, 251) [b] 
(2942.049999999264, 251.1) [b] 
(2941.9999999992638, 251.2) [b] 
(2941.9499999992636, 251.3) [b] 
(2941.7999999992635, 251.4) [b] 
(2941.7499999992633, 251.5) [b] 
(2941.649999999263, 251.7) [b] 
(2941.4999999992624, 251.8) [b] 
(2941.399999999262, 252) [b] 
(2941.249999999262, 252.2) [b] 
(2941.1499999992616, 252.3) [b] 
(2941.0999999992614, 252.4) [b] 
(2940.9499999992613, 252.5) [b] 
(2940.8499999992614, 252.7) [b] 
(2940.7499999992615, 252.8) [b] 
(2940.6999999992613, 252.9) [b] 
(2940.549999999261, 253) [b] 
(2940.499999999261, 253.2) [b] 
(2940.399999999261, 253.4) [b] 
(2940.149999999261, 253.5) [b] 
(2940.099999999261, 253.7) [b] 
(2939.9999999992606, 253.8) [b] 
(2939.89999999926, 254) [b] 
(2939.84999999926, 254.1) [b] 
(2939.79999999926, 254.3) [b] 
(2939.6999999992595, 254.5) [b] 
(2939.6499999992593, 254.6) [b] 
(2939.5499999992594, 254.7) [b] 
(2939.449999999259, 255) [b] 
(2939.2499999992588, 255.2) [b] 
(2939.1499999992584, 255.3) [b] 
(2939.099999999258, 255.6) [b] 
(2939.049999999258, 255.7) [b] 
(2938.999999999258, 255.8) [b] 
(2938.8999999992575, 255.9) [b] 
(2938.8499999992573, 256.1) [b] 
(2938.699999999257, 256.4) [b] 
(2938.549999999257, 256.5) [b] 
(2938.399999999257, 256.6) [b] 
(2938.349999999257, 257) [b] 
(2938.2499999992565, 257.1) [b] 
(2938.1999999992563, 257.2) [b] 
(2938.149999999256, 257.3) [b] 
(2937.999999999256, 257.6) [b] 
(2937.8999999992557, 257.7) [b] 
(2937.8499999992555, 257.8) [b] 
(2937.7999999992553, 258) [b] 
(2937.749999999255, 258.1) [b] 
(2937.6499999992548, 258.3) [b] 
(2937.5999999992546, 258.4) [b] 
(2937.3999999992543, 258.6) [b] 
(2937.349999999254, 258.9) [b] 
(2937.199999999254, 259) [b] 
(2937.149999999254, 259.1) [b] 
(2937.0999999992537, 259.3) [b] 
(2936.8999999992534, 259.4) [b] 
(2936.749999999253, 259.6) [b] 
(2936.6999999992527, 259.8) [b] 
(2936.6499999992525, 259.9) [b] 
(2936.5999999992523, 260) [b] 
(2936.4999999992524, 260.1) [b] 
(2936.349999999252, 260.3) [b] 
(2936.2999999992517, 260.5) [b] 
(2936.2499999992515, 260.6) [b] 
(2936.1999999992513, 260.7) [b] 
(2936.099999999251, 260.9) [b] 
(2936.0499999992508, 261) [b] 
(2935.9999999992506, 261.1) [b] 
(2935.9499999992504, 261.2) [b] 
(2935.8499999992505, 261.4) [b] 
(2935.7999999992503, 261.5) [b] 
(2935.69999999925, 261.6) [b] 
(2935.59999999925, 261.7) [b] 
(2935.54999999925, 262.1) [b] 
(2935.3999999992493, 262.2) [b] 
(2935.349999999249, 262.6) [b] 
(2935.299999999249, 262.7) [b] 
(2935.0999999992487, 262.8) [b] 
(2935.0499999992485, 263.1) [b] 
(2934.9999999992483, 263.3) [b] 
(2934.949999999248, 263.4) [b] 
(2934.799999999248, 263.5) [b] 
(2934.699999999248, 263.6) [b] 
(2934.649999999248, 264) [b] 
(2934.5999999992478, 264.1) [b] 
(2934.4499999992477, 264.2) [b] 
(2934.3999999992475, 264.6) [b] 
(2934.3499999992473, 264.7) [b] 
(2934.299999999247, 264.9) [b] 
(2934.249999999247, 265) [b] 
(2934.1999999992468, 265.1) [b] 
(2934.1499999992466, 265.3) [b] 
(2934.0999999992464, 265.5) [b] 
(2934.049999999246, 265.6) [b] 
(2933.999999999246, 265.7) [b] 
(2933.899999999246, 265.9) [b] 
(2933.799999999246, 266.1) [b] 
(2933.6999999992463, 266.2) [b] 
(2933.649999999246, 266.4) [b] 
(2933.4999999992456, 266.5) [b] 
(2933.4499999992454, 266.6) [b] 
(2933.399999999245, 266.8) [b] 
(2933.299999999245, 267) [b] 
(2933.2499999992447, 267.2) [b] 
(2933.0999999992446, 267.3) [b] 
(2932.999999999244, 267.5) [b] 
(2932.8999999992443, 267.6) [b] 
(2932.849999999244, 267.9) [b] 
(2932.799999999244, 268) [b] 
(2932.6999999992436, 268.1) [b] 
(2932.6499999992434, 268.2) [b] 
(2932.4999999992433, 268.6) [b] 
(2932.3999999992434, 268.7) [b] 
(2932.2999999992435, 268.9) [b] 
(2932.1999999992436, 269.3) [b] 
(2932.1499999992434, 269.4) [b] 
(2932.099999999243, 269.5) [b] 
(2932.049999999243, 270) [b] 
(2931.9499999992427, 270.1) [b] 
(2931.8999999992425, 270.2) [b] 
(2931.749999999242, 270.7) [b] 
(2931.6999999992418, 270.9) [b] 
(2931.5999999992414, 271.4) [b] 
(2931.549999999241, 271.5) [b] 
(2931.299999999241, 272) [b] 
(2931.249999999241, 272.2) [b] 
(2931.199999999241, 272.7) [b] 
(2931.0999999992405, 272.8) [b] 
(2930.9999999992406, 273.3) [b] 
(2930.89999999924, 273.5) [b] 
(2930.84999999924, 274.1) [b] 
(2930.79999999924, 274.2) [b] 
(2930.7499999992397, 274.7) [b] 
(2930.6999999992395, 274.9) [b] 
(2930.599999999239, 275.3) [b] 
(2930.549999999239, 275.6) [b] 
(2930.449999999239, 275.9) [b] 
(2930.399999999239, 276) [b] 
(2930.3499999992387, 276.2) [b] 
(2930.2999999992385, 276.6) [b] 
(2930.1999999992386, 276.7) [b] 
(2930.1499999992384, 276.9) [b] 
(2930.099999999238, 277.2) [b] 
(2929.9999999992383, 277.4) [b] 
(2929.949999999238, 277.6) [b] 
(2929.899999999238, 277.9) [b] 
(2929.8499999992378, 278.1) [b] 
(2929.749999999238, 278.4) [b] 
(2929.6999999992377, 278.5) [b] 
(2929.5999999992378, 278.7) [b] 
(2929.5499999992376, 279.1) [b] 
(2929.4999999992374, 279.2) [b] 
(2929.449999999237, 279.4) [b] 
(2929.399999999237, 279.8) [b] 
(2929.349999999237, 279.9) [b] 
(2929.249999999237, 280.1) [b] 
(2929.1999999992368, 280.5) [b] 
(2929.1499999992366, 280.6) [b] 
(2929.0999999992364, 280.7) [b] 
(2928.999999999236, 281.2) [b] 
(2928.949999999236, 281.4) [b] 
(2928.849999999236, 281.9) [b] 
(2928.749999999236, 282) [b] 
(2928.699999999236, 282.2) [b] 
(2928.5999999992355, 282.6) [b] 
(2928.5499999992353, 282.9) [b] 
(2928.449999999235, 283.3) [b] 
(2928.3999999992348, 283.5) [b] 
(2928.299999999235, 283.9) [b] 
(2928.2499999992347, 284) [b] 
(2928.1499999992343, 284.7) [b] 
(2927.999999999234, 285.4) [b] 
(2927.949999999234, 286) [b] 
(2927.899999999234, 286.1) [b] 
(2927.7999999992335, 286.7) [b] 
(2927.699999999233, 287.4) [b] 
(2927.649999999233, 288) [b] 
(2927.5999999992328, 288.1) [b] 
(2927.4499999992327, 288.7) [b] 
(2927.3999999992325, 289.3) [b] 
(2927.2999999992326, 289.4) [b] 
(2927.2499999992324, 290) [b] 
(2927.199999999232, 290.1) [b] 
(2927.149999999232, 290.6) [b] 
(2927.099999999232, 290.8) [b] 
(2927.0499999992317, 291.2) [b] 
(2926.9999999992315, 291.5) [b] 
(2926.9499999992313, 291.8) [b] 
(2926.899999999231, 292.2) [b] 
(2926.799999999231, 292.5) [b] 
(2926.649999999231, 292.9) [b] 
(2926.599999999231, 293.1) [b] 
(2926.499999999231, 293.6) [b] 
(2926.449999999231, 293.7) [b] 
(2926.3499999992305, 294.3) [b] 
(2926.24999999923, 295) [b] 
(2926.19999999923, 295.6) [b] 
(2926.09999999923, 295.7) [b] 
(2925.84999999923, 296.3) [b] 
(2925.79999999923, 296.9) [b] 
(2925.7499999992297, 297) [b] 
(2925.6499999992293, 297.6) [b] 
(2925.599999999229, 298.3) [b] 
(2925.499999999229, 298.4) [b] 
(2925.449999999229, 298.9) [b] 
(2925.399999999229, 299.1) [b] 
(2925.3499999992287, 299.6) [b] 
(2925.2999999992285, 299.7) [b] 
(2925.1999999992286, 300.2) [b] 
(2925.1499999992284, 300.4) [b] 
(2925.099999999228, 300.9) [b] 
(2925.049999999228, 301.1) [b] 
(2924.999999999228, 301.5) [b] 
(2924.9499999992277, 301.8) [b] 
(2924.8999999992275, 302.3) [b] 
(2924.8499999992273, 302.4) [b] 
(2924.799999999227, 302.9) [b] 
(2924.749999999227, 303.1) [b] 
(2924.649999999227, 303.5) [b] 
(2924.599999999227, 303.8) [b] 
(2924.5499999992267, 304.3) [b] 
(2924.4999999992265, 304.4) [b] 
(2924.4499999992263, 305.1) [b] 
(2924.3499999992264, 305.8) [b] 
(2924.299999999226, 306.5) [b] 
(2924.249999999226, 307.1) [b] 
(2924.199999999226, 307.8) [b] 
(2924.1499999992257, 308.5) [b] 
(2924.0999999992255, 309.1) [b] 
(2923.9999999992256, 309.8) [b] 
}}}{legend pos=north east}
\end{center}
\vfill
\end{frame}


\begin{frame}[fragile]
\frametitle{Job Shop with Time Lags}
\vfill
\begin{itemize}
	\item 150 tasks

	\vfill\uncover<3->{
	\begin{itemize}
	\item CP Optimizer has \memph{an extremely good primal heuristic!} 
	\end{itemize}
	}
\end{itemize}
\vfill
\begin{center}
\cactus{Objective}{CPU time}{CPOptimizer, Tempo, Tempo$^*$}{{{ 
(1450.9999999999986, 5.0) [a] 
(1449.9999999999986, 5.02) [a] 
(1449.4999999999986, 5.04) [a] 
(1449.1999999999987, 5.05) [a] 
(1447.2999999999986, 5.08) [a] 
(1446.0499999999986, 5.1) [a] 
(1445.7499999999986, 5.29) [a] 
(1445.1499999999987, 5.4) [a] 
(1445.0999999999988, 5.41) [a] 
(1444.7999999999986, 5.42) [a] 
(1443.9999999999984, 5.45) [a] 
(1443.8499999999983, 5.46) [a] 
(1443.3499999999983, 5.5) [a] 
(1443.2499999999984, 5.54) [a] 
(1443.1499999999985, 5.57) [a] 
(1441.7999999999986, 5.58) [a] 
(1439.2499999999986, 5.59) [a] 
(1438.6499999999987, 5.75) [a] 
(1438.0999999999988, 5.77) [a] 
(1437.6999999999987, 5.83) [a] 
(1435.7499999999986, 5.86) [a] 
(1435.2999999999986, 5.87) [a] 
(1434.8499999999985, 5.89) [a] 
(1434.3499999999985, 6.12) [a] 
(1433.9499999999985, 6.13) [a] 
(1433.8999999999985, 6.18) [a] 
(1433.4999999999984, 6.38) [a] 
(1433.3999999999985, 6.43) [a] 
(1433.1499999999985, 6.44) [a] 
(1433.0999999999985, 6.59) [a] 
(1432.9499999999985, 6.65) [a] 
(1432.7499999999984, 6.69) [a] 
(1431.2499999999984, 6.88) [a] 
(1431.1999999999985, 7.0) [a] 
(1430.5499999999986, 7.07) [a] 
(1430.3999999999987, 7.09) [a] 
(1429.8999999999987, 7.34) [a] 
(1427.9999999999989, 7.41) [a] 
(1427.449999999999, 7.5) [a] 
(1426.149999999999, 7.52) [a] 
(1425.549999999999, 7.59) [a] 
(1423.9499999999991, 7.64) [a] 
(1420.6999999999991, 7.65) [a] 
(1420.3999999999992, 7.68) [a] 
(1419.499999999999, 7.73) [a] 
(1418.8999999999992, 7.75) [a] 
(1418.7499999999993, 7.77) [a] 
(1418.6499999999994, 7.94) [a] 
(1416.9499999999994, 7.99) [a] 
(1416.1499999999994, 8.11) [a] 
(1416.0999999999995, 8.56) [a] 
(1415.8999999999994, 8.68) [a] 
(1415.1999999999994, 8.74) [a] 
(1414.0999999999995, 8.75) [a] 
(1413.9499999999994, 8.78) [a] 
(1412.7499999999993, 8.99) [a] 
(1412.5999999999992, 9.06) [a] 
(1412.4999999999993, 9.19) [a] 
(1412.0999999999995, 9.49) [a] 
(1412.0499999999995, 9.5) [a] 
(1411.8499999999995, 9.54) [a] 
(1411.7999999999995, 10.07) [a] 
(1410.4999999999995, 10.32) [a] 
(1410.4499999999996, 10.34) [a] 
(1410.3999999999996, 10.49) [a] 
(1409.7999999999997, 10.5) [a] 
(1409.6999999999998, 10.63) [a] 
(1409.1499999999999, 10.64) [a] 
(1408.9999999999998, 10.7) [a] 
(1407.5499999999997, 10.75) [a] 
(1406.5499999999997, 10.84) [a] 
(1405.3999999999996, 11.0) [a] 
(1405.3499999999997, 11.11) [a] 
(1403.9999999999998, 11.2) [a] 
(1403.8999999999999, 11.54) [a] 
(1403.55, 11.56) [a] 
(1403.45, 11.7) [a] 
(1403.4, 11.86) [a] 
(1403.3500000000001, 12.04) [a] 
(1402.55, 12.06) [a] 
(1400.3, 12.15) [a] 
(1400.2, 12.19) [a] 
(1400.15, 12.22) [a] 
(1399.8500000000001, 12.24) [a] 
(1398.8000000000002, 12.28) [a] 
(1398.1000000000001, 12.45) [a] 
(1397.9000000000003, 12.75) [a] 
(1396.9500000000003, 13.15) [a] 
(1396.0500000000002, 13.54) [a] 
(1395.8500000000001, 13.55) [a] 
(1395.6000000000001, 13.58) [a] 
(1395.3500000000001, 13.61) [a] 
(1395.3000000000002, 13.92) [a] 
(1395.2000000000003, 14.4) [a] 
(1394.6000000000004, 14.49) [a] 
(1394.5000000000005, 14.86) [a] 
(1392.9500000000005, 14.97) [a] 
(1392.8500000000006, 15.18) [a] 
(1391.4500000000005, 15.43) [a] 
(1391.3000000000004, 15.54) [a] 
(1390.6500000000003, 15.77) [a] 
(1390.3500000000004, 15.9) [a] 
(1389.7000000000005, 15.99) [a] 
(1388.8000000000004, 16.17) [a] 
(1388.7500000000005, 16.38) [a] 
(1388.6000000000004, 16.42) [a] 
(1388.5000000000005, 16.74) [a] 
(1387.9000000000005, 16.76) [a] 
(1387.1000000000006, 16.78) [a] 
(1387.0000000000007, 16.89) [a] 
(1386.9500000000007, 16.92) [a] 
(1386.8500000000008, 17.13) [a] 
(1386.4000000000008, 17.15) [a] 
(1386.2500000000007, 17.19) [a] 
(1386.0000000000007, 17.26) [a] 
(1385.1500000000008, 17.28) [a] 
(1385.1000000000008, 17.3) [a] 
(1385.0500000000009, 17.65) [a] 
(1384.700000000001, 17.7) [a] 
(1384.350000000001, 18.0) [a] 
(1384.200000000001, 18.13) [a] 
(1383.650000000001, 18.24) [a] 
(1383.600000000001, 18.29) [a] 
(1383.100000000001, 18.3) [a] 
(1382.600000000001, 18.31) [a] 
(1382.400000000001, 18.35) [a] 
(1381.650000000001, 18.6) [a] 
(1380.300000000001, 19.0) [a] 
(1380.2500000000011, 19.05) [a] 
(1380.0000000000011, 19.24) [a] 
(1379.2000000000012, 19.33) [a] 
(1379.1000000000013, 20.11) [a] 
(1377.7000000000012, 20.28) [a] 
(1376.3000000000013, 20.7) [a] 
(1376.0500000000013, 20.72) [a] 
(1375.9500000000014, 21.06) [a] 
(1375.2000000000014, 21.07) [a] 
(1375.1000000000015, 21.08) [a] 
(1374.9000000000015, 21.11) [a] 
(1373.8000000000015, 21.13) [a] 
(1373.6500000000015, 21.74) [a] 
(1373.1000000000015, 21.97) [a] 
(1372.6500000000015, 22.28) [a] 
(1372.6000000000015, 22.94) [a] 
(1371.5000000000016, 23.58) [a] 
(1371.1500000000017, 23.6) [a] 
(1370.8000000000018, 23.63) [a] 
(1370.2000000000019, 23.81) [a] 
(1369.100000000002, 24.07) [a] 
(1367.550000000002, 24.09) [a] 
(1367.050000000002, 24.15) [a] 
(1366.450000000002, 26.22) [a] 
(1366.3500000000022, 26.27) [a] 
(1366.3000000000022, 27.36) [a] 
(1366.2000000000023, 28.07) [a] 
(1366.1000000000024, 28.12) [a] 
(1365.1000000000024, 28.27) [a] 
(1364.9500000000023, 28.28) [a] 
(1364.9000000000024, 29.22) [a] 
(1364.7500000000023, 29.26) [a] 
(1362.7000000000023, 29.37) [a] 
(1362.3500000000024, 30.42) [a] 
(1362.1000000000024, 30.45) [a] 
(1362.0000000000025, 30.72) [a] 
(1361.7000000000025, 30.97) [a] 
(1361.5500000000025, 31.01) [a] 
(1360.9500000000025, 31.27) [a] 
(1360.8000000000025, 31.36) [a] 
(1360.7000000000025, 32.65) [a] 
(1360.4500000000025, 32.85) [a] 
(1360.3500000000026, 33.31) [a] 
(1359.9500000000025, 33.4) [a] 
(1358.8000000000027, 33.48) [a] 
(1358.6000000000029, 33.67) [a] 
(1358.500000000003, 34.36) [a] 
(1357.8500000000029, 34.48) [a] 
(1357.750000000003, 34.54) [a] 
(1357.200000000003, 34.55) [a] 
(1356.700000000003, 34.97) [a] 
(1356.000000000003, 35.0) [a] 
(1355.6000000000029, 35.02) [a] 
(1355.1000000000029, 35.06) [a] 
(1355.050000000003, 35.16) [a] 
(1355.000000000003, 35.48) [a] 
(1354.500000000003, 35.55) [a] 
(1354.3500000000029, 36.64) [a] 
(1353.6500000000028, 36.7) [a] 
(1353.4500000000028, 36.72) [a] 
(1353.3500000000029, 37.4) [a] 
(1352.8500000000029, 37.58) [a] 
(1352.250000000003, 37.78) [a] 
(1351.000000000003, 37.87) [a] 
(1350.150000000003, 38.38) [a] 
(1350.0500000000031, 39.26) [a] 
(1349.9500000000032, 39.31) [a] 
(1349.8500000000033, 40.21) [a] 
(1349.3500000000033, 40.22) [a] 
(1348.8000000000034, 41.18) [a] 
(1348.7000000000035, 41.94) [a] 
(1348.6500000000035, 42.22) [a] 
(1348.5500000000036, 42.31) [a] 
(1348.0500000000036, 42.38) [a] 
(1346.9500000000035, 42.59) [a] 
(1346.8500000000035, 42.74) [a] 
(1346.5000000000036, 43.26) [a] 
(1346.4000000000037, 43.46) [a] 
(1346.2000000000037, 43.8) [a] 
(1346.1000000000038, 44.77) [a] 
(1346.0000000000039, 46.4) [a] 
(1345.900000000004, 47.02) [a] 
(1345.550000000004, 47.09) [a] 
(1345.000000000004, 47.37) [a] 
(1344.9500000000041, 50.68) [a] 
(1344.6500000000042, 50.86) [a] 
(1344.6000000000042, 50.99) [a] 
(1344.0500000000043, 52.49) [a] 
(1343.9500000000044, 53.85) [a] 
(1343.9000000000044, 53.89) [a] 
(1343.6000000000045, 54.11) [a] 
(1343.5000000000045, 54.16) [a] 
(1343.3500000000045, 55.42) [a] 
(1343.0500000000045, 55.44) [a] 
(1342.8500000000045, 55.58) [a] 
(1342.7500000000045, 57.83) [a] 
(1342.5500000000045, 58.43) [a] 
(1342.4000000000044, 58.47) [a] 
(1340.6500000000044, 58.81) [a] 
(1339.4500000000044, 58.86) [a] 
(1339.1000000000045, 61.52) [a] 
(1338.7500000000045, 62.39) [a] 
(1338.6500000000046, 66.47) [a] 
(1338.2000000000046, 66.48) [a] 
(1338.0500000000045, 69.06) [a] 
(1338.0000000000045, 69.4) [a] 
(1337.9000000000046, 72.47) [a] 
(1337.8000000000047, 76.18) [a] 
(1337.7500000000048, 76.47) [a] 
(1337.5000000000048, 79.0) [a] 
(1337.4000000000049, 80.63) [a] 
(1336.7500000000048, 81.91) [a] 
(1335.7000000000048, 81.92) [a] 
(1335.5000000000048, 81.93) [a] 
(1335.2000000000048, 83.77) [a] 
(1335.100000000005, 85.51) [a] 
(1334.850000000005, 85.52) [a] 
(1334.500000000005, 88.86) [a] 
(1334.400000000005, 89.54) [a] 
(1334.3000000000052, 92.58) [a] 
(1334.2000000000053, 93.22) [a] 
(1334.0500000000052, 99.37) [a] 
(1333.8500000000051, 99.92) [a] 
(1333.7500000000052, 102.38) [a] 
(1333.2000000000053, 103.29) [a] 
(1333.1000000000054, 103.3) [a] 
(1333.0000000000055, 103.83) [a] 
(1331.4500000000055, 103.84) [a] 
(1330.9000000000055, 103.9) [a] 
(1329.7000000000055, 104.14) [a] 
(1328.4000000000055, 108.95) [a] 
(1328.2000000000055, 113.52) [a] 
(1327.9000000000055, 113.9) [a] 
(1327.8000000000056, 113.95) [a] 
(1326.0500000000056, 114.83) [a] 
(1326.0000000000057, 114.84) [a] 
(1325.9500000000057, 117.46) [a] 
(1325.9000000000058, 117.72) [a] 
(1325.4500000000057, 118.22) [a] 
(1325.0500000000059, 122.29) [a] 
(1324.9000000000058, 124.22) [a] 
(1324.8000000000059, 124.56) [a] 
(1324.4000000000058, 126.05) [a] 
(1323.1500000000058, 131.68) [a] 
(1322.5500000000059, 132.84) [a] 
(1322.450000000006, 138.09) [a] 
(1322.250000000006, 140.27) [a] 
(1322.200000000006, 140.97) [a] 
(1322.0500000000059, 143.17) [a] 
(1321.9000000000058, 143.37) [a] 
(1321.8500000000058, 144.13) [a] 
(1321.750000000006, 145.47) [a] 
(1321.700000000006, 148.77) [a] 
(1321.600000000006, 149.04) [a] 
(1321.550000000006, 153.89) [a] 
(1321.050000000006, 154.85) [a] 
(1319.7500000000061, 154.88) [a] 
(1319.550000000006, 157.4) [a] 
(1319.050000000006, 157.83) [a] 
(1318.9500000000062, 163.42) [a] 
(1318.7000000000062, 163.92) [a] 
(1318.6500000000062, 183.34) [a] 
(1318.6000000000063, 192.7) [a] 
(1318.5000000000064, 197.13) [a] 
(1318.4500000000064, 197.43) [a] 
(1318.3500000000065, 211.81) [a] 
(1318.2500000000066, 211.84) [a] 
(1317.7000000000066, 212.01) [a] 
(1317.6000000000067, 212.47) [a] 
(1317.5000000000068, 213.47) [a] 
(1317.100000000007, 226.31) [a] 
(1317.000000000007, 228.12) [a] 
(1316.9000000000071, 228.53) [a] 
(1316.8000000000072, 242.32) [a] 
(1316.7500000000073, 243.56) [a] 
(1316.6500000000074, 261.71) [a] 
(1316.5000000000073, 307.11) [a] 
(1316.4500000000073, 308.69) [a] 
(1316.2500000000073, 319.87) [a] 
(1316.0000000000073, 319.88) [a] 
(1315.9000000000074, 330.9) [a] 
(1315.8000000000075, 330.95) [a] 
(1315.7000000000075, 357.9) [a] 
(1315.6000000000076, 375.0) [a] 
(1315.5000000000077, 397.25) [a] 
(1315.0000000000077, 410.98) [a] 
(1314.9500000000078, 420.5) [a] 
(1314.8500000000079, 422.26) [a] 
(1314.750000000008, 441.54) [a] 
(1314.700000000008, 441.7) [a] 
(1314.600000000008, 445.38) [a] 
(1314.5000000000082, 461.93) [a] 
(1314.350000000008, 468.88) [a] 
(1314.150000000008, 470.13) [a] 
(1314.100000000008, 471.56) [a] 
(1314.0000000000082, 474.53) [a] 
(1313.9000000000083, 474.54) [a] 
(1313.5500000000084, 481.24) [a] 
(1313.5000000000084, 489.72) [a] 
(1313.4000000000085, 501.36) [a] 
(1312.9500000000085, 503.9) [a] 
(1312.8500000000085, 523.83) [a] 
(1312.5000000000086, 526.47) [a] 
(1312.4000000000087, 550.04) [a] 
(1312.2500000000086, 583.52) [a] 
(1312.1500000000087, 586.16) [a] 
(1312.0000000000086, 586.87) [a] 
(1311.9500000000087, 600.4) [a] 
(1311.7000000000087, 600.42) [a] 
(1311.6000000000088, 600.49) [a] 
(1311.5000000000089, 622.6) [a] 
(1311.450000000009, 626.9) [a] 
(1311.200000000009, 636.87) [a] 
(1311.100000000009, 652.07) [a] 
(1310.900000000009, 661.14) [a] 
(1310.300000000009, 661.59) [a] 
(1310.250000000009, 669.77) [a] 
(1310.2000000000091, 675.36) [a] 
(1310.050000000009, 682.64) [a] 
(1309.900000000009, 696.54) [a] 
(1309.850000000009, 704.88) [a] 
(1309.750000000009, 706.11) [a] 
(1309.7000000000091, 713.46) [a] 
(1309.6000000000092, 802.92) [a] 
(1309.4500000000091, 824.38) [a] 
(1309.3500000000092, 829.5) [a] 
(1309.2500000000093, 842.8) [a] 
(1309.2000000000094, 846.17) [a] 
(1308.9000000000094, 857.93) [a] 
(1308.8000000000095, 863.59) [a] 
(1308.6000000000095, 882.22) [a] 
(1308.5000000000095, 923.3) [a] 
(1308.4000000000096, 987.13) [a] 
(1308.3000000000097, 997.56) [a] 
(1308.1500000000096, 1113.1) [a] 
(1308.0500000000097, 1138.39) [a] 
(1307.8500000000097, 1142.96) [a] 
(1307.7500000000098, 1185.32) [a] 
(1307.6500000000099, 1236.59) [a] 
(1307.4500000000098, 1257.43) [a] 
(1307.4000000000099, 1278.13) [a] 
(1307.30000000001, 1314.03) [a] 
(1307.1500000000099, 1314.27) [a] 
(1307.10000000001, 1314.42) [a] 
(1307.00000000001, 1383.82) [a] 
(1306.95000000001, 1397.36) [a] 
(1306.8500000000101, 1417.42) [a] 
(1306.7500000000102, 1526.74) [a] 
(1306.6500000000103, 1531.81) [a] 
(1305.9000000000103, 1553.36) [a] 
(1305.7500000000102, 1553.38) [a] 
(1305.6500000000103, 1553.65) [a] 
(1305.5500000000104, 1567.01) [a] 
(1305.5000000000105, 1703.49) [a] 
(1305.4500000000105, 1734.11) [a] 
(1305.4000000000106, 1786.86) [a] 
(1305.3500000000106, 1801.92) [a] 
(1305.2500000000107, 1877.2) [a] 
(1305.1500000000108, 2086.06) [a] 
(1305.0500000000109, 2098.8) [a] 
(1305.000000000011, 2153.36) [a] 
(1304.950000000011, 2205.03) [a] 
(1304.650000000011, 2208.42) [a] 
(1303.950000000011, 2210.15) [a] 
(1303.900000000011, 2318.45) [a] 
(1303.850000000011, 2330.32) [a] 
(1303.7500000000111, 2481.59) [a] 
(1303.7000000000112, 2485.18) [a] 
(1302.9000000000112, 2532.22) [a] 
(1302.7500000000111, 2532.98) [a] 
(1302.6500000000112, 2761.31) [a] 
(1302.6000000000113, 2903.86) [a] 
(1302.5500000000113, 3036.52) [a] 
(1302.4500000000114, 3110.39) [a] 
(1302.4000000000115, 3111.31) [a] 
(1301.4000000000115, 3117.22) [a] 
(1301.3500000000115, 3143.61) [a] 
(1300.9500000000114, 3214.4) [a] 
(1300.3500000000115, 3218.34) [a] 
(1300.2500000000116, 3218.36) [a] 
(1300.1500000000117, 3281.7) [a] 
(1300.0000000000116, 3395.36) [a] 
(1299.8000000000116, 3424.06) [a] 
(1299.1000000000115, 3425.93) [a] 
(1299.0500000000116, 3622.25) [a] 
(1299.0000000000116, 3802.92) [a] 
(1298.9000000000117, 4396.66) [a] 
(1298.8000000000118, 4442.37) [a] 
(1298.5500000000118, 4470.43) [a] 
(1297.7500000000118, 4472.12) [a] 
(1297.6000000000117, 4472.14) [a] 
(1296.9000000000117, 4476.89) [a] 
(1296.8500000000117, 4506.49) [a] 
(1296.8000000000118, 4551.31) [a] 
(1295.9500000000119, 4556.5) [a] 
(1295.8000000000118, 4557.11) [a] 
(1295.6500000000117, 4605.38) [a] 
(1295.6000000000117, 4605.81) [a] 
(1295.5500000000118, 4710.33) [a] 
(1295.2500000000118, 4746.71) [a] 
(1294.5500000000118, 4749.97) [a] 
(1293.5500000000118, 4841.58) [a] 
(1293.4500000000119, 5000.15) [a] 
(1293.350000000012, 5032.65) [a] 
(1292.650000000012, 5044.18) [a] 
(1292.5000000000118, 5045.36) [a] 
(1292.4500000000119, 5248.27) [a] 
(1291.4500000000119, 5272.22) [a] 
(1291.400000000012, 5393.58) [a] 
(1290.400000000012, 5443.15) [a] 
(1290.350000000012, 5510.89) [a] 
(1290.250000000012, 5556.18) [a] 
(1289.500000000012, 5560.96) [a] 
(1289.350000000012, 5561.89) [a] 
(1289.2000000000119, 5592.65) [a] 
(1289.100000000012, 5626.02) [a] 
(1288.4500000000119, 5653.58) [a] 
(1288.3000000000118, 5653.93) [a] 
(1288.2000000000119, 5656.37) [a] 
(1287.400000000012, 5675.81) [a] 
(1287.2500000000118, 5676.72) [a] 
(1287.150000000012, 6581.68) [a] 
(1286.350000000012, 6602.63) [a] 
(1286.2000000000119, 6603.23) [a] 
(1286.0500000000118, 6605.54) [a] 
(1285.7500000000118, 6795.33) [a] 
(1285.0500000000118, 6798.15) [a] 
(1285.0000000000118, 7822.36) [a] 
(1284.150000000012, 7845.73) [a] 
(1284.0000000000118, 7846.47) [a] 
},{
(2196.85, 406.1) [b] 
(2196.1, 407) [b] 
(2188.85, 426.1) [b] 
(2188.2, 430.9) [b] 
(2187.1499999999996, 443.2) [b] 
(2185.0499999999997, 450.1) [b] 
(2184.95, 450.2) [b] 
(2184.5, 450.3) [b] 
(2183.4, 453) [b] 
(2181.7000000000003, 453.2) [b] 
(2173.7000000000003, 453.3) [b] 
(2173.55, 453.7) [b] 
(2172.6500000000005, 454.5) [b] 
(2171.3500000000004, 455.2) [b] 
(2170.2000000000003, 455.4) [b] 
(2169.9500000000003, 455.5) [b] 
(2169.6000000000004, 455.8) [b] 
(2169.5000000000005, 456) [b] 
(2166.2000000000003, 456.3) [b] 
(2165.9500000000003, 456.4) [b] 
(2165.7000000000003, 456.7) [b] 
(2164.05, 456.9) [b] 
(2162.3, 457.7) [b] 
(2162.25, 457.8) [b] 
(2160.85, 458.2) [b] 
(2160.7, 458.8) [b] 
(2160.2, 459.2) [b] 
(2159.85, 460.2) [b] 
(2159.5499999999997, 461.7) [b] 
(2159.4999999999995, 462.1) [b] 
(2158.2499999999995, 462.3) [b] 
(2157.6499999999996, 463.9) [b] 
(2156.3499999999995, 464.4) [b] 
(2156.2499999999995, 467.9) [b] 
(2156.0999999999995, 469.8) [b] 
(2155.5999999999995, 471.7) [b] 
(2154.9999999999995, 475.1) [b] 
(2154.9499999999994, 478.1) [b] 
(2153.8499999999995, 482.5) [b] 
(2153.7499999999995, 487.5) [b] 
(2153.6499999999996, 505.4) [b] 
(2153.5499999999997, 511.1) [b] 
(2153.0499999999997, 522.6) [b] 
(2152.45, 531.8) [b] 
(2152.3999999999996, 555) [b] 
(2131.8999999999996, 597.1) [b] 
(2130.0999999999995, 597.5) [b] 
(2129.5499999999993, 597.6) [b] 
(2129.399999999999, 597.7) [b] 
(2128.749999999999, 597.9) [b] 
(2128.0499999999993, 598) [b] 
(2126.999999999999, 598.1) [b] 
(2126.899999999999, 598.3) [b] 
(2122.1999999999994, 598.7) [b] 
(2120.2499999999995, 599.7) [b] 
(2118.149999999999, 600) [b] 
(2115.7999999999993, 600.3) [b] 
(2114.7999999999993, 601.1) [b] 
(2114.6999999999994, 601.4) [b] 
(2113.749999999999, 601.5) [b] 
(2112.599999999999, 603.6) [b] 
(2112.349999999999, 604.6) [b] 
(2111.799999999999, 607.6) [b] 
(2111.349999999999, 609.9) [b] 
(2109.899999999999, 611.6) [b] 
(2109.749999999999, 616.2) [b] 
(2109.249999999999, 618.6) [b] 
(2108.999999999999, 623.5) [b] 
(2100.5499999999993, 623.8) [b] 
(2100.0499999999993, 624.9) [b] 
(2094.8499999999995, 627.5) [b] 
(2093.6499999999996, 633.3) [b] 
(2091.8999999999996, 633.8) [b] 
(2091.8499999999995, 633.9) [b] 
(2091.2999999999993, 634.1) [b] 
(2090.499999999999, 634.2) [b] 
(2087.499999999999, 634.8) [b] 
(2086.349999999999, 635.2) [b] 
(2081.599999999999, 635.9) [b] 
(2081.149999999999, 636) [b] 
(2078.7999999999993, 636.1) [b] 
(2077.149999999999, 636.4) [b] 
(2076.849999999999, 636.9) [b] 
(2076.199999999999, 637.1) [b] 
(2075.999999999999, 637.3) [b] 
(2075.6499999999987, 637.4) [b] 
(2075.199999999999, 637.5) [b] 
(2074.449999999999, 637.9) [b] 
(2073.749999999999, 638.1) [b] 
(2073.499999999999, 638.8) [b] 
(2072.849999999999, 639) [b] 
(2072.399999999999, 640.5) [b] 
(2070.5499999999993, 641.4) [b] 
(2069.649999999999, 641.5) [b] 
(2069.199999999999, 641.6) [b] 
(2068.549999999999, 643.2) [b] 
(2068.4999999999986, 645.9) [b] 
(2068.1999999999985, 646) [b] 
(2067.5999999999985, 649) [b] 
(2066.3999999999987, 651) [b] 
(2065.6499999999987, 652.1) [b] 
(2065.449999999999, 655.5) [b] 
(2065.249999999999, 659.9) [b] 
(2065.0499999999993, 667.6) [b] 
(2064.999999999999, 671.6) [b] 
(2064.749999999999, 687.5) [b] 
(2061.7999999999993, 687.8) [b] 
(2061.249999999999, 688.5) [b] 
(2061.099999999999, 701.3) [b] 
(2060.949999999999, 707.5) [b] 
(2060.749999999999, 711.6) [b] 
(2060.199999999999, 715) [b] 
(2059.449999999999, 724.1) [b] 
(2059.349999999999, 732) [b] 
(2058.899999999999, 734.9) [b] 
(2057.9499999999994, 736.1) [b] 
(2053.649999999999, 738.6) [b] 
(2053.149999999999, 739.4) [b] 
(2051.2999999999993, 739.5) [b] 
(2050.6999999999994, 739.6) [b] 
(2049.5499999999993, 740.5) [b] 
(2047.8999999999992, 740.6) [b] 
(2047.749999999999, 741) [b] 
(2047.399999999999, 741.8) [b] 
(2044.5499999999988, 742.7) [b] 
(2043.2999999999988, 742.9) [b] 
(2039.9499999999987, 743.1) [b] 
(2039.0499999999986, 743.2) [b] 
(2038.6499999999985, 743.3) [b] 
(2037.7999999999986, 743.5) [b] 
(2036.0999999999985, 743.7) [b] 
(2034.6999999999985, 744.2) [b] 
(2034.4999999999984, 744.3) [b] 
(2034.2999999999984, 744.9) [b] 
(2034.1499999999983, 745.1) [b] 
(2032.8499999999983, 745.3) [b] 
(2030.2999999999984, 746.3) [b] 
(2029.1499999999983, 746.5) [b] 
(2028.4999999999982, 746.9) [b] 
(2027.1499999999983, 747.4) [b] 
(2026.9999999999982, 747.9) [b] 
(2026.1999999999982, 748.5) [b] 
(2025.6999999999982, 748.7) [b] 
(2025.4999999999982, 748.8) [b] 
(2024.6499999999983, 749.3) [b] 
(2024.1499999999983, 749.5) [b] 
(2023.6999999999982, 750.4) [b] 
(2023.4499999999982, 750.9) [b] 
(2023.0999999999983, 751.1) [b] 
(2021.6999999999982, 752) [b] 
(2021.2999999999981, 752.5) [b] 
(2020.9499999999982, 754.4) [b] 
(2020.6499999999983, 757) [b] 
(2019.8999999999983, 761.8) [b] 
(2019.6999999999982, 764) [b] 
(2019.5999999999983, 766.7) [b] 
(1999.3499999999983, 770.5) [b] 
(1997.3499999999983, 771) [b] 
(1995.7499999999984, 771.2) [b] 
(1994.1499999999985, 771.7) [b] 
(1993.2499999999984, 771.8) [b] 
(1992.6499999999985, 772.2) [b] 
(1992.1999999999985, 772.3) [b] 
(1989.6999999999985, 772.9) [b] 
(1989.3499999999985, 773.1) [b] 
(1989.2999999999986, 773.5) [b] 
(1986.4999999999986, 773.6) [b] 
(1985.7999999999986, 773.7) [b] 
(1985.1999999999987, 774.5) [b] 
(1984.9999999999986, 775.3) [b] 
(1984.2999999999986, 775.7) [b] 
(1983.4499999999987, 776.1) [b] 
(1981.3499999999988, 776.6) [b] 
(1981.2999999999988, 776.7) [b] 
(1980.8499999999988, 777.8) [b] 
(1979.8999999999987, 778.8) [b] 
(1979.3999999999987, 779.3) [b] 
(1979.1499999999987, 783) [b] 
(1978.5999999999988, 783.8) [b] 
(1978.5499999999988, 786.7) [b] 
(1977.7499999999989, 788.5) [b] 
(1977.699999999999, 793.5) [b] 
(1977.599999999999, 793.6) [b] 
(1977.549999999999, 795.5) [b] 
(1976.149999999999, 797.1) [b] 
(1975.7499999999989, 805) [b] 
(1974.649999999999, 811.3) [b] 
(1974.599999999999, 815.4) [b] 
(1974.399999999999, 821.7) [b] 
(1973.4999999999989, 826.9) [b] 
(1973.149999999999, 846.4) [b] 
(1955.299999999999, 851.4) [b] 
(1954.1999999999991, 852.5) [b] 
(1953.999999999999, 852.6) [b] 
(1950.599999999999, 852.7) [b] 
(1944.949999999999, 852.8) [b] 
(1944.899999999999, 853) [b] 
(1944.799999999999, 853.3) [b] 
(1943.649999999999, 854.2) [b] 
(1943.099999999999, 854.6) [b] 
(1942.599999999999, 854.7) [b] 
(1940.999999999999, 855) [b] 
(1940.349999999999, 855.2) [b] 
(1939.749999999999, 855.4) [b] 
(1939.599999999999, 855.6) [b] 
(1938.949999999999, 856) [b] 
(1937.949999999999, 856.1) [b] 
(1937.7499999999989, 856.6) [b] 
(1934.599999999999, 856.7) [b] 
(1933.849999999999, 857.2) [b] 
(1933.249999999999, 858.1) [b] 
(1932.6999999999991, 859.7) [b] 
(1932.3499999999992, 866.4) [b] 
(1931.6999999999991, 867.3) [b] 
(1931.5999999999992, 871.1) [b] 
(1931.4999999999993, 874.8) [b] 
(1931.0999999999992, 876.1) [b] 
(1930.8999999999992, 882.1) [b] 
(1930.8499999999992, 883.8) [b] 
(1930.0999999999992, 885.6) [b] 
(1929.9499999999991, 890.7) [b] 
(1929.299999999999, 894.5) [b] 
(1928.9499999999991, 899.4) [b] 
(1928.249999999999, 901.3) [b] 
(1928.099999999999, 910.2) [b] 
(1927.799999999999, 914.3) [b] 
(1927.349999999999, 915) [b] 
(1927.249999999999, 916.6) [b] 
(1912.6499999999992, 928.3) [b] 
(1911.5999999999992, 928.7) [b] 
(1911.4999999999993, 929.5) [b] 
(1910.4999999999993, 929.8) [b] 
(1909.4999999999993, 929.9) [b] 
(1908.7999999999993, 930.4) [b] 
(1907.7999999999993, 930.6) [b] 
(1907.6999999999994, 931.6) [b] 
(1907.2999999999993, 931.8) [b] 
(1907.2499999999993, 931.9) [b] 
(1903.0999999999992, 933.5) [b] 
(1902.2999999999993, 935.2) [b] 
(1902.1499999999992, 935.4) [b] 
(1901.5499999999993, 936.1) [b] 
(1899.7999999999993, 936.8) [b] 
(1899.3499999999992, 936.9) [b] 
(1899.0499999999993, 937) [b] 
(1895.3999999999992, 937.1) [b] 
(1894.999999999999, 937.7) [b] 
(1891.849999999999, 938.7) [b] 
(1891.549999999999, 938.8) [b] 
(1890.9499999999991, 938.9) [b] 
(1890.749999999999, 941.1) [b] 
(1889.549999999999, 941.4) [b] 
(1889.149999999999, 942.7) [b] 
(1888.899999999999, 944.6) [b] 
(1888.449999999999, 944.9) [b] 
(1888.2999999999988, 946.9) [b] 
(1888.1499999999987, 948.2) [b] 
(1887.2499999999986, 948.8) [b] 
(1887.1499999999987, 949.7) [b] 
(1887.0499999999988, 950.1) [b] 
(1886.8499999999988, 951.8) [b] 
(1886.5499999999988, 952.7) [b] 
(1885.7499999999989, 953.4) [b] 
(1885.699999999999, 954.5) [b] 
(1884.7499999999989, 956.5) [b] 
(1883.649999999999, 959.6) [b] 
(1883.349999999999, 959.7) [b] 
(1882.799999999999, 985.1) [b] 
(1882.549999999999, 985.3) [b] 
(1882.499999999999, 988.8) [b] 
(1882.3999999999992, 992.7) [b] 
(1882.249999999999, 993.6) [b] 
(1881.799999999999, 1020) [b] 
(1881.749999999999, 1028) [b] 
(1881.549999999999, 1032) [b] 
(1881.049999999999, 1041) [b] 
(1880.599999999999, 1046) [b] 
(1879.599999999999, 1070) [b] 
(1879.249999999999, 1083) [b] 
(1872.299999999999, 1118) [b] 
(1867.499999999999, 1122) [b] 
(1865.499999999999, 1127) [b] 
(1861.549999999999, 1131) [b] 
(1860.499999999999, 1132) [b] 
(1854.349999999999, 1133) [b] 
(1851.199999999999, 1135) [b] 
(1849.3999999999987, 1136) [b] 
(1847.6499999999987, 1137) [b] 
(1844.949999999999, 1138) [b] 
(1843.449999999999, 1139) [b] 
(1842.1499999999987, 1140) [b] 
(1841.6499999999987, 1141) [b] 
(1817.4999999999986, 1143) [b] 
(1812.7999999999986, 1144) [b] 
(1810.9499999999985, 1146) [b] 
(1810.7499999999986, 1147) [b] 
(1810.1999999999987, 1148) [b] 
(1802.7999999999986, 1149) [b] 
(1802.3999999999985, 1150) [b] 
(1801.3999999999985, 1151) [b] 
(1801.2999999999986, 1152) [b] 
(1800.3999999999987, 1153) [b] 
(1799.7499999999986, 1154) [b] 
(1797.1499999999987, 1156) [b] 
(1796.2999999999988, 1159) [b] 
(1796.0999999999988, 1162) [b] 
(1795.9999999999989, 1164) [b] 
(1794.449999999999, 1166) [b] 
(1793.899999999999, 1171) [b] 
(1793.7499999999989, 1172) [b] 
(1793.4999999999989, 1177) [b] 
(1792.699999999999, 1178) [b] 
(1792.349999999999, 1180) [b] 
(1791.799999999999, 1185) [b] 
(1790.9499999999991, 1189) [b] 
(1790.549999999999, 1201) [b] 
(1787.649999999999, 1217) [b] 
(1786.9999999999989, 1220) [b] 
(1786.199999999999, 1225) [b] 
(1780.199999999999, 1226) [b] 
(1777.649999999999, 1227) [b] 
(1774.299999999999, 1228) [b] 
(1766.9999999999993, 1229) [b] 
(1766.9499999999994, 1230) [b] 
(1765.6999999999994, 1231) [b] 
(1747.549999999999, 1232) [b] 
(1736.199999999999, 1233) [b] 
(1733.749999999999, 1234) [b] 
(1730.749999999999, 1235) [b] 
(1728.6499999999992, 1236) [b] 
(1725.3999999999992, 1237) [b] 
(1723.8499999999992, 1238) [b] 
(1720.8999999999994, 1239) [b] 
(1720.1499999999996, 1240) [b] 
(1718.4999999999995, 1241) [b] 
(1716.9999999999995, 1242) [b] 
(1716.0499999999995, 1243) [b] 
(1714.7499999999995, 1244) [b] 
(1713.6999999999996, 1245) [b] 
(1712.1499999999996, 1246) [b] 
(1710.4499999999996, 1247) [b] 
(1709.9999999999995, 1248) [b] 
(1709.1499999999994, 1250) [b] 
(1708.8499999999995, 1251) [b] 
(1708.6999999999994, 1252) [b] 
(1708.3499999999995, 1253) [b] 
(1707.2499999999995, 1255) [b] 
(1706.6999999999996, 1256) [b] 
(1706.4999999999995, 1258) [b] 
(1706.1499999999996, 1259) [b] 
(1704.0499999999997, 1261) [b] 
(1703.2499999999998, 1267) [b] 
(1702.7499999999998, 1271) [b] 
(1702.5499999999997, 1274) [b] 
(1702.4499999999998, 1281) [b] 
(1702.35, 1286) [b] 
(1701.9499999999998, 1291) [b] 
(1701.3499999999997, 1292) [b] 
(1701.0499999999997, 1300) [b] 
(1700.9499999999998, 1301) [b] 
(1700.85, 1306) [b] 
(1680.6999999999998, 1313) [b] 
(1678.0499999999997, 1314) [b] 
(1675.9999999999995, 1315) [b] 
(1667.8999999999994, 1316) [b] 
(1666.2499999999995, 1317) [b] 
(1663.9499999999996, 1318) [b] 
(1663.3499999999997, 1319) [b] 
(1662.3499999999997, 1320) [b] 
(1662.1499999999999, 1321) [b] 
(1661.6499999999999, 1322) [b] 
(1659.35, 1323) [b] 
(1659.1, 1324) [b] 
(1658.8999999999999, 1327) [b] 
(1657.9499999999998, 1328) [b] 
(1657.2499999999998, 1329) [b] 
(1656.9999999999998, 1333) [b] 
(1656.1999999999998, 1337) [b] 
(1655.9999999999998, 1339) [b] 
(1655.8999999999999, 1346) [b] 
(1655.6499999999999, 1354) [b] 
(1655.4499999999998, 1355) [b] 
(1655.35, 1364) [b] 
(1655.0, 1365) [b] 
(1654.95, 1370) [b] 
(1654.9, 1376) [b] 
(1654.1000000000001, 1378) [b] 
(1653.95, 1388) [b] 
(1653.8500000000001, 1389) [b] 
(1653.5000000000002, 1402) [b] 
(1652.4500000000003, 1453) [b] 
(1652.0500000000002, 1467) [b] 
(1651.65, 1486) [b] 
(1649.8500000000001, 1496) [b] 
(1646.95, 1506) [b] 
(1639.6000000000001, 1507) [b] 
(1638.0500000000002, 1510) [b] 
(1636.0500000000002, 1511) [b] 
(1626.6000000000004, 1512) [b] 
(1624.1000000000004, 1514) [b] 
(1622.3500000000001, 1515) [b] 
(1617.8500000000001, 1516) [b] 
(1616.6000000000001, 1517) [b] 
(1615.7, 1518) [b] 
(1613.4, 1519) [b] 
(1613.15, 1520) [b] 
(1612.6000000000001, 1521) [b] 
(1612.1000000000001, 1524) [b] 
(1611.6000000000001, 1525) [b] 
(1611.15, 1526) [b] 
(1610.45, 1529) [b] 
(1610.4, 1533) [b] 
(1609.9, 1534) [b] 
(1609.2, 1535) [b] 
(1608.5, 1536) [b] 
(1608.35, 1538) [b] 
(1577.05, 1542) [b] 
(1575.55, 1543) [b] 
(1575.35, 1544) [b] 
(1573.85, 1545) [b] 
(1573.15, 1546) [b] 
(1572.1000000000001, 1547) [b] 
(1571.1000000000001, 1548) [b] 
(1568.8500000000001, 1549) [b] 
(1568.5000000000002, 1550) [b] 
(1568.4000000000003, 1551) [b] 
(1567.8500000000004, 1554) [b] 
(1567.0500000000004, 1555) [b] 
(1566.7500000000005, 1557) [b] 
(1566.6000000000004, 1558) [b] 
(1566.3500000000004, 1560) [b] 
(1565.7000000000003, 1561) [b] 
(1565.6000000000004, 1565) [b] 
(1565.0000000000002, 1569) [b] 
(1564.8000000000002, 1572) [b] 
(1564.7500000000002, 1576) [b] 
(1564.6000000000001, 1577) [b] 
(1564.5000000000002, 1582) [b] 
(1542.5000000000002, 1583) [b] 
(1537.4500000000003, 1584) [b] 
(1536.2500000000002, 1587) [b] 
(1535.6000000000001, 1588) [b] 
(1534.65, 1589) [b] 
(1533.6, 1590) [b] 
(1530.55, 1591) [b] 
(1528.55, 1592) [b] 
(1528.25, 1593) [b] 
(1528.1000000000001, 1594) [b] 
(1527.65, 1595) [b] 
(1526.3999999999999, 1597) [b] 
(1525.1999999999998, 1598) [b] 
(1524.0999999999997, 1600) [b] 
(1523.8499999999997, 1601) [b] 
(1523.3499999999997, 1602) [b] 
(1521.7999999999997, 1603) [b] 
(1521.1999999999998, 1606) [b] 
(1520.9499999999998, 1607) [b] 
(1520.8999999999999, 1608) [b] 
(1520.85, 1612) [b] 
(1520.75, 1615) [b] 
(1520.3, 1621) [b] 
(1519.55, 1630) [b] 
(1519.5, 1632) [b] 
(1519.25, 1633) [b] 
(1519.2, 1634) [b] 
(1518.75, 1635) [b] 
(1518.7, 1636) [b] 
(1518.2, 1643) [b] 
(1518.15, 1646) [b] 
(1517.9, 1649) [b] 
(1517.8000000000002, 1661) [b] 
(1517.6000000000001, 1666) [b] 
(1517.15, 1668) [b] 
(1516.9, 1672) [b] 
(1516.6000000000001, 1676) [b] 
(1516.5000000000002, 1689) [b] 
(1516.3500000000001, 1695) [b] 
(1516.0000000000002, 1700) [b] 
(1515.9000000000003, 1707) [b] 
(1515.6500000000003, 1728) [b] 
(1500.4000000000003, 1741) [b] 
(1499.7500000000002, 1742) [b] 
(1498.9500000000003, 1743) [b] 
(1492.6000000000001, 1746) [b] 
(1487.0500000000002, 1747) [b] 
(1482.2500000000002, 1748) [b] 
(1480.8500000000004, 1749) [b] 
(1480.2500000000005, 1750) [b] 
(1477.9500000000005, 1751) [b] 
(1477.0500000000006, 1752) [b] 
(1475.7000000000005, 1756) [b] 
(1474.0000000000005, 1758) [b] 
(1473.6000000000004, 1761) [b] 
(1473.5500000000004, 1764) [b] 
(1473.4000000000003, 1766) [b] 
(1473.0000000000002, 1770) [b] 
(1472.6000000000001, 1771) [b] 
(1472.5500000000002, 1773) [b] 
(1472.5000000000002, 1780) [b] 
(1471.3500000000001, 1784) [b] 
(1471.3000000000002, 1785) [b] 
(1471.15, 1787) [b] 
(1470.75, 1791) [b] 
(1470.1000000000001, 1799) [b] 
(1469.4, 1806) [b] 
(1445.8999999999999, 1808) [b] 
(1438.1500000000003, 1809) [b] 
(1434.4500000000005, 1810) [b] 
(1433.2500000000005, 1811) [b] 
(1431.2500000000005, 1812) [b] 
(1431.1500000000005, 1813) [b] 
(1429.1000000000006, 1814) [b] 
(1427.5500000000006, 1815) [b] 
(1427.3000000000006, 1817) [b] 
(1426.2000000000007, 1820) [b] 
(1425.8500000000008, 1824) [b] 
(1425.5500000000009, 1825) [b] 
(1425.450000000001, 1828) [b] 
(1425.200000000001, 1832) [b] 
(1424.700000000001, 1852) [b] 
(1423.850000000001, 1859) [b] 
(1423.400000000001, 1868) [b] 
(1423.350000000001, 1895) [b] 
(1423.100000000001, 1907) [b] 
(1422.150000000001, 1943) [b] 
(1422.000000000001, 1960) [b] 
(1421.950000000001, 2002) [b] 
(1421.650000000001, 2043) [b] 
(1420.950000000001, 2065) [b] 
(1397.500000000001, 2222) [b] 
(1391.5000000000011, 2223) [b] 
(1389.4500000000012, 2224) [b] 
(1385.4000000000012, 2225) [b] 
(1382.8000000000013, 2226) [b] 
(1381.4500000000014, 2227) [b] 
(1380.6000000000015, 2228) [b] 
(1380.5000000000016, 2229) [b] 
(1380.3000000000015, 2233) [b] 
(1379.7000000000016, 2238) [b] 
(1378.8000000000015, 2241) [b] 
(1378.7500000000016, 2242) [b] 
(1378.3500000000015, 2245) [b] 
(1377.8000000000015, 2253) [b] 
(1377.5500000000015, 2257) [b] 
(1377.3500000000015, 2267) [b] 
(1376.8500000000015, 2310) [b] 
(1376.7500000000016, 2344) [b] 
(1376.4000000000017, 2362) [b] 
(1376.3500000000017, 2445) [b] 
(1375.3000000000018, 2510) [b] 
(1368.5000000000018, 3012) [b] 
(1362.3000000000018, 3024) [b] 
(1362.1500000000017, 3025) [b] 
(1357.2500000000016, 3027) [b] 
(1355.2000000000016, 3028) [b] 
(1353.7000000000016, 3029) [b] 
(1352.6500000000015, 3030) [b] 
(1350.6000000000015, 3031) [b] 
(1349.6000000000015, 3032) [b] 
(1347.5500000000015, 3033) [b] 
(1345.5500000000015, 3034) [b] 
(1339.1500000000015, 3035) [b] 
(1339.0500000000015, 3036) [b] 
(1339.0000000000016, 3037) [b] 
(1338.5500000000015, 3039) [b] 
(1337.4500000000016, 3041) [b] 
(1337.4000000000017, 3043) [b] 
(1337.3500000000017, 3044) [b] 
(1336.6000000000017, 3047) [b] 
(1336.5000000000018, 3048) [b] 
(1336.150000000002, 3049) [b] 
(1334.9500000000019, 3051) [b] 
(1334.4500000000019, 3054) [b] 
(1333.7500000000018, 3057) [b] 
(1333.400000000002, 3061) [b] 
(1332.650000000002, 3064) [b] 
(1332.2500000000018, 3081) [b] 
(1331.900000000002, 3087) [b] 
(1330.7500000000018, 3094) [b] 
(1330.7000000000019, 3315) [b] 
(1330.350000000002, 3353) [b] 
(1329.650000000002, 3364) [b] 
(1304.9500000000019, 3448) [b] 
(1304.850000000002, 3449) [b] 
(1303.850000000002, 3450) [b] 
(1298.750000000002, 3451) [b] 
(1297.200000000002, 3453) [b] 
(1293.8000000000022, 3454) [b] 
(1292.2000000000023, 3457) [b] 
(1291.2000000000023, 3458) [b] 
(1290.3000000000022, 3459) [b] 
(1290.1000000000022, 3462) [b] 
(1290.0500000000022, 3467) [b] 
(1290.0000000000023, 3470) [b] 
(1289.5500000000022, 3473) [b] 
(1288.3500000000022, 3479) [b] 
(1287.1500000000021, 3485) [b] 
(1287.1000000000022, 3493) [b] 
(1286.8000000000022, 3509) [b] 
(1286.6500000000021, 3514) [b] 
(1286.4000000000021, 3521) [b] 
(1286.250000000002, 3525) [b] 
(1286.100000000002, 3545) [b] 
(1285.7000000000019, 3558) [b] 
(1285.600000000002, 3589) [b] 
(1285.100000000002, 3597) [b] 
(1284.150000000002, 3761) [b] 
(1284.0000000000018, 3814) [b] 
},{
(2151.9500000000003, 6.232) [c] 
(2148.05, 6.599) [c] 
(2136.9500000000003, 6.708) [c] 
(2126.65, 7.408) [c] 
(2126.4500000000003, 7.689) [c] 
(2124.7500000000005, 7.785) [c] 
(2124.4000000000005, 7.811) [c] 
(2124.0000000000005, 7.974) [c] 
(2123.3500000000004, 8.148) [c] 
(2123.2000000000003, 8.149) [c] 
(2113.3, 8.223) [c] 
(2102.1000000000004, 8.297) [c] 
(2095.05, 8.371) [c] 
(2094.8500000000004, 8.444) [c] 
(2094.5000000000005, 8.507) [c] 
(2093.9000000000005, 8.674) [c] 
(2093.3500000000004, 8.707) [c] 
(2082.2500000000005, 8.787) [c] 
(2072.3500000000004, 8.805) [c] 
(2072.2500000000005, 8.983) [c] 
(2058.6500000000005, 9.595) [c] 
(2058.3000000000006, 9.627) [c] 
(2058.100000000001, 10.24) [c] 
(2057.8000000000006, 10.28) [c] 
(2057.6500000000005, 10.29) [c] 
(2043.5500000000006, 10.45) [c] 
(2042.5000000000007, 10.49) [c] 
(2041.6500000000008, 10.5) [c] 
(2041.6000000000008, 10.53) [c] 
(2040.8000000000009, 10.55) [c] 
(2040.6000000000008, 10.56) [c] 
(2040.250000000001, 10.57) [c] 
(2038.8500000000008, 10.69) [c] 
(2038.250000000001, 10.72) [c] 
(2038.1000000000008, 10.76) [c] 
(2036.8000000000009, 10.82) [c] 
(2025.5500000000009, 10.89) [c] 
(2011.250000000001, 10.91) [c] 
(2009.700000000001, 10.92) [c] 
(1999.8000000000009, 10.94) [c] 
(1999.500000000001, 11.48) [c] 
(1997.4000000000008, 11.52) [c] 
(1997.1000000000008, 11.54) [c] 
(1993.6000000000008, 11.69) [c] 
(1992.750000000001, 11.85) [c] 
(1991.6000000000008, 11.95) [c] 
(1988.8500000000008, 12.05) [c] 
(1988.500000000001, 12.1) [c] 
(1988.250000000001, 12.15) [c] 
(1987.500000000001, 12.16) [c] 
(1987.0500000000009, 12.22) [c] 
(1986.950000000001, 12.23) [c] 
(1983.100000000001, 12.32) [c] 
(1982.2500000000011, 12.38) [c] 
(1981.800000000001, 12.63) [c] 
(1981.7500000000011, 12.66) [c] 
(1980.800000000001, 12.87) [c] 
(1978.9500000000012, 12.89) [c] 
(1970.9500000000012, 13.02) [c] 
(1969.6000000000013, 13.07) [c] 
(1965.2500000000011, 13.1) [c] 
(1964.850000000001, 13.14) [c] 
(1961.5000000000011, 13.17) [c] 
(1961.2500000000011, 13.19) [c] 
(1960.7500000000011, 13.2) [c] 
(1959.2500000000011, 13.22) [c] 
(1958.850000000001, 13.24) [c] 
(1958.500000000001, 13.25) [c] 
(1958.3000000000009, 13.26) [c] 
(1957.6000000000008, 13.27) [c] 
(1956.6500000000008, 13.3) [c] 
(1956.3500000000008, 13.34) [c] 
(1955.1500000000008, 13.4) [c] 
(1954.8500000000008, 13.41) [c] 
(1949.2000000000007, 13.66) [c] 
(1948.0500000000006, 13.72) [c] 
(1947.1000000000006, 13.75) [c] 
(1947.0000000000007, 13.77) [c] 
(1945.8500000000006, 13.83) [c] 
(1945.1000000000006, 13.86) [c] 
(1945.0000000000007, 13.92) [c] 
(1944.3500000000006, 13.96) [c] 
(1942.4500000000005, 14.01) [c] 
(1942.3500000000006, 14.15) [c] 
(1941.6000000000006, 14.17) [c] 
(1940.3000000000006, 14.21) [c] 
(1938.4500000000007, 14.26) [c] 
(1936.0000000000007, 14.33) [c] 
(1926.9000000000008, 14.34) [c] 
(1925.7000000000007, 14.35) [c] 
(1925.0500000000006, 14.38) [c] 
(1924.2500000000007, 14.39) [c] 
(1923.9500000000007, 14.55) [c] 
(1923.8000000000006, 14.64) [c] 
(1923.1000000000006, 14.65) [c] 
(1921.7500000000007, 14.66) [c] 
(1920.9000000000008, 14.67) [c] 
(1915.5000000000007, 14.69) [c] 
(1911.1000000000006, 14.75) [c] 
(1909.0000000000007, 14.77) [c] 
(1908.4000000000008, 14.78) [c] 
(1908.2500000000007, 14.83) [c] 
(1906.3000000000006, 14.86) [c] 
(1905.9000000000005, 14.87) [c] 
(1904.7000000000005, 14.88) [c] 
(1904.6500000000005, 14.91) [c] 
(1903.8000000000006, 14.92) [c] 
(1895.4500000000005, 15.01) [c] 
(1894.9000000000005, 15.1) [c] 
(1894.4000000000005, 15.18) [c] 
(1893.9500000000005, 15.22) [c] 
(1892.3500000000006, 15.35) [c] 
(1892.1500000000005, 15.38) [c] 
(1892.0000000000005, 15.39) [c] 
(1889.1500000000005, 15.49) [c] 
(1888.8500000000006, 15.53) [c] 
(1886.6500000000005, 15.56) [c] 
(1886.6000000000006, 15.63) [c] 
(1886.1000000000006, 15.65) [c] 
(1884.4500000000005, 15.66) [c] 
(1880.2000000000005, 15.67) [c] 
(1880.0500000000004, 15.7) [c] 
(1880.0000000000005, 15.8) [c] 
(1879.1000000000004, 15.88) [c] 
(1878.3000000000004, 15.89) [c] 
(1878.0000000000005, 15.92) [c] 
(1877.8000000000004, 15.95) [c] 
(1877.4500000000005, 15.98) [c] 
(1872.6500000000005, 16.02) [c] 
(1869.8000000000006, 16.06) [c] 
(1869.5000000000007, 16.08) [c] 
(1868.7500000000007, 16.1) [c] 
(1867.9000000000008, 16.11) [c] 
(1855.5000000000007, 16.14) [c] 
(1855.3500000000006, 16.23) [c] 
(1852.8500000000006, 16.24) [c] 
(1851.8500000000006, 16.27) [c] 
(1850.7000000000005, 16.35) [c] 
(1850.2000000000005, 16.39) [c] 
(1849.9000000000005, 16.46) [c] 
(1848.8000000000006, 16.55) [c] 
(1848.6000000000006, 16.64) [c] 
(1848.3000000000006, 16.74) [c] 
(1848.0500000000006, 16.81) [c] 
(1847.8000000000006, 16.94) [c] 
(1847.3500000000006, 17) [c] 
(1846.1500000000005, 17.02) [c] 
(1845.0000000000005, 17.11) [c] 
(1843.9500000000005, 17.17) [c] 
(1842.9500000000005, 17.22) [c] 
(1842.8500000000006, 17.39) [c] 
(1842.5000000000007, 17.59) [c] 
(1842.4500000000007, 17.6) [c] 
(1841.8500000000008, 17.72) [c] 
(1841.3000000000009, 17.88) [c] 
(1841.250000000001, 17.96) [c] 
(1841.000000000001, 17.97) [c] 
(1840.8500000000008, 18.02) [c] 
(1840.000000000001, 18.04) [c] 
(1838.700000000001, 18.14) [c] 
(1838.500000000001, 18.21) [c] 
(1837.700000000001, 18.3) [c] 
(1836.950000000001, 18.33) [c] 
(1836.900000000001, 18.4) [c] 
(1836.300000000001, 18.48) [c] 
(1820.4500000000012, 18.54) [c] 
(1815.5000000000011, 18.6) [c] 
(1815.2000000000012, 18.62) [c] 
(1811.050000000001, 18.64) [c] 
(1810.350000000001, 18.65) [c] 
(1807.850000000001, 18.69) [c] 
(1806.600000000001, 18.71) [c] 
(1805.650000000001, 18.87) [c] 
(1804.0000000000011, 18.91) [c] 
(1803.9500000000012, 19.02) [c] 
(1800.1500000000012, 19.04) [c] 
(1798.5500000000013, 19.11) [c] 
(1797.5500000000013, 19.12) [c] 
(1797.2000000000014, 19.18) [c] 
(1796.5500000000013, 19.3) [c] 
(1796.4000000000012, 19.33) [c] 
(1796.2000000000012, 19.35) [c] 
(1780.800000000001, 19.43) [c] 
(1780.650000000001, 19.45) [c] 
(1780.500000000001, 19.47) [c] 
(1779.950000000001, 19.48) [c] 
(1777.150000000001, 19.51) [c] 
(1776.050000000001, 19.52) [c] 
(1775.600000000001, 19.55) [c] 
(1773.850000000001, 19.61) [c] 
(1772.5000000000011, 19.66) [c] 
(1764.4500000000012, 19.71) [c] 
(1763.3500000000013, 20) [c] 
(1762.1500000000012, 20.03) [c] 
(1761.2500000000011, 20.16) [c] 
(1759.1500000000012, 20.28) [c] 
(1759.0000000000011, 20.38) [c] 
(1757.350000000001, 20.45) [c] 
(1757.150000000001, 20.48) [c] 
(1756.600000000001, 20.54) [c] 
(1756.550000000001, 20.55) [c] 
(1754.100000000001, 20.56) [c] 
(1752.050000000001, 20.65) [c] 
(1751.9500000000012, 20.66) [c] 
(1749.8500000000013, 20.69) [c] 
(1749.1500000000012, 20.7) [c] 
(1748.4500000000012, 20.84) [c] 
(1748.1500000000012, 20.9) [c] 
(1746.6000000000013, 20.92) [c] 
(1746.0500000000013, 20.98) [c] 
(1745.7500000000014, 21.05) [c] 
(1745.5500000000013, 21.06) [c] 
(1744.4500000000014, 21.17) [c] 
(1739.8000000000013, 21.19) [c] 
(1731.4000000000012, 21.2) [c] 
(1730.5000000000011, 21.23) [c] 
(1728.9500000000012, 21.33) [c] 
(1727.9500000000012, 21.38) [c] 
(1727.7000000000012, 21.46) [c] 
(1723.1000000000013, 21.5) [c] 
(1722.8500000000013, 21.53) [c] 
(1722.2500000000014, 21.57) [c] 
(1721.5000000000014, 21.62) [c] 
(1721.0000000000014, 21.64) [c] 
(1720.8000000000013, 21.7) [c] 
(1720.3500000000013, 21.71) [c] 
(1719.7000000000012, 21.73) [c] 
(1719.550000000001, 21.77) [c] 
(1719.350000000001, 21.78) [c] 
(1718.900000000001, 21.8) [c] 
(1717.700000000001, 21.82) [c] 
(1717.350000000001, 21.83) [c] 
(1715.100000000001, 21.86) [c] 
(1714.450000000001, 21.92) [c] 
(1710.950000000001, 21.93) [c] 
(1710.600000000001, 21.94) [c] 
(1710.0000000000011, 21.96) [c] 
(1709.4000000000012, 21.97) [c] 
(1708.9500000000012, 21.98) [c] 
(1708.7500000000011, 22) [c] 
(1708.350000000001, 22.02) [c] 
(1708.0000000000011, 22.04) [c] 
(1706.0000000000011, 22.05) [c] 
(1705.4500000000012, 22.08) [c] 
(1703.2000000000012, 22.09) [c] 
(1703.1500000000012, 22.1) [c] 
(1702.7000000000012, 22.13) [c] 
(1700.7500000000011, 22.15) [c] 
(1699.4500000000012, 22.16) [c] 
(1699.3500000000013, 22.24) [c] 
(1698.1000000000013, 22.26) [c] 
(1697.7000000000012, 22.29) [c] 
(1696.800000000001, 22.38) [c] 
(1696.7500000000011, 22.4) [c] 
(1696.550000000001, 22.41) [c] 
(1694.800000000001, 22.42) [c] 
(1694.600000000001, 22.43) [c] 
(1694.2500000000011, 22.44) [c] 
(1692.800000000001, 22.45) [c] 
(1691.9500000000012, 22.46) [c] 
(1691.1500000000012, 22.53) [c] 
(1689.4500000000014, 22.55) [c] 
(1689.3000000000013, 22.56) [c] 
(1688.4000000000012, 22.57) [c] 
(1687.6000000000013, 22.6) [c] 
(1687.5000000000014, 22.61) [c] 
(1683.4500000000012, 22.62) [c] 
(1682.7500000000011, 22.65) [c] 
(1682.2500000000011, 22.67) [c] 
(1681.4500000000012, 22.68) [c] 
(1680.9500000000012, 22.75) [c] 
(1678.900000000001, 22.76) [c] 
(1678.350000000001, 22.77) [c] 
(1678.100000000001, 22.8) [c] 
(1677.5000000000011, 22.84) [c] 
(1675.9000000000012, 22.94) [c] 
(1674.7500000000011, 22.96) [c] 
(1674.2000000000012, 23) [c] 
(1673.8500000000013, 23.02) [c] 
(1673.0000000000014, 23.05) [c] 
(1672.6000000000013, 23.06) [c] 
(1672.3500000000013, 23.08) [c] 
(1669.6000000000013, 23.09) [c] 
(1669.3500000000013, 23.1) [c] 
(1668.9000000000012, 23.11) [c] 
(1668.7500000000011, 23.14) [c] 
(1666.9500000000012, 23.15) [c] 
(1664.800000000001, 23.2) [c] 
(1663.800000000001, 23.22) [c] 
(1663.4500000000012, 23.23) [c] 
(1661.800000000001, 23.33) [c] 
(1661.650000000001, 23.34) [c] 
(1659.750000000001, 23.38) [c] 
(1658.6000000000008, 23.39) [c] 
(1657.1500000000008, 23.43) [c] 
(1656.000000000001, 23.48) [c] 
(1655.950000000001, 23.56) [c] 
(1655.700000000001, 23.6) [c] 
(1655.350000000001, 23.67) [c] 
(1655.300000000001, 23.69) [c] 
(1653.1000000000013, 23.72) [c] 
(1652.1000000000013, 23.77) [c] 
(1650.9000000000012, 23.78) [c] 
(1649.1500000000012, 23.79) [c] 
(1648.4500000000012, 23.81) [c] 
(1648.2000000000012, 23.9) [c] 
(1647.7000000000012, 23.92) [c] 
(1647.550000000001, 23.95) [c] 
(1646.9500000000012, 23.98) [c] 
(1646.800000000001, 24.03) [c] 
(1644.7500000000011, 24.11) [c] 
(1643.550000000001, 24.13) [c] 
(1643.350000000001, 24.25) [c] 
(1642.5000000000011, 24.31) [c] 
(1641.6000000000013, 24.32) [c] 
(1641.4000000000012, 24.35) [c] 
(1639.9500000000012, 24.39) [c] 
(1639.4000000000012, 24.4) [c] 
(1638.9500000000012, 24.42) [c] 
(1638.3500000000013, 24.45) [c] 
(1638.1000000000013, 24.51) [c] 
(1637.4000000000012, 24.54) [c] 
(1636.2500000000011, 24.64) [c] 
(1634.2500000000011, 24.67) [c] 
(1633.0000000000011, 24.79) [c] 
(1632.600000000001, 24.93) [c] 
(1632.450000000001, 25.02) [c] 
(1632.000000000001, 25.16) [c] 
(1631.950000000001, 25.19) [c] 
(1631.150000000001, 25.21) [c] 
(1629.900000000001, 25.24) [c] 
(1629.300000000001, 25.35) [c] 
(1628.850000000001, 25.37) [c] 
(1627.700000000001, 25.39) [c] 
(1627.200000000001, 25.4) [c] 
(1626.750000000001, 25.47) [c] 
(1625.900000000001, 25.48) [c] 
(1625.650000000001, 25.6) [c] 
(1625.400000000001, 25.64) [c] 
(1624.900000000001, 25.68) [c] 
(1622.100000000001, 25.74) [c] 
(1621.300000000001, 25.77) [c] 
(1621.150000000001, 25.83) [c] 
(1621.100000000001, 26.01) [c] 
(1620.900000000001, 26.06) [c] 
(1619.450000000001, 26.15) [c] 
(1619.000000000001, 26.16) [c] 
(1615.150000000001, 26.18) [c] 
(1614.650000000001, 26.21) [c] 
(1613.950000000001, 26.37) [c] 
(1602.750000000001, 26.4) [c] 
(1602.200000000001, 26.47) [c] 
(1602.0500000000009, 26.6) [c] 
(1601.950000000001, 26.61) [c] 
(1601.750000000001, 26.68) [c] 
(1601.5500000000009, 26.7) [c] 
(1599.8000000000009, 26.72) [c] 
(1598.950000000001, 26.83) [c] 
(1598.700000000001, 26.85) [c] 
(1597.8000000000009, 26.87) [c] 
(1596.500000000001, 26.97) [c] 
(1595.900000000001, 26.98) [c] 
(1595.600000000001, 27.05) [c] 
(1589.950000000001, 27.16) [c] 
(1582.500000000001, 27.19) [c] 
(1573.3500000000008, 27.29) [c] 
(1572.9500000000007, 27.3) [c] 
(1572.0500000000006, 27.31) [c] 
(1571.6000000000006, 27.32) [c] 
(1570.8000000000006, 27.42) [c] 
(1570.1500000000005, 27.57) [c] 
(1569.6500000000005, 27.58) [c] 
(1569.5000000000005, 27.61) [c] 
(1569.3500000000004, 27.71) [c] 
(1568.6000000000004, 27.76) [c] 
(1567.7500000000005, 27.82) [c] 
(1567.7000000000005, 27.83) [c] 
(1566.0000000000005, 27.88) [c] 
(1565.9000000000005, 27.9) [c] 
(1563.2000000000005, 27.94) [c] 
(1563.1500000000005, 28.06) [c] 
(1560.6000000000006, 28.08) [c] 
(1546.0500000000006, 28.12) [c] 
(1541.5500000000006, 28.16) [c] 
(1540.5500000000006, 28.21) [c] 
(1539.6000000000006, 28.28) [c] 
(1538.6500000000005, 28.29) [c] 
(1538.3500000000006, 28.34) [c] 
(1537.0000000000007, 28.37) [c] 
(1534.7000000000007, 28.43) [c] 
(1532.5500000000006, 28.55) [c] 
(1532.5000000000007, 28.61) [c] 
(1532.2500000000007, 28.67) [c] 
(1531.1500000000008, 28.69) [c] 
(1531.1000000000008, 28.87) [c] 
(1530.6500000000008, 28.88) [c] 
(1530.6000000000008, 28.93) [c] 
(1530.3500000000008, 29.06) [c] 
(1529.750000000001, 29.17) [c] 
(1529.3500000000008, 29.21) [c] 
(1529.250000000001, 29.24) [c] 
(1528.450000000001, 29.25) [c] 
(1528.400000000001, 29.36) [c] 
(1528.250000000001, 29.38) [c] 
(1525.5500000000009, 29.48) [c] 
(1525.250000000001, 29.53) [c] 
(1525.150000000001, 29.58) [c] 
(1523.300000000001, 29.59) [c] 
(1523.2500000000011, 29.67) [c] 
(1523.1500000000012, 29.78) [c] 
(1522.8500000000013, 29.84) [c] 
(1522.7000000000012, 29.86) [c] 
(1522.0000000000011, 29.88) [c] 
(1521.9500000000012, 29.9) [c] 
(1521.800000000001, 29.94) [c] 
(1519.900000000001, 29.95) [c] 
(1519.400000000001, 29.96) [c] 
(1518.700000000001, 29.97) [c] 
(1516.600000000001, 30.3) [c] 
(1516.200000000001, 30.39) [c] 
(1515.750000000001, 30.44) [c] 
(1515.700000000001, 30.46) [c] 
(1515.3000000000009, 30.47) [c] 
(1514.3000000000009, 30.49) [c] 
(1514.250000000001, 30.57) [c] 
(1513.0500000000009, 30.65) [c] 
(1512.250000000001, 30.72) [c] 
(1511.6000000000008, 30.75) [c] 
(1510.9500000000007, 31) [c] 
(1510.1500000000008, 31.03) [c] 
(1509.9000000000008, 31.05) [c] 
(1509.2000000000007, 31.06) [c] 
(1508.6500000000008, 31.17) [c] 
(1506.8500000000008, 31.32) [c] 
(1506.1000000000008, 31.33) [c] 
(1505.7000000000007, 31.48) [c] 
(1505.3000000000006, 31.5) [c] 
(1503.9500000000007, 31.53) [c] 
(1503.8000000000006, 31.58) [c] 
(1501.7500000000007, 31.77) [c] 
(1501.4000000000008, 31.78) [c] 
(1500.6000000000008, 31.82) [c] 
(1498.4000000000008, 31.83) [c] 
(1498.2000000000007, 31.84) [c] 
(1497.7000000000007, 31.89) [c] 
(1497.1500000000008, 31.9) [c] 
(1497.1000000000008, 31.96) [c] 
(1496.8000000000009, 31.99) [c] 
(1496.6500000000008, 32.06) [c] 
(1496.3500000000008, 32.09) [c] 
(1495.8500000000008, 32.14) [c] 
(1495.8000000000009, 32.36) [c] 
(1493.6500000000008, 32.38) [c] 
(1493.6000000000008, 32.39) [c] 
(1493.4000000000008, 32.41) [c] 
(1490.6000000000008, 32.53) [c] 
(1490.3500000000008, 32.59) [c] 
(1490.1500000000008, 32.67) [c] 
(1489.3500000000008, 32.68) [c] 
(1489.000000000001, 32.85) [c] 
(1487.750000000001, 32.88) [c] 
(1487.3000000000009, 32.89) [c] 
(1486.500000000001, 33.01) [c] 
(1485.8500000000008, 33.12) [c] 
(1485.0500000000009, 33.15) [c] 
(1484.8500000000008, 33.2) [c] 
(1483.9000000000008, 33.21) [c] 
(1483.8500000000008, 33.27) [c] 
(1483.8000000000009, 33.31) [c] 
(1483.0500000000009, 33.34) [c] 
(1482.950000000001, 33.5) [c] 
(1482.8000000000009, 33.54) [c] 
(1482.000000000001, 33.63) [c] 
(1481.150000000001, 33.66) [c] 
(1480.950000000001, 33.71) [c] 
(1480.500000000001, 33.73) [c] 
(1480.000000000001, 33.75) [c] 
(1478.6000000000008, 33.82) [c] 
(1477.500000000001, 33.92) [c] 
(1477.450000000001, 33.93) [c] 
(1476.550000000001, 33.99) [c] 
(1476.300000000001, 34.01) [c] 
(1475.850000000001, 34.04) [c] 
(1472.600000000001, 34.19) [c] 
(1472.5000000000011, 34.28) [c] 
(1471.550000000001, 34.31) [c] 
(1470.950000000001, 34.42) [c] 
(1470.750000000001, 34.66) [c] 
(1470.250000000001, 34.79) [c] 
(1468.950000000001, 34.84) [c] 
(1468.800000000001, 34.96) [c] 
(1468.550000000001, 35.07) [c] 
(1467.400000000001, 35.15) [c] 
(1463.200000000001, 35.18) [c] 
(1462.0500000000009, 35.33) [c] 
(1461.6500000000008, 35.43) [c] 
(1459.8000000000009, 35.59) [c] 
(1459.3500000000008, 35.68) [c] 
(1457.1000000000008, 35.84) [c] 
(1456.7000000000007, 35.89) [c] 
(1456.5000000000007, 36.06) [c] 
(1456.2000000000007, 36.13) [c] 
(1455.0000000000007, 36.17) [c] 
(1454.5500000000006, 36.18) [c] 
(1452.2000000000005, 36.19) [c] 
(1451.2500000000005, 36.24) [c] 
(1451.2000000000005, 36.33) [c] 
(1451.1000000000006, 36.34) [c] 
(1450.8000000000006, 36.37) [c] 
(1450.0500000000006, 36.4) [c] 
(1449.8500000000006, 36.41) [c] 
(1448.7500000000007, 36.44) [c] 
(1446.9500000000007, 36.46) [c] 
(1444.8500000000008, 36.47) [c] 
(1443.0500000000009, 36.5) [c] 
(1442.8000000000009, 36.51) [c] 
(1442.6500000000008, 36.54) [c] 
(1442.2500000000007, 36.57) [c] 
(1442.2000000000007, 36.59) [c] 
(1441.4500000000007, 36.77) [c] 
(1440.7500000000007, 36.98) [c] 
(1435.6000000000006, 37) [c] 
(1435.5000000000007, 37.06) [c] 
(1435.1500000000008, 37.07) [c] 
(1434.7500000000007, 37.09) [c] 
(1431.9500000000007, 37.13) [c] 
(1431.8000000000006, 37.14) [c] 
(1431.0500000000006, 37.15) [c] 
(1428.9000000000005, 37.18) [c] 
(1428.7000000000005, 37.3) [c] 
(1428.5000000000005, 37.33) [c] 
(1428.3500000000004, 37.51) [c] 
(1427.9000000000003, 37.53) [c] 
(1427.8500000000004, 37.54) [c] 
(1427.3000000000002, 37.64) [c] 
(1427.2500000000002, 37.69) [c] 
(1427.1500000000003, 37.81) [c] 
(1426.8000000000004, 37.87) [c] 
(1421.4000000000003, 37.96) [c] 
(1421.1000000000004, 38) [c] 
(1421.0000000000005, 38.11) [c] 
(1420.7500000000005, 38.23) [c] 
(1420.5500000000004, 38.24) [c] 
(1420.0000000000005, 38.25) [c] 
(1418.7500000000005, 38.51) [c] 
(1413.3000000000004, 38.61) [c] 
(1412.6000000000004, 38.66) [c] 
(1412.2000000000003, 38.69) [c] 
(1411.8000000000002, 38.76) [c] 
(1411.7500000000002, 38.87) [c] 
(1411.6000000000001, 38.96) [c] 
(1411.15, 39) [c] 
(1410.95, 39.09) [c] 
(1410.2, 39.11) [c] 
(1408.6, 39.36) [c] 
(1407.6999999999998, 39.37) [c] 
(1406.9499999999998, 39.69) [c] 
(1406.4499999999998, 39.82) [c] 
(1406.3999999999999, 39.86) [c] 
(1405.85, 39.96) [c] 
(1404.5, 40.02) [c] 
(1404.3, 40.13) [c] 
(1404.2, 40.22) [c] 
(1402.6000000000001, 40.26) [c] 
(1401.1500000000003, 40.35) [c] 
(1400.0500000000004, 40.57) [c] 
(1399.7500000000005, 40.64) [c] 
(1399.3000000000004, 40.66) [c] 
(1398.8000000000004, 40.68) [c] 
(1398.6500000000003, 40.76) [c] 
(1398.4500000000003, 40.78) [c] 
(1398.2500000000002, 40.79) [c] 
(1396.8000000000004, 40.83) [c] 
(1396.4500000000005, 40.91) [c] 
(1396.4000000000005, 41.05) [c] 
(1395.9000000000005, 41.29) [c] 
(1395.8000000000006, 41.41) [c] 
(1394.3000000000006, 41.54) [c] 
(1394.2000000000007, 41.74) [c] 
(1394.0500000000006, 41.85) [c] 
(1392.9000000000005, 41.9) [c] 
(1391.2000000000005, 41.99) [c] 
(1389.8000000000004, 42.05) [c] 
(1389.3500000000004, 42.09) [c] 
(1388.9000000000003, 42.11) [c] 
(1388.4500000000003, 42.12) [c] 
(1388.1500000000003, 42.17) [c] 
(1387.3000000000004, 42.22) [c] 
(1387.2000000000005, 42.31) [c] 
(1387.1000000000006, 42.44) [c] 
(1386.1500000000005, 42.48) [c] 
(1385.6000000000006, 42.6) [c] 
(1385.2000000000005, 42.72) [c] 
(1385.1000000000006, 42.92) [c] 
(1384.7500000000007, 43) [c] 
(1383.2500000000007, 43.17) [c] 
(1382.3500000000006, 43.22) [c] 
(1382.0500000000006, 43.26) [c] 
(1381.9000000000005, 43.49) [c] 
(1381.7000000000005, 43.63) [c] 
(1381.3000000000004, 43.74) [c] 
(1379.6500000000003, 43.76) [c] 
(1379.4500000000003, 43.77) [c] 
(1378.8000000000002, 43.85) [c] 
(1378.0500000000002, 43.92) [c] 
(1378.0000000000002, 43.99) [c] 
(1377.7000000000003, 44.14) [c] 
(1377.0500000000002, 44.38) [c] 
(1374.65, 44.4) [c] 
(1374.25, 44.41) [c] 
(1372.35, 44.46) [c] 
(1370.9499999999998, 44.69) [c] 
(1370.8999999999999, 44.79) [c] 
(1370.35, 44.81) [c] 
(1370.1499999999999, 45) [c] 
(1369.4499999999998, 45.28) [c] 
(1369.2999999999997, 45.43) [c] 
(1368.5999999999997, 45.59) [c] 
(1367.1999999999996, 45.64) [c] 
(1366.6999999999996, 45.74) [c] 
(1366.4999999999995, 45.83) [c] 
(1365.6999999999996, 45.87) [c] 
(1364.8999999999996, 45.89) [c] 
(1364.6999999999996, 46.11) [c] 
(1364.0999999999997, 46.17) [c] 
(1363.3499999999997, 46.3) [c] 
(1362.7499999999998, 46.55) [c] 
(1362.3999999999999, 46.73) [c] 
(1362.3, 47.05) [c] 
(1362.05, 47.31) [c] 
(1360.7499999999998, 47.32) [c] 
(1360.6499999999999, 47.4) [c] 
(1360.3, 47.56) [c] 
(1360.05, 47.63) [c] 
(1359.7, 47.66) [c] 
(1358.25, 47.67) [c] 
(1357.4, 47.84) [c] 
(1357.25, 47.96) [c] 
(1355.6, 48.07) [c] 
(1354.6499999999999, 48.36) [c] 
(1353.6499999999999, 48.42) [c] 
(1352.85, 48.46) [c] 
(1352.5, 48.53) [c] 
(1352.35, 48.73) [c] 
(1352.3, 48.93) [c] 
(1352.25, 49.06) [c] 
(1351.45, 49.37) [c] 
(1351.3500000000001, 49.55) [c] 
(1350.7500000000002, 49.91) [c] 
(1350.5000000000002, 50.03) [c] 
(1350.3000000000002, 50.84) [c] 
(1349.8500000000001, 51.39) [c] 
(1349.8000000000002, 51.57) [c] 
(1349.7500000000002, 51.67) [c] 
(1349.1000000000001, 51.74) [c] 
(1349.0000000000002, 51.83) [c] 
(1348.3000000000002, 52.35) [c] 
(1347.4500000000003, 53.02) [c] 
(1347.1000000000004, 53.18) [c] 
(1346.5000000000005, 53.49) [c] 
(1346.4500000000005, 53.61) [c] 
(1345.9500000000005, 53.69) [c] 
(1345.9000000000005, 53.97) [c] 
(1345.5000000000005, 54.08) [c] 
(1344.7500000000005, 54.15) [c] 
(1344.1000000000004, 54.19) [c] 
(1344.0500000000004, 54.21) [c] 
(1343.2500000000005, 54.22) [c] 
(1343.1000000000004, 54.32) [c] 
(1342.6000000000004, 54.44) [c] 
(1342.1500000000003, 54.59) [c] 
(1341.3500000000004, 54.68) [c] 
(1340.9500000000003, 54.94) [c] 
(1340.8500000000004, 55.38) [c] 
(1340.4000000000003, 55.85) [c] 
(1340.2000000000003, 55.93) [c] 
(1340.0000000000002, 56.41) [c] 
(1339.9500000000003, 57.24) [c] 
(1339.9000000000003, 57.45) [c] 
(1339.3500000000004, 57.54) [c] 
(1339.2500000000005, 57.79) [c] 
(1339.2000000000005, 57.87) [c] 
(1338.7000000000005, 57.91) [c] 
(1338.5500000000004, 58.18) [c] 
(1338.5000000000005, 58.61) [c] 
(1338.3500000000004, 59.09) [c] 
(1338.0500000000004, 59.32) [c] 
(1337.4500000000005, 59.89) [c] 
(1337.4000000000005, 60.37) [c] 
(1336.6500000000005, 60.97) [c] 
(1335.4500000000005, 61.5) [c] 
(1334.8000000000004, 62.02) [c] 
(1334.6000000000004, 62.07) [c] 
(1334.4000000000003, 62.28) [c] 
(1334.3500000000004, 62.61) [c] 
(1333.5000000000005, 63.33) [c] 
(1333.1500000000005, 63.48) [c] 
(1332.7500000000005, 63.67) [c] 
(1332.1500000000005, 63.78) [c] 
(1332.1000000000006, 64.09) [c] 
(1331.6500000000005, 64.32) [c] 
(1331.4000000000005, 64.92) [c] 
(1331.1500000000005, 65.22) [c] 
(1330.7500000000005, 65.33) [c] 
(1329.8000000000004, 65.76) [c] 
(1329.7000000000005, 65.95) [c] 
(1328.7500000000005, 66.09) [c] 
(1327.7500000000005, 66.31) [c] 
(1327.7000000000005, 67.91) [c] 
(1327.4500000000005, 68.71) [c] 
(1327.1500000000005, 69.04) [c] 
(1326.6500000000005, 70.6) [c] 
(1326.1500000000005, 71.11) [c] 
(1326.0500000000006, 71.44) [c] 
(1325.0500000000006, 71.78) [c] 
(1324.6000000000006, 72.27) [c] 
(1324.0000000000007, 72.73) [c] 
(1323.4500000000007, 72.8) [c] 
(1323.3500000000008, 73.53) [c] 
(1322.6500000000008, 73.65) [c] 
(1322.6000000000008, 75.9) [c] 
(1322.0500000000009, 76.45) [c] 
(1321.8000000000009, 77.04) [c] 
(1321.0500000000009, 77.62) [c] 
(1320.450000000001, 77.64) [c] 
(1320.200000000001, 79.04) [c] 
(1319.950000000001, 80.58) [c] 
(1319.700000000001, 80.79) [c] 
(1319.5500000000009, 81.44) [c] 
(1319.1000000000008, 82.45) [c] 
(1318.750000000001, 83.39) [c] 
(1318.6000000000008, 84.68) [c] 
(1318.250000000001, 85.29) [c] 
(1317.700000000001, 86.47) [c] 
(1317.500000000001, 86.5) [c] 
(1317.150000000001, 88.12) [c] 
(1317.100000000001, 88.57) [c] 
(1316.7500000000011, 88.62) [c] 
(1316.4000000000012, 89) [c] 
(1316.2000000000012, 89.03) [c] 
(1316.1000000000013, 89.25) [c] 
(1316.0000000000014, 89.98) [c] 
(1315.6000000000013, 90.27) [c] 
(1315.4500000000012, 92.58) [c] 
(1315.1500000000012, 92.91) [c] 
(1315.1000000000013, 94.79) [c] 
(1315.0000000000014, 94.89) [c] 
(1314.9000000000015, 95.23) [c] 
(1314.8500000000015, 95.4) [c] 
(1314.5500000000015, 96.88) [c] 
(1314.1000000000015, 97.28) [c] 
(1313.1500000000015, 98.56) [c] 
(1312.8500000000015, 98.74) [c] 
(1312.5000000000016, 100.9) [c] 
(1312.4500000000016, 102.9) [c] 
(1312.3000000000015, 103.6) [c] 
(1312.1500000000015, 104.7) [c] 
(1311.9500000000014, 107.8) [c] 
(1311.8500000000015, 108.5) [c] 
(1311.7000000000014, 108.9) [c] 
(1311.5500000000013, 109.5) [c] 
(1311.5000000000014, 109.8) [c] 
(1311.2000000000014, 113.3) [c] 
(1310.8000000000013, 113.6) [c] 
(1310.5000000000014, 113.8) [c] 
(1310.4000000000015, 119.9) [c] 
(1310.2000000000014, 122.8) [c] 
(1309.8500000000015, 123.9) [c] 
(1309.7500000000016, 124.9) [c] 
(1309.4000000000017, 130.4) [c] 
(1309.3500000000017, 132.3) [c] 
(1308.8500000000017, 132.7) [c] 
(1308.7000000000016, 134.5) [c] 
(1308.2500000000016, 135.1) [c] 
(1307.8500000000015, 138.6) [c] 
(1307.6500000000015, 139.4) [c] 
(1307.3000000000015, 140.4) [c] 
(1307.2000000000016, 141.4) [c] 
(1306.9000000000017, 142.5) [c] 
(1306.8500000000017, 144.6) [c] 
(1306.6000000000017, 144.8) [c] 
(1306.2500000000018, 145.9) [c] 
(1305.8500000000017, 146.9) [c] 
(1305.6000000000017, 154.6) [c] 
(1304.9000000000017, 155) [c] 
(1304.4500000000016, 159.7) [c] 
(1304.0500000000015, 164.2) [c] 
(1303.1500000000015, 164.9) [c] 
(1303.1000000000015, 165.1) [c] 
(1303.0000000000016, 171) [c] 
(1301.9000000000017, 177) [c] 
(1301.3500000000017, 181.4) [c] 
(1301.2000000000016, 185.4) [c] 
(1301.0500000000015, 185.6) [c] 
(1300.0000000000016, 188.4) [c] 
(1299.9000000000017, 194.1) [c] 
(1299.5500000000018, 200.3) [c] 
(1298.6000000000017, 203) [c] 
(1298.2000000000016, 204.1) [c] 
(1297.6500000000017, 218.1) [c] 
(1296.5500000000018, 222) [c] 
(1296.5000000000018, 222.6) [c] 
(1295.8000000000018, 233.6) [c] 
(1294.9000000000017, 234.7) [c] 
(1294.8500000000017, 235) [c] 
(1294.8000000000018, 235.3) [c] 
(1294.1000000000017, 237.4) [c] 
(1294.0000000000018, 239.3) [c] 
(1293.7000000000019, 245.2) [c] 
(1293.5500000000018, 246.8) [c] 
(1293.1500000000017, 257.3) [c] 
(1293.1000000000017, 261.5) [c] 
(1293.0500000000018, 263.2) [c] 
(1292.9500000000019, 267.7) [c] 
(1292.900000000002, 269.7) [c] 
(1292.5000000000018, 280.7) [c] 
(1291.8000000000018, 284.7) [c] 
(1291.1000000000017, 286.2) [c] 
(1291.0500000000018, 287.6) [c] 
(1290.3500000000017, 290) [c] 
(1290.0500000000018, 291.9) [c] 
(1289.0000000000018, 294) [c] 
(1288.3000000000018, 323) [c] 
(1287.9500000000019, 337.7) [c] 
(1287.2500000000018, 337.9) [c] 
(1286.8500000000017, 341.8) [c] 
(1286.8000000000018, 366.7) [c] 
(1286.7500000000018, 396.6) [c] 
(1286.0500000000018, 396.8) [c] 
(1285.7000000000019, 398.9) [c] 
(1285.0000000000018, 404.2) [c] 
(1284.7000000000019, 490.4) [c] 
(1284.0000000000018, 549.9) [c] 
}}}{legend pos=north east}
\end{center}
\vfill
\end{frame}



\begin{frame}[fragile]
\frametitle{Lessons}
\vfill\pause
\begin{itemize}
	\item The initial bounds are not a problem 
	\vfill
	\begin{itemize}
		\item LNS, local search, GNN, there are many things to try
	\end{itemize}

	\vfill\pause
	\item The scalability is more worrying\pause, but:
	\vfill
	\begin{itemize}
		\item The main issue is the number of binary disjuncts (i.e., $\geq 12000$ in \texttt{swv16})

		\vfill\pause
		\begin{itemize}
			\item We need to scan them all to select a choicepoint (easy fix with a Heap structure)

			\vfill\pause
			\item \memph{We need to propagate them to detect unfeasible difference logic constraints (may require a dedicated algorithm)}
		\end{itemize}
\vfill\pause
		\item Some gains can be made by making Edge-Finding more incremental
	\end{itemize}
\end{itemize}
\vfill
\end{frame}




\section{Future Work}

\begin{frame}
\frametitle{Transition Times (Traveling Salesman Problem)}

\pause
\begin{itemize}
	\item The \memph{Minimum Spanning Tree} gives a good lower bound

	\vfill\pause
	\item But spanning tree of which graph? (initially all edges may be feasible)

	\vfill\pause
	\begin{itemize}
		\item We need to know which edges are not possible anymore

		\vfill\pause
		\item \memph{Transitive Reduction}: can be computed incrementally as easily as the transitive closure

		\vfill\pause
		\item Computing a global lower bound in this way is easy. \pause Two challenges:

		\vfill
		\begin{itemize}
			\item Computing ``local'' bounds (e.g., only among the predecessors of a tasks): how many and which ones?

			\vfill
			\item Computing an explanation is not trivial: \memph{deleted edges} (but hopefully not all of them)
		\end{itemize}
	\end{itemize}
\end{itemize}

\end{frame}


\begin{frame}
\frametitle{Optional Taks / Flexible Scheduling}

\begin{itemize}
	\item Easy to model with Boolean variables

	\vfill\pause
	\item Easy to propagate (say) the Edge-Finding rule: just ignore optional tasks

	\vfill\pause
	\item Challenge: deduce from (say) the Edge-Finding rule that a given optional task is not possible

	\vfill\pause
	\begin{itemize}
		\item Tentatively insert optional tasks in the theta tree

		\vfill
		\begin{itemize}
			\item If it would entail an overload, the task cannot exist

			\vfill
			\item Otherwise, ignore it (remove it from the theta tree)
		\end{itemize}
	\end{itemize}

	\vfill\pause
	\item Change every resource propagator to handle optional tasks in this way
\end{itemize}

\end{frame}




\begin{frame}
\frametitle{Cumulative Resources}

\vfill\pause

\begin{itemize}
	\item The same principles apply

	\vfill
	\item More complex model:

	\vfill\pause
	\begin{itemize}
		\item With \memph{disjunctive} resources, we only need a Boolean disjunct for every pair of \memph{tasks}

		\vfill
		\item With \memph{cumulative} resources we need a disjunct for every pair of \memph{events} (starts and ends of tasks)
	\end{itemize}

	\vfill\pause
	\item More complex propagators and explanations (there is a lot of prior work)

\end{itemize}

\vfill

\end{frame}




\begin{frame}
\frametitle{Long Term}

\vfill\pause
\begin{itemize}
	\item General Constraint Programming (not difficult, just tedious)

	\vfill\pause
	\item More complex theories/semantic for Boolean variables

	\vfill\pause
	\begin{itemize}
		\item Linear Arithmetic: instead of a difference logic constraint, Boolean variables can be given a general linear inequation as semantic

		\vfill
		\item Simplex reasoning instead of Bellman-Ford
	\end{itemize}
\end{itemize}

\vfill

\end{frame}









%% Cross between a CP/SAT solver and a SMT solver

%%% difference logic <-> temporal graph 

%% Disjunctive non-preemptive scheduling

%%% I believe it is already better than CPO on some very combinatorial instances

%%% But it is easy to find instances where CPO is better

%% Transition times TSP/VRP

%% Optional tasks / flexible scheduling

%% Cumulative resources

%% General constraint programming

%% SAT Modulo Linear Arithmetic

% \end{frame}


%% x - y <= k OR x <= p
%% y - x <= -k-1 AND -x <= -p-1









\end{document}




